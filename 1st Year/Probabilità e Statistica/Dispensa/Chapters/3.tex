\section{Elementi di probabilità}
La probabilità è una branca della matematica che si occupa dello studio e descrizione degli \textbf{esperimenti aleatori}, ovvero delle inferenze il cui esito non è del tutto prevedibile. Esistono due metodi per l'espressione del concetto di probabilità:
\begin{itemize}
	\item \textbf{Approccio frequentista}; Determinazione della probabilità mediante esperimenti ripetuti. Risulta quindi come il rapporto fra il totale in cui si è esperito un esito e il totale degli esperimenti.
	\item \textbf{Approccio soggettivista}; Dove la probabilità è vista come un livello di fiducia nel verificarsi di un dato esito. È na roba da filosofi, non fa per noi.
\end{itemize}
Abbiamo parlato di una totalità di esperimenti; questi vengono formalmente chiamati \textbf{eventi} $E$ e detengono informazioni riguardo al loro esito. Ogni evento è un sottoinsieme dello \textbf{spazio campionario} $S$, che li comprende tutti.
\begin{eg}
	\textbf{Spazio campionario}\newline
	Un esempio di spazio campionario è dato dalla totalità dei valori delle facce di un dado, mentre gli eventi sono i singoli valori usciti da un esperimento.
	\begin{center}
		$S = \{1, 2, 3, 4, 5, 6\}$, $E = \{4\}$
	\end{center}
\end{eg}
Alle operazioni logiche sulle affermazioni corrispondono quelle insiemistiche,le quali si mostrano mediante diagrammi di eulero venn. Siano $A, B \subseteqq S$ due eventi:
\begin{itemize}
	\item \textbf{Intersezione}\newline
	Quando "Avviene $A$ e avviene $B$"
	% TODO ----- AGGIUNGI DIAGRAMMA
	\item \textbf{Unione}\newline
	Quando "Avviene $A$ oppure $B$"
	% TODO ----- AGGIUNGI DIAGRAMMA
	\item \textbf{Sottrazione}\newline
	Quando "Avviene $A$, ma non $B$"
	% TODO ----- AGGIUNGI DIAGRAMMA
	\item \textbf{Complementare}\newline
	Quando "Non avviene $A$"
\end{itemize}
Inoltre, se gli insiemi $A,B$ sono tali che la loro intersezione sia vuota, si dicono \textbf{incompatibili}.

%

\section{Calcolo della probabilità}
Fortunatamente esiste una concezione standard sulle caratteristiche assunte dalla probabilità. Associamo infatti ad ogni evento $E$ sullo spazio campionario $S$, un valore denotato con $P(E)$, detto \textbf{probabilità dell'evento} $E$. Il comportamento della funzione è dato dai seguenti \textbf{assiomi di Kolmogorov}:
\begin{definition}
	\textbf{Assiomi di Kolmogorov}\newline
	\begin{enumerate}
		\item $P(A)$ è un valore compreso fra $0$ e $1$.
		\item $P(S) = 1$.
		\item Se $A$ e $B$ sono incompatibili, allora $P(A \cup B) = P(A) + P(B)$.
	\end{enumerate}
	Questi detengono inoltre le seguenti proprietà:
	\begin{itemize}
		\item Siano $A, B$ due eventi tali che $A \subseteqq B$, allora:
		\begin{center}
			$\begin{cases}
				B = S \implies S / A = A^c \implies P(A^c) = 1 - P(A)\\
				P(B / A) = P(B) - P(A)
			\end{cases}$
		\end{center}
		\item Se $A_1, A_2, ..., A_k$ sono eventi a due a due incompatibili, quindi disgiunti, allora:
		\begin{center}
			\[P\left(\bigcup_{i = 1}^k A_i\right) = \sum_{i = 1}^{k} P(A_i)\]
		\end{center}
		\item Siano $A, B$ due eventi generici, allora:
		\begin{center}
			$P(A \cup B) = P(A) + P(B) - P(A \cap B)$
		\end{center}
	\end{itemize}
\end{definition}
Un primo caso di studio per la probabilità è il suo calcolo ad \textbf{esiti equiprobabili}; ciò significa che ogni evento ha la stessa chance di avvenire rispetto agli altri. L'esempio classico è il lancio di un dado; implicando che questo non sia truccato, ogni faccia ha $\frac{1}{6}$ di possibilità di uscire. Formalmente la definiamo con la seguente scrittura:
\begin{center}
	$P(A) = \dfrac{|A|}{|S|}$
\end{center}
\begin{eg}
	Quali sono le probabilità che lanciando due volte un dado esca il valore 7?\newline
	Innanzitutto dobbiamo chiederci quale sia lo spazio campionario e gli eventi. Sappiamo che è un dado, quindi avremo rispettivamente:
	\begin{itemize}
		\item $S = \{1, 2, 3, 4, 5, 6\}$
		\item $E_1 = \{1\}, ..., E_6 = \{6\}$
	\end{itemize}
	Ora, potremmo fare bruteforcing facendoci del male, ma il trucco per questi esercizi (entro certi limiti) è disegnare una tabella dei risultati, prendere il totale di quante volte si presenta il valore richiesto e poi applicare la formula dell'approccio frequentista. In questo caso:
	\begin{center}
		\begin{tabular}{|c||c|c|c|c|c|c|}
			\hline
			- & 1 & 2 & 3 & 4 & 5 & 6\\
			\hline
			\hline
			1 & 2 & 3 & 4 & 5 & 6 & 7\\
			\hline
			2 & 3 & 4 & 5 & 6 & 7 & 8\\
			\hline
			3 & 4 & 5 & 6 & 7 & 8 & 9\\
			\hline
			4 & 5 & 6 & 7 & 8 & 9 & 10\\
			\hline
			5 & 6 & 7 & 8 & 9 & 10 & 11\\
			\hline
			6 & 7 & 8 & 9 & 10 & 11 & 12\\
			\hline
		\end{tabular}
	\end{center}
	Notiamo che il valore $7$ compare $6$ volte ed il totale degli esiti ottenibili è $6 \times 6 = 36$. Il risultato sarà dato quindi da:
	\begin{center}
		$\dfrac{6}{36} = \dfrac{1}{6}$, soluzione dell'esercizio.
	\end{center}
\end{eg}

%

\section{Probabilità condizionata}

%

\section{Variabili aleatorie}

%

\section{Distribuzioni congiunte}

%

\section{Classi notevoli di variabili aleatorie}

%

\section{Statistiche campionarie}