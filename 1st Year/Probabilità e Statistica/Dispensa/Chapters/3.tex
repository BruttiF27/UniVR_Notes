\section{Elementi di probabilità}
La probabilità è una branca della matematica che si occupa dello studio e descrizione degli \textbf{esperimenti aleatori}, ovvero delle inferenze il cui esito non è del tutto prevedibile. Esistono due metodi per l'espressione del concetto di probabilità:
\begin{itemize}
	\item \textbf{Approccio frequentista}; Determinazione della probabilità mediante esperimenti ripetuti. Risulta quindi come il rapporto fra il totale in cui si è esperito un esito e il totale degli esperimenti.
	\item \textbf{Approccio soggettivista}; Dove la probabilità è vista come un livello di fiducia nel verificarsi di un dato esito. È na roba da filosofi, non fa per noi.
\end{itemize}
Abbiamo parlato di una totalità di esperimenti; questi vengono formalmente chiamati \textbf{eventi} $E$ e detengono informazioni riguardo al loro esito. Ogni evento è un sottoinsieme dello \textbf{spazio campionario} $S$, che li comprende tutti.
\begin{eg}
	\textbf{Spazio campionario}\newline
	Un esempio di spazio campionario è dato dalla totalità dei valori delle facce di un dado, mentre gli eventi sono i singoli valori usciti da un esperimento.
	\begin{center}
		$S = \{1, 2, 3, 4, 5, 6\}$, $E = \{4\}$
	\end{center}
\end{eg}
Alle operazioni logiche sulle affermazioni corrispondono quelle insiemistiche,le quali si mostrano mediante diagrammi di eulero venn. Siano $A, B \subseteqq S$ due eventi:
\begin{itemize}
	\item \textbf{Intersezione}\newline
	Quando "Avviene $A$ e avviene $B$"
	% TODO ----- AGGIUNGI DIAGRAMMA
	\item \textbf{Unione}\newline
	Quando "Avviene $A$ oppure $B$"
	% TODO ----- AGGIUNGI DIAGRAMMA
	\item \textbf{Sottrazione}\newline
	Quando "Avviene $A$, ma non $B$"
	% TODO ----- AGGIUNGI DIAGRAMMA
	\item \textbf{Complementare}\newline
	Quando "Non avviene $A$"
\end{itemize}
Inoltre, se gli insiemi $A,B$ sono tali che la loro intersezione sia vuota, si dicono \textbf{incompatibili}.

%

\section{Calcolo della probabilità}
La probabilità è una funzione che riceve un evento $A$ o direttamente lo spazio campionario e provoca i seguenti risultati:
\begin{enumerate}
	\item $P(A)$ è un valore compreso fra $0$ e $1$
	\item $P(S) = 1$
	\item Se $A$ e $B$ sono incompatibili, allora $P(A \cup B) = P(A) + P(B)$
\end{enumerate}
Questi vengono chiamati \textbf{assiomi di Kolmogorov} e ne traiamo le seguenti proprietà:

% TODO ----- INSERISCI LE PROPRIETÀ

%

\section{Probabilità condizionata}

%

\section{Variabili aleatorie}

%

\section{Distribuzioni congiunte}

%

\section{Classi notevoli di variabili aleatorie}

%

\section{Statistiche campionarie}