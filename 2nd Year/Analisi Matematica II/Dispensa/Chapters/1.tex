\section{Modelli differenziali}
Passiamo da uno studio numerico ad uno particolarmente più astratto. Analisi matematica 2 è una materia molto importante non solo per consolidare le nozioni del predecessore che compongono il toolset necessario per lavorare qui, ma anche perché renderà il resto delle materie di stampo matematico più comprensibili e approcciabili.\par
Iniziamo riprendendo le funzioni; ne hai viste di ogni tipo, da sole, composte, inverse etc... e adesso lavorerai con famiglie di funzioni.
\begin{definition}
	\textbf{Equazione differenziale ordinaria}\newline
	Definiamo equazione differenziale di ordine $n$ un'equazione del tipo:
	\begin{center}
		$F(t, y, y', y'', ..., y^n)$
	\end{center}
	Dove $y(t)$ è la funzione incognita ed $F$ è una funzione assegnata delle $n+2$ variabili $(t, y, y', y'', ..., y^n)$ a valori reali. Diremo inoltre il suo \textbf{ordine} l'ordine massimo di derivata che compare.
\end{definition}
Come potrai immaginare, l'esistenza di un'equazione implica l'esistenza di una soluzione. Non sarà bello, ma per ottenerla sarà necessario l'aiuto degli integrali.
\begin{definition}
	\textbf{Curva integrale dell'equazione differenziale}\newline
	Diciamo curva integrale o soluzione dell'equazione nell'intervallo $I \subset \mathbb{R}$ una funzione $\phi(t)$, definita almeno in $I$ e a valori reali, per cui risulti:
	\begin{center}
		$F(t, \phi(t), \phi'(t), \phi''(t), ..., \phi^n(t)) = 0$, $\forall t \in I$
	\end{center}
	In merito, ci servirà ottenere l'\textbf{integrale generale}, ovvero una formula che rappresenti la famiglia di tutte le soluzioni dell'equazione.
\end{definition}
Concetti di base ottenuti; benvenuto in analisi matematica 2.

%

\section{Equazioni differenziali del primo ordine}
\begin{definition}
	\textbf{Equazione differenziale del primo ordine}\par
	\noindent Si dice tale qualunque equazione differenziale si presenti con un'incognita, una funzione e una singola derivata. Avrà infatti la forma:
	\begin{center}
		$F(t, y, y') = 0$, con $F$ funzione assegnata di $t, y, y'$ a valori reali
	\end{center}
	Tali equazioni si risolvono attraverso l'integrazione delle stesse. Queste prenderanno una forma generale del tipo:
	\begin{center}
		$y'(t) = f(t)$, con soluzioni $y(t) = \int f(t)dt+c$, dove $c \in \mathbb{R}$
	\end{center}
\end{definition}
\noindent Essendo che l'equazione ha infinite curve integrali distinte dalla costante arbitraria $c$, ne traiamo che l'insieme delle soluzioni di un'equazione differenziale del primo ordine è costituito da più funzioni, dipendenti dal parametro $c: t \to \phi(t;c)$. Questa scrittura è l'\textbf{integrale generale} menzionato prima.\par
Qua sorge una domanda importante: sarebbe possibile determinarne una curva integrale precisa? Il quesito ha soluzione nell'aggiunta della \textbf{condizione di Cauchy}, da qui in poi riferita come \textit{restrizione}. Infatti, applicandola all'integrale generale di un'equazione differenziale, ci consente di determinare il valore della costante arbitraria in sua funzione. Questa aggiunta forma il sovramenzionato costrutto:
\begin{definition}
	\textbf{Problema di Cauchy}\newline
	Chiamiamo problema di Cauchy il processo risolutivo di un'equazione differenziale che detiene una condizione supplementare, con lo scopo ultimo di trovare una soluzione precisa. Assume la forma:
	\begin{center}
		$\begin{cases}
			F(t, y, y') = 0\\
			y(t_0) = y_0
		\end{cases}$
	\end{center}
\end{definition}
\noindent Osserviamo le modalità di risoluzione dei suddetti problemi. Per quanto riguarda il lavoro sulle equazioni differenziali si tratta sempre di determinare prima la curva integrale, per poi applicare ad essa la restrizione del problema.
\begin{eg}
	\textbf{Risolvere il seguente problema di Cauchy}
	\begin{center}
		$\begin{cases}
			y' = -e^{-x}\\
			y(0) = 3
		\end{cases}$
	\end{center}
	Per prima cosa è necessario trovare la curva integrale dell'equazione differenziale, quindi procediamo con l'integrazione.
	\begin{center}
		\[\int -e^{-x} dx = e^{-x}+c\]
	\end{center}
	Abbiamo trovato la soluzione generale. Non basta: troviamo quel valore della costante $c$ in funzione della condizione per far sì che torni. Per applicarla, sostituiamo alle $x$ il valore $0$ e poniamo l'equazione uguale a $3$:
	\begin{center}
		$e^{-0} + c = 3 \implies 1+c = 3 \implies c = 2$		
	\end{center}
	Abbiamo trovato il valore richiesto. Sostituiamolo alla costante nella soluzione generale dell'equazione differenziale per trovare la specifica.
	\begin{center}
		$Soluzione: y(x) = e^{-x} + 2$
	\end{center}
\end{eg}

\noindent Lavorando con questo costrutto è cosa buona e giusta ordinare gli elementi dell'equazione. La forma standard più chiara, detta \textbf{forma normale}, vede la derivata uguale al resto dei dati, ovvero:
\begin{center}
	$y'(t) = f(t, y(t))$
\end{center}
Il procedimento osservato per i problemi di Cauchy è generale e varrà per tutti gli argomenti ad esso inerenti, seppur si possano trovare alcune differenze per i casi particolari trattati nelle successive sezioni.

%

\subsection{Equazioni a variabili separabili}
\begin{definition}
	\textbf{Equazioni differenziali a variabili separabili}\par
	\noindent Questo è un caso particolare di equazioni differenziali ordinarie del primo ordine. La derivata è data dal prodotto di due funzioni $a, b$, la prima continua su un intervallo $I \subset \mathbb{R}$ e la seconda su un intervallo $J \subset \mathbb{R}$. Si presentano nella forma:
	\begin{center}
		$y' = a(t)b(y)$
	\end{center}	
\end{definition}
Da questa definizione, vedendo che si parla di prodotti, è necessario distinguere due istanze di lavoro:
\begin{itemize}
	\item Se il numero $\overline{y}$ è una soluzione dell'equazione $b(y) = 0$, la funzione costante $y(t) = \overline{y}$ è una soluzione valida, detta \textbf{integrale singolare}. Il secondo membro si annulla perché $b(\overline{y}) = 0$, come anche il primo, perché la derivata di una costante è $0$.
	\item Supponendo $b(y) \neq 0$ abbiamo il seguente caso più comune ed elaborato, ovvero la forma:
	\begin{center}
		$a(t) = \dfrac{y'}{b(y)}$
	\end{center}
\end{itemize}
\noindent È necessario ampliare il discorso sul secondo caso. Prendiamo un'ipotetica soluzione $y(t)$, allora l'equazione soddisferà la seguente identità, la quale prendendo gli integrali definiti di ambo i membri fa ottenere:
\begin{center}
	\[\dfrac{y'(t)}{b(y(t))} = a(t) \implies \int\dfrac{y'(t)}{b(y(t))}dt = \int a(t)dt+c\]
\end{center}
Nell'integrale di sinistra è consentito effettuare un cambio di variabile $y = y(t); dy = y'(t)dt$, ottenendo:
\begin{center}
	\[\int \dfrac{dy}{b(y)} = \int a(t)dt+c\]
\end{center}
Che risulta essere l'integrale generale dell'equazione differenziale.\par
Inoltre, se la funzione $B(y)$ è una primitiva di $\frac{1}{b(y)}$ e $A(t)$ una primitiva di $a(t)$, allora l'integrale generale è assegnato dalla seguente equazione in \textbf{forma implicita}:
\begin{center}
	$B(y) = A(t)+c$, con $c$ costante arbitraria.
\end{center}
\noindent Adesso ragioniamo; in che modo il problema di Cauchy si adatta a questo tipo di equazioni? Abbiamo una forma apposita:
\begin{theorem}
	\textbf{Problema di Cauchy per ED a variabili separabili}
	\begin{center}
		$\begin{cases}
			y' = a(t)b(y)\\
			y(t_0) = y_0
		\end{cases}$
	\end{center}
	Dove $a$ è continua in un intorno $I$ di $t_0$ e $b$ è continua in un intorno $J$ di $y_0$. Esisteranno quindi:
	\begin{itemize}
		\item Intorno $I' \subset I$ di $t_0$.
		\item Funzione continua $y$ definita su $I'$.
		\item Funzione derivata $y'$ continua su $I'$, soluzione del problema.
	\end{itemize}
	Inoltre, se anche $b'$ è una funzione continua su $J$ oppure $b$ ha un rapporto incrementale\footnote{Inserire spiegazione} limitato in J (anche se non è derivabile), allora la soluzione è \textbf{unica}.
\end{theorem}

% TODO ----- Inserire nota su cos'è un rapporto incrementale

\begin{eg}
	\textbf{Risolvere il problema di Cauchy}
	\begin{center}
		$\begin{cases}
			y' = ty^3\\
			y(0) = 1
		\end{cases}$
	\end{center}
	\begin{enumerate}
		\item \textbf{Calcola l'integrale generale dell'equazione differenziale}\newline
		Anzitutto, poniamo $y \neg 0$, poiché a noi serve trovare l'integrale generale, non quello singolare. Procediamo con la separazione delle variabili:
		\begin{center}
			$y' = ty^3 \implies \dfrac{dy}{dt} = ty^3 \implies \dfrac{dy}{y^3} = t dt$
		\end{center}
		Adesso possiamo procedere ad integrare le due parti distinte, quindi:
		\begin{center}
			\[\int\dfrac{dy}{y^3} = \int tdt+c \implies -\dfrac{1}{2y^2} = \dfrac{t^2}{2}+c\]
		\end{center}
		Effettuiamo i passaggi algebrici per ricavare la funzione y:
		\begin{itemize}
			\item $-\dfrac{1}{2y^2} = \dfrac{t^2}{2}+c \implies \left(-\dfrac{1}{2y^2}\right)^{-1} = \left(\dfrac{t^2}{2}+c\right)^{-1}$
			\item $-2y^2 = \dfrac{2}{t^2+2c} \implies -y^2 = \dfrac{1}{t^2+2c} \implies y^2 = -\dfrac{1}{t^2-2c}$
		\end{itemize}
		Che porta infine ad avere la soluzione generale:
		\begin{center}
			$y = \pm\dfrac{1}{\sqrt{-2c-t^2}}$
		\end{center}
		\item \textbf{Imponi la condizione di Cauchy}\newline
		Notiamo che il valore posto della condizione è positivo, di conseguenza la soluzione di $y$ è tale che appartiene all'intervallo $(0, +\infty)$ e considereremo solamente la radice positiva della soluzione generale.\par
		Detto ciò, determiniamo il valore di $y(t)$ applicando la condizione; per renderci la vita ulteriormente facile, applichiamo una piccola sostituzione sulla costante. D'altronde è arbitraria. Dopodiché diamole il valore della condizione.
		\begin{center}
			for $k = -2c \implies y(t) = \dfrac{1}{\sqrt{k-t^2}} \implies y(t) = \dfrac{1}{\sqrt{1-t^2}}$
		\end{center}
		Soluzione specifica dell'equazione.
	\end{enumerate}
\end{eg}

% TODO ----- Rielabora da qua

\subsection{Equazioni lineari del primo ordine}











Tipo di equazioni differenziali ordinarie, dove $F$ è lineare in $y$ e $y'$, le cui soluzioni sono espresse mediante uno spazio vettoriale di dimensione 1. Si presentano nella forma:
\begin{center}
	$a_1(t)y'(t) + a_2(t)y(t) = g(t)$
\end{center}
Con $a_1$, $a_2$ funzioni continue su un intervallo. L'equazione può presentarsi in due forme diverse, date $a$ ed $f$ funzioni continue sull'intervallo $I \subset R$:
\begin{itemize}
	\item \textbf{Equazione completa o forma normale}; scrivibile se il coefficiente $a_1(t)$ non si annulla.
	\begin{center}
		$y'(t) + a(t)y(t) = f(t)$
	\end{center}
	Otteniamo la sua soluzione aggiungendone una particolare alla sua curva integrale, definita da un valore conosciuto della costante arbitraria $c$.
	\item \textbf{Equazione omogenea}; ottenibile ponendo $f(t) = 0$, presenta la seguente forma:
	\begin{center}
		$z'(t) + a(t)z(t) = 0$
	\end{center}
	Notiamo che le $y$ sono sostituite dalle $z$ per chiarire il tipo di equazione presa in esame.
\end{itemize}
Capiamo quindi che il procedimento da attuare si compone di due passi principali: prima la ricerca di un integrale generale dell'equazione omogenea, poi trovare la soluzione particolare da aggiungere a quella completa. Vediamo un esempio:
\begin{eg}
	\textbf{Risoluzione equazione differenziale lineare del primo ordine}\par
	\noindent Consideriamo la seguente semplice equazione:
	\begin{center}
		$y'(t) + 3y(t) = 2t$, dove $a(t) = 3$, $f(t) = 2t$ e la primitiva $A(t) = 3t$
	\end{center}
	\begin{enumerate}
		\item \textbf{Ricerca della soluzione generale dell'equazione omogenea}\par
		\noindent L'equazione omogenea ha forma $y'(t)+a(t)y(t) = 0$, di conseguenza prenderemo quella iniziale ponendo $f(t) = 0$, per poi moltiplicare tutto per l'esponenziale alla primitiva $A(t)$ ed infine ricavare $z'(t)$
		\begin{center}
			$z'(t) + 3z(t) = 0 \implies z'(t)e^{3t} + 3z(t)e^{3t} = 0 \implies z'(t)e^{3t} = 0$
		\end{center}
		Avendo ottenuto una funzione differenziale, necessitiamo di integrarla per ottenere quello che ci serve.
		\begin{center}
			\[\int z'(t)e^{3t} dt = z(t)e^{3t}+c \implies z(t) = ce^{-3t}\]
		\end{center}
		Come dici? Hai sbagliato il segno? Forse devo ricordarti il significato di costante \textbf{arbitraria}. Può essere quello che voglio, quando lo voglio.
		\item \textbf{Metodo di variazione delle costanti per trovare il valore di c}\par 
		\noindent Il metodo consiste nel trasformare la costante in una funzione. Abbiamo che la soluzione appena trovata è di forma $\overline{y} = ce^{-3t}$, la quale dovrà prendere il posto delle $y$ nell'equazione iniziale, dando la forma:
		\begin{center}
			$e^{-3t}(c'(t) -3c(t)) + 3c(t)e^{-3t} = 2t \implies e^{-3t}c'(t) = 2t$
		\end{center}
		Ora dobbiamo trovare la funzione $c(t)$, quindi spostiamo gli elementi ed integriamo le due parti.
		\begin{center}
			\[c'(t) = 2te^{3t} \implies c(t) = \int 2te^{3t} dt \implies c(t) = \dfrac{2}{3}te^{3t} - \dfrac{2}{9}e^{3t}\]
		\end{center}
		Ora, sperando che quello che sto per fare non sia un abusivismo notazionistico, rimuovo l'esponenziale dall'equazione, siccome la funzione della costante è arbitraria e può essere quello che voglio, ottenendo:
		\begin{center}
			$c(t) = \dfrac{2}{3}t - \dfrac{2}{9}$
		\end{center}
		\item \textbf{Composizione dell'integrale generale dell'equazione completa}\par 
		\noindent Adesso abbiamo tutte le equazioni necessarie per scrivere la soluzione:
		\begin{itemize}
			\item $z(t) = ce^{-3t}$
			\item $c(t) = \dfrac{2}{3}t - \dfrac{2}{9}$
		\end{itemize}
		\begin{center}
			Soluzione: $y(t) = ce^{-3t} + \dfrac{2}{3}t - \dfrac{2}{9}$
		\end{center}
	\end{enumerate}
\end{eg}

%

\section{Equazioni differenziali lineari del secondo ordine}
Un'equazione differenziale è tale se si presenta nella forma:
\begin{center}
	$a_2(t)y'' + a_1(t)y' + a_0(t)y = f(t)$
\end{center}
\noindent Dove i coefficienti $a_i$ ed il termine noto $f$ sono funzioni definite in un intervallo $I$ dove sono continue. Come in quelle del primo ordine, se il termine noto è $0$, l'equazione si dice \textbf{omogenea}, altrimenti è \textbf{completa}. Inoltre, se le funzioni $a_i$ sono costanti, la chiameremo \textbf{a coefficienti costanti}, in caso contrario sarà a \textbf{coefficienti variabili}. Infine, se $a_2(t) \equiv 1$ si dirà in \textbf{forma normale}.\par
La chiamiamo lineare perché introducendo un operatore apposito $L$ al primo membro abbiamo che risulta essere un operatore lineare tra gli spazi di funzioni\footnote{Si tratta del concetto di \textbf{classi di funzioni}. La classe C di una funzione indica l'appartenenza della stessa all'insieme delle funzioni derivabili con continuità per n numero di volte.}. Notiamo infatti che:
\begin{center}
	$L:C^2(I) \to C^0(I)$\par 
	$L:y \to Ly$
\end{center}

% TODO ----- Ricomincia da qua. Da rivedere anche il testo sovrastante della sezione.



\subsection{Equazioni lineari omogenee a coeffcienti costanti}
\subsection{Equazioni lineari non omogenee a coefficienti costanti}
Comprende "Metodo di somiglianza"