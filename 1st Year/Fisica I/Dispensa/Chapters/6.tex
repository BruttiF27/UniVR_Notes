\section{Pressione e profondità}
Fino ad ora abbiamo studiato i meccanismi della materia fisica. Con l'introduzione dei \textbf{fluidi} è necessario cambiare leggermente forma mentis, pur sempre ricordando che le Leggi di Newton regnano sovrane, e quindi introdurre nuovi concetti. Con fluido intendiamo un insieme di molecole disposte in modo casuale, le quali si possono anche allontanare fra di loro; sulle quali è ancora possibile applicare delle forze.\par
Propriamente, non è presente forza di attrito fra i piani delle molecole, infatti una forza è applicabile esclusivamente in senso non parallelo alla superficie del fluido. Negli esercizi verrà poi utilizzato il \textbf{fluido ideale}, ovvero un liquido con le seguenti caratteristiche: \begin{itemize}
	\item \textbf{Non viscoso}, ovvero privo di qualunque forza di attrito.
	\item \textbf{Stazionario}, quindi tutte le parti del fluido che passano in un punto hanno la stessa velocità.
	\item \textbf{Incomprimibile}, dunque la sua densità è costante.
	\item \textbf{Irrotazionale}, semplicemente non ha momento angolare e non rotea.
\end{itemize}
\noindent Notiamo come la superficie del liquido e le forze, grazie alla massa, su di essa esercitate sono strettamente legate. Ciò significa che all'aumentare della superficie del liquido, si dà la possibilità a più massa di esercitare la forza peso. Questa forza viene definita \textbf{pressione} $P$; è un valore scalare misurato in \textbf{Pascal} $[1Pa = 1N/m^2]$ ed è definito come il rapporto fra la forza esercitata $F$ e l'area o superficie $A$ del liquido selezionata: \[P = \frac{F}{A} \implies F = PA\]
\noindent La pressione è un valore che dipende da quanta massa esercita un peso ad una determinata altezza; ciò significa che prendendo una sezione specifica all'interno del liquido, bisognerà considerare la somma delle forze esercitate fino a quel punto\footnote{Insieme, naturalmente, anche alla risultante}. Questa posizione è detta \textbf{profondità}; ne consegue che più scendiamo in profondità, maggiore sarà la pressione percepita.\par
\noindent Ricapitolando, per avere un modello completo delle forze, considerando una sezione del liquido in un contenitore, bisogna conoscere: \begin{itemize}
	\item La \textbf{profondità} $d+h$ della sezione del fluido, con $d$ \textbf{distanza} della sezione dalla superficie e $h$ \textbf{altezza} della forma sezionata.
	\item La \textbf{forza peso} $-P_0A$ esercitata dalla superficie e tutto il liquido sopra la sezione.
	\item La \textbf{forza peso} $-Mg$ esercitata dalla sezione, dove la massa $M = \rho V = \rho Ah$, con $\rho$ \textbf{densità}, $V$ \textbf{volume} e $h$ altezza.
	\item La \textbf{forza risultante} $PA$ che mantiene in equilibrio la sezione.
\end{itemize} 
\noindent La sommatoria di queste forze in gioco deve essere sempre uguale a zero; la scrittura seguente che descrive questa relazione è inoltre utile per ricavare la pressione facilmente: \[PA - P_0A - Mg = 0\implies PA - P_0A - \rho Ahg = 0 \implies P = P_0 + \rho gh\]
\noindent Quest'ultima formula per la pressione è infatti chiamata \textbf{Legge di Stevino}, la quale fa dipendere la pressione in base alla profondità. È grazie a questa considerazione che possiamo spiegare il comportamento dei fluidi in vasi comunicanti. Consideriamo infatti due recipienti collegati; riempiendone uno dei due il liquido si trasferirà anche all'altro, fin quando non raggiungeranno la stessa altezza.\par
Se invece consideriamo due vasi comunicanti con aree diverse $A_1, A_2$, dove la prima è più grande della seconda e poniamo sulle superfici due pistoni che possono esercitare rispettivamente due forze $F_1, F_2$, avremo a che fare con due pressioni diverse, date da: $P_1 = \frac{F_1}{A_1}, P_2 = \frac{F_2}{A_2}$.\par
Quando un pistone esercita una forza il liquido nel suo vaso scenderà ne eserciterà un'altra verso l'alto nel secondo recipiente, ragion per cui, per la Legge di Stevino possiamo dire: \[P = \frac{F_1}{A_1} = \frac{F_2}{A_2}\]
\noindent Le leve idrauliche sfruttano proprio questa formula; esercitando forza da un vaso più piccolo è possibile muovere molto più peso dall'altro, perché cerca sempre di ribilanciare le altezze. Un altro esempio di utilizzo di questa legge è il \textbf{barometro di Torricelli}, che vede supposti due recipienti, il primo, più grande, è riempito di liquido, mentre il secondo, più piccolo e privo di aria, è inserito nell'altro.\par
In assenza di aria, non è esercitata pressione atmosferica; quindi abbassando il recipiente in quello più grande possiamo osservare che il liquido inizia a salire a causa della forza risultante. Viene usato per misurare la pressione in Bar, ma continueremo ad usare il Pascal.

%

\section{Spinta di Archimede}
Se si prova a spingere un pallone sott'acqua, è normale vedere che questo pone resistenza e risale verso l'alto. Questo accade a causa del \textbf{principio di Archimede}, il quale afferma che ogni corpo immerso parzialmente o totalmente in un fluido viene \textbf{spinto} verso l'alto da una forza uguale al peso del fluido spostato dal corpo. È definita come: \[B = \rho_{fluido}\cdot gV \implies B = M_{fluido}\cdot g = F_{fluido}\]
Con questo principio ci è possibile determinare se un corpo galleggerà o affonderà quando immerso in un liquido; basta guardare la densità di entrambi, se l'oggetto è meno denso sale, altrimenti scende. Più nello specifico: \begin{itemize}
	\item \textbf{Corpo completamente immerso}: Il volume $V_{sp}$ spostato sarà uguale a quello del corpo immerso $V_{corpo}$ e l'intensità della spinta è data da: \[B = \rho_{fluido}\cdot gV_{corpo}\]
	\noindent Inoltre, il corpo di massa $M$ e densità $\rho_{corpo}$ esercita una forza peso $F_{corpo} = Mg = \rho_{corpo}\cdot gV_{corpo}$, la cui risultante sarà: \[B - F_{corpo} = (\rho_{fluido} - \rho_{corpo})gV_{corpo}\]
	\item \textbf{Corpo galleggiante}: Il corpo sposta un quantitativo di liquido pari alla sua massa immersa; in questi termini la risultante bilancia l'oggetto facendolo, appunto, galleggiare e possiamo dire quindi: \[F_{corpo} = B \implies \frac{V_{sp}}{V_{corpo}} = \frac{\rho_{corpo}}{\rho_{fluido}}\]
\end{itemize}

%

\section{Dinamica dei fluidi}
Finora abbiamo considerato i fluidi in uno stato di quiete, ma è possibile fare asserzioni anche quando questi sono in movimento ad una data velocità. Supponiamo un tubo le cui due estremità hanno dimensioni diverse; la prima, più piccola $A_1$, mentre la seconda, più grande e posta più in alto $A_2$. Dentro questo tubo scorre un fluido ideale e definiamo il movimento di una sua particella come \textbf{linea di corrente}.\par
Noi sappiamo che in un certo intervallo di tempo si sposta una relativa quantità di fluido da $A_1$ ad $A_2$, ed essendo questo il medesimo concetto visto con le leggi del moto possiamo definire semplicemente l' \textbf{equazione di continuità dei fluidi}: \[A_1v_1 = A_2v_2 = const\]
\noindent Questa espressione stabilisce che per un fluido ideale, in tutti i punti del tubo, è costante il prodotto fra l'area e la velocità del liquido. In particolare, per brevità, definiamo questo prodotto come \textbf{portata}.\newline

\noindent Proviamo adesso a ragionare in termini energetici con la stessa situazione: come avviene la trasformazione dell'energia? Anzitutto possiamo ricavare forza esercitata e area grazie alle due pressioni, infatti: \[F = PA; A = \frac{F}{P}\]
\noindent Come al solito, usiamo la definizione di lavoro e sostituiamo l'espressione della forza dentro all'integrale: \[L = \int F\cdot ds = PA\cdot s \implies PA\cdot\Delta x\]
\noindent Se applichiamo l'integrale al tubo preso in esame, risultano le seguenti due espressioni: \[\begin{cases}
	L = P_1A_1\cdot\Delta x_1\\
	L = -P_2A_2\cdot\Delta x_2
\end{cases}\]
\noindent Il lavoro della seconda estremità è negativo perché la pressione che agisce da quel lato è contraria all'altra; in ogni caso, a noi interessa il calcolo della variazione dell'energia totale: \[E_{tot} = E_{cin} + E_{pot} \implies E_{tot} = \frac{1}{2}mv^2 + mgh\]
\noindent E per ottenerlo bisogna osservare il comportamento del fluido alle estremità del tubo, ricavandone le relative energie: \[\begin{cases}
	E_1 = \frac{1}{2}mv_1^2 + mgh\\
	E_2 = \frac{1}{2}mv_2^2 + mgh
\end{cases}\]
\noindent Molto semplicemente, sottraiamo l'energia alla seconda estremità a quella della prima, lo stesso concetto di energia finale sottratta all'energia iniziale: \[\Delta E = E_2-E_1 = \frac{1}{2}mv_2^2 + mgh - \frac{1}{2}mv_1^2 + mgh = (P_1 - P_2)v\]
\noindent Se eguagliamo le due energie, tramite alcuni passaggi algebrici otteniamo una forma che consente di bilanciare pressione, altezza e fluido:
\begin{equation}
	\begin{split}
		P_1V + \frac{1}{2}mv_1^2 + mgh_1 &= P_2V + \frac{1}{2}mv_2^2 + mgh_2\\
		P_1 + \frac{1}{2}\left(\frac{m}{V}\right)v_1^2 + \frac{m}{V}gh_1 &= P_2 + \frac{1}{2}\left(\frac{m}{V}\right)v_2^2 + \frac{m}{V}gh_2\\
		P_1 + \frac{1}{2}\rho v_1^2 + \rho gh_1 &= P_2 + \frac{1}{2}\rho v_2^2 + \rho gh_2
	\end{split}
\end{equation}
\noindent Quest'ultima formula è infatti chiamata \textbf{equazione di Bernoulli}: \[P_1 + \frac{1}{2}\rho v_1^2 + \rho gh_1 = P_2 + \frac{1}{2}\rho v_2^2 + \rho gh_2 = const\]
\noindent Ci consente di fare ulteriori asserzioni; supponiamo ora infatti un recipiente di altezza $h$, fino ad essa riempito d'acqua. Presenta un foro alla sua base e bisogna calcolare la velocità di uscita del liquido. Qui possiamo usare l'equazione di Bernoulli perché abbiamo dimostrato che la differenza di energia è costante sia per la superficie che per il foro: \[P_1 + \frac{1}{2}\rho v_1^2 + \rho gh_1 = P_0 + \frac{1}{2}\rho v_0^2 + \rho gh_0 \implies gh_1 = v_0^2 \implies v_0 = \sqrt{2gh_1}\]
\noindent La pressione è uguale da ambo le parti perché parliamo di un recipiente circondato dall'atmosfera. Quindi possiamo dire che $P_1=P_0$, rendendoli intercambiabili e semplificabili, stesso destino vale per le densità $\rho$. Si può inoltre approssimare la velocità della superficie del recipiente a zero, perché è praticamente nulla. Inoltre, $gh_0 = 0$, perché la base del recipiente è presa come riferimento nel modello.

%

\section{Esercizi svolti}

\begin{comment}
	```
	# Esempio di esercizio
	Supponiamo di avere un tubo di 2.5cm di acqua che serve per riempire un secchio di 30l. Serve 1min per riempire il secchio.
	Spostiamo il tubo ad altezza 1m da terra e aggiungiamo alla sua estremità dove esce l'acqua un ugello di 0.5cm^2. Calcolare la distanza coperta dall'acqua che esce dal tubo.
	
	Per risolvere l'esercizio consigliabile ragionare come un proiettile con la sua gittata.
	In questi termini possiamo quindi usare le leggi del moto per i movimenti in due dimensioni:
	- y = y_0 + v_0t - 1/2gt^2
	- x = x_0 + v_0t
	
	Abbiamo già la posizione di y che è 1m da terra; ora bisogna trovare il tempo t, per il quale serve v_{0y}.
	Capiamo che v_{0y} = 0, perché l'acqua è sparata in orizzontale. Dunque: \[y = y_0 + v_0t - 1/2gt^2 \implies 1/2gt^2 = 1 \implies t^2 = 2m/g \implies t = \sqrt{2m/g} = 0.45s\]
	
	Per quanto riguarda l'asse delle x, abbiamo x_0 = 0 dato il punto di riferimento; ma ci manca velocità e tempo.
	Per la velocità bisogna calcolare quella della situazione iniziale dell'ugello da 2.5cm. Dopodiché, siccome A_1v_1 = A_2v_2 sarà semplice sostituire. Procediamo:
	Sappiamo che 1L = 1000cm^3, quindi fa passare in 1min 30000cm^3. La portata è dunque \frac{30L * 1000cm^3}{60s} = 500cm^3/s. Possiamo dirlo grazie alla legge di continuità dei fluidi: \[A_1v_1 = A_2v_2 \implies A_1\frac{s_1}{t_1} = A_2\frac{s_2}{t_2}\]
	
	Quindi dal primo tubo con ugello di 2.5cm escono 500cm^3/sec di acqua.
	Adesso bisogna calcolare v_2, che è v_{0x} nella legge del moto scritta prima.
	Ma tu guarda, basta ricavarlo con passaggi algebrici, lmao: \[A_1v_1 = A_2v_2 \implies v_2 = \frac{A_1v_1}{A_2} = \frac{500cm^3/sec}{0.5cm^2} = 1000cm/sec = 10m/sec\]
	Adesso abbiamo tutto per calcolare la posizione: \[x = x_0 + v_0t \implies x = 0 + (10m/s)(0.45s) \implies x = 4.5m\], che è la soluzione.
	```
\end{comment}