\section{Definizione e calcolo}
Il \textbf{Determinante} di una matrice è un numero utile per la descrizione di alcune proprietà algebriche e e geometriche della stessa. Si ottiene di base in questi tre modi:
\begin{definition}
    \textbf{Calcolo del determinante di una matrice}\par
    Chiariamo che serve necessariamente avere una matrice quadrata, dove $n$ indica il numero di righe o colonne.
    \begin{itemize}
        \item Se $n = 1$, la matrice avrà una singola entrata.\par
        $A = (a)$, \textit{det}$A = a$.
        \item Se $n = 2$, la matrice avrà quattro entrate\par$A =$
        $\begin{pmatrix}
            a & b\\
            c & d
        \end{pmatrix}$, \textit{det}$A = ad - bc$.
        \item Se $n \geq 3$, la matrice avrà nove entrate o più. Vale la formula generale\par
        $A = (a_{ij})_{1\leq ij \leq n}$, \textit{det}$A = \sum^{n}_{j = 1}(-1)^{j+1} \times a_{1j} \times $\textit{det}$A_{1j}$.\par
        Dove $A_{1j}$ è la matrice ottenuta da $A$ cancellando la prima riga e la colonna $j$. Tuttavia siccome la scrittura del terzo caso è illeggibile (Cristo ti sfido a capirla), propongo un esempio pratico.
    \end{itemize}
\end{definition}
\begin{eg}
    Calcolo del determinante di una matrice $A$ dove $n = 3$.\par
    \begin{center}
    $\begin{pmatrix}
        1 & 2 & 3\\
        0 & 1 & 3\\
        1 & 2 & 0
    \end{pmatrix} = \begin{pmatrix}
        a_{11} & a_{12} & a_{13}\\
        a_{21} & a_{22} & a_{23}\\
        a_{31} & a_{32} & a_{33}
    \end{pmatrix}$
    \end{center}
    Tieni a mente la formula e sostituisci:\par
    \begin{center}
    \textit{det}$A = (-1)^{1+1} \times 1_{11} \times$ \textit{det}$\begin{pmatrix}
        \cancel{1} & \cancel{2} & \cancel{3}\\
        \cancel{0} & 1 & 3\\
        \cancel{1} & 2 & 0
    \end{pmatrix} + (-1)^{2+1} \times 2_{12} \times$ \textit{det}$\begin{pmatrix}
        \cancel{1} & \cancel{2} & \cancel{3}\\
        0 & \cancel{1} & 3\\
        1 & \cancel{2} & 0
    \end{pmatrix} + (-1)^{3+1} \times 3_{13} \times$ \textit{det}$\begin{pmatrix}
        \cancel{1} & \cancel{2} & \cancel{3}\\
        0 & 1 & \cancel{3}\\
        1 & 2 & \cancel{0}
    \end{pmatrix} = +1\times1(1\times0) - (2\times3)] + [-2\times2(0\times0)-(1\times3)] + [1\times3(0\times2)-(1\times1)] =$\par
    $-6+6-3 = -3$
    \end{center}
    So che può sembrare un casino, ma se lo leggi piano e con calma potrai capirne i segreti.
\end{eg}
Tuttavia quello del calcolo del determinante è un processo particolarmente tedioso, ed è per questo che introduciamo i due teoremi seguenti.
\subsection{Teorema di Sarrus}
A detta del Dr. Er Lucertola, nemmeno Sarrus usava Sarrus, ma nel caso in cui ti trovassi davanti ad una matrice quadrata 3x3, riesce a semplificare di molto i calcoli attraverso il seguente algoritmo:
\begin{definition}
    \textbf{Teorema di Sarrus}\par
    Data una matrice quadrata 3x3, è possibile calcolare il determinante attraverso la formula:\par
    \begin{center}
        det$A = (a_{11}\times a_{22}\times a_{33}) + (a_{12}\times a_{23}\times a_{31}) + (a_{13}\times a_{21}\times a_{32}) - (a_{13}\times a_{22}\times a_{31}) - (a_{11}\times a_{23}\times a_{32}) - (a_{12}\times a_{21}\times a_{33})$.
    \end{center}
    In forma matriciale, dove $+$ indica gli elementi da sommare e $-$ quelli da sottrarre, per una visione più chiara:
    \begin{center}
        $\begin{pmatrix}
        +a_{11} & +a_{12} & +a_{13}-\\
        a_{21} & +a_{22}- & +a_{23}-\\
        a_{31}- & a_{32}- & +a_{33}-
    \end{pmatrix} \begin{pmatrix}
        a_{11}- & a_{12}-\\
        +a_{21}- & a_{22}\\
        +a_{31} & +a_{32}
    \end{pmatrix}$
    \end{center}
    Qua è da trovare una forma per mostrarlo diversa, sai.
\end{definition}
\begin{eg}
    \textbf{Calcolo del determinante di una matrice mediante il teorema di Sarrus}
    \begin{center}
      $A = \begin{pmatrix}
        1 & 2 & 3\\
        0 & 1 & 3\\
        1 & 2 & 0
    \end{pmatrix} \begin{pmatrix}
        1 & 2\\
        0 & 1\\
        1 & 2
    \end{pmatrix}$
    \end{center}
    \begin{center}
    \textit{det}$A = 0+6+0-3-6-0 = -3$
    \end{center}
    Che è corretto.
\end{eg}
\subsection{Teorema di Laplace}
L'algoritmo di Laplace è invece universalmente utile, per questo è il più forte, mio padre. Vale per ogni matrice quadrata nxn e il determinante può essere sviluppato per ogni riga o colonna.
\begin{definition}
    \textbf{Algoritmo di Laplace}\par
    Data una qualunque matrice quadrata nxn, è possibile ottenere il determinante di una matrice $A$ in uno di questi due modi:
    \begin{itemize}
        \item Se sviluppi per riga $i$\par
        INSERISCI FORMULA
        \item Se sviluppi per colonna $j$\par
        INSERISCI FORMULA
    \end{itemize}
    MANCA TUTTO IL RESTO.
\end{definition}
\section{Utilizzi del determinante}
\section{Esercizi}

5.1.2 Teorema di Laplace

5.2 UTILIZZI DEL DETERMINANTE
5.2 Determinante e la trasposta
5.2 Proposizione: Det di matrici triangolari
5.2 Teorema: Det di un prodotto con una matrice elementare
5.2 Corollario: invertibile se e solo se Det non-nullo
5.2 Corollario: Det di un prodotto di matrici
5.2 Teorema: invertibile se e solo se Det non-nullo (2x2)
5.2 La regola di Cramer