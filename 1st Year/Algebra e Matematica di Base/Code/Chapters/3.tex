\section{Principio di induzione sui naturali}
L'insieme dei numeri naturali $\mathbb{N}$ è il più importante di tutta la matematica e la sua dinamica si basa su due principi:
\begin{itemize}
    \item Un numero dato, spesso $0$, dal quale partire.
    \item Una funzione successore $Succ(n)$, che permette di ottenere il numero conseguente.
\end{itemize}
Conosci già gli elementi di $\mathbb{N}$; si tratta di un intervallo che comprende i numeri da $[0, +\infty)$, ne consegue che è possibile dimostrare una proprietà $\phi(n)$ per tutti i numeri naturali usando l'induzione, che in questo caso chiameremo $IND_{\mathbb{N}}$. Il suo funzionamento non differisce da una classica induzione:
\begin{enumerate}
    \item Parti da un caso base $\phi(0)$ e provalo vero.
    \item Supponi un $n \in \mathbb{N}$ e prova l'ipotesi $\phi(n+1)$.
    \item Concludi che $\forall n \in \mathbb{N}$ vale $\phi(n)$.
\end{enumerate}
Per i nostri scopi useremo inoltre i seguenti assiomi, i quali renderanno più semplice la risoluzione degli esercizi:
\begin{prop}
    \textbf{Assiomi di Peano}\par
    \begin{itemize}
    \item $0 \notin \mathbb{N}$, quindi la funzione $Succ(n)$ non è suriettiva.
    \item $Succ(m) = Succ(m) \implies m = n \forall m,n \in \mathbb{N}$, quindi la funzione $Succ(n)$ è iniettiva.
    \item Sia $\phi(n)$ una formula sui numeri naturali, allora vale:\par
    $[\phi(0) \land \forall n \in \mathbb{N}.(\phi(n) \implies \phi(Succ(n))] \implies \forall n \in \mathbb{N}.(\phi(n))$
\end{itemize}
\end{prop}

\begin{eg}
    \textbf{Dimostrazione con $IND_{\mathbb{N}}$}\par
    INSERISCI ESEMPIO con $\forall n \in \mathbb{N}.[(1+2024)^n \geq 1+n\times 2024]$.
\end{eg}

Considera che è possibile operare anche in un insieme dei naturali dove non fa parte lo zero, formalmente definito come $\mathbb{N^*} = \mathbb{N} / \{0\}$. Il procedimento per le dimostrazioni non cambia, semplicemente il passo base sarà con il numero $1$. Abbiamo inoltre un algoritmo equivalente a quanto visto denominato $IND_<$.

\begin{eg}
    \textbf{Dimostrazione con $IND_{\mathbb{N}^*}$}\par
    Definiamo l'insieme numerico di lavoro come $\mathbb{N}^* = \mathbb{N} \setminus \{0\} = \{1, 2, ..., n\}$\par
    Tesi da provare: $\theta(n) = \forall n \in \mathbb{N}^*.(1+2+...+n=\dfrac{n(n+1)}{2})$
    \begin{itemize}
        \item \textbf{Passo base}:\par
        Testiamo se la tesi vale sostituendo $n$ a $1$\par
        \begin{center}
            $\theta(1) \iff 1 = \dfrac{1(1+1)}{2} = 1$, che è vera.
        \end{center}
        \item \textbf{Passo induttivo}:\par
        Espandiamo il ragionamento per $\theta(n+1)$. Va sostituito $(n+1)$ alla singola $n$ presente nella tesi iniziale.
        \begin{center}
            $\theta(n+1) \iff (1+2+...+n+(n+1)) = \dfrac{(n+1)((n+1)+1)}{2} = \dfrac{(n+1)(n+2)}{2}$
        \end{center}
        Adesso proviamo che il risultato ottenuto è valido:
        \begin{center}
            $\dfrac{n(n+1)}{2}+(n+1) = \dfrac{n(n+1) + 2(n+1)}{2} = \dfrac{n^2+3n+2}{2} = \dfrac{(n+1)(n+2)}{2}$
        \end{center}
    \end{itemize}
    Come volevasi dimostrare.
\end{eg}

\begin{eg}
    \textbf{Dimostrazione con $IND_<$}\par
    INSERISCI ESEMPIO.
\end{eg}

%

\section{Principali operazioni ed elementi}
Dove è sempre possibile utilizzare l'induzione per la dimostrazione dei teoremi, risulta particolarmente comodo provare determinate relazioni e operazioni per \textbf{Ricorsione}. Anch'essa è composta di due casi:
\begin{enumerate}
    \item Caso base, da dove inizia la dimostrazione.
    \item Caso ricorsivo, il quale avanza tenendo conto dei valori precedentemente ottenuti.
\end{enumerate}
L'esempio più semplice, spesso usato anche nella programmazione, è la formalizzazione del concetto di fattoriale.
\begin{eg}
    \textbf{Dimostrazione con ricorsione}\par
    \begin{itemize}
        \item Passo base\par
        Se $n = 0 \implies 1$
        \item Passo ricorsivo\par
        Se $n > 0 \implies (n-1)! \times n$ 
    \end{itemize}
    Proviamo a ragionare come si comporta tale funzione quando sostituiamo alla $n$ i valori presi in esame. Otterremo che:
    \begin{center}
        $0! = (0-1)! \times1 = 1$\par
        $1! = (1-1)!\times1 = 1\times1 = 1$\par
        $2! = (2-1)!\times2 = 1\times2 = 2$\par
        $3! = (3-2)!\times3 = 2\times3 = 6$\par
        $4! = (4-3)!\times4 = 6\times4 = 24$\par
    \end{center}
\end{eg}
Adesso è ora di distruggere tutto ciò che è stato insegnato alle scuole elementari sulla matematica; definiamo tutte le operazioni elementari:
\begin{itemize}
    \item \textbf{Funzione Successore}\par
        La funzione più semplice ma anche la più importante, che consente di poter ragionare induttivamente sull'insieme dei naturali.
        \begin{definition}
            \textbf{Funzione numero successore}\par
            \begin{center}
                Supponiamo la formula: $\forall n \in \mathbb{N}.\{1+n = succ(n)\}$\par
                Diciamo quindi che $[1+n = succ(n)] = \phi(n)$.
            \end{center}
            \begin{itemize}
                \item Caso base\par
                 $\phi(0) \iff 1+0 = succ(0) \iff 1+0 = 1$, vero.
                \item Passo induttivo\par
                $\phi(n) \implies \phi(n+1) = \phi(succ(n)) \iff 1+succ(n) = succ(succ(n))$\par
                $1+succ(n) = succ(n+1) = succ(succ(n))$ per definizione.\par
            \end{itemize}
            Formula dimostrata.
        \end{definition}
        Questa prova ci consente di ampliare il nostro arsenale con le operazioni elementari, in quanto esse sono basate su di essa e ogni numero naturale di può scrivere come la somma di $1_n$.    
    \item \textbf{Addizione}\par
        L'operazione elementare di base usa la definizione di funzione successore. Viene definita ricorsivamente come:
        \begin{definition}
            \textbf{Addizione fra due numeri $m,n$}\par
            \begin{itemize}
                \item \textbf{Passo base}\par
                $m+0 = m$
                \item \textbf{Passo ricorsivo}\par
                $m+succ(n) = succ(m+n)$
            \end{itemize}
        \end{definition}
        
        \begin{eg}
            Dimostrare che $2+3 = 5$\par
            Molto semplicemente utilizziamo i casi della definizione per andare avanti:
            \begin{itemize}
                \item $2+3 = 2+succ(2) = succ(2+2)$
                \item $succ(2+2) = succ(2+succ(1)) = succ(succ(2+1))$
                \item $succ(succ(2+1)) = succ(succ(2+succ(0))) = succ(succ(succ(2))) = 5$.
            \end{itemize}
        \end{eg}
            
        L'addizione porta necessariamente con sé le sue proprietà, che sono:
        \begin{enumerate}
            \item $0+m = m$, $m+0 = m$
            \begin{proof}
                $\phi_0(m) := 0+m = m \in \mathbb{N}$\par
                \begin{itemize}
                    \item $\phi_0(0) \iff 0+0=0$, vale.
                    \item $\phi_0(m) \implies \phi_0(succ(m)) \iff 0+succ(m) = succ(m)$\par\quad
                    $0+succ(m) = succ(m+0) = succ(m)$ per definizione.
                \end{itemize}
                CVD.
            \end{proof}
            \item $succ(m) = m+1 = 1+m$\par
            Dimostrata nella definizione 3.2
            \item $m+succ(n) = succ(m)+n$
            \begin{proof}
                $\forall n \in N, \forall m \in N.[(m+succ(n) = succ(m)+n]$
                \begin{center}
                    Diciamo innanzitutto che $\forall m \in N.[(n+succ(n) = succ(m)+n] = \phi(n)$
                \end{center}
                \begin{itemize}
                    \item $\phi(0) \iff \forall m \in \mathbb{N}.[m+succ(0) = succ(0)+m]$\par\quad
                    $m+succ(0) = succ(m+0) = succ(m) = succ(m)+0$. Vale.
                    \item $\phi(n) \implies \phi(succ(n)) \iff \forall m \in \mathbb{N}.[m+succ(succ(n)) = succ(m) + succ(n)]$.\par\quad
                    Sia ora $m \in \mathbb{N}.succ(m)+succ(n) = succ(m+n) = succ(m+succ(n)) = m + succ(succ(n))$.
                \end{itemize}
                La formula vale per definizione della funzione successore. CVD.
            \end{proof}
            \item $(m+n)+l = m+(n+l)$
            \begin{proof}
                INSERIRE DIMOSTRAZIONE
            \end{proof}
            \item $m+n = n+m$
            \begin{proof}
                $\forall m,[\forall n\in \mathbb{N}.(m+n = n+m)]$, $[] = \phi(m)$
                
                \begin{itemize}
                    \item $\phi(0) \iff 0+n = n+0$. Vale per definizione della proprietà n.1.
                    \item $\phi(m) \implies \phi(succ(m)) \iff \forall n \in \mathbb{N}.[succ(m)+n = n+succ(m)]$, dove $[] = \theta(m)$.
                \end{itemize}
                Effettuiamo una seconda induzione sulla formula $\theta(n)$:
                \begin{itemize}
                    \item $\theta(0) \iff succ(m)+0 = 0+succ(m)$, formula vera.
                    \item $\theta(n) \implies \theta(succ(n)) \iff succ(m)+succ(n) = succ(n)+succ(m)=$\par\quad
                    1. $=succ(succ(m)+n)$\par\quad
                    2. $=succ(m+succ(n))$\par\quad
                    3. $=succ(succ(n)+m)$\par\quad
                    4. $=succ(n+succ(m))$\par
                \end{itemize}
                Formula dimostrata. CVD.
            \end{proof}
            \item $m+n = 0 \implies m=0 \land n=0$
            \begin{proof}
                $\forall m\in N.[\forall n\in N.(m+n = 0 \implies m = 0 = n)], [] = \phi(m)$
                \begin{itemize}
                    \item $\phi(0) \iff \forall n\in N . (0 + n = 0 \implies 0 = n)$, vale per definizione.
                    \item $\phi(m) \implies \phi(succ(m)) \iff \forall n\in N.[succ(m) + n = 0 \implies$\par\quad
                    $succ(m) = 0 = n]$.
                \end{itemize}
                A questo punto ragioniamo. La formula $succ(m)+n = 0$ non potrà mai essere vera poiché viola l'assioma di Peano 1. Raggiungiamo quindi un'assurdità e la formula risulta falsa.\par\quad
                Tuttavia, la formula $\phi(m)$ è della forma $P \implies Q$. Noi abbiamo provato che $P$ è falsa, di conseguenza $Q$ risulta vera grazie al connettivo. Abbiamo quindi dimostrato la formula.
            \end{proof}
            \item $m+n = l+n \implies m=l$
            \begin{proof}
                $m,l \in \mathbb{N}.\forall n\in N.[(m+n = l+n \implies m = l)], [] = \phi(n)$.
                \begin{itemize}
                    \item $\phi(0) \iff m+0 = l+0 \implies m=l$, vale.
                    \item $\phi(n) \implies \phi(succ(n)) \iff m+succ(n) = l+succ(n) \implies$\par\quad
                    $succ(m+n) = succ(l+n) \implies m+n = l+n$
                \end{itemize}
                Formula dimostrata in quanto abbiamo raggiunto l'ipotesi induttiva\par
                $m+n = l+n \implies m = l$.
            \end{proof}
            \item $m+n = n \implies m=0$
            \begin{proof}
                $\forall n \in \mathbb{N}.[m+n = n \implies m = 0], [] = \phi(m)$.
                \begin{itemize}
                    \item $\phi(0) \iff 0+n = n$, vale per definizione.
                    \item $\phi(m) \implies \phi(succ(m)) \iff succ(m)+n = 0$, impossibile per Peano 1.
                \end{itemize}
                Formula dimostrata poiché $P\implies Q$, dove $P=0$, $Q=1$.
            \end{proof}
        \end{enumerate}
    \item \textbf{Ordinamento, concetto di minimo}\par
    Partiamo dal presupposto che per parlare di ordinamento abbiamo bisogno di definire cosa rende un numero maggiore o minore di un altro. Assorbito tal concetto, diciamo che ogni sottoinsieme non vuoto di $\mathbb{N}$ ha necessariamente un elemento minimo, formalmente:
    \begin{center}
        $\exists n \in N$, dove $n \in A, \forall \alpha \in A.(n \leq \alpha)$.
    \end{center}
    \begin{proof}
        \textbf{Concetto di minimo $MIN$}\par
        Iniziamo supponendo un insieme $A$ non vuoto ed un elemento $n \in A$. Diciamo che:
        \begin{itemize}
            \item $0 \in A$, se non lo è, passa al numero successivo, altrimenti hai trovato il minimo.
            \item $1 \in A$, se non lo è, passa ancora al successivo, altrimenti hai trovato il minimo. Ripeti il processo fin quando non trovi il numero.
        \end{itemize}
    \end{proof}
    Tuttavia, attenzione: il concetto di minimo non vale per i sovrainsiemi numerici di $\mathbb{N}$, a meno che non venga preso un insieme proprio. Puoi inoltre usare lo stesso ragionamento per trovare il valore massimo in un insieme i cui estremi sono definiti.\par\quad
    Un'altra particolarità del principio di minimo è che implica la validità del principio del terzo escluso nei numeri naturali:
    \begin{theorem}
        $MIN \implies PEM_P$\par 
    \end{theorem}
    \begin{proof}
        \textbf{Ricorsione}\par
        Sia una formula $P$, dimostriamo che vale il principio del terzo escluso, ovvero $(P \lor \neg P)$.\par
        \begin{center}
            $A_P = \{1\} \cup \{x \in \mathbb{N} | x = 0 \land P\}$\par
            $1 \in A_P$, quindi $A_P$ non è vuoto.
        \end{center}
        Non essendo un insieme vuoto, avrà per forza un minimo. CVD.
    \end{proof}
    Possiamo ottenere uno stesso risultato anche induttivamente, effettuando il seguente ragionamento:
    \begin{proof}
        $IND_\mathbb{N}$\par
        Sia ora $n = min(A_P)$, abbiamo che $[n = 0 \lor \exists m \in \mathbb{N}.(n = succ(m))]$.\par
        Chiamiamo il minimo $P$ e dimostriamo che vale $(P \lor \neg P)$:
        \begin{center}
            $0 \in A_P \implies P$\par
            Se il minimo è 1, P non è valida, di conseguenza vale $\neg P$, quindi:\par
             $P_A.(0=0)\implies (0\in A_P) \implies 0 = minA_P$.\par
             Qui hai trovato l'assurdo ricavando che il minimo è $0$.
        \end{center}
        CVD, vale $(P \lor \neg P)$.
    \end{proof}
    \item \textbf{Moltiplicazione}\par
    La moltiplicazione è la seconda operazione elementare che consente di esprimere ogni singolo numero naturale. Viene definita ricorsivamente come segue:
    \begin{definition}
        \textbf{Moltiplicazione} - $m,n \in N$\par
        \begin{itemize}
            \item $m\times0 = 0$
            \item $m\times succ(n) = m\times n+m$
        \end{itemize} 
    \end{definition}
    Seguono esempi pratici per miglior comprensione:
    \begin{eg}
        \textbf{Dimostrare che $2\times 3 = 6$}\par
        Il senso dei passaggi è che, in base alla definizione ricorsiva della moltiplicazione e a quella della funzione successivo, puoi ridurre tutti i numeri nella formula al minimo ed arrivare a una somma.
        \begin{center}
            $2\times 3 = 2\times succ(2) = 2\times 2 + 2 =$ $2\times succ(1) + 2 =$\par
            $2\times 1 + 2 + 2 = 2\times succ(0) + 2 + 2 = 2\times 0 + 2 + 2 + 2 =$\par
            $0 + 2 + 2 + 2 = 6$.
        \end{center}
        Questo procedimento può tranquillamente valere anche per numeri incogniti. Infatti, se vogliamo dimostrare che $m\times 1 = m$, opereremo allo stesso modo:
        \begin{center}
            $m \times 1 = m \times succ(0) = m \times 0 + m = 0 + m = m$
        \end{center}
    \end{eg}
        \begin{itemize}
            \item \textbf{Proprietà distributiva}\par
            La moltiplicazione ha giustamente la sua proprietà distributiva, ovvero rende possibile "distribuire" l'operazione a più valori se ambo devono essere moltiplicati. 
            \begin{prop}
                La proprietà distributiva viene definita ricorsivamente e ha i seguenti casi:
                \begin{enumerate}
                    \item $m(l+n) = m\times l+m\times n$
                    \item $m(l\times n) = (m\times l)n$
                    \item $m\times n = n\times m$
                    \item $m\times n = 0 \implies m = 0 \lor n = 0$
                \end{enumerate}
            \end{prop}
            \begin{proof}
                La proprietà è definita tramite induzione come segue:
                \begin{center}
                    Supponiamo che $\forall m \in N . [\forall n \in N.(m*n = 0 \implies m= 0 \lor n = 0]$
                \end{center}
                \begin{itemize}
                    \item Il caso base vale per identità. Bella fortuna.\par
                    $\phi(0) \iff \forall n \in N . (0\times n = 0 \implies 0=0 \lor n = 0)$.
                    \item Passo induttivo\par
                    $\phi(n) \implies \phi(succ(n))\iff$\par
                    $\quad \forall n \in N .[(succ(n)\times n = 0)\implies succ(m) = 0 \lor  n = 0]$, $[] = \theta(n)$
                    \item Abbiamo ottenuto la formula $\theta(n)$, sulla quale possiamo operare:\par
                    $\theta(n) \iff succ(m)\times n = 0 \implies succ(m) = 0 \lor n = 0)$.
                    \item Il suo caso base vale fortunatamente per identità:\par
                    $\theta(0) \iff succ(m) \times 0 = 0 \implies succ(m) = 0 \lor 0=0$.
                    \item Passo induttivo.\par
                    $\theta(n) \implies \theta(succ(n)) \iff$\par
                    $\quad succ(n) = 0 \implies succ(m = 0 \lor succ(n)) = 0$.
                    \item Proviamo a verificare la proprietà\par
                    $succ(m) \times succ(n) = 0 \iff succ(m)\times n + succ(m) = 0 \implies$\par
                    $succ(m)\times n = 0 \land succ(m) = 0$.\par
                    Questa formula risulta falsa per l'assioma di Peano 1, infatti il successore di $0$ è $1$, non $0$.
                \end{itemize}
                La formula qui è in ogni caso dimostrata grazie al funzionamento del connettivo $\implies$. Il falso implica il vero. CVD
            \end{proof}
        \end{itemize}
    \item \textbf{Elevamento a potenza}\par
    Contrariamente a quello che si può pensare, l'elevamento a potenza è un'operazione fattibile nell'insieme $\mathbb{N}$ e anche questa viene definita ricorsivamente:
    \begin{definition}
        \textbf{Elevamento a potenza}
        \begin{itemize}
            \item $m^0 = 1$
            \item $m^{succ(n)} := m^n * m$
        \end{itemize}
    \end{definition}
    Vediamo un semplice esempio usando la definizione.
    \begin{eg}
        Dimostrare che $2^3 = 8$\par
        Come visto nella moltiplicazione, anche qui è necessario vedere i numeri sotto la lente della funzione successivo, per poi usare la definizione di moltiplicazione e ottenere il risultato.\par
        $2^3 = 2^{succ(2)} = 2^2 \times 2 = 2^{succ(1)} \times 2 = 2^1 \times 2 \times 2 = 2^{succ(0)}\times2\times2 = 2^0\times2\times2\times2 =$\par
        \begin{center}
            $1\times2\times2\times2 = 8$. CVD
        \end{center}
    \end{eg}
\end{itemize}

%

\section{Costruzione di interi e razionali}
Finora ci siamo concentrati sull'insieme numerico dei numeri naturali $\mathbb{N}$, ma come è possibile definire gli altri sovrainsiemi che compongono la matematica? Partiamo dal basso per ora e cerchiamo di definire i \textbf{Numeri interi}, appartenenti all'insieme $\mathbb{Z}$:
\begin{definition}
    \textbf{Insieme dei numeri interi relativi $\mathbb{Z}$}\par
    Fondamentalmente l'unione dell'insieme dei naturali con quello dei loro opposti, formalmente:
    \begin{center}
        $\mathbb{Z} = \mathbb{N} \cup \{0\} \cup \mathbb{N}^-$
    \end{center}
    Dove i rispettivi insiemi sono:
    \begin{itemize}
        \item $\mathbb{N} = \{0,1,2,...,n\}$.
        \item $\mathbb{N}^- = \{-1,-2,...,-n\}$.
    \end{itemize}
    Il ragionamento vale pure per l'insieme dei naturali senza $0$, $\mathbb{N}^*$. Importante ricordare che siccome stai unendo due opposti, se provi a trovare un'intersezione fra i due, otterrai un insieme vuoto.
    \begin{center}
        $N^* \sim N_*, N^* \cap N_* = \emptyset$
    \end{center}
\end{definition}
Abbiamo aggiunto all'equazione dello studio i numeri negativi: come possiamo rappresentarli? Ebbene, serviranno due numeri.
\begin{eg}
    Rappresentazione dei numeri negativi.\par
    \begin{center}
        $-1 = 0-1 = 1-2$\par
        $-2 = 0-2 = 1-3$\par
        e così via.
    \end{center}
    Questa scrittura vale anche con 0, 1, 2 e altri. Tutti i numeri si possono infatti rappresentare mediante una relazione di sottrazione. Ci consente di rendere due coppie scritte in modo diverso uguali grazie alla definizione di funzione.
\end{eg}
Adesso approfondiamo un simbolo raramente utilizzato negli studi della matematica alle scuole dell'obbligo: $\sim$, indicante l'equivalenza. Ci permette di identificare gli oggetti disuguali che però sono equivalenti. Diamone una definizione formale:
\begin{definition}
    \textbf{Equivalenza}\par
    Una relazione si dice equivalente quando rispetta questi tre criteri:
    \begin{enumerate}
        \item La relazione è riflessiva.
        \item La relazione è simmetrica.
        \item La relazione è transitiva.
    \end{enumerate}
    Supponiamo ora le coppie $(m,n)\sim(m',n')$. Se queste sono equivalenti, come indicato dalla tilde, sarà possibile utilizzare i classici criteri di equivalenza per le equazioni. Varranno quindi le seguenti scritture \textit{equivalenti}:
    \begin{center}
        $m-n = m'-n' \sim m + n' = m' + n$
    \end{center}
    Diremo inoltre che $\sim$ è un'equivalenza su $\mathbb{N}\times\mathbb{N}$.
\end{definition}
\begin{proof}
    \textbf{Relazione di equivalenza}
    \begin{itemize}
        \item La relazione è riflessiva.\par
        $(m,n) \sim (m,n) \iff m+n = m+n$.
        \item La relazione è simmetrica.\par
        $(m,n) \sim (m',n') \implies (m',n') \sim (m,n) \implies m+n' = m'+n \implies$\par
        \quad$m'+n = m+n'$. 
        \item La relazione è transitiva.\par
        $(m,n) \sim (m',n') \iff m+n' = m'+n \land (m',n')\sim(m'',n'') \iff$\par
        \quad$m'+n'' = m''+n' \land (m,n) \sim (m'',n'') \iff m+n'' = m'' + n$\par
        Abbiamo ora che:\par
        $m' + n'' = m''+n' \implies m'+n+n'' = m''+n'+n$\par
        \quad$\implies m'+n'+n'' = m''+n'+n$\par
        \quad$\implies m+n'' = m''+n$. Un termine vale l'altro. Transitività dimostrata.
    \end{itemize}
\end{proof}
Sapendo tutto ciò, ci è possibile dare una definizione rigorosa dell'insieme dei relativi:
\begin{definition}
    \textbf{Definizione rigorosa dell'insieme $\mathbb{Z}$}\par
    Definiamo $\mathbb{Z}$ l'insieme delle classi di equivalenza $\mathbb{Z}:=(\mathbb{N}\times\mathbb{N})/\sim$, ovvero:
    \begin{center}
        $\mathbb{Z} = \{[(m,n)]_{\sim} | (m,n) \in \mathbb{N}\times\mathbb{N}\} = \biggl\{\{(m',n') \in \mathbb{N}\times\mathbb{N} | (m',n') \sim (m,n)\} | (m,n) \in \mathbb{N}\times\mathbb{N}\biggr\}$.
    \end{center}
    Come visto prima, ci è possibile definire ogni numero tramite sottrazioni. Vedi queste coppie come se fossero una funzione di sottrazione e avrai capito tutto.
    \begin{itemize}
        \item $0_{\mathbb{Z}} = [(0,0)]_\sim$
        \item $1_{\mathbb{Z}} = [(1,0)]_\sim$
        \item $-1_{\mathbb{Z}} = [(0,1)]_\sim$
        \item $2_{\mathbb{Z}} = [(2,0)]_\sim$
        \item $-2_{\mathbb{Z}} = [(0,2)]_\sim$
    \end{itemize}
    Puoi potenzialmente andare avanti ad infinitum con questa scrittura.
\end{definition}
A partire da questa definizione ci è possibile ragionare induttivamente per eseguire utili dimostrazioni, come nel seguente esempio:
\begin{eg}
    Dimostrare che: $[(m,n)]_{\sim}:= \{(m',n') \in NxN | (m',n') \sim (m,n)\}$\par
    Per avere un'equivalenza da un punto di vista di valore con numeri effettivi, bisogna trovare due operazioni che ritornino lo stesso numero. Ora ragionando induttivamente:
    \begin{itemize}
        \item $0_Z .= [(0,0)]_{\sim} = [(1,1)]_{\sim} \iff (0,0) \sim (1,1)$, vera perché $0-0=0$, $1-1=0$.
        \item $1_Z := [(1,0)]_{\sim} = [(2,1)]_{\sim} \iff (1,0) \sim (2,1)$, vera perché $1-0=1$, $2-1=1$.
    \end{itemize}
    La dimostrazione vale perché abbiamo dimostrato precedentemente che si può andare avanti all'infinito.
\end{eg}
\begin{prop}
    Osserviamo inoltre che l'insieme $\mathbb{N}$ è incorporato in $\mathbb{Z}$.\par
    Definiamo quindi la funzione $i: \mathbb{N} \to \mathbb{Z}$ dalla regola:
    \begin{center}
        \item $m \to i(m) := m_{\mathbb{Z}} := [(m,0)]_{\sim} := \{m',n' \in \mathbb{N}\times \mathbb{N} | m' = m+n'\}$.
    \end{center}
    Nel caso in cui valga $m = l$, avremo il seguente sviluppo:
    \begin{itemize}
        \item $i(m) = i(l) \iff [(m,0)]_{\sim} = [(l,0)]_{\sim}$
        \item $\iff (m,0) \sim (l,0)$
        \item $\iff m+0 = l+0$
        \item $\iff m = l$.
    \end{itemize}
    Puoi osservare che la funzione $i$ è iniettiva perché $\mathbb{N}$ si può identificare in $i(\mathbb{N})$ in quanto scritture equivalenti, ed in questo senso, otteniamo che $\mathbb{N} \subseteq \mathbb{Z}$.
\end{prop}

%

\section{Fattorizzazione e teorema fondamentale dell'aritmetica}
\section{Congruenze}
\section{Domande di teoria}
\section{Esercizi}
\begin{enumerate}
    \item Dimostrare la funzione $pred(n):\mathbb{N}\to\mathbb{N}$\par
    \begin{proof}
        Definiamo la funzione predecessore nei naturali ricorsivamente con i seguenti casi:
        \begin{itemize}
            \item $pred(0) = 0$
            \item $pred(succ(n)) = n$
        \end{itemize}
        Data la particolarità del caso $0$, lavoreremo su $\mathbb{N^*}$. Supponiamo la seguente formula:
        \begin{center}
            $\forall n \in \mathbb{N^*}.[(succ(pred(n))] = n, [] = \theta(n)$
        \end{center}
        \begin{itemize}
            \item $\theta(1) \iff succ(pred(succ(0)) = succ(0)$
            \item $\theta(n) \implies \theta(succ(n)) \iff succ(pred(succ(n))) = succ(n)$.
        \end{itemize}
        Per definizione abbiamo che $pred(succ(n)) = n$. Quindi la formula è dimostrata.
    \end{proof}
    \item Dimostrare che $1 \times 1 = 1$.\par
    \begin{proof}
        Banalmente sostituisci i termini nella definizione di moltiplicazione e vedi i numeri come i successori del loro valore precedente
        \begin{center}
            $1 \times 1 = 1 \times succ(0) = 1 \times 0 + 1 = 1$
        \end{center} 
    \end{proof}
    \item Dimostrare che $1 \times m = m$.\par
    \begin{proof}
        Questa volta ragioniamo per induzione
        \begin{center}
            Supponiamo innanzitutto che $\forall m \in N . (1\times m = m)$\par
        \end{center}
        \begin{itemize}
            \item Il caso base vale banalmente per definizione di moltiplicazione\par
            $\phi (0) = 1 \times 0 = 0$.
            \item Per gli assiomi di Peano, diciamo che:\par
            $\phi(m) \implies \phi(succ(m)) \iff 1 \times succ(m) = succ(m)$.
            \item Mentre per la definizione del caso ricorsivo della moltiplicazione abbiamo:\par
            $1\times succ(m) = 1\times m + 1 = m+1 = succ(m)$.
            \item Avendo dimostrato il passo induttivo, possiamo dire che vale anche per $m$:\par
            $\phi(m) \iff 1 \times m = m$. CVD
        \end{itemize}
    \end{proof}
    \item Dimostrare che $0 \times m = 0$.\par
    \begin{proof}
        Anche qui risulta comodo ragionare per induzione\par
        \begin{center}
            Supponiamo che $\forall m \in N(0\times m = 0)$.
        \end{center}
        \begin{itemize}
            \item Abbiamo che il caso base vale per definizione:\par
            $\phi(0) \iff 0\times 0 = 0$.
            \item Usiamo ancora gli assiomi di Peano per implicare la funzione successiva.\par
            $\phi(m) \implies \phi(succ(m)) \iff 0 \times succ(m) = 0$.
            \item Usiamo la definizione di moltiplicazione e sostituiamo i termini:\par
            $0 \times succ(m) = 0 \times m + 0 = 0+0 = 0$. CVD
        \end{itemize}
    \end{proof}
    \item Dimostrare induttivamente e ricorsivamente che  $m < n$.\par
    Per entrambe le dimostrazioni valgono le seguenti supposizioni:
    \begin{enumerate}
        \item $m < n \implies m*l < n*l$, dove $l >= 1$
        \item $m < n \iff \exists k \in N^*(n = m+k), k >= 1$
    \end{enumerate}
    \begin{proof}
        Per induzione\par
        AGGIUSTA
        \begin{itemize}
            \item $\forall l \in N^*[(m*l < n*l)]$, $m,n . m < n$, $\phi(l) = []$
            \item $\phi(1) \iff m*l < n*l \iff m < n$ vale
            \item $\phi(l) \implies \phi(succ(l))$
            \item $\phi(succ(l)) \iff m * succ(l) < n * succ(l)$
            \item $m * succ(l) = m * l + m$
            \item $n * succ(l) = n*l +n$
            \item $m*l < n*l = m < n$
        \end{itemize}
    \end{proof}
    \begin{proof}
        Per ricorsione\par
        AGGIUSTA
        \begin{itemize}
            \item $n = m+k, k <= 1$
            \item $m*l < n*l$
            \item $n*l = (m+k)l = m*l + k*l, k*l >= 1$
            \item $k >= 1, l >= 1, k*l >= 1$ 
            \item $k = 1 + \sigma$, $l = 1 + \sigma'$
            \item $k*l = (1+\sigma)(1+\sigma') = 1+\sigma' + \sigma + \sigma * \sigma' >= 1$
        \end{itemize}
    \end{proof}
    \item Dimostrare che $m <= m*n$, dove $n >= 1$.\par
    \begin{proof}
        AGGIUSTA - Per lavorare su questa dimostrazione risulta utile fissare un $m$. Detto ciò:
        \begin{center}
            Supponiamo la formula: $m <= n \iff \exists k \in N . (n = m+k), k >= 0$.
        \end{center}
        \begin{itemize}
            \item $\forall n \in N^*.[(m <= m*n)], \theta(n) = []$
            \item $\theta(1) \iff m <= m*1 = m$ vale
            \item $\theta(n) \implies \theta(succ(n)) \iff m <= m *succ(n) \iff m <= m(n+1) = m*n+m$
            \item $\theta(n) \iff m <= m*n <= m*n + n = n*succ(m)$
    \end{itemize}
    \end{proof}    
\end{enumerate}

%

\section{Appunti}

\begin{itemize}
    \item 21-11-24\par	
	
			

				


Es 28.
		Sia n \in N, dimostriamo che \exists k \in N . (n < k < succ(n)) \implies \bot		Dimostra quindi che non c'è niente fra n e il suo successore nei naturali.
		Dimostra quindi \neg A, dove A = la formula appena scritta.
		
		Supponiamo \exists k \in N . (n < k < succ(n))
		
		- n < k \iff \exists l \in N* . (k = n + l)
			l > 0 \iff l >= 1 \iff l = 1+\sigma, \sigma >= 0.
			k = n+1 + \isgma
			
			k < succ(n) \iff \exists m \in N* . (succ(n) = k+m), m >= 1
			
		- Succ(n) = n + 1 + \sigma) + m \iff n+1 = (n+1) + \sigma + m \iff \sigma + m = 0 \implies \sigma = 0 = m
		
		Siccome abbiamo trovato che m >= 1, siamo incappati in un'assurdità provando che \sigma = 0 = m. La formula è quindi falsa e abbiamo provato l'assurdo.
		Inoltre, n+1 è stato semplificato grazie ai criteri di equivalenza. Sì, a quanto pare si possono usare.
		

Prop. 1		Siano m, n, m', n' \in N
		
		- m = m' \land n = n' \land m <= n \implies m' <= n'
		- m = 0 \lor m > 0
		- l = 1 \lor l > 1, \forall l \in N*
		- n < succ(n)
		- n < m \implies succ(n) <= m
		- m < n \lor m = n \lor n < m
		- m <= n \iff m = n \lor m < n
		- m < n \iff m <= n \land m != n
		- \neg(n < n)
		- m < n \implies m != n
		- m != n \imploies m < n \lor n < mù
		- m > 1 \implies pred(m) >= 1
		
- Il principio di numero minimo -
	Siano un relazione d'ordine (x, <=) . x_0 \in X, A \subseteq X
	x_0 è un elemento minimo di A \iff
		- x_0 \in A
		- \forall \alpha \in A . (x_0 <= \alpha)
		
	X = {0,1,2} . 0 è un elemento minimo di X perché 0 è in X e \forall x \in X x >= 0.
	
	Decidere un elemento minimo dipende dal sottoinsieme. Ogni altro elemento deve essergli maggiore o uguale.
	
Prop. (X, <=) X insieme parzialmente ordinato, dove x_0, y_0 \in X, A \subseteq X
	Se x_0, y_0 sono elementi minimi di A, allora x_0 = y_0. Questo perché esiste solamente un minimo.
	
	Dim.
		x_0 \iff \forall a \in A, a >= x_0
		y_0 \iff \forall a \in A, a >= y_0
		
		Transitività?
		if a >= x_0 \land a >= y_0 \implies x_0 = y_0.
		
Definizione, minimo MIN
		Ogni sottoinsieme non vuoto di N ha un elemento minimo.
		
		Dim. per induzione IND_<
			- \phi(0) \lND [\PHI(0) \land \phi(1) ... \land(\phi(n) \implies \phi(n+1)] \implies \forall n \in N . (\phi(n))
			
			\nullset != A \subseteq X
			
			- \phi(n) \iff n \notin A		Supponiamo che A non abbia un minimo e proviamo a dimostrare l'assurdità di questa ipotesi
			
			- \phi(0) \iff 0 \notin A vale perché se 0 \in a allora 0 Min(A).
			- 0 \notin A \land 1 \notin A \land ... \land n \notin A \implies n+1 \notin A \iff \neg(n+1 \in A);
				Se n+1 \in A, allora n+1 = min(A)
				
			Guarda la foto da telefono.
			
		C'è una prova più semplice, che segue:
			Supponi A insieme non vuoto e che n \in A.
			
			- 0 \in A. Se 0 non è in A, passa al successivo, altrimenti è il minimo
			- 1 \in A. Se 1 non è in A, passa al successivo, altrimenti è il minimo e così via.
		
		Il principio del minimo non regge per l'insieme R, a meno che non venga preso un insieme proprio.
    \item 05-12-24\par
    Ex 29.
	\forallm \in N.[(\foralln\inN, \foralll \in N.(m(n+l) = mn + ml))] [] = \phi(m)
	
	\phi(0) \iff \foralln \in N, \foralll \in N.(0(n+l) = 0*n+0*l) vale
	\phi(m) \implies \phi(succ(m)) \iff \forall n \in N, \forall l \in N.(succ(m)(n+l) = succ(m)n + succ(m)l)
	
	Lemma:
	\foralln \in N.(succ(m)*n = m*n+n)
		{
			m+0 = 0
			m+succ(n) = succ(m+n)
		}
		{
			m*0 = 0
			m*succ(n) = m*n+m
		}
		{
			m^0 = 1
			m^{succ(n)} = m^n*n
		}
		
		\theta(0) \iff succ(m)*0 = m*0+0 vale
		\theta(n) \implies \theta(succ(n)) \iff succ(m)*succ(n) = m*succ(n) + succ(n)
			succ(m) * succ(n) = succ(m)*n + succ(m) = (m*n+n) + succ(m) = m*n + m + succ(m)
			
			n+succ(m) = n+m+1
			m+succ(n) = m+n+1
			
		m*n + m + succ(m) = m*n+m+succ(n)
		
	Per il lemma, succ(n)(n+l) = m(n+l)+n+l = mn+ml+nl = (m*n+l)+(m*l+l) = succ(n)*n + succ(m)*l 	La formula \phi(n) vale.
	
Ex. 30
	\forall m \in N, \forall n \in N.[(m*n = n*m)] [] = \phi(m)
	
	\phi(0) \iff \foralln \in N.(0*n = n*0) la formula vale
	\phi(m) \implies \phi(succ(m)) \iff \forall n \in N.[(succ(m)*n = n*succ(m))]
	
		for n \in n : succ(m)*n = m*n+n. Questo ultimo risultato è uguale all'ultima ipotesi ottenuta fra le quadre.
			m*n = n*m per iporesi induttiva
			
	Formula dimostrata.
	
Ex. 30i
	for m \in N, \foralln \in N, \forall l \in N.](m*n)*l = m*(n*l)] [] = \phi(n)
	
	- \phi(0) \iff \forall l \in N.[(m*n)*l = m(0*l)]	Il caso base vale
	- \phi(n) \implies \phi(succ(n)) \iff \forall l \in N.[(m*succ(n))*l = m(succ(n)*l)]
		Fissiamo l \in N.
		
		(m*succ(n))*l = (m*n+m)l per definizione. Essendo poi la moltiplicazione commutativa
			= l(m*n+m), che per distributività diventa
			= (m*n)l + m*l = m(n*l+l) = m(succ(n)*l)
			
			Formula dimostrata.

Ex.31-1
	m*l = n*l \implies m = n, m,n,l \in N
	
	Per prop già vista 2.6.4 abbimo che m<n \implies m*l < n*l. Supponiamo quindi m<n \forall l \in N.
	Quindi abbiamo che m*l < n*l; tuttavia questa è un'assurdità, quindi m<n è falsa e quindi m<=n.
	
	Rimangono ora solo m=n \lor m>n, quindi m>=n.
	Allo stesso modo, se n<m avremo un risultato uguale e arriveremo ad un'assurdità.
	
	Varrà solo l'istanza m=n. Abbiamo infatti che m<=n<=m \implies m=n. La relazione è antisimmetrica.
	
Ex.31-2
	m<n \land m'<n'
	---------------
	m*m' < n*n'
	
	m<n \implies m*m' < n*m'
	Per ipotesi m' < n' \implies m'*n < n'*n
	m<n \implies m*l < n*l
	
	m*m' < n*m' = m'*n < n*m = n*n'. Abbiamo quindi dimostrato l'ipotesi sotto la barra. [m*m' < n*n'].
	
Ex.32-1
	m^0 = m^{succ(0))} = m^0*m = 1*m = m.
	
	fissiamo m,n \in N e dimostriamo che \forall l \in N.[(m^{n+l} = m^n*m^l)] [] = \phi(l)
	
	- \phi(0) \iff m^{n+0} = m^n*m^0 = m^n * 1, che vale.
	- \phi(l) \implies \phi(succ(l)) \iff m^{n+succ(l)} = m^n*m^{succ(l)}
	
		m^{n+succ(l)} = m^{succ(n+l)} per definizione. Ora usiamo la definizione di addizione.
			= m^{n+l}*m.	Per ipotesi induttiva abbiamo poi il seguente risultato
			= (m^n*m^l)m = m^n*(m^l*m) = m^n * m^{succ(l)}, quindi \phi(succ(l)) vale.
			
Definita una relazione 0 \sim m.NxN
	(m,n) \sin (k,l) \iff m+l =_N k+l*m.
	
	Z è l'insieme delle classi di equivalenza (definizione già data)
	se n \in N abbiamo pluspotenza i. i*N \to Z
	Abbiamo infatti che i(m) := [(m,0)]_\sim (m-0) = m.
	
Esercizio da esame:
	Dimostrare che 0_Z != 1_Z == 0_Z = 1_Z \implies \bot
	
	0_Z = [(0,0)]_\sim
	1_Z = [(1,0)]_\sim
	
	Stiamo dimostrando che una cosa è falsa e in questo caso non uguale ad un'altra. Quindi andiamo di equivalenza.
	0_Z = 1_Z \iff [(0,0)]_\sim = [(1,0)]_\sim \iff (0,0) \sim (1,0) \iff 0+0 = 1+0 \iff 0 = 1.
	0 = 1 \implies \bot per assioma di peano 1. Formula dimostrata.
	
Dimostrare l'addizione in Z
[(m,n)]_\sim + [(k,l)]_\sim = [(m+k, n+l)]_\sim		MOLTO IMPORTANTE STA Formula
	Perché m-n + k-l = (m+k)-(k+l)
	
	Esempio pratico:
	3_Z + (-5)_Z = [(3,0)] + [(0,5)] = 3+0 + 0-5 = 3-5 = [(3,5)] = -2 = [(0,2)].
	
(m,n) \sim (m',n') \land (k,l) \sim (k',l') \implies [(m,n)]_\sim + [(k,l)]_\sim = [(m',n')]_\sim + [(k',l')]_\sim
	= [(m+k, n+l)]_\sim = [(m'+k', n'+l')]_\sim
	= (m+k, n+l) \sim (m'+k', n'+l') \iff [m+k+n'+l' = m'+k'+n+l]		Dimostrare [] e le due ipotesi iniziali.
	
	(m+n') + (k+l') = (m'+n)+(k'+l)	Formula dimostrata nonostante l'ordine dei termini grazie alla commutatività dell'addizione.
	

Provare ora che 
	i:N\toZ	im|->i(m) := [(m,0]_\sim rispetta l'addizione come i(m+l) = i(m) + i(l)

	I(m+l) = [(m+l, 0)]
	i(m) = [(m,0)]
	i(l) = [(l,0)]
	
	[m,0] + [l,0] = [m+l,0+0]
	
Provare che z+0_z = 0_z+z = z, dove z \in Z

	z = [(m,n)]
	0_Z = [(0,0)]
	
	[m,n] + [0,0] = [m,n], la seconda parte vale per commutatività.
	
Dimostrare che
	z + z' = z' + z, dove z, z' \in Z
	
	z = [m,n]
	z' = [m',n']
	
	[m,n] + [m',n'] = [m',n'] + [m,n] = [m+m', n+n'] = [m'+m, n'+n] vale per commutatività.
		= [m',n'] + [m,n], addizione vale su Z, dimostrata per definizione.
\end{itemize}
