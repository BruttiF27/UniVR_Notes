\section{Dipendenza e Indipendenza Lineare}
§6. Dipendenza e indipendenza lineare (vedi [GS, Capitolo II])
6.1 Proposizione
6.2 Definizione: linearmente dipendente
6.3 Teorema: linear indipendenza
6.4 Definizione: base
6.5 Osservazione: base `e come un sistema di coordinate
6.6 Base di C(U ), U una matrice in forma ridotta
11/04/24
6.7 Proposizione: base, insieme di generatori minimo, insieme massimamente linearmente indipen-
dente
6.8 Teorema: esistenza della base
6.9 Teorema di Steinitz: ogni insieme linearmente indipendente pu`o essere completato a una base
6.10 Corollario: ogni base ha lo stesso numero di elementi
6.11 Definizione: dimensione
6.12 Corollario
6.13 Proposizione: dimensioni di sottospazi
\section{Applicazioni Lineari}
§7. Applicazioni lineari (vedi [GS, Capitolo II])
7.1 Definizione: applicazione lineare
7.2 Esempi e matrice associata a un’applicazione lineare (rispetto alla base canonica)
15/04/24
7.3 Definizione: isomorfismo
7.4 Definizione: applicazione delle coordinate rispetto a una base
7.5 Applicazione delle coordinate Kn → Kn
7.6 Teorema: l’applicazione delle coordinate `e un isomorfismo
7.7 Osservazione: isomorfismi e dimensione
7.8 Corollario: due spazi vettoriali sono isomorfismi se e solo se hanno la stessa dimensione
18/04/24
7.9 Teorema e definizione: matrice del cambio di base
7.10 Teorema e definizione: matrice associata a f rispetto a basi
\section{Rango e Nullità}
§8. Rango e nullit`a (vedi [GS, Capitolo II])
8.1 Spazio nullo e immagine di un’applicazione lineare
22/04/24
8.2 Teorema: nullit`a + rango
8.3 Dimensione di C(A)
8.4 Dimensione di N (A)
8.5 Procedimento per determinare basi di C(A) e N (A)
8.6 Proposizione e definizione: rango di un’applicazione lineare
8.7 Teorema: insieme di soluzioni di un sistema lineare
\section{Esercizi}