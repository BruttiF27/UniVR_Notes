\section{Sistemi Lineari Tempo invariante}
Possiamo ora dare una definizione rigorosa di sistema: si tratta di un modello matematico o formulazione di un processo o fenomeno fisico che permette di trasformare un'entrata in un'uscita determinata. Ne esistono principalmente di due tipi:
\begin{itemize}
	\item \textbf{Single Input Single Output}
	\item \textbf{Multiple Input Multiple Output}, non visti in questo programma.
\end{itemize}
\noindent In base a questi poi ci sono i sistemi dinamici, ovvero con memoria, i quali salvano i dati grazie agli stati. Rammento infine che, come i segnali, i sistemi possono essere a tempo continuo e discreto.\par
Per quanto riguarda i sistemi LTI, approfondiamo le proprietà menzionate nel capitolo precedente:
\begin{itemize}
	\item \textbf{Linearità}: Partendo da un sistema con due entrate, otterrò in output un risultato equivalente alla combinazione lineare dei singoli input. Formalmente: \[da rivedere\]
	\item \textbf{Tempo-invarianza}: Traslando l'input del sistema prima o dopo nel tempo avrò come output lo stesso risultato traslato prima o dopo nel tempo: \[u(t)\to v(t) \implies u(t+\tau)\to v(t+\tau), \forall \tau \in \mathbb{R}\]
	\item \textbf{Causalità}: L'effetto non precede mai la causa. L'uscita $v(t)$ all'istante $\tau$ deve dipendere esclusivamente dall'ingresso $u(t)$ per $t \leq \tau$. Per comodità considereremo solo sistemi a riposo, ovvero con $\tau = 0$.
\end{itemize}
\noindent Richiamiamo all'attenzione il concetto di stabilità per formalizzarlo definitivamente. Definiamo:\newline

\noindent \textbf{Sistema asintoticamente stabile}\par
\noindent Un sistema si dice tale se dopo aver dato l'input, l'output va a morire fino ad arrivare a $0$. Formalmente:
\begin{center}
	$\exists\tau\in\mathbb{R}$ tale che $u(t) = 0$, $\forall t \geq \tau \implies lim_{t\to \infty}v(t) = 0$
\end{center}

\noindent\textbf{Sistema BIBO-stabile}\par
\noindent

\begin{comment}
	--- Rivedere ---
	
	1. Se ho un sistema con a_1u_1(t)+b_1u_2(t) [la somma degli input u_n crea un altro input.] e li metto dentro al sistema, otterrò un risultato che è combinazione lineare dei singoli input, quindi: a_1u_1(t \to a_2u_2)
	
	
	
	--- Riprendi da qui ---
	
	- Proprietà di stabilità BIBO: Un sistema si dice tale se \exists\tau\in\mathbb{R} e M_u>0\in R tale che se |u(t)| < M_u, \forall t \geq \tau \to \exists M_v > 0: |v(t)| < M_v \forall t \geq \tau. WhatTheFuck
	Diciamo di avere un input che oscilla, ad un certo \tau continua a farlo semprre entro una certa ampiezza. Se lo butto dentro al sistema otterrò un segnale in uscita che sarà dopo \tau all'interno di un certo intervallo di ampiezza.
	Quindi, esiste un \tau istante di tempo ed un vincolo massimo e minimo, ambo in ampiezza, paralleli all'asse delle x. Dopo \tau, l'input rimane all'interno di questi vincoli.
	Se la stessa cosa vale per l'output, si dice bibostabile.
	
	Ogni sistema asintoticamente stabile è necessariamente anche bibostabile, ma non viceversa.
	
	Come modelliamo i sistemi?
	Descriveremo i sistemi tramite equazioni differenziali in input e in output.
	Passaggi da fare:
	- Scrivere tutte le equazioni che descrivono le leggi in atto
	- Scrivere estesa l'equazione differenziale del sistema. Input a dx, Output a sx.
	
	Forma generale dei modelli?
	a_n*\frac{\partial^nv(t)}{\partial t^n}+a_{n-1}*\frac{\partial^{n-1}v(t)}{\partial t}+ ... + \frac{\partial^1v(t)}{\partial t} + a_0v(t) = b_m*\frac{\partial^mu(t)}{\partial t}+b_{m-1}*\frac{\partial^{m-1}u(t)}{\partial t} + ... + b_1*\frac{\partial u(t)}{\partial t} + b_0u(t)
	
	u(t) è l'input, v(t) è l'output.
	b_n, a_n \neq 0 sono due coefficienti normali.
	
	In forma più compatta, sono due sommatorie: \sum_{i=0}^na_i \frac{\partial^i v(t)}{\partial t} = \sum_{j=0}^mb_j\frac{\partial^ju(t)}{\partial t}
	n ed m danno l'ordine delle equazioni differenziali e di solito n\geq m.
	I sistemi si scrivono con questa equazione. La scatoletta, proprio.
	
	Se n>m, si dice strettamente proprio
	Se n\geq m si dice proprio
\end{comment}