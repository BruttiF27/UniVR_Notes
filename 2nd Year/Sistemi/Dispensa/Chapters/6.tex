\section{Serie di Fourier}
Come si sarà capito, i sistemi LTI trasformano fasori o segnali sinusoidali $u(t)$ in un output $v(t)$ sinusoidale e alla stessa frequenza, ma con ampiezza e fase diverse. Questa dinamica rende lo studio delle risposte di un sistema notevolmente semplificata, poiché basterebbe determinare come vengono modificate ampiezza e fase; questo è un concetto fondamentale della teoria dei sistemi, perché è possibile \textbf{rappresentare ogni segnale periodico sufficientemente regolare come somma di sinusoidi}.\par
Quindi, presa l'entrata in forma periodica, serve studiare i cambiamenti su ampiezza e fase per poter descrivere appropriatamente il comportamento di un sistema. Utilizzeremo la \textbf{serie di Fourier} con segnali periodici e più in avanti la \textbf{trasformata di Fourier} con segnali non periodici.\newline

\noindent Definiamo anzitutto \textbf{segnale periodico} una funzione $v(t) = v(t + T_0)$ di valori reali o complessi, con $\forall t\in \mathbb{R}$, la quale ha un periodo $T_0$, una frequenza $f_0$ ed una pulsazione $\omega_0 = \frac{2\pi}{T_0} = 2\pi f_0$.\par
Le funzioni di questo tipo sono utilizzate come base di espansione per la rappresentazione dei segnali e devono soddisfare opportune condizioni di regolarità; in particolare, dato un segnale periodico $v(t)$ diciamo che è \textbf{sommabile} o \textbf{al quadrato sommabile} rispettivamente se: \[\int_{t_0}^{t_0+T_0} |v(t)| dt < \infty; \int_{t_0}^{t_0+T_0} |v(t)|^2 dt < \infty\]
\noindent Tali condizioni garantiscono l’esistenza dei coefficienti della serie di Fourier e la convergenza della rappresentazione del segnale.
\begin{definizione}
	\textbf{Rappresentazione dei segnali in serie di Fourier}\par
	\noindent Sia $v(t)$ una funzione a valori reali o complessi, con $t\in \mathbb{R}$, periodica di periodo $T_0$ e quindi frequenza $f_0$. Se: \begin{itemize}
		\item $v(t)$ è generalmente continua.
		\item $v(t)$ è generalmente derivabile e la derivata prima è generalmente continua e limitata su ogni intervallo $[t_0 , t_0 + T_0)$.
	\end{itemize}
	\noindent Allora la funzione è sia sommabile che al quadrato sommabile su un periodo ed è sviluppabile in serie di Fourier. In tal caso diciamo che per ogni punto $t$ in cui la funzione è continua si ha la seguente \textbf{equazione di sintesi}: \[v(t) = \sum_{k=-\infty}^{\infty} a_ke^{jk2\pi f_0t} = \sum_{k=-\infty}^{\infty}a_ke^{jk\omega_0t}\]
	\noindent Con il coefficiente $a_k$ dato dall'\textbf{equazione di analisi}: \[a_k = \frac{1}{T_0}\int_{t_0}^{t_0+T_0} v(t) e^{-jk\omega_0t}dt\]
	\noindent Inoltre, per ogni punto in cui non è continua, vale: \[\frac{v(t^-) + v(t)^+}{2} = \sum_{k=-\infty}^{+\infty}a_ke^{jk\omega_0t}\]
\end{definizione}
\noindent Per approfondire, spieghiamo come è possibile arrivare all'equazione di analisi a partire dall'equazione di sintesi. Dato un segnale periodico $x(t)$ che ammette la rappresentazione in serie di Fourier, per determinare i singoli coefficienti $a_k$ si sfrutta l'ortogonalità delle esponenziali complesse su un periodo $T_0$. Moltiplicando entrambi i membri dell'equazione per $e^{-jN\omega_0t}$ e integrando sul periodo, otteniamo: \[\int_{t_0}^{t_0+T_0} x(t)\cdot e^{-jN\omega_0t} dt = \sum_{k=-\infty}^{+\infty} a_k \int_{t_0}^{t_0+T_0} e^{j(k-N)\omega_0t} dt\]
\noindent Per l’ortogonalità delle esponenziali complesse, l’integrale è nullo per $k \neq N$ e vale $T_0$ per $k=N$, da cui segue l'equazione di analisi: \[a_N = \frac{1}{T_0}\int_{t_0}^{t_0+T_0} x(t)\cdot e^{-jN\omega_0t} dt\]
\noindent Possiamo dunque rappresentare un segnale periodico come una somma infinita di sinusoidi con frequenze multiple della frequenza fondamentale, indicizzate dall'indice armonico $k$.\par
Nella rappresentazione complessa compaiono anche frequenze negative, che non hanno un significato fisico autonomo, ma sono necessarie per garantire che il segnale risultante sia reale, grazie alla simmetria coniugata dei coefficienti. Inoltre, poiché la serie di Fourier utilizza esponenziali complessi, è possibile applicare la formula di Eulero: \[e^{jk\omega_0t} = \cos(k\omega_0t) + j\sin(k\omega_0t)\]
\noindent Questo con lo scopo di ottenere una rappresentazione reale. Esistono più scritture per la serie di Fourier, ma useremo esclusivamente quella sinusoidale per la formula di Eulero, notevolmente più intuitiva rispetto alle altre: \[v(t) = a_0 + \sum_{k=1}^{+\infty} 2|a_k|\cdot \cos(2\pi f_0kt + arg(a_k))\]
\noindent Nel caso di segnali periodici è utile considerare le proprietà di parità delle funzioni, poiché impongono vincoli sulla forma della serie di Fourier e ne semplificano la rappresentazione:
\begin{itemize}
	\item \textbf{Segnali dispari} $v(-t) = -v(t)$: Per segnali reali e dispari, il coefficiente medio $a_0$ è nullo e la serie di Fourier contiene esclusivamente termini di tipo seno, assumendo la forma: \[v(t) = 0 + \sum_{k=1}^{\infty} B_k\sin(\omega_0kt)\]
	\noindent Con coefficienti reali $B_k$.\par
	Nella rappresentazione complessa, ciò equivale a coefficienti $B_k$ puramente immaginari e antisimmetrici, ovvero $B_{-k} = -B_k$. Come esempio, consideriamo il segnale periodico definito da: \[v(t) = \begin{cases}
		-1, & t \in \left(-\frac{T_0}{2},0\right)\\
		1, & t \in \left(0,\frac{T_0}{2}\right)
	\end{cases}\]
	\noindent I coefficienti della serie di Fourier si ottengono tramite l’equazione di analisi e risultano:
	\[B_k = \frac{1}{j\pi k}\left(1 - (-1)^k\right)\]
	\noindent I quali sono non nulli solo per $k$ dispari.
	\item \textbf{Segnali pari} $v(-t) = v(t)$: Per segnali reali e pari, i coefficienti $A_k$ sono reali e simmetrici, quindi $A_{-k} = A_k$ e la serie di Fourier contiene esclusivamente termini di tipo coseno, assumendo la forma: \[v(t) = v_0 + \sum_{k=1}^{+\infty} A_k\cos(\omega_0kt)\]
	\noindent Come esempio, consideriamo un segnale reale e pari con periodo $T_0$.	I coefficienti della serie di Fourier sono: \[A_k = \begin{cases}
		\frac{1}{2}, & k=0\\
		\frac{\sin\left(\frac{\pi k}{2}\right)}{\pi k}, & k\neq 0
	\end{cases}\]
\end{itemize}
\noindent Nella pratica non è possibile sommare un numero infinito di termini, quindi si utilizza una \textbf{serie troncata}, considerando solo un numero finito $2N+1$ di componenti: \[v_N(t) = \sum_{k=-N}^{N} a_k e^{j\omega_0kt}\]
\noindent Con $a_k \in \mathbb{C}$, $\omega_0, t \in \mathbb{R}$. La troncatura, tuttavia, introduce un errore di approssimazione, che può essere quantificato tramite l'\textbf{errore quadratico medio}: \[MSE(v(t), v_N(t)) = \frac{1}{T_0}\int_{t_0}^{t_0+T_0} |v(t) - v_N(t)|^2 dt\]
\noindent Il quale misura l'energia dell'errore tra il segnale originale e l'approssimazione. Inoltre, quando il segnale presenta discontinuità, la serie troncata mostra oscillazioni locali vicino ai punti di salto. Questo comportamento è noto come \textbf{fenomeno di Gibbs}.

%

\section{Trasformata di Fourier}
La \textbf{Trasformata di Fourier} consente di effettuare l'analisi in frequenza dei segnali non periodici,  generalizzando l'omonima serie per ogni tipo di segnale con la seguente idea di fondo: \begin{enumerate}
	\item Preso un segnale non periodico $v(t)$ compreso in un intervallo $[-T_1, T_1]$, possiamo considerare quest'ultimo come l'intero periodo e dire che la funzione $\tilde{v}(t)$ è in esso periodica.
	\item Applichiamo la serie di Fourier al neo-periodico segnale $\tilde{v}(t)$: \[\tilde{v}(t) = \sum_{k=-\infty}^{+\infty} a_k e^{j k \omega_0 t}, \quad a_k = \frac{1}{T_0} \int_{-T_0/2}^{T_0/2} \tilde{v}(t) e^{-j k \omega_0 t} dt\]
	\item Leghiamo i coefficienti alla frequenza continua e otteniamo: \[T_0a_k = V(k\omega_0), \quad \text{dove} \quad V(\omega) = \int_{-\infty}^{\infty} v(t) e^{-j\omega t}dt\]
	\noindent $V(\omega)$ è l'inviluppo di $T_0a_k$ e corrisponde alla trasformata di Fourier.
\end{enumerate}
\noindent Facendo poi tendere il periodo $T_0 \to \infty$, il passo tra le frequenze $\omega_0 \to 0$ e la somma discreta diventa un integrale continuo: \[v(t) = \frac{1}{2\pi} \int_{-\infty}^{\infty} V(\omega) e^{j \omega t} d\omega\]
\noindent che è l'\textbf{Antitrasformata di Fourier}. In questo modo, $V(\omega)$ assume il ruolo di equazione di analisi e $v(t)$ quello di equazione di sintesi. Formalmente le definiamo come: \begin{definizione}
	\textbf{Trasformata di Fourier}\par
	\noindent Sia $v(t)$, con $t\in\mathbb{R}$ una funzione a valori reali o complessi. La relativa TdF è data da: \[\mathcal{F}[v(t)](f) = \int_{-\infty}^{\infty} v(t) e^{-j2\pi ft}dt = V(f)\]
	\noindent $V:\mathbb{R}\to\mathbb{C}$, funzione della variabile reale $f$. Corrisponde all'equazione di analisi.
\end{definizione}
\begin{definizione}
	\textbf{Antitrasformata di Fourier}\par
	\noindent Data una funzione $V:\mathbb{R}\to\mathbb{C}$, definiamo la relativa Anti-TdF: \[\mathcal{F}^{-1}[V(f)](t) = \int_{-\infty}^{\infty} V(f)e^{j2\pi ft}df = v(t)\]
	\noindent $v:\mathbb{R}\to\mathbb{C}$, funzione della variabile reale $t$. Corrisponde all'equazione di sintesi.
\end{definizione}
\noindent Possiamo quindi mettere in relazione i segnali non periodici con quelli periodici, mostrando come la serie di Fourier diventi l'omonima trasformata nel limite $T_0\to \infty$. Infatti, se il segnale $\tilde{v}(t)$ è periodico, la sua TdF non è continua, bensì consiste in un \textbf{treno di impulsi}: \[\tilde{V}(\omega) = 2\pi \sum_{k=-\infty}^{\infty} a_k \delta(\omega - k \omega_0)\]
\noindent Dove $a_k$ sono i coefficienti di fourier e l'elemento fra le tonde il treno di impulsi. In tal caso, l'antitrasformata si ricava come: \[\tilde{v}(t) = \frac{1}{2\pi}\int_{-\infty}^{+\infty} \tilde{V}(\omega)e^{j\omega t}d\omega = \frac{1}{2\pi}\sum_{-\infty}^{+\infty} 2\pi a_k\int_{-\infty}^{+\infty} \delta(\omega - \omega_0k)e^{j\omega t}d\omega\]
\noindent Riassumendo, abbiamo i seguenti casi di studio possibili: \begin{itemize}
	\item \textbf{Segnale non periodico} $v(t)$: Il segnale ammette una Trasformata di Fourier continua $V(\omega)$, che ne descrive il contenuto in frequenza: \[V(\omega) = \int_{-\infty}^{\infty} v(t) e^{-j\omega t} dt\]
	\noindent La rappresentazione nel dominio del tempo si ottiene tramite l’antitrasformata: \[v(t) = \frac{1}{2\pi} \int_{-\infty}^{\infty} V(\omega) e^{j\omega t} d\omega\]
	\item \textbf{Segnale periodico} $\tilde{v}(t)$ di periodo $T_0$: Il segnale è descritto tramite la serie di Fourier: \[\tilde{v}(t) = \sum_{k=-\infty}^{\infty} a_k e^{j k \omega_0 t}, \quad \omega_0 = \frac{2\pi}{T_0}\]
	\noindent La sua Trasformata di Fourier non è continua, ma consiste in un \textbf{treno di impulsi}: \[\tilde{V}(\omega) = 2\pi \sum_{k=-\infty}^{\infty} a_k,\quad \delta(\omega - k\omega_0)\]
	\item \textbf{Relazione tra segnali periodici e non periodici}: Se $v(t)$ rappresenta un singolo periodo di un segnale periodico $\tilde{v}(t)$, allora i coefficienti della serie di Fourier sono proporzionali ai campioni della TdF di $v(t)$: \[a_k = \frac{1}{T_0} V(k\omega_0)\]
	\noindent Al crescere del periodo $T_0$, lo spettro discreto si infittisce e tende allo spettro continuo del segnale non periodico. Quindi si torna a tempo continuo facendo tendere $T_0\to \infty$.
\end{itemize}
\noindent La trasformata di Fourier è tuttavia soggetta a delle condizioni di esistenza. Sia $v(t)$, con $t\in \mathbb{R}$ un segnale a valori reali o complessi; se almeno una delle seguenti condizioni è vera, allora la funzione accetta la trasformata: \begin{enumerate}
	\item $v(t)$ è sommabile e a variabile limitata su ogni intervallo finito di $\mathbb{R}$. Quindi è esprimibile come differenza di funzioni limitate non decrescenti. \[\int_{-\infty}^{+\infty} |v(t)|dt < \infty\]
	\item $v(t)$ è un segnale di energia, ovvero che \[\int_{-\infty}^{+\infty} |v(t)|^2 dt < \infty\]
	\item $v(t)$ è un segnale di potenza, con \[\lim_{t\to\infty} \frac{1}{2\pi}\int_{-\infty}^{+\infty} |v(t)|^2 dt < \infty\]
	\noindent In questo caso bisognerà finestrare il segnale (moltiplicare il segnale per una finestra rettangolare), considerandolo in un intervallo (la finestra) ben preciso.
\end{enumerate}

%

\section{Trasformate notevoli e proprietà della TdF}
Segue la lista delle trasformate notevoli da usare nello svolgimento degli esercizi: \begin{itemize}
	\item \textbf{Impulso}: \[\mathcal{F}[A\delta_0(t)](f) = A\]
	\item \textbf{Esponenziale complesso causale}: Supponendo che $Re(\lambda) < 0 $ e siccome $\delta_{-1}=0$ per $t<0$, abbiamo: \[\mathcal{F}[Ae^{j\Phi}e^{\lambda t}\delta_{-1}(t)](f) = \frac{Ae^{j\Phi}}{j2\pi f - \lambda}\]
	\item \textbf{Finestra rettangolare di altezza A e supporto T}: \[\mathcal{F}\left[A\Pi\left(\frac{t}{T}\right)\right](f) = ATsinc(fT)\]
	\noindent Dove $sinc()$ è una funzione oscillante non periodica che all'aumentare/diminuire delle frequenze tende a convergere verso l'asse reale delle $x$. per supporto invece si intende la "larghezza" del segnale.
	\item \textbf{Funzione costante}: Se $v(t)=A$, la sua versione finestrata è data da $v_T(t) = A\Pi(t/T)$, dunque: \[\mathcal{F}[v_T(t)](f) = \mathcal{F}\left[A\Pi\left(\frac{t}{T}\right)\right](f) = ATsinc(fT)\]
	\noindent E allora: \[\mathcal{F}[v(t)](f) = A\delta(f)\]
	\item \textbf{Funzione fasore}: Data $v(t) = Ae^{j2\pi f_0t}$, la moltiplichiamo per la finestra, ottenendo $v_T(t) = Ae^{j2\pi f_0t}\Pi(t/T)$. Quindi: \[\mathcal{F}[v_T(t)](f) = \mathcal{F}\left[Ae^{j2\pi f_0t}\Pi\left(\frac{t}{T}\right)\right](f) = ATsinc((f-f_0)T)\]
	\noindent Inoltre: \[\mathcal{F}[v(t)](f) = A\delta(f-f_0)\]
	\item \textbf{Funzione seno}: Sia $v(t) = A\sin(2\pi f_0t)$. Per la formula di Eulero otterremo due impulsi specchiati, rappresentabili nel piano immaginario. \[\mathcal{F}[v(t)](f) = \frac{A}{2j}(\delta(f-f_0) - \delta(f+f_0))\]
	\item \textbf{Funzione coseno}: Sia $v(t) = A\cos(2\pi f_0t)$. Per la formula di Eulero otterremo due impulsi specchiati rappresentabili nel piano reale. \[\mathcal{F}[v(t)](f) = \frac{A}{2}(\delta(f-f_0) + \delta(f+f_0))\]
\end{itemize}
\noindent E qui la lista delle proprietà, non poco simili a quelle viste per Laplace: \begin{enumerate}
	\item \textbf{Linearità}: \[av_1(t) + bv_2(t) \implies aV_1(f) + bV_2(f)\]
	\item \textbf{Riflessione e coniugazione}: \[v(-t) \implies V(-f)\quad \overline{v(t)} \implies \overline{V(-f)}\quad \overline{v(-t)} \implies \overline{V(f)}\]
	\item \textbf{Convoluzione nel dominio del tempo}: La convoluzione nel dominio del tempo diventa un prodotto in quello delle frequenze. \[[v_1 * v_2](t) \implies V_1(f)\cdot V_2(f)\]
	\item \textbf{Traslazione nel dominio del tempo}: \[v(t-\tau) \implies e^{-j2\pi f\tau}V(f)\]
	\item \textbf{Modulazione del segnale nel dominio del tempo}: \[v_1(t)\cdot v_2(t) \implies [V_1*V_2](f)\]
	\item \textbf{Traslazione nel dominio delle frequenze}: \[e^{j2\pi f_0t}v(t) \implies V(f-f_0)\]
\end{enumerate}

%

\section{Campionamento e replicazione}
Il \textbf{campionamento} è l’operazione che consente di passare da un segnale a tempo continuo a una rappresentazione discreta nel tempo. Prima di introdurlo bisogna avere presenti alcuni concetti preliminari: supponiamo un segnale continuo $v(t)$, e sia $T_c > 0$ il relativo \textbf{periodo di campionamento}, con \textbf{frequenza di campionamento} $f_c = 1/T_c$; definiamo il \textbf{treno di impulsi}: \[\hat{\delta}_{T_c} = \sum_{k = -\infty}^{+\infty} \delta(t - kT_c)\]
\noindent Ed è una sequenza infinita di impulsi traslati nel tempo ad intervalli $T_c$. In particolare, per Fourier, abbiamo le seguenti relazioni: \[\hat{\delta}_{T_c}(t) \implies \frac{1}{T_c}\sum_{k=-\infty}^{+\infty} \delta\left(f - \frac{k}{T_c}\right) \implies \frac{1}{T_c}\hat{\delta}_{\frac{1}{T_c}}(f)\]
\noindent Possiamo concludere che la TdF del treno di impulsi ideale è sempre un treno di impulsi, ma in frequenza, dove gli impulsi hanno l’area $1/T$ e sono equispaziati alla frequenza $1/T$. In questi termini possiamo definire il segnale campionato.
\begin{definizione}
	\textbf{Campionamento}\par
	\noindent Dato un segnale $v(t)$ con $t\in \mathbb{R}$, $0 < T_c\in \mathbb{R}$, definiamo campionamento del segnale $v(t)$ come: \[[samp_{T_c}v](t) = \sum_{k=-\infty}^{+\infty} v(kT_c)\]
	\noindent Il quale, per proprietà del campionamento dell'impulso diventa: \[[samp_{T_c}v](t) = \sum_{k=-\infty}^{+\infty} v(t)\delta(t-kT_c) = v(t)\hat{\delta}_{T_c}(t)\]
	\noindent Il risultato è dunque un segnale continuo costituito da impulsi separati in intervalli di $T_c$, le cui ampiezze sono uguali ai singoli campioni presi dall'originale.
\end{definizione}
\noindent Quando si lavora con segnali ripetuti periodicamente nel tempo allora si parla di \textbf{replicazione}, la quale è formalmente definita come:
\begin{definizione}
	\textbf{Replicazione}\par
	\noindent Sia un segnale $v(t)$ con $t\in\mathbb{R}$ e periodo di campionamento $0 < T_c\in\mathbb{R}$; definiamo replicazione del segnale $v(t)$ come: \[[rep_{T_c}v](t) = \sum_{k=-\infty}^{\infty} v(t-kT_c) = \sum_{k=-\infty}^{+\infty} v(t) * \delta(t-kT_c) = [v * \hat{\delta}_{T_c}](t)\]
\end{definizione}
\noindent Queste due operazioni sono duali nel dominio delle frequenze, campionando nel tempo si otterrà una replicazione in frequenza: \[[samp_{T_c}v](t) \iff \frac{1}{T_c}[rep_{T_c}V](f)\]
\noindent Mentre se si replica nel tempo si otterrà un campionamento in frequenza: \[[rep_{T_c}v](t) \iff \frac{1}{T_c}[samp_{T_c}V](f)\]
\noindent Tuttavia, attenzione: non sempre è possibile revertire il segnale a quello originale dopo il processo di campionamento. La condizione che definisce questa dinamica è data dal \textbf{teorema del campionamento Shannon-Nyquist}, dove supponiamo un segnale $v_0(t)$ limitato in banda, ovvero $V_0(f) = 0$ per $|f| > B$. Se la frequenza di campionamento soddisfa: \[f_c > 2B\]
\noindent Allora il segnale può essere riportato all'originale tramite un filtro di \textbf{ricostruzione} $H_r$, definito come: \[H_r(f) = T_c\Pi\left(\frac{f}{2f_L}\right) = \frac{1}{f_c}\Pi\left(\frac{f}{2f_L}\right)\]
\noindent Con $B < f_L < f_c - B$. Quindi il segnale originale sarà uguale alla convoluzione fra il segnale campionato e l'antitrasformata del filtro: \[v_0(t) = [(samp_{T_c}v_0) * h_r](t)\]
\noindent In caso di $f_c < 2B$, invece, si ha quello che è noto come \textbf{fenomeno di aliasing}, e non si ritornerà ad un segnale uguale all'originale.

%

\section{Esercizi svolti}

\begin{comment}
	``` Esempio - Lez29
	Dato un sistema a blocchi vogliamo capire come viene alterato dagli operatori. Dovremo passare dal dominio del tempo a quello delle frequenze con la TdF. Allo schema a blocchi è aggiunto l'operatore di convoluzione.
	
	Abbiamo le funzioni:
	- u(t) = 3cos(6\pi t) + cos(2\pi t)
	- w(t) = 2cos(4\pi t)
	- h(t) = 4\sinc(4t)
	- (output) b(t) = ?
	
	1. Applichiamo le TdF per passare al dominio delle frequenze
	A è l'ampiezza, f_0 è dato dalla formula Acos(2\pif_0t).
	- U(f) = \frac{3}{2}(\delta(f - 3) + \delta(f + 3)) + \frac{1}{2}(\delta(f - 1) + \delta(f + 1))
	- W(f) = \frac{2}{2}(\delta(f-2) + \delta(f+2))
	
	// ATsinc(tT) = A\Pi(\frac{f}{T})
	- H(f) = 1\cdot \Pi(\frac{f}{4})
	
	2. Disegnare grafici al posto di fare conti manuali (vedi foto da telefono)
	
	3. Effettua l'operazione di convoluzione fra U(f) e W(f) come mostrato nello schema a blocchi (Vedi foto da telefono)
	
	Ricorda che per fare convoluzione bisogna invertire un segnale e scorrerlo addosso all'altro.
	
	```
\end{comment}