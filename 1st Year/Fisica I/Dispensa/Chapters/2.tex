\section{Moti in una dimensione}
Iniziamo il percorso con la \textbf{cinematica}, la quale tratta il moto dei corpi, visti come un punto materiale, da un punto di vista descrittivo, ignorando le interazioni con l'ambiente circostante. Ciò avviene con il relativo asse di riferimento, il quale per ora sarà rappresentato da un singolo asse.\par
Pensando ad un corpo in movimento ci si aspetta che abbia una \textbf{posizione} in determinati punti e che si possa \textbf{spostare} sul piano, magari con una determinata \textbf{velocità}. Questi concetti sono strettamente legati e fanno riferimento punto scelto nell'asse. Sono definiti come:
\begin{itemize}
	\item \textbf{Posizione} $x$: Punto occupato, istante per istante, dal punto materiale rispetto ad un altro punto di riferimento, scelto come origine.
	\item \textbf{Spostamento} $\Delta x = (x_f - x_i)[m]$: Variazione della posizione di un punto materiale, da quella iniziale $x_i$, a quella finale $x_f$, in un certo intervallo di tempo.
	\item \textbf{Velocità} $v = \frac{\Delta x}{\Delta t}[m/s]$: Detta anche velocità media, è il rapporto fra lo spostamento del punto materiale e l'intervallo di tempo in cui ha compiuto tale movimento.
\end{itemize}
\noindent Potrebbe venirci l'idea o la necessità di misurare la velocità in frazioni più piccoli del moto, ottenendo un valore più preciso della stessa. Seguendo questo ragionamento, possiamo portarlo all'estremo e minimizzare questo intervallo, quindi il tempo. Ciò consente di ottenere la \textbf{velocità istantanea}, definita come il limite della stessa quando il tempo tende a zero: \[v_x = \lim_{\Delta t\to 0}\frac{\Delta x}{\Delta t}\]
\noindent Una dinamica particolare ma molto utile della fisica, è che talvolta le componenti delle equazioni sono interpretabili come funzioni o riscrivibili in termini di altre. Prendiamo infatti lo spostamento; questo è un valore dipendente dal tempo, quindi esprimibile in sua funzione. Se riprendiamo infatti la formula della velocità istantanea possiamo dire che, ricavando $v_f = v_i + \Delta t$: \[v_x = \lim_{\Delta t\to 0} \frac{x(t_f) - x(t_i)}{\Delta t} \implies \lim_{\Delta t\to 0} \frac{x(t_i+\Delta t) - x(t_i)}{\Delta t} = \frac{dx}{dt}\]
\noindent Facendo attenzione, si noterà che questa è una funzione incrementale, quindi una derivata. Abbiamo appena provato che la velocità istantanea è la derivata della funzione di spostamento rispetto al tempo.\par
Chiaramente, è possibile andare anche all'indietro, applicando l'integrale alla velocità ricaviamo lo spostamento a partire dalla prima. \[\frac{dx}{dt} = v_x \implies \int dx = \int v_xdt \implies x = x_0 + v_xt\]
\noindent Con questi passaggi abbiamo ottenuto la funzione del tempo $x(t)$, più comunemente conosciuta come \textbf{legge oraria}, la quale descrive il moto di un punto materiale con velocità costante.\newline

\noindent Lo studio dei fenomeni fisici non si piega ad ogni caratteristica in esso presente; infatti, come premesso nell'introduzione, verranno usati modelli di analisi per approssimarli ad una forma simile e riproducibile. La legge del moto appena ottenuta si applica con il modello di riferimento del \textbf{moto rettilineo uniforme}, esprimente la sua dinamica. Come diretta conseguenza, la velocità istantanea sarà sempre uguale indipendentemente dall'istante colto.\newline

% TODO AGGIUNGI GRAFICO MOTO RETTILINEO UNIFORME

\noindent E se la velocità cambiasse nel tempo? Non ci si può aspettare che in ogni fenomeno fisico i corpi si muovano a velocità costante; questa infatti può variare, e quando accade, si dice che il corpo \textbf{accelera}. Questa, come visto per la velocità, è misurata in un determinato intervallo di tempo, e quindi si può anche studiare nei singoli istanti tramite il limite. Definiamo:
\begin{itemize}
	\item \textbf{Accelerazione} $a = \frac{\Delta v_x}{\Delta t}[m/s^2]$: Rapporto fra la variazione di velocità rispetto al tempo. Va a $0$ quando la velocità del corpo è massima ed è negativa quando la velocità in direzione positiva decresce.
	\item \textbf{Accelerazione istantanea} $a_x = \lim_{\Delta t \to 0} \frac{\Delta v_x}{\Delta t}$: Il limite dell'accelerazione media per il tempo che tende a zero. Coglie il valore in un determinato istante di tempo.
\end{itemize}
\noindent Attuando un ragionamento simile a prima con il calcolo differenziale, possiamo vedere $v(t)$ come la funzione della velocità nel tempo e dire, sapendo che $t_f = t_i + \Delta t$: \[a_x = \frac{\Delta v}{\Delta t} = \frac{v(t_f) - v(t_i)}{\Delta t} \implies \frac{v(t_i + \Delta t) - v(t_i)}{\Delta t} = \frac{dv}{dt}\]
\noindent Quindi, l'accelerazione è la derivata della funzione di velocità rispetto al tempo. Come già visto, grazie all'integrale possiamo anche ottenere l'equazione che esprime la \textbf{velocità} con accelerazione costante: \[\int dv(t) = \int a dt \implies v(t) = v_0 + at\]
\noindent Come anche quella per lo \textbf{spostamento} negli stessi termini: \[\int v_0 + at dt = \int v_0 dt + \int at dt \implies x(t) = x_0 + v_0t + \frac{1}{2} at^2\]
\noindent Quest'ultima legge del moto è utilizzata nel modello di riferimento \textbf{punto materiale ad accelerazione costante}, ed è fondamentalmente un'estensione di quanto visto prima. Qui l'accelerazione media è numericamente uguale a quella istantanea in qualunque intervallo di tempo.\newline

% TODO AGGIUNGI GRAFICO PUNTO MATERIALE AD ACCELERAZIONE COSTANTE

\noindent Un caso differente di quanto appena visto è invece il \textbf{moto di caduta}. Tutti i corpi sotto effetto della gravità terrestre cadono con la stessa accelerazione costante: $g = 9,8m/s^2$. È possibile usare le equazioni cinematiche del precedente modello di analisi, l'unica differenza sta nel moto, che ha direzione verticale, ed essendo che l'accelerazione va verso il basso, bisognerà indicare la costante gravitazionale con segno negativo.

%

\section{Moti in due dimensioni}
Prima di iniziare abbiamo la necessità di aggiungere un attributo alle grandezze. Precedentemente, essendo stato il moto in una singola dimensione, potevamo ignorare questa caratteristica e lavorare con quelle che sono chiamate \textbf{grandezze scalari}, le quali indicano un solo valore numerico.\par
Lavorando ora in due dimensioni dobbiamo considerare anche la \textbf{direzione} del moto. Andremo quindi ad aggiungere questo attributo alle variabili e le chiameremo \textbf{grandezze vettoriali}. Sintatticamente non cambia molto, infatti le formule rimangono le stesse, eccezion fatta che ora le grandezze sono vettoriali, ma è possibile utilizzarle per nuovi modelli di analisi.\newline

\noindent \textbf{- Moto in due dimensioni con accelerazione costante}\par
\noindent Il punto materiale è specificato dal vettore posizione $\overrightarrow{r} = x\hat{i}+y\hat{j}$. Ciò significa che il moto può essere modellizzato in due moti indipendenti lungo rispettivamente l'asse $x$ e l'asse $y$ e quindi, per ottenere i vettori finali richiesti, sarà necessario effettuare la somma fra le parti. Supponendo che l'accelerazione abbia effetto in ambo le dimensioni, possiamo scrivere la legge del moto completa come segue: \[x = x_0+v_{0x}t+\frac{1}{2}at^2; y = y_0+v_{0y}t+\frac{1}{2}at^2\]
\noindent Dove naturalmente, le $x_0, y_0$ sono le posizioni iniziali, si tiene conto della velocità in un certo istante, come anche dell'accelerazione. Tenere a mente che velocità e accelerazione sono ora considerate grandezze vettoriali.\newline

% TODO Aggiungi grafico per il moto in due dimensioni con acc costante

\noindent \textbf{- Moto dei proiettili}\par
\noindent Per proiettile si intende un punto materiale che è lanciato in una certa direzione, ed è sempre influenzato dalla gravità per poi arrivare a terra. Ne consegue che il movimento crea una parabola. L'unica differenza rispetto al modello di analisi precedente è come il ruolo dell'accelerazione è assunto dalla costante gravitazionale, esattamente come nel moto di caduta, solo in due dimensioni. Ci sono tuttavia alcuni punti che è interessante analizzare.\par
Per prima cosa, è necessario notare come il movimento del punto materiale nell'asse delle $x$ ha sempre velocità costante, mentre in quello delle $y$ influisce l'accelerazione gravitazionale. Dunque possiamo vedere il primo come un moto rettilineo uniforme ed il secondo come un uniformemente accelerato, dandoci:
\begin{itemize}
	\item \textbf{Velocità iniziale di $x$}: $v_{0x} = v_0cos\theta$
	\item \textbf{Velocità iniziale di $y$}: $v_{0y} = v_0sin\theta$
	\item \textbf{Legge di $x$}: $x = x_0 + v_{0x}t$
	\item \textbf{Legge di $y$}: $y = y_0 - v_{0y}t + \frac{1}{2}gt^2$
\end{itemize}
\noindent Questo modello di analisi introduce anche due nuovi concetti nelle forme di \textbf{altezza massima} $h_{max} = \frac{v_{0y}^2}{2g}$, con relativo istante di tempo, e \textbf{gittata orizzontale}. La prima rappresenta il massimo della funzione e si ottiene ragionando sulla velocità del vettore verticale. Quando questa è uguale a zero, ne consegue che il proiettile è arrivato al massimo. La seconda, invece, è la distanza raggiunta dal proiettile una volta ritornato a terra dopo il lancio.\par
Per quanto riguarda il primo concetto, possiamo subito ottenere il valore dell'istante del tempo in cui la particella raggiunge il suo massimo, grazie alla legge del moto diciamo: \[t_{max} = \frac{v_0sin\theta}{g} = \frac{v_{0y}}{g}\]
\noindent Era la componente mancante per la legge del moto, pensa te. Sostituendola alla $t$, otteniamo quindi la posizione delle $y$ all'istante del tempo dove la funzione ha massimo: \[y = h_{max} = \frac{v_{0y}}{g} - \frac{1}{2}g\left(\frac{v_{0y}}{g}\right)^2 \implies h_{max} = \frac{v_{0y}^2}{2} - \frac{1}{2}\left(\frac{v_{0y}^2}{g}\right) = \frac{v_{0y}^2}{2g}\]
\noindent Per quanto riguarda la gittata, invece, possiamo fare un ragionamento sulla natura del fenomeno e notare che a prescindere dal modo in cui è lanciato, il moto sarà sempre una parabola perfettamente simmetrica. Dunque, il tempo che ha usato per arrivare al suo massimo è la metà di quello usato per ritoccare terra. Più semplicemente, $2t$. Sostituendolo alla legge oraria usata per $x$, otterremo infatti: \[R = \frac{2(v_{0x}v_{0y})}{g}\]

% TODO Aggiungi grafico per il moto di proiettile

\noindent \textbf{- Punto materiale in moto circolare uniforme}\par
\noindent Questo modello di analisi vede un punto materiale muoversi con una velocità scalare costante in senso circolare. Qui il vettore velocità, sempre tangente alla traiettoria, cambia continuamente direzione; inoltre, l'accelerazione è perpendicolare alla traiettoria e punta verso il centro del cerchio. I concetti da ricordare qui sono:
\begin{itemize}
	\item \textbf{Accelerazione centripeta} $a_c = \frac{v^2}{r} = r\omega^2$: Accelerazione con direzione perpendicolare al vettore della velocità, verso il centro della circonferenza.
	\item \textbf{Periodo del moto} $T=\frac{2\pi r}{v}$: Intervallo di tempo richiesto al moto per compiere un giro completo.
	\item \textbf{Velocità angolare} $\omega = \frac{2\pi}{T}$: Prodotto fra la frequenza e la lunghezza della circonferenza, è misurata in radianti.
\end{itemize}
\noindent Esiste inoltre una relazione fra la velocità angolare e la velocità con cui il punto si muove lungo la traiettoria circolare: \[\omega = 2\pi\frac{v}{2\pi r} = \frac{v}{r} \implies v = r\omega\]
\noindent Questo ci è particolarmente comodo, perché in questo modo possiamo esprimere l'accelerazione centripeta con una formula molto più semplice e compatta: \[a_c = \frac{v^2}{r} \implies a_c = \frac{(r\omega)^2}{r} = r\omega^2\]
\noindent Presta attenzione; se i vettori possono essere rappresentati attraverso le loro componenti rispetto agli assi, allora è possibile anche ottenere quello dell'\textbf{accelerazione totale}, considerando quello dell'accelerazione centripeta e quello dell'accelerazione tangente alla circonferenza. Più precisamente abbiamo:
\begin{itemize}
	\item \textbf{Accelerazione radiale} $a_r = -a_c = -\frac{v^2}{r}$: L'inverso dell'accelerazione centripeta, sicché sia un valore positivo.
	\item \textbf{Accelerazione tangenziale} $a_t = \left|\frac{dv}{dt}\right|$: Come detto dal nome, il vettore tangente al vettore della velocità istantanea, quindi una derivata.
	\item \textbf{Accelerazione totale} $\overrightarrow{a} = \overrightarrow{a_r}+\overrightarrow{a_t}$: La somma totale delle due accelerazioni appena viste.
\end{itemize}
\noindent Come ultimo concetto del capitolo, supponiamo di avere due osservatori $A,B$ di uno stesso fenomeno posti in posizioni diverse. Ciò significa che osserveranno l'evento con due origini differenti. Questo è un tipo di problema spesso utilizzato e necessita dei concetti di velocità e accelerazione \textbf{relative}.\par
Definiamo qui un tempo $t=0$ dove le origini coincidono; andando avanti nel tempo fino a $t$ si troveranno ad una distanza $v_{BA}t$ l'una dall'altra, sotto il punto di vista dell'osservatore $B$. Diciamo di volere la posizione del vettore della posizione relativa ad $A$. Avremo che: \[\overrightarrow{r}_A = \overrightarrow{r}_B + \overrightarrow{v}_{BA}t\]
\noindent Essendo questa la formula che mostra la posizione finale e quindi uno spostamento, possiamo ragionare con il calcolo differenziale per poter ottenere anche la velocità e l'accelerazione, derivando rispettivamente una o due volte la formula.

%

\section{Esercizi svolti}
\textbf{Esercizio 1: Moto rettilineo uniforme}\par
\noindent Una studiosa misura la velocità di un atleta che corre a ritmo costante su una strada rettilinea. Fa partire il cronometro quando arriva in un dato punto e lo ferma quando arriva $20m$ più avanti. Registra un tempo di $4,0s$.
\begin{itemize}
	\item \textbf{Qual è la velocità dell'atleta?}\par
	\noindent Richiesta esplicita, possiamo prendere direttamente la formula apposita senza usare criteri di equivalenza. \[v_x = \frac{\Delta s}{\Delta t} = \frac{20m - 0m}{4,0s} = 5,0m/s\]
	\item \textbf{Se l'atleta continua a correre per altri $10s$, quale sarà la sua posizione allora?}\par
	\noindent Fondamentalmente ci sta chiedendo la posizione dell'atleta alla fine di questi $10s$. Possiamo prendere anche qui la formula senza cambiare nulla. \[x_f = x_i+v_xt = 0m + 5m/s \times 10s = 50m\]
\end{itemize}

\noindent\textbf{Esercizio 2: Punto materiale ad accelerazione costante}\par
\noindent Un aereo atterra alla velocità di $140mi/h$.
\begin{itemize}
	\item \textbf{Qual è l'accelerazione dell'aereo, se il cavo di arresto lo ferma in $2s$?}\par
	\noindent Essendo l'accelerazione supposta costante, possiamo utilizzare direttamente la formula. Bisogna solo convertire le miglie orarie a metri al secondo: $140mi/h = \frac{140}{2,237} \approx 63m/s$. \[a_x = \frac{\Delta v_x}{\Delta t} = \frac{0-63m/s}{2,0s-0} = -32m/s^2\]
	\item  \textbf{Se l'aereo aggancia il cavo quando si trova in $x_i = 0$, quale sarà la sua posizione finale?}\par
	\noindent Molto semplicemente si tratta di una sostituzione dei dati alla formula per la posizione finale: \[x_f = x_i + \frac{1}{2}(v_{xi}+v_{xf})t = 0 + \frac{1}{2}(63m/s^2+0)2,0s = 63m\]
\end{itemize}

\noindent\textbf{Esercizio 3: Corpo in caduta libera}\par
\noindent Dal tetto di un palazzo una pietra è lanciata verso l'alto e la sua velocità iniziale è di $20m/s$. Il lancio avviene da un'altezza di $50m$ rispetto al suolo, per poi cadere a terra.
\begin{itemize}
	\item \textbf{Se $t_A$ è l'istante iniziale in cui la pietra lascia la mano del lanciatore, trovare l'istante in cui la pietra raggiunge la massima altezza.}\par
	\noindent Ci aspettiamo che, essendo la pietra lanciata verso l'alto, la velocità sia inizialmente positiva. Una volta raggiunta la massima altezza questa sarà a zero, mentre nel cadere avrà valore negativo. L'accelerazione avrà la stessa dinamica.\par
	Per trovare l'istante preciso in cui l'altezza è massima bisogna prima ricavare l'equazione per ottenere $t$: \[v_{yf} = v_{yi}+a_yt \implies t = \frac{v_{yf} - v_{yi}}{a_y} = \frac{0-20m/s}{-9,8m/s^2} = 2,04s\]
	\item \textbf{Trovare l'altezza massima raggiunta dalla pietra}\par
	\noindent Risulta comodo considerare due istanze, la seconda come finale. Sapendo questo possiamo usare la formula apposita: \[y_f = y_i + v_{xi}t + \frac{1}{2}a_yt^2 = 0+(20m/s)(2,04m) + \frac{1}{2}(-9,8m/s^2)(2,04s)^2 = 20,4m\]
	\item \textbf{Calcolare la velocità della pietra quando ripassa per l'altezza da cui era stata lanciata}\par
	\noindent Il punto iniziale è lo stesso, mentre il finale è la medesima altezza mentre la pietra cade verso il basso. Sarà necessario utilizzare l'equazione in base alla posizione: \[v_{yf}^2 = v_{yi}^2 + 2a_y(y_f - y_i) = (20m/s)^2 + 2(-8,8m/s^2)(0) = 400m^2/s^2\]
	\noindent Vogliamo tuttavia la velocità a grado uno, quindi usiamo la radice quadrata: \[\sqrt{v_{yf}^2} = \sqrt{400m^2/s^2} \implies v_{yf} = -20m/s\]
	\noindent Sebbene il risultato dalla radice quadrata sia positivo, bisogna tenere a mente che il movimento della pietra nella posizione finale è verso il basso, ragion per cui ha velocità negativa.
	\item \textbf{Trovare velocità e posizione della pietra al tempo $t = 5s$.}\par
	\noindent Il punto iniziale è sempre il solito, mentre adesso quello finale è la posizione a tempo $t=5s$. Prima calcoliamo la velocità in quest'ultima: \[v_{yf} = v_{yi} + a_yt = 20m/s +(-9,8m/s^2)5s = -29m/s\]
	\noindent E adesso la posizione effettiva: \[y_f = y_i + v_{yi}t + \frac{1}{2}a_yt = 0+(20m/s)5s + \frac{1}{2}(-9,8m/s^2)5s = -22,5m\]
\end{itemize}

% Esercizio su moto unif. accelerato
% proiettile
% circolare uniforme
% velocità e accelerazione relative