\section{Diagonalizzazione di una Matrice}
§10. Diagonalizzazione di una matrice (vedi [GS, Capitolo V])
10.1 Proposizione: Propriet`a delle matrici simili
10.2 Teorema: diagonalizabile se e solo se esiste una base di autovettori
10.3 Corollario: autovettori distinti diagonalizzabile
10.4 Osservazione: non `e necessario
10.5 Lemma
10.6 Teorema: condizioni per diagonalizzabilit`a
09/05/24
10.7 Algoritmo per diagonalizzazione
10.8 Osservazione: diagonalizzazione su R
10.9 Teorema Spettrale
\section{Basi Ortonormali}
§11. Basi ortonormali (vedi [GS, Capitolo III])
11.1 Definizione: matrice coniugata, H-trasposta e prodotto interno
11.2 Definizione: norma (euclidea)
11.3 Interpretazione geometrica in R2
11.4 Definizione: ortogonale
11.5 Proposizione: insieme ortogonale `e linearmente indipendente
11.6 Osservazione: coefficienti per base ortogonale
23/05/24
11.7 Definizione: ortonormale
11.8 Algoritmo di Gram-Schmidt
11.9 Corollario: ogni sottospazio possiede una base ortonormale
\section{Esercizi}