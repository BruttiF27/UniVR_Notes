\section{Moto in una dimensione}
Questa sezione si occuperà di \textbf{Cinematica}. Tratta il moto dei corpi da un punto di vista descrittivo, ignorando l'ambiente circostante. Negli esercizi verrà utilizzato il \textbf{punto materiale}, un corpo con una massa, ma dalle dimensioni infime, e i \textbf{diagrammi di moto}, che mostreranno lo spostamento del punto.

%

\subsection{Grandezze}
\begin{itemize}
    \item Posizione $\Delta x = x_f - x_i$
    \item Velocità e velocità istantanea $v_m = \Delta x/\Delta t$, $\lim_{\Delta t \to 0} \Delta x/\Delta t$
    \item Velocità scalare e velocità scalare istantanea $v_x = d/\Delta t$, $\lim_{\Delta t \to 0} dx/dt$
    \item Accelerazione media ed istantanea $a_m = \Delta v_x/\Delta t$, $a_x = \lim_{\Delta t \to 0} \Delta v_x/\Delta t = dv_x/dt$
\end{itemize}

%

\subsection{Moto rettilineo uniforme}
Legge oraria, oltre alle formule viste prima: $x_f = x_i + v_x(t_f - t_i)$.
\subsection{Moto rellilineo uniformemente accelerato}

\subsection{Punto materiale con accelerazione costante}
\subsection{Corpo in caduta libera}

\subsection{Calcolo differenziale applicato alla cinematica}

%

\subsection{Domande di teoria}
\begin{enumerate}
    \item \textbf{Che cosa sono posizione, velocità e velocità scalare?}\par
    La posizione x di un punto materiale è il punto occupato istante per istante rispetto ad un'origine; introduce il concetto di spostamento; una quantità vettoriale indicata con $s$ oppure $\Delta x$ che esprime la variazione della posizione in un certo intervallo di tempo con la seguente formula: $\Delta x = x_f - x_i$, dove la posizione finale $x_f = x_i + vel_x\times t$.\par\quad
    Un secondo concetto che introduce è la distanza; è una quantità scalare che indica il valore di tutti i movimenti effettuati, come un contapassi.\par\quad
    La velocità è il valore dato dal rapporto fra lo spostamento ed il l'intervallo di tempo. Possiamo ottenere rispettivamente: 
    \begin{itemize}
        \item Velocità vettoriale media; $v_x = \Delta x/\Delta t$.
        \item Velocità scalare media; .
        \item Velocità vettoriale istantanea; .
        \item Velocità scalare istantanea; .
    \end{itemize}  
	\item \textbf{Che cos'è l'accelerazione?}\par
    Una particella, spostandosi, varia la propria velocità. Se è positiva, significa che la velocità sta aumentando, se è negativa, significa che diminuisce, mentre se è nulla, la velocità è massima. Definiremo in modo simile alla velocità le seguenti formule:
    \begin{itemize}
        \item Accelerazione media; .
        \item Accelerazione istantanea; 
    \end{itemize}
    
    INSERIRE LEGGE ORARIA PERCHé NON CAPISCO ORA STO FRITTO.
    MANCA INOLTRE IL MOTO DI CADUTA.
\end{enumerate}

%

\subsection{Esercizi I}

% 

\section{Moto in due dimensioni}
\subsection{Domande di teoria}
\subsection{Esercizi II}