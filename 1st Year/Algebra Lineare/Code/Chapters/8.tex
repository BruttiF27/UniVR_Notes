\section{Diagonalizzazione di una Matrice}
§10. Diagonalizzazione di una matrice (vedi [GS, Capitolo V])
10.1 Proposizione: Propriet`a delle matrici simili
10.2 Teorema: diagonalizabile se e solo se esiste una base di autovettori
10.3 Corollario: autovettori distinti diagonalizzabile
10.4 Osservazione: non `e necessario
10.5 Lemma
10.6 Teorema: condizioni per diagonalizzabilit`a
09/05/24
10.7 Algoritmo per diagonalizzazione
10.8 Osservazione: diagonalizzazione su R
10.9 Teorema Spettrale
\section{Basi Ortonormali}
§11. Basi ortonormali (vedi [GS, Capitolo III])
11.1 Definizione: matrice coniugata, H-trasposta e prodotto interno
11.2 Definizione: norma (euclidea)
11.3 Interpretazione geometrica in R2
11.4 Definizione: ortogonale
11.5 Proposizione: insieme ortogonale `e linearmente indipendente
11.6 Osservazione: coefficienti per base ortogonale
23/05/24
11.7 Definizione: ortonormale
11.8 Algoritmo di Gram-Schmidt
11.9 Corollario: ogni sottospazio possiede una base ortonormale
\section{Esercizi}
\section{Appunti}
Non so molto delle matrici diagonalizzabili, né delle matrici simili. Tuttavia so come determinare se una base è ortogonale o ortonormale.\par\quad
Per il primo caso bisogna partire da uno spazio vettoriale, diciamo che ha quattro vettori. Per ognuno di questi è necessario verificare l'ortogonalità, ovvero moltiplicare ogni coppia realizzabile. Prendendo l'esempio dell'insieme da quattro vettori, dovremo effettuare moltiplicazioni per:
\begin{itemize}
    \item v1, v2
    \item v1, v3
    \item v1, v4
    \item v2, v3
    \item v2, v4
    \item v3, v4
\end{itemize}
Se anche una sola della somma di prodotti di questi vettori dovesse risultare diversa da $0$, allora l'insieme preso sotto esame NON SARÀ ortogonale.\newline

Passiamo invece alla base ortonormale. Per lavorarci è necessario introdurre il concetto di norma. Non lo so che cos'è, ma si calcola per il singolo vettore.\par\quad
Diciamo di avere il seguente vettore $v$. La norma è data dalla seguente formula:
\begin{center}
    $v = \begin{pmatrix}
        -1\\
        0\\
        i\\
        -1
    \end{pmatrix} \implies ||v|| = \sqrt{-1^2+0^2+i^2-1^2} = \sqrt{1+1-1} = 1$
\end{center}
Se e solo se valgono le seguenti condizioni possiamo dire che uno spazio vettoriale è una base ortonormale del suo insieme di definizione, sia esso $\mathbb{R}^n$ oppure $\mathbb{C}^n$:
\begin{itemize}
    \item Lo spazio vettoriale è base ortogonale.
    \item Tutti i vettori dello spazio hanno norma uguale ad $1$.
\end{itemize}




