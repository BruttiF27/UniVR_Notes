\section{Rotazione di un corpo rigido attorno a un asse fisso}
Nello scorso capitolo è stato introdotto il concetto di centro di massa, il quale ci consente di iniziare a vedere i corpi non più come particelle, bensì aventi una forma. Tuttavia, ora che i corpi possiedono uno spessore, avranno anche una dinamica di \textbf{rotazione} che si attiva quando sono in moto.\par
Supponiamo dunque un corpo che ruota, il quale avrà necessariamente un punto alla sua estremità che misura una distanza $R$ dal centro di massa. Ruotando, effettua uno spostamento $\Delta s$ in un intervallo di tempo $\Delta t$ ad una velocità $\frac{\Delta s}{\Delta v}$. Lo spostamento crea un arco; grazie alla geometria euclidea possiamo dire che $\Delta s = R\theta$, dove $\theta$ è l'ampiezza dell'angolo. Ciò fa ricavare quella che è conosciuta come \textbf{velocità angolare}, ovvero la variazione dell'angolo rispetto al tempo: \[v = \frac{\Delta s}{\Delta t} = \frac{R\Delta\theta}{\Delta t} = R\omega\] 
\noindent La velocità angolare è costante, perché non dipende dalla posizione del punto nel raggio. Facendone la derivata otteniamo poi l'\textbf{accelerazione angolare} $\alpha$: \[\alpha = \frac{dv}{dt} = \frac{d(R\omega)}{dt} = R\frac{d\omega}{dt} = R\frac{d^2\theta}{dt^2} = R\alpha\]
\noindent In breve, al posto dello spazio, velocità e accelerazione abbiamo le loro controparti angolari. Sostituendole alle equazioni del moto otteniamo il comportamento relativo agli angoli. Prendiamo per esempio un corpo composto da tante piccole massette $m_i$, ognuna con una posizione $s_i$; possiamo passare dal moto traslazionale all'angolare dicendo $s=R\theta$ e sarà possibile riutilizzare tutte le leggi del moto viste in precedenza.\par
È corretto pensare che, in quanto movimento, un corpo possa resistere alla rotazione. Questa capacità è misurata con il \textbf{momento di inerzia} $I$, in due modi diversi in base al numero di masse del corpo. Se questo è finito, usiamo: \[I = \sum_i m_ir_i^2\]
\noindent Mentre se parliamo di un numero continuo, avremo: \[I = \int r_i^2\cdot dm\]
\noindent dove $r_i$ è la posizione di ogni massa e $dm$ una massa infinitesima. La misura del momento di inerzia dipende anche dalla forma dell'oggetto, vedasi l'esempio di lanciare un coltello, che rotea più facilmente in base al modo in cui è lanciato. Infatti l'energia cinetica dipende anche dal momento di inerzia, ed è descritta come: \[K = \frac{1}{2}I\omega^2\]
\noindent Supponiamo ora di applicare una forza ad un corpo rigido attaccato a un asse fisso da una delle sue estremità; questo allora ruoterà attorno all'asse. In termini simili, chiamiamo la lunghezza del corpo a partire dall'asse \textbf{braccio} $r$, e la capacità di una forza di porre in rotazione un corpo è detta \textbf{momento della forza} $\overrightarrow{\tau}$, determinato da: \[\overrightarrow{\tau} = rFsin\theta = Fd\]
\noindent Dove $r$ è la distanza fra l'asse di rotazione ed il punto in cui è applicata la forza, e $d$ la distanza fra l'asse di rotazione e il vettore della forza. Il $\sin\theta$ è poi l'angolazione che determina l'efficacia della forza. Difatti, esercitando una forza parallela al braccio, non ruota, mentre se è perpendicolare si avrà massimo effetto.\par
A partire dalla definizione di momento della forza, possiamo sostituire la definizione di forza vista nella seconda legge di Newton con le sue componenti angolari: \[\tau_i = r_iF_i = r_im_ia_i = r_im_ir_i\alpha = r_i^2m_i\alpha\]
\noindent In necessità di calcolare il momento totale, si farà la sommatoria dei $\tau$, che risulta essere: \[\tau_{tot} = I\alpha\]
\noindent In parole povere, stiamo sostituendo le variabili da termini di traslazione in termini rotazionali:
\begin{itemize}
	\item $K = \frac{1}{2}mv^2 \implies K = \frac{1}{2}I\omega^2$
	\item $F = ma \implies \tau I\alpha$
\end{itemize}
\noindent Quindi il momento di inerzia prende il posto della massa e il momento di una forza sostituisce la forza. Abbiamo tuttavia un terzo elemento; il \textbf{momento angolare}, usato al posto della quantità di moto: \[\overrightarrow{L} = \overrightarrow{r} \times \overrightarrow{p}\]
\noindent Il momento angolare si conserva, quindi è costante in assenza di forze.

%

\section{Legge di gravitazione universale}
La \textbf{legge di gravitazione universale} di Newton afferma che ogni punto materiale attrae ogni altro punto materiale con una forza direttamente proporzionale al prodotto delle loro masse, ed inversamente proporzionale al quadrato della loro distanza reciproca.\par
Più semplicemente, ogni corpo esercita un'attrazione gravitazionale su altri oggetti. La legge esprime un valore vettoriale con la formula: \[F_g = G\frac{m_1m_2}{r^2}\]
\noindent Dove $m_1, m_2$ sono le due masse, $r$ è la distanza fra di loro, e $G$ è la \textbf{costante di gravitazione universale}, corrispondente al valore $6.7*10^{-11}[N\cdot m^2/kg^2]$. Essendo quest'ultimo un valore molto piccolo, è banale capire che solo corpi enormi sono capaci di esercitare questa forza attrattiva in modo significativo e percepibile. Infatti normalmente una delle due masse sarà indicata come $M_T$, ovvero massa della terra.\par
Tuttavia, attenzione: questa legge esprime una forza, ed in quanto tale possiamo sostituire nell'equazione la sua definizione data dalla seconda legge di Newton. In questi termini possiamo ricavare l'accelerazione, qui chiamata \textbf{di caduta libera}, ed è quella del singolo corpo: \[ma = G\frac{M_T\cdot m}{r^2} \implies a = G\frac{M_T}{r^2} \implies g = G\frac{M_T}{r^2}\]
\noindent Possiamo naturalmente ragionare in termini energetici anche con la gravità. Sostituiamo quindi la definizione di attrazione gravitazionale alla forza nell'integrale discusso nelle sezioni precedenti, ottenendo la formula per l'energia potenziale: \[L = \Delta E = \int F\cdot ds = \int_{r_1}^{r_2} G\frac{M_T\cdot m}{r^2} = -\frac{G\cdot M_T\cdot m}{r}\Big|_{r_1}^{r_2}\]
\noindent Possiamo notare come più ci si allontana dalla massa, minore sarà il valore della forza gravitazionale, fino ad arrivare a zero. Passiamo ora però ad analizzare il moto dei pianeti in un sistema solare; questo è descritto tramite le tre \textbf{leggi di Keplero}, le quali affermano che:
\begin{enumerate}
	\item Tutti i pianeti percorrono un'orbita ellitica ed il sole occupa uno dei due fuochi. Il punto dell'orbita più vicino al sole è detto perielio, mentre quello più lontano è l'afelio.
	\item Nel moto, il raggio vettore che unisce il sole ad un pianeta descrive aree uguali in tempi uguali. Ciò significa che, dati due intervalli di tempo uguali, un pianeta copre distanze diverse in base alla vicinanza che ha col sole. In ogni caso, l'area coperta dallo spostamento sarà la stessa.\par
	Ne consegue che l'energia si conserva, perché il prodotto vettoriale fra la distanza $r$ e la quantità di moto della massa deve sempre dare lo stesso valore.
	\item Il quadrato del periodo di rivoluzione è direttamente proporzionale al cubo del semiasse maggiore dell'orbita. Dunque tanto più grande è l'orbita, tanto più tempo ci si impiegherà a percorrerla.\par
	Questa relazione si ricava sostituendo nella seconda legge di Newton la forza con quella gravitazionale e la massa con quella del pianeta. Dunque: \[F_g = M_pa \implies \frac{G\cdot M_s\cdot M_p}{r^2} = M_p\left(\frac{v^2}{r}\right)\]
	\noindent Abbiamo visto come nei moti circolari la velocità è data da $2\pi r/T$, con $T$ periodo di rivoluzione; cerchiamo quindi di isolare la velocità e sostituirla con questo valore: \[\frac{G\cdot M_s\cdot M_p}{r^2} = M_p\left(\frac{v^2}{r}\right) \implies \frac{G\cdot M_{s}}{r} = v^2 \implies \frac{G\cdot M_{s}}{r} = \frac{(2\pi r)^2}{T^2}\]
	\noindent E adesso possiamo isolare il quadrato del periodo di rivoluzione per verificare la legge: \[\frac{G\cdot M_{s}}{r} = \frac{(2\pi r)^2}{T^2} \implies T^2 = (2\pi r)^2 \frac{r}{G\cdot M_s} = \frac{4\pi^2}{G\cdot M_s}r^3\]
	\noindent Il quadrato del periodo $T$ è quindi direttamente al cubo del semiasse maggiore $r^3$.
\end{enumerate}
\noindent Restiamo sempre in ambito energetico e supponiamo di avere un pianeta ed un corpo di massa $m$, lanciato verso l'alto a partire dal terreno. Possiamo sfruttare l'energia per determinare il valore minimo della velocità necessaria per arrivare ad una determinata distanza dal centro del pianeta.\par
Riprendiamo quindi la formula dell'energia totale e sostituiamo le componenti con le loro relative definizioni: \[E_{tot} = K + U_g \implies E_{tot} = \frac{1}{2}mv^2 -\frac{G\cdot M\cdot m}{r}\]
\noindent In caso di un sistema isolato costituito solo dalle due masse $M,m$ l'energia meccanica del sistema è anche il valore dell'energia totale, in quanto si conserva, e possiamo dire che si trasferisce da una all'altra in caso di movimenti: \[\Delta K + \Delta U_g = 0 \implies E_i = E_f\]
\noindent Dunque, se il corpo $m$ si sposta da un punto ad un altro l'energia totale del sistema è costante, e possiamo dire, sostituendo le definizioni alle energie iniziale e finale: \[\frac{1}{2}mv_i^2 -\frac{G\cdot M\cdot m}{r_i} = \frac{1}{2}mv_f^2 -\frac{G\cdot M\cdot m}{r_f}\]
\noindent Quando l'oggetto è ad altezza minima, quindi si trova sulla superficie della Terra, diciamo: $v=v_i$, $r=r_i=r_T$, mentre quando raggiunge la quota massima: $v=v_f=0$, $r=r_f=r_{max}$. In questi termini, possiamo sostituire questi valori all'equazione che eguaglia l'energia iniziale a quella finale e ricavare il valore della velocità con alcuni passaggi algebrici: \[\frac{1}{2}mv_i^2 - \frac{G\cdot M_T\cdot m}{r_T} = 0 - \frac{G\cdot M_T\cdot m}{r_{max}} \implies v_i^2 = 2GM_T\left(\frac{1}{r_T} - \frac{1}{r_{max}}\right)\]
\noindent Questa espressione è utilizzata per calcolare la \textbf{velocità di fuga}, ovvero la velocità minima che il corpo deve avere sulla superficie terrestre per poter sfuggire all'influenza del campo gravitazionale del pianeta. \[v_{fuga} = \sqrt{\frac{2GM_T}{r_T}}\]
\noindent Ne consegue che se la velocità non è sufficiente per sfuggire, il corpo ritornerà verso la terra ed il suo moto descriverà una parabola.

%

\section{Esercizi svolti}