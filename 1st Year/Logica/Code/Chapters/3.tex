La \textbf{Logica del Primo Ordine}, o dei Predicati è un'espansione della Logica Proposizionale. La novità principale è la presenza di due nuovi simboli detti \textbf{Quantificatori}, i quali legano più di ogni altro connettivo e ci consentiranno di descrivere strutture matematiche:
\begin{itemize}
    \item \textbf{Per ogni} - $\forall$\par
    Introduce una proprietà valida per ogni variabile a lui legata.
    \item \textbf{Esiste} - $\exists$\par
    Introduce l'esistenza di un elemento con una data proprietà.
\end{itemize}
Da qui in poi intenderò "logica dei predicati" con l'acronimo \textbf{FOL}, ovvero First Order Logic.

%

\section{Sintassi}
Essendo la FOL un'espansione della logica proposizionale, è corretto pensare che tutte le entità sintattiche studiate finora siano presenti anche in questo linguaggio.\par
Più precisamente è formato da:
\begin{itemize}
    \item \textbf{Connettivi}; $\land, \lor, \to, \bot$.
    \item \textbf{Quantificatori}; Visti poco fa, $\forall, \exists$.
    \item \textbf{Simboli di variabili}; Insieme numerabile $Var = \{v_0, v_1, ...\}$, i cui elementi si dicono variabili. Queste verranno indicate con una lettera ed indicizzate.
    \item \textbf{Simboli di relazione}; Sono tutti i simboli n-ari utilizzati per le relazioni.
    \begin{eg}
        Arietà delle relazioni\par
        Ritengo necessario approfondire, quindi ascoltami bene. Abbiamo un insieme di simboli che indicano relazioni fra variabili di un numero potenzialmente infinito, ragion per cui n-arie. In precedenza ho menzionato il significato di \textbf{arietà}, ovvero quante variabili può contenere una data relazione. Diremo, per capire meglio:
        \begin{itemize}
            \item \textbf{Relazione un-aria} $R^{(1)}$\par
            Relazione che coinvolge una singola variabile, come una semplice proprietà. L'essere un numero pari è una proprietà.
            \item \textbf{Relazione bin-aria} $R^{(2)}$\par
            Relazione che coinvolge due variabili. In questo caso possiamo parlare di operazioni come l'addizione o la moltiplicazione; infatti per eseguirle abbiamo bisogno di due variabili. Le operazioni menzionate sono dunque relazioni binarie.
        \end{itemize}
    \end{eg}
    \item \textbf{Simboli di funzione}; Simboli n-ari utilizzati come argomento per le variabili nell'insieme di lavoro. Prendiamo per esempio ogni $n \in \mathbb{N}$; avremo di conseguenza un insieme delle funzioni $F^{n}$ dove alla $n$ sostituiremo il numero apposito.
    \item \textbf{Simbolo opzionale}; Relazione binaria non appartenente a $R^{(2)}$. Se esiste un $n \geq 1$ tale che $F^{(n)} \neq \emptyset$, allora l'alfabeto $L$ deve contenere il simbolo $=$.
    \item \textbf{Simbolo ausiliare}; Parentesi tonde "$()$", punto "$.$" e virgola "$,$".
\end{itemize}
Avremo infine un insieme delle \textbf{costanti}, corrispondente a $F^{(0)}$ e denotato con \textbf{C} e l'insieme \textbf{L-stringhe}, il quale contiene tutte le stringhe componibili usando i simboli appena visti.\par\quad
Un'ulteriore novità della FOL sono i \textbf{termini}; il punto di partenza per l'ottenimento delle formule logiche.
\begin{definition}
    \textbf{L-termini}\par
    Dato l'alfabeto $L$, l'insieme $TERM_L$ dei termini è il più piccolo insieme $X$ tale che:
    \begin{itemize}
        \item $Var \in X$.
        \item $C \in X$.
        \item Se $t_1, ..., t_n \in X$ e $f$ è un simbolo di funzione n-aria, allora $f(t_1, ..., t_n) \in X$.
    \end{itemize}
\end{definition}
\begin{definition}
    \textbf{L-formule}\par
    Dato l'alfabeto $L$, l'insieme $FORM_L$ delle formule è il più piccolo insieme $X$ tale che:
    \begin{itemize}
        \item $\bot \in X$.
        \item Se $t_1, t_2 \in TERM_L$, allora $t_1 = t_2 \in X$.
        \item Se $P$ è una relazione k-aria e $t_1,...,t_k \in TERM_L$, allora $P(t_1,...,t_k) \in X$.
        \item Se $\phi, \psi \in X$, allora $\phi \circ \psi \in X$, dove $circ \in \{\land, \lor, \to\}$.
        \item Se $x \in Var$ e $\phi \in X$, allora $Qx.(\phi) \in X$, dove $Q \in \{\forall, \exists\}$.
    \end{itemize}
\end{definition}
Ovviamente l'esistenza di questi due concetti implica la presenza di \textbf{sottotermini} e \textbf{sottoformule}, dove tuttavia spiegarli risulta ridondante. Parliamo ora invece delle istanze in cui si possono trovare le variabili; ovvero \textbf{libere} e \textbf{legate}. Considera la seguente formula:
\begin{center}
    $\forall x.(P(x,y)) \to \forall z.(Q(z,x))$
\end{center}
Puoi osservare che nel primo ramo la variabile $x$ è legata dal quantificatore per ogni, mentre $y$ no, risultando quindi libera. Nel secondo ramo invece $z$ risulta quella legata. Tieni a mente inoltre che se una variabile è legata in un ramo, ciò vale solo per la specifica zona.
\begin{definition}
    \textbf{Occorrenze libere, insieme FV}\par
    Sia $\phi \in FORM_L$, allora si dice che un'occorrenza di $x \in Var$ è libera se non occorre in una sottoformula del tipo $Qx.y$, dove $Q$ è il quantificatore.\par
    Indichiamo infine l'insieme delle variabili libere con la notazione $FV[\phi]$, dove nelle parentesi quadre si inserisce il simbolo indicante una formula. Infatti con la notazione si intende il concetto di "Variabili libere nella formula $\phi$".
\end{definition}
\begin{definition}
    \textbf{Insieme delle variabili libere}\par
    INSERT DEFINITION
\end{definition}
\begin{definition}
    \textbf{Insieme delle variabili chiuse}\par
    INSERTO DEFINITION
\end{definition}


% - Continua da qua - Pag.73 dispense

\section{Sistemi Deduttivi}
formalizzazione, contromodello, struttura di Peano

%

\section{Semantica}

%

\section{Esercizi}









%%%%%%%%%%%%%%%%%%%%%%%%%%%%
\section{Appunti}
\subsection{Strutture e Tipi di Similarità}
\subsection{Derivazione Semantica}
\subsection{Deduzione Naturale}
\subsection{Teoremi di Correttezza e Completezza}