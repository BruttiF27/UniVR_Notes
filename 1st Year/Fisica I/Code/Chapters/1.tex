\section{Grandezze Fisiche}
Si dice \textbf{grandezza fisica} una proprietà misurabile mediante un dispositivo apposito\footnote{Come i chili sono misurati con le bilance}.\par
Si esprimono nella formula "valore $\times$ unità di misura". Distinguiamo le:
\begin{itemize}
    \item \textit{Fondamentali}; Concetti indipendenti l'uno dall'altro indefinibili in termini di altre grandezze.
    \item \textit{Derivate}; Definibili mettendo in relazione le grandezze fondamentali.
\end{itemize}
\begin{table}[h]
    \centering
    \begin{tabular}{c | c}
       \textbf{Grandezze fondamentali} & \textbf{Grandezze derivate}\\
       &\\
       Lunghezza $[L]$ & Superficie $[L^2]$ \\
       Massa $[M]$ & Volume $[L^3]$\\
       Tempo $[t]$ & Velocità $[L/t]$\\
       Intensità di corrente $[i]$ & Accelerazione $[L/t^2]$\\
       Temperatura assoluta $[T]$ & Forza $[M\times L/t^2]$\\
       & Pressione $[(M\times L/t^2)/L^2]$\\
    \end{tabular}
    \caption{Grandezze Fisiche}
    \label{tab:my_label}
\end{table}
Noi gente sana di mente utilizziamo l'insieme di unità di misura \textbf{Sistema Internazionale}, gruppo di grandezze particolarmente comodo perché i multipli e i sottomultipli sono ottenibili moltiplicando o dividendo per 10.\par\quad
Nello svolgimento degli esercizi è bene tenere a mente queste linee guida per la scrittura dei risultati:
\begin{itemize}
    \item \textit{Cifre significative}; Se il numero è decimale, è possibile ignorare tutti gli 0 a partire da destra finché non si trova una cifra diversa.\par Ex. $0,0250 = 0,025$.
    \item \textit{Notazione scientifica}; Utile per contrarre numeri molto grandi o piccoli. Si moltiplica o divide il dato fino alla forma $1,0 \times 10^n$ oppure $1,0 \times 10^-n$ se decimale.\par Ex. $27522 = 2,7 \times 10^4$.
\end{itemize}

%

\subsection{Analisi dimensionale}
L'\textbf{analisi dimensionale} è un accertamento sulla grandezza fisica risultante dell'equazione presa in esame. Funziona considerando le unità di misura come variabili in un'equazione.
\begin{eg}
    Prendiamo la seguente formula indicante una velocità: $v = at$.\par
    Per controllare se è dimensionalmente corretta, si sostituiscono ai valori nell'equazione le loro unità di misura. Se le misure sono concordanti, la formula sarà corretta.\par
    Abbiamo quindi: $v = L/T$, $a = L/T^2$, $t = T$.\newline

    Otteniamo la seguente equazione: $\dfrac{L}{T} = \dfrac{L}{\cancel{T^2}}\times \cancel{T} \to \dfrac{L}{T} = \dfrac{L}{T}$
\end{eg}

%

\section{Vettori}
I \textbf{vettori}, indicati con $\overrightarrow{A}$, sono oggetti nel piano cartesiano definiti mediante due misure: la distanza da un punto detto \textit{origine} e la direzione orientata relativamente ad un \textit{asse di riferimento}. Utilizzano \textbf{grandezze vettoriali}, espresse con un valore e una direzione, e possono essere rappresentati in un sistema di coordinate cartesiane o polari. Useremo le seguenti formule per ottenere le coordinate:
\begin{itemize}
    \item Coordinate cartesiane in funzione delle polari; $x = rcos(\theta)$, $y = rsin(\theta)$.
    \item Coordinate polari in funzione delle cartesiane; $tan(\theta) = \dfrac{y}{x}$, $r = \sqrt{x^2 + y^2}$.
\end{itemize}
La grandezza del vettore si dice \textbf{modulo} ed è un valore sempre positivo. La presenza di valori implica che è possibile operarci; è possibile effettuare:
\begin{itemize}
    \item \textit{Somma fra vettori}; $\overrightarrow{A} + \overrightarrow{B} = \overrightarrow{C}$
    \begin{itemize}
        \item \textit{Metodo punta-coda} - INSERIRE IMMAGINE\par
        Il vettore risultante è ottenuto attaccando la testa del primo vettore $A$ alla coda dell'altro $B$. Traccia una linea dalla coda di $A$ alla testa di $B$ e hai fatto.
        \item \textit{Somma delle parti} - INSERIRE IMMAGINE\par
        I vettori possono essere visti come la somma delle proprie componenti, ovvero nella forma $\overrightarrow{A} = A_x\hat{i} + A_y\hat{j}$, dove le varie A sono i moduli delle proiezioni dati dalle formule $A_x = Acos(\theta), A_y = Asin(\theta)$, mentre le lettere con accento circonflesso i vettori unitari che determinano la direzione.\par
        Ciò rende possibile utilizzare le formule polari viste prima:\par$A = \sqrt{A_x^2 + A_y^2}$, $\theta = tan^-1(\dfrac{A_y}{A_x})$.
    \end{itemize}
    Questa operazione gode di proprietà commutativa e associativa.
    \item \textit{Sottrazione fra vettori}; $\overrightarrow{A} - \overrightarrow{B} = \overrightarrow{C}$\par
    La sottrazione usa la definizione di \textbf{opposto}, definito come $\overrightarrow{A} + \overrightarrow{-A} = 0$. Si tratta di un vettore parallelo al primo, ma con verso opposto.\par
    Attacca le code dei due vettori e traccia una linea che collega le teste: questo è il tuo risultato.
    \item \textit{Moltiplicazione con scalare}; $n\overrightarrow{A}$\par
    Abbastanza autoesplicativo, serve solo a modificare il modulo ed eventualmente direzione se è negativo.
\end{itemize}

%

\section{Domande di teoria}
\begin{enumerate}
    \item \textbf{Quali sono le tre grandezze fondamentali della fisica?}
    \begin{itemize}
        \item Massa [Kg]; Quantità di materia contenuta in un oggetto.
        \item Lunghezza [m]; Distanza fra due punti nello spazio.
        \item Tempo [s]; Specifica istanza di tempo misurata rispetto ad una iniziale.
    \end{itemize}
    Tutte le altre grandezze sono derivate da queste tre mediante combinazioni matematiche.
    \item \textbf{In cosa consiste l'analisi dimensionale?}\par
    Si tratta di un controllo effettuato sull'equazione sulla quale si lavora per avere la conferma che la grandezza risultante sia quella desiderata. Si attua mediante il calcolo letterale. Se è stato fatto correttamente i due membri dell'equazione avranno la stessa dimensione.
    \item \textbf{Che scopo ha la notazione scientifica?}\par
    La notazione scientifica risulta utile per scrivere un numero particolarmente grande o piccolo in forma ridotta e quindi più leggibile. Consigliabile approssimare solo una volta ottenuto il risultato finale. Ex. $1500 = 1,5\times10^3$.
    \item \textbf{Cos'è un vettore? Che grandezza usa? Com'è definito e quali operazioni si possono fare?}\par
    Un vettore $\overrightarrow{A}$ è un oggetto definito mediante due grandezze: distanza da un punto detto origine e direzione orientata relativamente ad un asse di riferimento. Utilizza le grandezze vettoriali, definite tramite un valore con la sua unità di misura e una direzione in funzione di un asse di riferimento.\par
    Le operazioni effettuabili sono:
    \begin{itemize}
        \item Somma fra vettori; svolta tramite metodo testa-coda oppure con la somma fra le due proiezioni su asse x e y.
        \item Sottrazione fra vettori; una somma algebrica che utilizza la definizione di opposto per il vettore interessato.
        \item Moltiplicazione con scalare; influisce sulla grandezza (o modulo) del vettore senza modificare verso e direzione.
    \end{itemize}
    \item \textbf{Qual è lo scopo dei vettori unitari?}\par
    I vettori unitari \^{i}, \^{j}, \^{k}, sono vettori unidimensionali di modulo 1; servono a dare la direzione di un vettore normale. Contribuiscono a trovare il vettore completo nella seguente formula: $\overrightarrow{A} = A_xi + A_yj + A_zk$.
\end{enumerate}

\section{Esercizi}

\begin{theorem}
    Here goes a theorem.
\end{theorem}

\begin{proof}
        Here goes the proof
\end{proof}


\begin{corollary}
    Here goes a collorary
\end{corollary}

\begin{eg}
    Here goes an example
\end{eg}

\begin{note}
    Here goes a note 
\end{note}


\begin{lemma}
    Here goes a lemma
\end{lemma}

\begin{prop}
    Here goes a proposition
\end{prop}


\begin{definition}
    Here goes a definition 
\end{definition}







