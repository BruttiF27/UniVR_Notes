La statistica si occupa della raccolta, descrizione ed analisi dei dati e ci aiuta a trarre delle conclusioni in base a quanto ottenuto.\par
Anzitutto, allo statista è richiesta l'ideazione dell'algoritmo ideale di valutazione per la raccolta dei dati, dopodiché, dato un sottoinsieme della \textbf{popolazione}\footnote{Indicato con $M$, si tratta dell'insieme più grande che contiene ogni elemento. Presenta inoltre le caratteristiche reali, oggetto di studio ultimo degli statisti.}, si effettuano delle \textbf{inferenze}, le quali saranno poi \textbf{descritte} mediante appositi grafici e tabelle.\par
Queste ultime due parole in neretto non sono evidenziate a caso, infatti distinguono le due parti della statistica, nostro oggetto di studio:
\begin{itemize}
	\item \textbf{Statistica descrittiva}; Si occupa dell'illustrazione e sintetizzazione dei dati.
	\item \textbf{Statistica inferenziale}; Si occupa della ricerca e l'ottenimento dei dati.
\end{itemize}
Ci concentreremo poi sullo calcolo della \textbf{probabilità}, concetto strettamente legato alla statistica, in quanto ci consente di fare assunzioni sul risultato di un dato evento, come il lancio di un dado. Definiamo l'insieme di tali ipotesi come \textbf{modello probabilistico} e risulta utile per definire non solo le aspettative, ma anche per capire quali siano i risultati probabili dell'evento.\par
L'esame sarà di tipo informatizzato e comprenderà una parte di teoria come una parte di lavoro con il linguaggio di programmazione R.