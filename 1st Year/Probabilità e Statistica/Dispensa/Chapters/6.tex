\section{Introduzione}
R è linguaggio di programmazione. Viene utilizzato per l'analisi statistica dei dati e la loro visualizzazione, statistica descrittiva, machine learning, manipolazione dei dati, datamining e anche nella ricerca scientifica.\par
Si tratta di un codice open-source, la cui sintassi è estremamente flessibile e a tratti simile a C. R è poi diviso in pacchetti, ovvero insiemi di funzioni create da altri utenti. In particolare vedremo:
\begin{itemize}
	\item \textbf{ggplot2}; usato per la creazione dei grafici.
	\item \textbf{dplyr}; usato per la manipolazione dei dati.
\end{itemize}
I files con dominio ".r" sono detti \textbf{R-Scripts}; infatti ciò che andremo a scrivere e far elaborare dal software sono fondamentalmente dei testbench per ritornare determinati risultati o comportamenti. Per lavorare su di essi è necessario l'uso del terminale, e dove è possibile utilizzare l'interfaccia di R innata, è consigliato, e così faremo nel corso, l'ambiente di sviluppo \textbf{R Studio}. Spiegare un IDE esula dallo scopo della dispensa, arrangiati e leggi le relative documentazioni.\newline

Installato R Studio avremo davanti il terminale, l'ambiente dei dati e una terza interfaccia variabile per i grafici, manuale e tant'altro. Abbiamo nominato dei pacchetti con i quali andremo a lavorare; questi devono essere installati con il seguente comando:
\begin{verbatim}
	> install.package("nomePacchetto")
\end{verbatim}
Abbiamo tutto e adesso possiamo iniziare a lavorare col linguaggio.

%

\section{Funzioni di base}
\subsection{Variabili e funzioni di uso generale}
esempi di codice, variabili, operazioni, media (mean, trim)/moda/mediana, ?funz, n:k (sequenze), NA, na.rm, 
\subsection{Vettori e matrici}