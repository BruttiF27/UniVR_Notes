\section{Programmazione orientata agli oggetti}
Il corso si propone di fornire le conoscenze necessarie alla comprensione, sviluppo e correzione di software realizzato in un linguaggio di programmazione orientato agli oggetti e di fornire le competenze relative alla strutturazione di progetti software di medie dimensioni.\par
Lo scopo e l'utilità della OOP è quello di gestire appropriatamente un programma con molte linee di codice. Rende il codice estremamente modulare, riutilizzabile e di conseguenza risulta più facile mantenerlo. Per il modo in cui sono trattati gli oggetti, inoltre, il codice è reso particolarmente sicuro.\newline

\noindent Familiarizziamo ora con i due concetti base della OOP: \textbf{classi} e \textbf{oggetti}. Le prime definiscono una struttura dati, alla quale è possibile associare dati, detti \textbf{attributi}, e funzioni, chiamate \textbf{metodi}. Dalle classi, che in parole povere fungono da tipo di dato, è possibile ottenere gli oggetti, delle loro istanze, con stessi dati e metodi. Infatti, il processo di creazione di un oggetto è detto \textbf{istanziazione}.\par
Il workflow della OOP si basa sulle interazioni fra gli oggetti; queste avvengono grazie alle loro \textbf{interfacce}, set di messaggi che l'oggetto può ricevere, mappate ad un metodo nello stesso. Se si riceve un messaggio al di fuori dell'interfaccia, è detto illegale e viene bloccato. Un iter di lavoro generale per una corretta programmazione a oggetti è:
\begin{enumerate}
	\item Identificare i componenti.
	\item Definire l'interfaccia dei componenti.
	\item Definire le modalità con cui le interfacce consentono l'interazione fra oggetti.
	\item Minimizzare le relazioni fra i componenti.
\end{enumerate}
\noindent Generalmente, è buona pratica documentare il programma ad ogni sviluppo, ma con grandi quantità di dati, la cosa può risultare tediosa. Per questa ragione si usa uno strumento utile per la documentazione delle classi: l'\textbf{Unified Modeling Language}, il quale consente di definire graficamente e quindi documentare un sistema orientato agli oggetti. Verrà approfondito più avanti nel corso.

%

\section{Il linguaggio Java}
Il linguaggio utilizzato nel corso sarà Java, nato nel '95 e proprietà di Oracle, è diventato lo standard nel mercato del lavoro per il software development. La sua caratteristica principale è la portabilità, ovvero è possibile eseguire programmi su qualunque architettura grazie alla \textbf{Java Virtual Machine}, la quale funge da interprete al codice compilato.\par
Più precisamente, dopo essere compilato, verrà creato in output un file con estensione ".class" contenente il \textbf{bytecode}. Questo codice è ciò che viene effettivamente interpretato in runtime da una parte della JVM, il compiler \textbf{Just In Time}, utile anche per ulteriori ottimizzazioni. In soldoni, a patto che la macchina abbia installata la JVM, sarà possibile eseguire i files compilati. Altre caratteristiche di Java sono:
\begin{itemize}
	\item Linguaggio fortemente tipizzato, ovvero è possibile dichiarare variabili di un determinato tipo di dato, le quali non lo possono cambiare una volta aggiunte al codice.
	\item Non ha manipolazioni esplicite di puntatori grazie alla filosofia dell'incapsulamento, rendendo meno probabili errori riguardanti la memoria.
	\item Controlla il runtime, rendendo impossibile avere array overflow.
	\item Il \textbf{Garbage collector} controlla eventuali leaks di memoria per tapparle.
	\item È possibile usare eccezioni per controllare gli errori.
	\item Linguaggio fortemente dinamico, poiché fa loading e linking in runtime. Inoltre, usa dimensioni di array dinamiche.
\end{itemize}
\noindent Dove è possibile installare sulla macchina solo la JVM, tipico se vuoi giocare a Minecraft, per scrivere programmi in Java è necessario usare il \textbf{Java Development Kit}, compreso di debugger, compiler, disassembler, ed un applicativo per la documentazione.\par
Per l'esecuzione dei programmi abbiamo poi il \textbf{Java Runtime Environment}, avente con sé librerie di classe, il compiler JIT precedentemente menzionato, la JVM e il Java application launcher.\newline

\noindent Si suppone che tu abbia frequentato il precedente corso di programmazione, tenuto con C. Conoscerlo faciliterà enormemente le cose, poiché la sintassi per le funzionalità base del linguaggio ne è in gran parte simile. Infatti si mantengono i tipi primitivi: \textbf{int}, \textbf{double}, \textbf{char}, aggiungendo \textbf{boolean} e \textbf{String}, quest'ultimo non primitivo, ma successivamente approfondito. Si mantengono inoltre altre funzionalità come il type casting. Riassumendo, diciamo che la sintassi base di Java è composta da:
\begin{itemize}
	\item \textbf{Parole chiave del linguaggio}: Hanno un significato speciale e non possono essere usate per la dichiarazione di variabili o funzioni.
	\item \textbf{Identificatori}: I nomi scelti per gli elementi di programmazione definiti nel linguaggio.
	\item \textbf{Operatori}: Simboli per effettuare operazioni.
	\item \textbf{Dati}: Valori delle variabili, le informazioni nel codice.
\end{itemize}
\noindent La vera novità di nostro interesse sono invece le classi e gli oggetti; lo scopo di questi concetti è l'astrazione, un modo per rendere più vicino al nostro pensiero la programmazione.\par
Supponiamo che una classe rappresenti gli animali, con delle caratteristiche generali come nome e verso, detti \textbf{campi}; a questo punto è possibile istanziare un oggetto con nome "cane" che esegue il verso "woof". La programmazione orientata agli oggetti si basa su questo workflow, espandibile con ulteriori concetti che andremo a vedere più avanti nella dispensa.