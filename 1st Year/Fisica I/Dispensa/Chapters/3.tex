\begin{comment}
	Da riorganizzare il contenuto
\end{comment}


\section{Leggi di Newton}
Il concetto che governa le dinamiche di questo mondo è la \textbf{forza}, della quale esistono due tipi:
\begin{itemize}
	\item \textbf{Forze di contatto}: Esercitate mediante il contatto fisico fra due oggetti.
	\item \textbf{Forze di campo}: Agiscono mediante lo spazio vuoto.
\end{itemize}
\noindent Le forze $\overrightarrow{F}$ sono annotate come vettori, poiché vanno in una determinata direzione con un certo valore scalare. Un altro concetto fondamentale per questa sezione è la \textbf{massa}, la quale misura quanta resistenza un corpo mostra ai cambiamenti della sua velocità. In linguaggio comune è chiamata "peso".\par
Diciamo di voler spingere un corpo di massa $3kg$ con una certa forza che produce un'accelerazione di $4m/s^2$. Se applichiamo la stessa forza ad un corpo di massa differente, così lo sarà l'accelerazione. Diciamo infatti che il modulo dell'accelerazione del corpo è inversamente proporzionale alla sua massa. A partire da questi concetti, possiamo definire le \textbf{Leggi di Newton}:
\begin{teorema}
	\textbf{Prima legge di Newton}\par
	\noindent Chiamata anche legge d'inerzia, definisce dei sistemi di riferimento detti \textbf{sistemi inerziali}. Afferma che se un corpo non interagisce con altri corpi, si può trovare un sistema di riferimento nel quale la sua accelerazione è nulla. Inoltre, quando su di un corpo non agiscono forze, la sua accelerazione è nulla. Risponde alla domanda "Cosa succede a un corpo se non gli vengono applicate forze?"
\end{teorema}
\begin{teorema}
	\textbf{Seconda legge di Newton}\par
	\noindent L'accelerazione di un corpo è dovuta alla forza risultante, ovvero la somma vettoriale delle forze, esercitata su un corpo. Risponde alla domanda "Cosa succede a un corpo se gli viene applicata una o più forze?" \[\sum\overrightarrow{F} = m\overrightarrow{a}\]
\end{teorema}
\noindent Grazie alla seconda legge, è possibile definire anche la forza gravitazionale esercitata dalla Terra. Molto semplicemente, come visto nel moto di caduta, si sostituisce la costante gravitazionale $g$ al posto dell'accelerazione. Chiamiamo questo valore \textbf{forza peso}.
\begin{teorema}
	\textbf{Terza legge di Newton}\par
	\noindent Ad ogni forza esercitata ne corrisponde una uguale e contraria. Infatti, una forza $\overrightarrow{F}_{12}$ esercitata da un corpo $1$ su un corpo $2$ è uguale in intensità ed è opposta in verso alla forza $\overrightarrow{F}_{21}$, esercitata dal secondo corpo sul primo. \[\overrightarrow{F}_{12} = -\overrightarrow{F}_{21}\]
\end{teorema}
\noindent Se consideri un oggetto poggiato su un tavolo, noterai che non viene accelerato, distruggendo ogni cosa nel suo cammino verso il centro della terra. Questo è perché il tavolo esercita una forza che annulla quella del peso, chiamata \textbf{forza normale}, poiché porta il corpo ad essere in quiete. Le forze sono infine misurate con l'unità \textbf{Newton}, dove $1N = 1kg \times m/s^2$. Passiamo ora ai modelli di analisi che utilizzano quanto appena visto:\newline

\noindent \textbf{- Punto materiale in equilibrio}\par
\noindent Se l'accelerazione di un corpo schematizzato come punto materiale è nulla, stiamo parlando di un corpo in equilibrio e la forza risultante deve necessariamente essere uguale a zero.\par
Supponiamo quindi un corpo con una determinata massa $m$, il quale è appeso con una fune che si dirama poi in altre due, attaccate al soffitto. Le forze risultanti si equivalgono ed il corpo rimane sospeso in aria. Dalla terza legge di Newton sappiamo che alla forza peso esercitata dal corpo ne corrisponde una uguale e contraria, ed in questo caso prende il nome di \textbf{tensione} $T$, dunque: \[P-T=0\]
\noindent Abbiamo supposto che la corda attaccata al corpo si dirami in altre due; ciò significa che il valore della tensione è la somma delle tensioni esercitate dalle due funi. Essendo infine le forze dei valori vettoriali, si possono ottenere le loro componenti coi metodi già visti.\newline

\noindent \textbf{- Sistema di carrucole}\par
\noindent Supponiamo una fune retta da una carrucola con due corpi attaccati alle sue estremità. Avremo necessariamente due masse diverse indipendenti e due tensioni legate una all'altra. Ciò crea un sistema di forze descritto come \[\begin{cases}
	T - m_1g = m_1a\\
	m_2g - T = m_2a
\end{cases}\]
\noindent Dove i segni negativi sono presenti per tenere conto del movimento in direzione inversa che compie un corpo rispetto all'altro. Quando uno sale, l'altro scende e viceversa.\par
Tuttavia finora abbiamo compiuto delle grandi approssimazioni. Nella vita vera, quando un corpo è posto su una superficie, viene attuata la \textbf{forza di attrito}, una resistenza al moto con direzione opposta rispetto al movimento del corpo. Questa è descritta con \[F_A = \mu\cdot mg\]
\noindent Dove $m$ è la massa, $g$ la costante gravitazionale, e $\mu$ la \textbf{costante di attrito}, determinata in base alle superfici e per la quale ne esistono due tipi:
\begin{itemize}
	\item \textbf{Attrito statico} $\mu_s$: Attuata finché il corpo è fermo.
	\item \textbf{Attrito dinamico} $\mu_d$: Attuata su corpi in movimento.
\end{itemize}
\noindent Che cosa cambia nei modelli di analisi visti in precedenza? Che fondamentalmente bisogna considerare una nuova forza nella sommatoria da calcolare. Supponiamo nuovamente un corpo di massa $m$ posto su un piano inclinato di angolo $\theta$ e che il corpo non si muova. Abbiamo la forza peso, le cui componenti sono date da $P_y = mgsin\theta$ e $P_x = mgcos\theta$, dove quest'ultima spinge il corpo contro la superficie.\par
La forza di attrito, come già menzionato, è scomponibile nelle sue componenti, e quella che impedisce il movimento al corpo, sta sull'asse delle $x$. È data da: $F_a = \mu_scos\theta$. Possiamo infine ricavare la costante di attrito con alcuni semplici passaggi algebrici: \[mgsin\theta - F_A = 0 \implies mgsin\theta - \mu_smgcos\theta = 0 \implies sin\theta = \mu_scos\theta \implies \mu_s = tan\theta\]


\begin{comment}	
	--- Concetto di energia
	Esistono tanti tipi di energia, come meccanica, cinetica e tant'altre. Cambiando lo stato di quiete si modifica l'energia di un corpo. La variazione di energia, ovvero quanta ne è consumata, si calcola con il lavoro L = \Delta E = Fscos\theta.
	Il lavoro è un valore scalare
	
	Energia potenziale. Se alzo un libro si immagazzina dell'energia in esso, e quando viene rilasciato, così sarà anche per l'energia sotto forma di energia cinetica. Lo rispiegherà più in seguito.
\end{comment}


% Estensione del modello moto circolare uniforme
% Moto circolare non uniforme
% Moto in sistemi di riferimento accelerati
% Moto in presenza di forze frenanti

\section{Concetto di energia}
\section{Esercizi svolti}