\section{Calcolo infinitesimale per le curve}
Prima di iniziare l'argomento è necessario rimettere in chiaro alcuni importanti concetti visti in precedenza con algebra lineare, come il \textbf{calcolo vettoriale}. Quanto verrà espresso ora si trova nello spazio $\mathbb{R}^n$, i cui elementi si dicono \textbf{vettori} e si indicano con $v = (x_1, x_2, ..., x_n)$, mentre le $x_i$ sono le \textbf{componenti del vettore}.\par
Diremo poi \textbf{modulo} o norma di un vettore la radice del prodotto scalare fra ogni sua componente alla seconda. Si scrive:
\begin{center}
	\[|v| = \sqrt{\sum_{i=1}^{n}x_i^2} \equiv \sqrt{x_1^2 + x_2^2 + ... + x_n^2}\]
\end{center}
\noindent Diciamo poi \textbf{versore} il vettore unitario indicato con:
\begin{center}
	\[vers(v) = \dfrac{v}{|v|} = 1\]
\end{center}
\noindent Lo spazio $R^n$ è poi dotato della \textbf{base canonica}, dove ogni vettore $e_i$ presenta un solo valore $1$ in una posizione $i$, con tutte le altre uguali a $0$
\begin{center}
	$e_1 = (1, 0, ..., 0, 0)$, $e_2 = (0, 1, ..., 0, 0)$, $e_n = (0, 0, ..., 0, 1)$
\end{center}
\noindent Ed ogni vettore può essere espresso come combinazione lineare degli elementi della base canonica:
\begin{center}
	\[v = \sum_{i=1}^{n}x_ie_i\]
\end{center}
\noindent Le operazioni classiche somma algebrica e prodotto per uno scalare portano al \textbf{prodotto scalare}, già menzionato prima e definito come:
\begin{center}
	\[\left<v_1, v_2\right> = (u_1,u_2,...,u_n)(w_1,w_2,...,w_n) = \sum_{i=1}^{n}u_iw_i\]
\end{center}
\noindent In particolare, se lo spazio $\mathbb{R}^n$ ha dimensione $2$ o $3$, abbiamo che funziona il \textbf{teorema del coseno}:
\begin{center}
	$\left<v_1, v_2\right> = |v_1||v_2|\cos(\theta) \implies \cos(\theta) = \dfrac{\left<v_1, v_2\right>}{|v_1||v_2|}$
\end{center}
\noindent dove theta è un angolo compreso fra i due vettori. Diremo inoltre che due vettori si dicono \textbf{ortogonali} quando il loro prodotto scalare è uguale a zero.\par 
Infine, richiamiamo il \textbf{prodotto vettoriale}, un procedimento simile a quello per il calcolo dell'inversa di una matrice. Sfortunatamente, solo la parte più tediosa:
\begin{center}
	$u \times v = \begin{pmatrix}
		i & j & k\\
		u_1 & u_2 & u_3\\
		v_1 & v_2 & v_3
	\end{pmatrix} = i(u_2v_3 - u_3v_2) - j(u_1v_3 - u_3v_1) + k(u_1v_2 - u_2v_1)$
\end{center}
\noindent In tal merito, se il prodotto vettoriale di due vettori è $0$, questi si dicono \textbf{paralleli}. Per quanto riguarda i prerequisiti di algebra, queste nozioni sono sufficienti, quindi soffermiamoci sullo scopo che vogliamo darci.\par 
Finora abbiamo lavorato con la nozione di funzione ad una singola variabile reale; dobbiamo estendere il concetto per far sì che comprenda anche i casi in cui si ricevano o ritornino più variabili. Lavoreremo infatti con funzioni del tipo $f:\mathbb{R} \to \mathbb{R}^n$ oppure $g:\mathbb{R}^n \to \mathbb{R}$. Nello specifico:
\begin{itemize}
	\item \textbf{Funzione di più variabili}; Funzione definita sullo spazio $A \subset \mathbb{R}^n$, con $n = 1$.
	\item \textbf{Funzione a valori reali}; Funzione che ha immagine $f(A) \subseteq \mathbb{R}$.
	\item \textbf{Funzione a valori vettoriali}; Funzione con immagine $f(A) \subset \mathbb{R}^n$, dove $n > 1$. 
\end{itemize}
\noindent Generalmente ci interesseremo massimo allo spazio $\mathbb{R}^3$, perché altrimenti la cosa diventerebbe estremamente dolorosa. Ricorda: se è a una dimensione, ha una sola variabile reale, se ne ha due, stai lavorando con vettori, mentre se ne ha tre con una matrice $3\times 3$ e così via.

\subsection{Calcolo differenziale e integrale per funzioni a valori vettoriali}
Anzitutto osserveremo casi di calcolo infinitesimale per funzioni a più variabili, che risulta essere il caso più semplice su cui lavorare; più nello specifico, vedremo che in dimensione $2$ e $3$, le funzioni assumeranno il significato geometrico di curva nel piano o nello spazio rispettivamente.
\begin{definition}
	\textbf{Limite per funzioni a più variabili}\par 
	\noindent Sia $r:I \to \mathbb{R}^m$, con $I$ intervallo dello spazio $\mathbb{R}$. Siano ora $t_0 \in I \land l \in \mathbb{R}^m$, con $l$ il valore del limite. Allora:
	\begin{center}
		\[\lim_{t \to t_0}|r(t) -l| = 0 \implies \lim_{t \to t_0}r(t) = l\]
	\end{center}
\end{definition}
\noindent Questa definizione viene utilizzata per le funzioni ad una variabile, tuttavia è estendibile anche a funzioni vettoriali immaginando che $r(t)$ comprenda ogni singola componente del vettore, quindi come $r(t) = [r_1(t), r_2(t), ..., r_n(t)]$, che produrrà altrettanti valori $l_i$. Infatti, se prendiamo una funzione vettoriale dello spazio $\mathbb{R}^2$ possiamo dire che:
\begin{center}
	\[\lim_{t \to t_0}|r(t)-l| = 0 \implies \lim_{t \to t_0} \left|\begin{pmatrix}
		r_1(t) - l_1\\
		r_2(t) - l_2
	\end{pmatrix}\right|\]
\end{center}
\noindent Che ci fa ottenere, finalmente, la forma generale del limite che ci serve per gli esercizi. Nonostante questa sia quella per lo spazio di dimensione $2$, è facilmente estendibile a dimensioni superiori:
\begin{center}
	\[\lim_{t \to t_0}\sqrt{(r_1(t) - l_1)^2 + (r_2(t) - l_2)^2} = 0 \implies \lim_{t \to t_0}(r_1(t) - l_1) + (r_2(t) - l_2) = 0\]
\end{center}
\begin{eg}
	\textbf{Calcolo di limite di funzione} $r:I\to\mathbb{R}^3$\par 
	\noindent Consideriamo la funzione: $r(t) = (\cos(t)+\pi, e^t-1, \sin(t^2))$. Per calcolarne il limite per $t \to t_0$, bisogna applicarlo per ogni singola componente del vettore. Avremo quindi che:
	\begin{center}
		\[lim_{t\to t_0} r(t) = \begin{cases}
			lim_{t\to t_0}(\cos(t)+\pi) = 1+\pi\\
			lim_{t\to t_0}(e^t-1) = 0\\
			lim_{t\to t_0}(\sin(t^2)) = 0
		\end{cases}\]
	\end{center}
	\noindent Se hai dubbi sui risultati ottenuti dai limiti, ripassati la teoria di analisi 1 perché sarà fondamentale. Il vettore di risultati ottenuti sarà:
	\begin{center}
		$lim_{t\to t_0}r(t) = (1+\pi, 0, 0)$
	\end{center}
\end{eg}
\noindent Fortunatamente, abbiamo che definizioni e proprietà dei limiti di funzioni vettoriali sono analoghe alle loro controparti unidimensionali; infatti:
\begin{itemize}
	\item Vale il teorema di unicità del limite.
	\item Vale il teorema sul limite della somma o del prodotto per una costante.
	\item La definizione di funzione continua in un punto o insieme è la medesima. Diciamo infatti che $r:I \subseteq \mathbb{R} \to \mathbb{R}^n$ è continua in $t_0$ se vale che:
	\begin{center}
		\[\lim_{t\to t_0}r(t) = r(t_0)\]
	\end{center}
	\noindent Generalmente, una funzione a valori vettoriali è continua se e solo se lo sono tutte le sue componenti.
\end{itemize}
\noindent Se vale il limite, sarà possibile utilizzarlo per dare la definizione di \textbf{derivata per le funzioni a valori vettoriali}, non molto dissimile da quella vista in termini unidimensionali.
\begin{definition}
	\textbf{Derivata di funzioni a valori vettoriali}\par 
	\noindent Sia $r$ una funzione tale che $r:I\to \mathbb{R}^m$, con $t_0 \in I$. Diciamo che $r$ è derivabile se esiste finito il limite di quanto segue:
	\begin{equation}
	\begin{split}
		r'(t) & = \lim_{t\to t_0}\dfrac{r(t_0+h - r(t_0))}{h}\\
		& = \left(\lim_{h \to 0}\dfrac{r_1(t_0+h-r_1(t))}{h}, \lim_{h\to 0}\dfrac{r_2(t_0+h-r_2(t))}{h}, ..., \lim_{h\to 0}\dfrac{r_m(t_0+h-r_m(t))}{h}\right)\\
		& = r'_1(t_0), r'_2(t_0), ..., r'_m(t_0)
	\end{split}
	\end{equation}
	\noindent Noterai che ogni scrittura è conseguenza diretta della precedente. Inoltre, se $r$ è derivabile in tutto lo spazio $I$ e la sua derivata è ivi continua, diremo che la funzione è di classe $C^1(I)$ e lo annotiamo come $r \in C^1(I)$. Tale concetto è estendibile a ordini successivi se la derivata continua ad essere continua.
\end{definition}
\noindent Ne esce naturalmente che gli integrali seguano una stessa dinamica.
\begin{definition}
	\textbf{Integrale di funzioni a valori vettoriali}\par 
	\noindent Sia $r$ una funzione tale che $r:[a,b]\to \mathbb{R}^m$, come prima. L'integrazione segue la relazione:
	\begin{center}
		\[\int_{a}^{b}r(t) dt = \left(\int_{a}^{b}r_1(t) dt, \int_{a}^{b}r_2(t) dt, ..., \int_{a}^{b}r_m(t) dt\right)\]
	\end{center}
	\noindent E diciamo che $r$ è integrabile nell'intervallo $[a,b]$ se e solo se tutte le sue componenti sono integrabili. Valgono inoltre il \textbf{teorema fondamentale del calcolo integrale}, con la funzione di classe $C^1([a,b])$:
	\begin{center}
		\[\int_{a}^{b}r'(t) dt = r(b)-r(a)\]
	\end{center}
	\noindent E se $r$ è integrabile abbiamo che vale:
	\begin{center}
		\[\int_{a}^{b} r(t) dt \leq \int_{a}^{b}|r(t)| dt\]
	\end{center}
\end{definition}

\subsection{Curve regolari e lunghezza di un arco di curva}
Possiamo iniziare il discorso facendo appello alla fisica. Se immaginiamo un punto materiale che si muove su uno spazio tridimensionale, la sua dinamica sarà espressa da una funzione di forma: $r:[t_1, t_2] \to \mathbb{R}^3$ e le coordinate dal vettore $r = (x,y,z)$.\par 
Dunque, dato il punto $t \in [t_1, t_2]$ dato in pasto alla funzione $r$, avremo che la funzione prenderà la forma generale di:
\begin{center}
	$r(t) = x(t)i + y(t)j + z(t)k \equiv r(t) = (x(t) + y(t) + z(t))$
\end{center}
\noindent Ovviamente il concetto vale anche per spazi bidimensionali, come n-dimensionali. Nel primo caso avremo una funzione $r:[t_1,t_2] \to \mathbb{R}^2$ con le coordinate date da $r = (x,y)$ e la funzione generale:
\begin{center}
	$r(t) = x(t)i + y(t)j \equiv r(t) = (x(t) + y(t))$
\end{center}
\noindent Possiamo ora introdurre il concetto di \textbf{arco di curva continua} $\gamma$. Rigorosamente, è la coppia costituita da due parti:
\begin{enumerate}
	\item Una funzione continua del tipo $r:I \to \mathbb{R}^m$, detta \textbf{parametrizzazione della curva}.
	\item Un insieme di punti di $\mathbb{R}^m$ costituente l'immagine della funzione $r$, chiamato \textbf{sostegno della curva}.
\end{enumerate}
\noindent Proprio grazie a questa divisione in componenti possiamo avere funzioni diverse con uno stesso sostegno. Inoltre, se esiste la derivata seconda, possiamo definire la \textbf{velocità scalare} con la formula:
\begin{center}
	$v(t) = |r'(t)|$
\end{center}
\noindent Un arco di curva continua può assumere le seguenti caratteristiche:
\begin{itemize}
	\item L'arco è \textbf{chiuso} se nell'intervallo di definizione $I = [a,b]$ il punto iniziale della curva coincide con quello finale, quindi $r(a) = r(b)$.
	\item L'arco è \textbf{semplice} se la curva non ripassa mai dallo stesso punto, quindi $t_1 \neq t_2 \implies r(t_1) \neq r(t_2)$.
	\item L'arco è \textbf{piano} se ha un piano che contiene il suo sostegno.
\end{itemize}
\noindent Altri tipi di archi di curve sono:
\begin{itemize}
	\item \textbf{Arco di curva regolare}\par 
	\noindent Sia la funzione $r:I\to \mathbb{R}^m$ tale che $\forall t \in I.(r \in C^1(I) \land r'(t) \neq 0)$. Abbiamo quindi che il vettore derivato esiste in ogni punto, varia con la continuità e non si annulla mai. Non a caso per queste curve è definito bene il \textbf{versore tangente} $T = \frac{r'(t)}{|r'(t)|}$, il quale dipende con continuità dalla variabile $t$.
	\item \textbf{Arco di curva regolare a tratti}\par 
	\noindent Sia $I \subset \mathbb{R}$; definiamo tale un arco di curva $r:I\to \mathbb{R}^m$ tale che sia continuo in $\mathbb{R}$ e l'intervallo possa essere diviso in un numero finito di sub-intervalli, su tutti i quali la $r$ sarà un arco di curva regolare.
\end{itemize}
\noindent L'espressione delle curve può infine avvenire nei seguenti tre modi:
\begin{itemize}
	\item Forma parametrica.
	\item Forma di grafico: $y = g(x)$.
	\item Forma implicita: $f(x,y) = 0$.
\end{itemize}
\noindent Adesso buttiamo dentro un concetto che ha bisogno di qualche premessa; la \textbf{lunghezza di un arco di curva}. Consideriamo la parametrizzazione di un arco di curva continua $\gamma$ ed una partizione $P = \{a = t_0, t_1, ..., t_{n-1}, t_n = b\}$ dell'intervallo $I = [a,b]$, con $(t_0 < t_1 < ... < t_n)$.\par 
Alla partizione $P$ è associata la poligonale\footnote{Una curva non tonda. Un angolo sotto questa visione potrebbe essere una poligonale.} inscritta in $\gamma$ creata dagli n-segmenti di estremi $r(t_{j-1})$ e $r(t_j)$, con $j = 1,...,n$.\par 
Sia ora $l(P)$ la lunghezza della poligonale data dalla formula:
\begin{center}
	\[l(P) = \sum_{j=1}^{n}|r(t_j) - r(t_{j-1})|\]
\end{center}
\noindent L'idea di fondo è che questa lunghezza approssimi per difetto quella di $\gamma$, consentendoci di ottenerla variando in ogni istanza possibile $P$ e prendendo l'estremo superiore di $l(P)$. In tal merito, abbiamo le definizioni:
\begin{definition}
	\textbf{Arco di curva rettificabile}\par 
	\noindent Diciamo che un arco di curva $\gamma$ è \textbf{rettificabile} se vale:
	\begin{center}
		$supl(P) = l(\gamma) < +\infty$
	\end{center}
	\noindent L'estremo superiore $sup$ è calcolato al variare di ogni partizione possibile dell'intervallo e $l(\gamma)$ è la lunghezza dell'arco di curva.
\end{definition}
\noindent Grazie al concetto di velocità scalare introdotto prima, è possibile semplificarci ulteriormente la vita per quanto riguarda rettificazioni con il seguente teorema:
\begin{theorem}
	\textbf{Rettificazione di arco di curva regolare}\par 
	\noindent Sia $r:[a,b] \to \mathbb{R}^m$ parametrizzazione di un arco di curva $\gamma$ regolare. Allora questo è rettificabile e si scrive generalmente:
	\begin{center}
		\[l(\gamma) = \int_{a}^{b}|r'(t)| dt\]
	\end{center}
	\noindent Mentre per rispettivamente uno spazio bidimensionale e tridimensionale valgono le seguenti scritture:
	\begin{center}
		\[l(\gamma) = \int_{a}^{b} \sqrt{x'(t)^2 + y'(t)^2} dt \land l(\gamma) = \int_{a}^{b} \sqrt{x'(t)^2 + y'(t)^2 + z'(t)^2} dt\]
	\end{center}
\end{theorem}
\noindent È inoltre possibile unire le curve; infatti, dati due archi $\gamma_1, \gamma_2$, con rispettive equazioni:
\begin{itemize}
	\item $r_1 = r_1(t)$ per $t \in [a,b]$
	\item $r_2 = r_2(t)$ per $t \in [b,c]$
\end{itemize}
\noindent E con condizione di raccordo $r_1(b) = r_2(b)$, abbiamo che la relazione $\gamma = \gamma_1 \cup \gamma_2$ è data da:
\begin{center}
	$r(t) = \begin{cases}
		r_1 = r_1(t), t \in [a,b]\\
		r_2 = r_2(t), t \in [b,c]
	\end{cases}$
\end{center}
\noindent In questi casi, abbiamo che se una proprietà vale per entrambi gli archi, si estenderà anche alla loro unione.



% TODO Riprendi da pag. 39 lunghezza di un grafico

\subsection{Cambiamenti di parametrizzazione e parametro arco}
\subsection{Integrali di prima specie}

%

\section{Calcolo infinitesimale per funzioni reali di più variabili}
\subsection{Grafico, linee di livello e domini}
\subsection{Limiti e continuità}
\subsection{Analisi delle forme di indeterminazione}