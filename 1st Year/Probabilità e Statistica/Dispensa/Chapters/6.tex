\section{Introduzione}
R è linguaggio di programmazione. Viene utilizzato per l'analisi statistica dei dati e la loro visualizzazione, statistica descrittiva, machine learning, manipolazione dei dati, datamining e anche nella ricerca scientifica.\par
Si tratta di un codice open-source, la cui sintassi è estremamente flessibile e a tratti simile a C. R è poi diviso in pacchetti, ovvero insiemi di funzioni create da altri utenti. In particolare vedremo:
\begin{itemize}
	\item \textbf{ggplot2}; usato per la creazione dei grafici.
	\item \textbf{dplyr}; usato per la manipolazione dei dati.
\end{itemize}
I files con dominio ".r" sono detti \textbf{R-Scripts}; infatti ciò che andremo a scrivere e far elaborare dal software sono fondamentalmente dei testbench per ritornare determinati risultati o comportamenti. Per lavorare su di essi è necessario l'uso del terminale, e dove è possibile utilizzare l'interfaccia di R innata, è consigliato, e così faremo nel corso, l'ambiente di sviluppo \textbf{R Studio}. Spiegare un IDE esula dallo scopo della dispensa, arrangiati e leggi le relative documentazioni.\par
Installato R Studio avremo davanti il terminale, l'ambiente dei dati e una terza interfaccia variabile per i grafici, manuale e tant'altro. Abbiamo nominato dei pacchetti con i quali andremo a lavorare; questi devono essere installati con il seguente comando:
\begin{verbatim}
	> install.package("nomePacchetto")
\end{verbatim}
Abbiamo tutto e adesso possiamo iniziare a lavorare col linguaggio.

%

\section{Funzioni di base}
Come ben saprai, ogni linguaggio di programmazione ha le sue caratteristiche base, ma una cosa che li accomuna è la necessità di renderlo chiaro, leggibile e funzionale. In tal merito, possiamo scrivere dei commenti con "\#", tenendo sempre a mente che è sempre meglio scrivere codice autoesplicativo. Inoltre, se mai dovessi avere problemi a ricordare la logica delle funzioni puoi utilizzare la sezione help di R Studio.

%
\subsection{Variabili e funzioni di uso generale}
La dichiarazione di una variabile è molto semplice; bisogna solamente tenere a mente tre restrizioni del linguaggio:
\begin{itemize}
	\item R è case-sensitive. Quindi "Palle" $\neq$ "palle".
	\item Le variabili dichiarate non possono iniziare con un numero.
	\item Le variabili dichiarate non possono essere dichiarate con un nome uguale ad una parola chiave.
\end{itemize}
Chiarito ciò, è possibile dichiarare variabili con le stesse dinamiche di C, infatti:
\begin{verbatim}
	> A <- 1+1		# Dichiaro A col valore 2
	> A				# Stampo la variabile A	a video
	> A <- NULL		# Rendo la variabile vuota
\end{verbatim}
Non c'è bisogno di utilizzare un tipo, in quanto l'allocazione della memoria è gestita dal linguaggio stesso. La variabile viene salvata nell'\textbf{ambiente} di lavoro, ed è utilizzabile fin quando non la si rimuove; cosa che può essere fatta manualmente o col comando "remove(nomeVariabile)".\par 
Ovviamente, se le variabili contengono valori è possibile effettuare operazioni aritmetiche. Inoltre, esistono due semplici funzioni per familiarizzare con la loro logica:
\begin{itemize}
	\item Per il calcolo della media: mean(x, ...)
	\item Per il calcolo della mediana: median(x, na.rm = FALSE, ...)
\end{itemize}
Nella funzione della mediana notiamo la parola "na.rm"; si tratta di un valore logico con il quale decidere se contare o meno i valori "NA", ovvero \textbf{Not Available}. Inoltre, finora abbiamo stampato le singole variabili, ma se volessimo aggiungere delle semplici stringhe? Guarda, un gattino!
\begin{verbatim}
	> A <- 27
	> cat("Guarda, il gattino numero", A)
\end{verbatim}
Il comando dato stamperà a video la stringa e il valore della variabile A. La sintassi è semplice da ricordare pensando a Java, oppure al significato del comando: \textbf{concatenazione}. Segue ora esempio di codice con quanto esposto finora:
\begin{verbatim}
	# Assegno i valori 24, 25, 28 alle rispettive variabili A, B, C
	> A <- 24
	> B <- 25
	> C <- 28
	
	# Calcolo la media e la mediana e le salvo in due variabili
	> mn <- mean(A, B, C)
	> md <- median(c(24, 25, 28))
	
	# Stampo a video i risultati
	> cat("Media:", mn, "Mediana:", md)
\end{verbatim}
La riga della mediana fa da apripista per il prossimo argomento.

%
\subsection{Vettori e matrici}
I vettori e matrici sono tali e quali a quelli che hai potuto apprezzare in programmazione C. Ciò significa che condividono le loro caratteristiche, anzi in alcune funzioni risulta più accessibile R, ma andiamo con ordine.\newline

\noindent Abbiamo più modi per dichiarare un vettore, ognuno con la propria ragion d'essere:
\begin{verbatim}
	# Vettore di elementi 5, 10, 15, 20, 25
	> v1 <- c(5, 10, 15, 20, 25)
	# Vettore di elementi in sequenza da 1 a 5	
	> v2 <- 1:5
	# Vettore di elementi da 5 a 25 a passi di 5
	> v3 <- seq(from = 5, to = 25, by = 5)
	# Vettore degli elementi di v3 ripetuti tre volte
	> v4 <- rep(v3, times = 3)
	# Accesso al valore del vettore v1 in posizione 1
	> v1[1]
\end{verbatim}
Come puoi vedere, è possibile utilizzare il vettore come una semplice variabile senza che il compilatore si lagni. Ciò significa che anche per le funzioni di base precedenti, come media e mediana, è possibile usare i vettori. Seguono ulteriori funzioni utili:
\begin{itemize}
	\item length(): Calcola la lunghezza del vettore.
	\item max(), min(): Calcolano valore massimo e minimo del vettore.
	\item sum(): Calcola la somma di tutti gli elementi del vettore.
	\item cumsum(): Calcola la somma cumulativa di ogni elemento.
\end{itemize}
Passiamo ora invece alle matrici; dove anche qui abbiamo vari modi per la dichiarazione e l'accesso:
\begin{verbatim}
	# Dichiarazione di una matrice, lega i due vettori come colonne.
	> mat1 <- cbind(c(1, 2, 3), c(1, 2, 3))
	# Dichiarazione di una matrice, lega i due vettori come righe.
	> mat2 <- rbind(c(1, 2, 3), c(1, 2, 3))
	# Accesso a elementi specifici
	> mat1[2:3,1:2]
	# Rimuove la prima riga, nessun comando per le colonne.
	> mat1[-1,]
	# Riempie una matrice di i righe e j colonne del valore n
	> mat3 <- matrix(n, i, j)
	# Crea una matrice identità di dimensione k
	> matIdt <- diag(k)
\end{verbatim}
Le operazioni utilizzabili dalle matrici seguono quelle viste in algebra lineare, vale a dire somma, sottrazione fra vettori o matrici e divisione, moltiplicazione per scalari, vettori o matrici. Il risultato deve essere salvato in una variabile diversa e l'operazione si scrive semplicemente come fossero due numeri. Seguono altre funzioni utili:
\begin{itemize}
	\item det(): calcola il determinante di una matrice.
	\item qr(): Calcola la decomposizione QR.
	\item solve(): Fa l'inversa di una matrice.
	\item nrow(), ncol(): Ritornano il numero di righe o colonne di una matrice.
	\item rowSums(), colSums(): Fanno la somma dei valori di ogni riga o colonna. 
	\item rowMeans(), colMeans(): Calcola la media dei valori di ogni riga o colonna. 
	\item dim(): Calcola la dimensione di una matrice.
\end{itemize}