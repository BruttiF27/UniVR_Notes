\section{Operazioni Matriciali}
\section{Tipologie di Matrici}
Quadrata, elementare, invertibili, inverse, formule per l'inversa di una matrice
\section{Matrici invertibili}


\section{Domande di teoria}
\begin{enumerate}
    \item \textbf{Che cosa e quali sono le operazioni matriciali?}
    \item \textbf{Quali sono i vari tipi di matrice?}
    \item \textbf{Che cosa rende una matrice invertibile e come si ottiene?}
    \item \textbf{Che cosa si intende per matrice trasposta? Quali sono le proprietà di questa forma?}
\end{enumerate}

%

\section{Argomenti}
3.1 Definizione e propriet`a: somma di due matrici
3.1 Definizione e propriet`a: prodotto di una matrice per uno scalare
3.1 Definizione e propriet`a: trasposta di una matrice
3.1 Definizione e propriet`a: prodotto di due matrici
3.1 Osservazione: sistema lineare in forma matriciale `e un prodotto di matrici
3.2 Definizioni: matrice quadrata, matrice diagonale, matrice triangolare inferiore/superiore
3.2 Matrici elementari
3.2 Moltiplicazione con matrici elementari

3.3 MATRICI INVERTIBILI
3.3 Definizione: matrice invertibile
3.3 Inverse di matrici elementari
3.3 Proposizione: un sistema lineare Ax = b `e equivalente al sistema lineare U x = c dove (U |c) `e
una forma ridotta di (A|b).
3.3 Proposizione: un sistema lineare ammette una soluzione se e solo se il rango `e massimo
3.3 Formula per l’inversa
3.3 Proposizione: invertible se e solo se esiste una sequenza di matrici elementari
3.3 Il calcolo della matrice inversa
3.3 Teorema della matrice invertibile