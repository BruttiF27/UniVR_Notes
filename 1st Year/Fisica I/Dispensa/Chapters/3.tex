\section{Leggi di Newton}
Il concetto che governa le dinamiche di questo mondo è la \textbf{forza}, della quale esistono due tipi:
\begin{itemize}
	\item \textbf{Forze di contatto}: Esercitate mediante il contatto fisico fra due oggetti.
	\item \textbf{Forze di campo}: Agiscono mediante lo spazio vuoto.
\end{itemize}
\noindent Le forze $\overrightarrow{F}$ sono annotate come vettori, poiché vanno in una determinata direzione con un certo valore scalare. Un altro concetto fondamentale per questa sezione è la \textbf{massa}, la quale misura quanta resistenza un corpo mostra ai cambiamenti della sua velocità. In linguaggio comune è chiamata "peso".\par
Diciamo di voler spingere un corpo di massa $3kg$ con una certa forza che produce un'accelerazione di $4m/s^2$. Se applichiamo la stessa forza ad un corpo di massa differente, così lo sarà l'accelerazione. Diciamo infatti che il modulo dell'accelerazione del corpo è inversamente proporzionale alla sua massa. A partire da questi concetti, possiamo definire le \textbf{Leggi di Newton}:
\begin{theorem}
	\textbf{Prima legge di Newton}\par
	\noindent Chiamata anche legge d'inerzia, definisce dei sistemi di riferimento detti \textbf{sistemi inerziali}. Afferma che se un corpo non interagisce con altri corpi, si può trovare un sistema di riferimento nel quale la sua accelerazione è nulla. Inoltre, quando su di un corpo non agiscono forze, la sua accelerazione è nulla. Risponde alla domanda "Cosa succede a un corpo se non gli vengono applicate forze?"
\end{theorem}
\begin{theorem}
	\textbf{Seconda legge di Newton}\par
	\noindent L'accelerazione di un corpo è dovuta alla forza risultante, ovvero la somma vettoriale delle forze, esercitata su un corpo. Risponde alla domanda "Cosa succede a un corpo se gli viene applicata una o più forze?" \[\sum\overrightarrow{F} = m\overrightarrow{a}\]
\end{theorem}
\noindent Grazie alla seconda legge, è possibile definire anche la forza gravitazionale esercitata dalla Terra. Molto semplicemente, come visto nel moto di caduta, si sostituisce la costante gravitazionale $g$ al posto dell'accelerazione. Chiamiamo questo valore \textbf{forza peso}.
\begin{theorem}
	\textbf{Terza legge di Newton}\par
	\noindent Ad ogni forza esercitata ne corrisponde una uguale e contraria. Infatti, una forza $\overrightarrow{F}_{12}$ esercitata da un corpo $1$ su un corpo $2$ è uguale in intensità ed è opposta in verso alla forza $\overrightarrow{F}_{21}$, esercitata dal secondo corpo sul primo. \[\overrightarrow{F}_{12} = -\overrightarrow{F}_{21}\]
\end{theorem}
\noindent Se consideri un oggetto poggiato su un tavolo, noterai che non viene accelerato, distruggendo ogni cosa nel suo cammino verso il centro della terra. Questo è perché il tavolo esercita una forza che annulla quella del peso, e la chiamiamo \textbf{forza normale} poiché porta il corpo ad essere in quiete. Le forze sono infine misurate con l'unità \textbf{Newton}, dove $1N = 1kg \times m/s^2$. Passiamo ora ai modelli di analisi che utilizzano quanto appena visto:\newline

\noindent \textbf{- Punto materiale in equilibrio}\par
\noindent Se l'accelerazione di un corpo schematizzato come punto materiale è nulla, stiamo parlando di un corpo in equilibrio. La forza risultante deve necessariamente essere uguale a zero.\newline

\noindent \textbf{- Punto materiale soggetto ad una forza risultante}\par
\noindent Se un corpo ha un'accelerazione, abbiamo la certezza che su di esso agisce una forza, ed infatti sarà possibile analizzarne la dinamica grazie alla seconda legge di Newton. In caso di una fune, diremo che la forza risultante sarà uguale al modulo della \textbf{tensione} $T$, ottenendo le seguenti formule: \[\sum F_x = T = ma_x; a_x = \frac{T}{m}\]

% TODO - Riprendi da forze di attrito



% Estensione del modello moto circolare uniforme
% Moto circolare non uniforme
% Moto in sistemi di riferimento accelerati
% Moto in presenza di forze frenanti

\section{Concetto di energia}
\section{Applicazioni}
\section{Esercizi svolti}