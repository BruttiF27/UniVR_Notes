\section{Sistemi Lineari Tempo invariante}
Possiamo ora dare una definizione rigorosa di sistema: si tratta di un modello matematico o formulazione di un processo o fenomeno fisico che permette di trasformare un'entrata in un'uscita determinata. Ne esistono principalmente di due tipi:
\begin{itemize}
	\item \textbf{Single Input Single Output}
	\item \textbf{Multiple Input Multiple Output}, non visti in questo programma.
\end{itemize}
\noindent In base a questi poi ci sono i sistemi dinamici, ovvero con memoria, i quali salvano i dati grazie agli stati. Rammento infine che, come i segnali, i sistemi possono essere a tempo continuo e discreto.\par
Per quanto riguarda i sistemi LTI, approfondiamo le proprietà menzionate nel capitolo precedente:
\begin{itemize}
	\item \textbf{Linearità}: Partendo da un sistema con due entrate, otterrò in output un risultato equivalente alla combinazione lineare dei singoli input.
	\item \textbf{Tempo-invarianza}: Traslando l'input del sistema prima o dopo nel tempo avrò come output lo stesso risultato traslato prima o dopo nel tempo: \[u(t)\to v(t) \implies u(t+\tau)\to v(t+\tau), \forall \tau \in \mathbb{R}\]
	\item \textbf{Causalità}: L'effetto non precede mai la causa. L'uscita $v(t)$ all'istante $\tau$ deve dipendere esclusivamente dall'ingresso $u(t)$ per $t \leq \tau$. Per comodità considereremo solo sistemi a riposo, ovvero con $\tau = 0$.
\end{itemize}
\noindent Richiamiamo all'attenzione il concetto di stabilità per formalizzarlo definitivamente. Definiamo:\newline

\noindent \textbf{Sistema asintoticamente stabile}\par
\noindent Un sistema si dice tale se dopo aver dato l'input, l'output va a morire fino ad arrivare a $0$. Formalmente:
\begin{center}
	$\exists\tau\in\mathbb{R}$ tale che $u(t) = 0$, $\forall t \geq \tau \implies lim_{t\to \infty}v(t) = 0$
\end{center}

\noindent\textbf{Sistema BIBO-stabile}\par
\noindent Supponiamo di avere un segnale che oscilla da un certo istante $\tau$ sempre all'interno di un certo intervallo di ampiezza. Se in output vale la stessa dinamica, il sistema si dice BIBO-stabile. Formalmente:
\begin{enumerate}
	\item Se $(\exists\tau \land M_u>0)\in\mathbb{R}$ tali che $|u(t)| < M_u$
	\item Allora: $\forall t\geq\tau \implies \exists M_v > 0$
	\item Infine, quest'ultimo è tale che: $|v(t)| < M_v \forall t \geq \tau$.
\end{enumerate}
\noindent Prendere nota del fatto che ogni sistema asintoticamente stabile è necessariamente anche BIBO-stabile, in quanto suo caso particolare, ma non vale il contrario.\par
Ora che conosciamo la teoria, bisogna capire come si modellano effettivamente i sistemi. Come già detto, vengono descritti mediante equazioni differenziali per input e output. Un buon iter di lavoro generale è prima scrivere tutte le equazioni che descrivono le leggi in atto, per poi estendere la scrittura con tutte le componenti richieste. Va ordinata con output a sinistra e input a destra. L'equazione differenziale generale per la rappresentazione dei modelli è: \[a_n\frac{\partial^nv(t)}{\partial t^n}+a_{n-1}\frac{\partial^{n-1}v(t)}{\partial t}+ ... + \frac{\partial^1v(t)}{\partial t} + a_0v(t) = b_m\frac{\partial^mu(t)}{\partial t}+b_{m-1}\frac{\partial^{m-1}u(t)}{\partial t} + ... + b_1\frac{\partial u(t)}{\partial t} + b_0u(t)\]
\noindent Qui, $b_n, a_n \neq 0$ sono due normali coefficienti. La buona notizia è che esiste una forma più compatta, grazie alla sommatoria: \[\sum_{i=0}^na_i \frac{\partial^i v(t)}{\partial t} = \sum_{j=0}^mb_j\frac{\partial^ju(t)}{\partial t}\]
\noindent Qui invece $n,m$ danno l'ordine delle equazioni differenziali, con generalmente $n\geq m$. Più precisamente, se $n>m$, si dice strettamente proprio e se $n\geq m$ si dice proprio. Nello schema con la black box, nel sistema si scrive proprio questa seconda equazione.