\section{Definizione}
Partiamo dal problema che fa nascere il senso di questo argomento: data un'applicazione lineare ed uno spazio vettoriale sul quale opera noi vogliamo trovare una base $B$ di $V$ tale che la matrice $A$ che rappresenta l'applicazione rispetto alla base sia la più semplice possibile.\par\quad
Questa forma ambita è la matrice quadrata diagonale di ordine $n = dimV$, i cui elementi al di fuori della diagonale principale sono nulli. Per i nostri scopi inseriremo il valore $\lambda$ a tutti gli elementi della diagonale principale. Da questa ricerca della base si arriva all'introduzione dei concetti di questa sezione: \textbf{Autovettori} e \textbf{Autovalori}.\newline

\textbf{- Che cosa sono autovettori e autovalori?}\par
Partiamo dalle definizioni; esiste un significato distinto di entrambi i concetti per applicazioni lineari e matrici ed è bene tenerli entrambi a mente.
\begin{definition}
    \textbf{Autovettori e autovalori di un'applicazione lineare}\par
    Sia $f:V\to V$ un'applicazione lineare. Un vettore tale che $v \in V$ si dice \textbf{autovettore} della funzione se:
    \begin{itemize}
        \item $v$ non è il vettore nullo.
        \item Esiste uno scalare $\lambda \in \mathbb{K}$ tale che $f(v) = \lambda v$.
    \end{itemize}
    Inoltre, lo scalare $\lambda$ è univocamente determinato dal vettore $v$ e si dice \textbf{autovalore} della funzione relativo all'autovettore $v$.
\end{definition}
Ottenendo gli autovalori possiamo dire di poter creare uno spazio che li racchiude tutti; lo chiamiamo \textbf{autospazio}. Per ogni $\lambda \in \mathbb{K}$ poniamo la seguente formula:
\begin{center}
    $V_\lambda = Ker(f-\lambda I) = \{v \in V:f(v) = \lambda v\}$
\end{center}
questo sottospazio vettoriale di $V$ si dice autospazio di $f$ relativo all'autovalore $\lambda$ ed i suoi elementi non nulli sono tutti gli autovettori dell'applicazione relativi all'autovalore a lei dato.\par\quad
Per le matrici il concetto si rivela più semplice e intuitivo, anche se alla fine gli elementi più utilizzati rimarranno gli autovalori.
\begin{definition}
    \textbf{Autovettori e autovalori di una matrice}\par
    Sia $A$ una matrice quadrata di ordine $n$. Un vettore $x \in \mathbb{C}^n$ non nullo si dice \textbf{autovettore} della matrice se esiste uno scalare $\lambda \in \mathbb{C}$ tale che valga:
    \begin{center}
        $Ax = \lambda x$
    \end{center}
    Come per le applicazioni, lo scalare $\lambda$ è univocamente determinato dal vettore $x$ e viene chiamato \textbf{autovalore} di $A$ relativo all'autovettore $x$.
\end{definition}
\textbf{- Come si ottengono autovettori e autovalori?}\par
Palle\newline

\textbf{- A cosa servono autovettori e autovalori?}\par



§9. Autovalori e autovettori (vedi [GS, Capitolo V])
9.1 Definizione: autovalore e autovettore
9.2 Osservazione: autovettori sono soluzioni di un sistema lineare
\section{Polinomio caratteristico}
9.3 Definizione: polinomio caratteristico
9.4 Teorema: autovalori sono radici e autovettori sono elementi di spazi nulli (autospazi)
9.5 Corollario: matrici su C possiedono autovalori
9.6 Definizioni: autospazio, moltiplicit`a algebrica e geometrica
9.7 Osservazione: se esiste una base B formata di autovettori di A, allora la matrice associata a A
rispetto a B nel dominio e codominio `e diagonale.
9.8 Proposizione: autovettori linearmente indipendenti
9.9 Definizioni: matrici simili, matrice diagonalizzabile


\section{Esercizi}
\section{Appunti}
Gli autovettori e gli autovalori si ottengono mediante la ricerca del polinomio caratteristico. Suppongo che il primo \textbf{SUPPOSIZIONE, VERIFICARE} sia lo stesso polinomio trovato, mentre è certo che gli autovalori siano le radici, quindi le soluzioni, dello stesso.\par\quad
La ricerca del polinomio ha luogo con la formula $det(A-\lambda I_n)$, dove:
\begin{itemize}
    \item $A$ è la matrice presa sotto esame.
    \item $\lambda$ è lo scalare che verrà moltiplicato alla matrice identità.
    \item $I_n$ è la matrice identità di ordine $n$.
\end{itemize}
 Per ordine si intende quanto è lunga. Essendo che normalmente si lavora con matrici quadrate $3x3$, l'ordine sarà, appunto, $3$.\par\quad
 Per il calcolo del determinante è consigliato semplificare per la riga o colonna che presenta più zeri, in tal modo da ridurre al minimo i calcoli e quindi i possibili sbagli. Fondamentalmente l'operazione consiste nella sottrazione fra la matrice sotto esame $A$ e una matrice diagonale i cui elementi della diagonale principale sono tutti $-\lambda$.\par\quad
 In termini ancora più semplici parliamo della matrice $A$ con aggiunto il termine $-\lambda$ alla sua diagonale principale.\par\quad
 Fatto ciò è possibile calcolare il determinante. Se ti trovi incasinato con le lambda è normale. Spesso è necessario scomporre il polinomio in forma più leggibile. Prega che non sia un grado sì alto da usare Ruffini. Bleah.\newline

 Trovati gli autovalori sarà sempre richiesto di calcolare molteplicità algebrica e geometrica. La prima è semplicissima, il suo valore corrisponde a quante volte è presente l'autovalore nel polinomio. La molteplicità geometrica invece richiede più passaggi, data la formula $m_g = n -rk(A - \lambda I_n)$:
 \begin{itemize}
     \item $n$ è l'ordine della matrice. Se hai una quadrata sarà uguale per righe e colonne. Per esempio se la matrice è $3x3 \to n = 3$. Semplice.
     \item Calcola il rango della matrice risultante dall'operazione $A - \lambda I_n$ tramite Gauss. Non fare cazzate, di solito è semplice.
 \end{itemize}
La molteplicità geometrica è necessariamente un valore che segue questa restrizione: $m_a \leq 1 \leq m_g$, quindi è necessariamente maggiore o uguale della sua controparte algebrica. Ciò dà un utile sicurezza per il controllo calcoli.\newline

Da questi dati è possibile determinare se una matrice è o meno \textbf{diagonalizzabile}. Qua termina la mia conoscenza.
