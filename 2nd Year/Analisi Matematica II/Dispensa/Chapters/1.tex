\section{Modelli differenziali}
Passiamo da uno studio numerico ad uno particolarmente più astratto. Analisi matematica 2 è una materia molto importante non solo per consolidare le nozioni del predecessore che compongono il toolset necessario per lavorare qui, ma anche perché renderà il resto delle materie di stampo matematico più comprensibili e approcciabili.\par
Iniziamo riprendendo le funzioni; ne hai viste di ogni tipo, da sole, composte, inverse etc... e adesso lavorerai con famiglie di funzioni.
\begin{definition}
	\textbf{Equazione differenziale ordinaria}\newline
	Definiamo equazione differenziale di ordine $n$ un'equazione del tipo:
	\begin{center}
		$F(t, y, y', y'', ..., y^n)$
	\end{center}
	Dove $y(t)$ è la funzione incognita ed $F$ è una funzione assegnata delle $n+2$ variabili $(t, y, y', y'', ..., y^n)$ a valori reali. Diremo inoltre il suo \textbf{ordine} l'ordine massimo di derivata che compare.
\end{definition}
Come potrai immaginare, l'esistenza di un'equazione implica l'esistenza di una soluzione. Non sarà bello, ma per ottenerla sarà necessario l'aiuto degli integrali.
\begin{definition}
	\textbf{Curva integrale dell'equazione differenziale}\newline
	Diciamo curva integrale o soluzione dell'equazione nell'intervallo $I \subset \mathbb{R}$ una funzione $\phi(t)$, definita almeno in $I$ e a valori reali, per cui risulti:
	\begin{center}
		$F(t, \phi(t), \phi'(t), \phi''(t), ..., \phi^n(t)) = 0$, $\forall t \in I$
	\end{center}
	In merito, ci servirà ottenere l'\textbf{integrale generale}, ovvero una formula che rappresenti la famiglia di tutte le soluzioni dell'equazione.
\end{definition}
Concetti di base ottenuti; benvenuto in analisi matematica 2.

%

\section{Equazioni differenziali del primo ordine}
Le equazioni differenziali del primo ordine si presentano con un'incognita, una funzione e solo una derivata. Quindi, nella forma:
\begin{center}
	$F(t, y, y') = 0$
\end{center}
Dove $F$ è la funzione assegnata di $t, y, y'$ a valori reali.\newline

Un esempio di tale forma è la ricerca delle primitive di una funzione $f$ continua su $I$. Prende infatti la forma:
\begin{center}
	$y'(t) = f(t)$, con soluzioni $y(t) = \int f(t)dt+c$, dove $c \in \mathbb{R}$
\end{center}
Essendo che l'equazione ha infinite soluzioni distinte dalla costante arbitraria $c$, ne traiamo che generalmente l'insieme delle soluzioni di un'equazione differenziale del primo ordine è costituito da più funzioni, dipendenti dal parametro $c: t \to \phi(t;c)$. Questa scrittura è l'\textbf{integrale generale} menzionato prima.\par
Abbiamo inoltre la condizione supplementare $y(t_0) = y_0$, la quale ci permette di selezionare una soluzione precisa. Ciò introduce un concetto specifico:

\begin{definition}
	\textbf{Problema di Cauchy}\newline
	Chiamiamo problema di Cauchy il processo di risoluzione di un'equazione differenziale che detiene una condizione supplementare. Assume la forma:
	\begin{center}
		$\begin{cases}
			F(t, y, y') = 0\\
			y(t_0) = y_0
		\end{cases}$
	\end{center}
\end{definition}

Può capitare che l'equazione prenda una forma dove la derivata è uguale al resto delle funzioni e variabili; chiamiamo questa scrittura \textbf{forma normale}.
\begin{center}
	$y'(t) = f(t, y(t))$
\end{center}

per concludere, vediamo adesso un esempio di risoluzione di un problema di Cauchy

% Inserisci esempio

\subsection{Equazioni a variabili separabili}
Le equazioni a variabili separabili sono un caso particolare di equazioni differenziali ordinarie del primo ordine, caratterizzate dalla presenza di una funzione $f$ prodotto di due funzioni, una della sola variabile $t$ e l'altra solo dell'incognita $y$. Si presentano nella forma:
\begin{center}
	$y' = a(t)b(y)$
\end{center}
con $a$ funzione continua su un intervallo $I \subset \mathbb{R}$ e $b$ funzione continua su un intervallo $J \subset \mathbb{R}$. Supponendo invece che $b(y) \neq 0$ otteniamo la scrittura:
\begin{center}
	$a(t) = \dfrac{y'}{b(y)}$
\end{center}
Prendendo ora un'ipotetica soluzione $y(t)$, l'equazione soddisfa la seguente identità, la quale prendendo gli integrali definiti di ambo i membri fa ottenere:
\begin{center}
	\[\dfrac{y'(t)}{b(y(t))} = a(t) \implies \int\dfrac{y'(t)}{b(y(t))}dt = \int a(t)dt+c\]
\end{center}
Nell'integrale di sinistra è consentito effettuare un cambio di variabile $y = y(t); dy = y'(t)dt$, ottenendo:
\begin{center}
	\[\int \dfrac{dy}{b(y)} = \int a(t)dt+c\]
\end{center}
Così abbiamo ottenuto l'integrale generale dell'equazione differenziale.\par
Inoltre, se la funzione $B(y)$ è una primitiva di $\dfrac{1}{b(y)}$ e $A(t)$ una primitiva di $a(t)$, allora l'integrale generale è assegnato dalla seguente equazione in \textbf{forma implicita}:
\begin{center}
	$B(y) = A(t)+c$, con $c$ costante arbitraria.
\end{center}
Tuttavia non sarà sempre possibile ricavare $y$ esplicitamente o ridurre l'equazione in forma normale. È esattamente per questo che esiste una forma generale del problema di Cauchy ad-hoc per questi casi.
\begin{theorem}
	\textbf{Problema di Cauchy per ED a variabili separabili}\newline
	\begin{center}
		$\begin{cases}
			y' = a(t)b(y)\\
			y(t_0) = y_0
		\end{cases}$
	\end{center}
	Dove $a$ è continua in un intorno $I$ di $t_0$ e $b$ è continua in un intorno $J$ di $y_0$. Esisteranno quindi:
	\begin{itemize}
		\item Intorno $I' \subset I$ di $t_0$.
		\item Funzione continua $y$ definita su $I'$.
		\item Funzione derivata $y'$ continua su $I'$, soluzione del problema.
	\end{itemize}
	Inoltre, se anche $b'$ è una funzione continua su $J$ oppure $b$ ha un rapporto incrementale\footnote{Inserire spiegazione} limitato in J (anche se non è derivabile), allora la soluzione è \textbf{unica}.
\end{theorem}

% TODO ----- Inserire nota su cos'è un rapporto incrementale

\begin{eg}
	\textbf{Risolvere il problema di Cauchy}
	\begin{center}
		$\begin{cases}
			y' = ty^3\\
			y(0) = 1
		\end{cases}$
	\end{center}
	\begin{enumerate}
		\item \textbf{Calcolare l'integrale generale dell'equazione differenziale}\newline
		Notiamo innanzitutto che $y = 0$ è integrale singolare\footnote{Soluzione che soddisfa anche l'equazione $F_{y'}(t, y, y') = 0$.} per l'equazione, ma a noi serve quello generale, quindi poniamo $y \neq 0$. Separando le variabili e svolgendo gli integrali abbiamo:
		\begin{center}
			\[\int\dfrac{dy}{y^3} = \int tdt+c \implies -\dfrac{1}{2y^2} = \dfrac{t^2}{2}+c\]
		\end{center}
		Svolgiamo i passaggi intermedi riguardanti i calcoli:
		\begin{itemize}
			\item $-\dfrac{1}{2y^2} = \dfrac{t^2}{2}+c \implies \left(-\dfrac{1}{2y^2}\right)^{-1} = \left(\dfrac{t^2}{2}+c\right)^{-1}$
			\item $-2y^2 = \dfrac{2}{t^2+2c} \implies -y^2 = \dfrac{1}{t^2+2c}$
			\item $y^2 = -\dfrac{1}{t^2-2c} \implies y^2 = \dfrac{1}{-t^2-2c}$
		\end{itemize}
		Che porta infine ad avere:
		\begin{center}
			$y = \pm\dfrac{1}{\sqrt{k-t^2}}$, $k = -2c$
		\end{center}
		\item \textbf{Imponi la condizione di Cauchy}\newline
		Sostituiamo alla $y$ lo zero dato dalla restrizione e l'uno al valore arbitrario $c$. Otterrai che:
		\begin{center}
			$y(0) = \dfrac{1}{\sqrt{1-t^2}}$
		\end{center}
		e hai ufficialmente finito.
	\end{enumerate}
\end{eg}

\subsection{Equazioni lineari del primo ordine}
Tipo di equazioni differenziali ordinarie, dove $F$ è lineare in $y$ e $y'$, le cui soluzioni sono espresse mediante uno spazio vettoriale di dimensione 1. Si presentano nella forma:
\begin{center}
	$a_1(t)y'(t) + a_2(t)y(t) = g(t)$
\end{center}
Con $a_1$, $a_2$ funzioni continue su un intervallo. Anch'esse detengono una forma normale, ottenibile se il coefficiente $a_1(t)$ non si annulla.
\begin{center}
	$y'(t) +a(t)y(t) = f(t)$
\end{center}
Anche qui $a$ ed $f$ sono continue sull'intervallo $I \subset R$.\newline

Diciamo inoltre:
\begin{itemize}
	\item \textbf{Equazione completa} se $f$ non è identicamente nulla; la soluzione si ottiene aggiungendo al suo integrale generale una sua soluzione particolare.
	\item \textbf{Equazione omogenea} se $f \equiv 0$; la soluzione è generalmente indicata con le z, quindi: $z'(t) + a(t)z(t) = 0$.
\end{itemize}
Capiamo quindi che il procedimento da attuare si compone di due passi: prima la ricerca di un integrale generale dell'equazione omogenea, poi trovare la soluzione particolare da aggiungere a quella completa.\newline

La \textbf{ricerca della soluzione dell'equazione omogenea} inizia moltiplicando ambo i membri dell'equazione per $e^{A(t)}$, dove $A(t)$ è una primitiva di $a(t)$, quindi vale $A'(t) = a(t)$.
\begin{center}
	$z'(t)+a(t)z(t) = 0 \implies z'(t)e^{A(t)}+a(t)z(t)e^{A(t)} = 0 \equiv [z(t)e^{A(t)}]'$
\end{center}
Ne ricaviamo che: $z(t) = ce^{A(t)} \implies z(t) = ce^{\int a(t)dt}$ e avremo che la forma della soluzione, dove $z_0$ è una soluzione particolare, sarà:
\begin{center}
	$z = cz_0$
\end{center}

% TODO ----- Continua da qua -----
%

\section{Equazioni differenziali lineari del secondo ordine}
Comprende "Generalità", "Struttura dell'integrale generale"
\subsection{Equazioni lineari omogenee a coeffcienti costanti}
\subsection{Equazioni lineari non omogenee a coefficienti costanti}
Comprende "Metodo di somiglianza"