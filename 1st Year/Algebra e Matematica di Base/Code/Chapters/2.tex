\section{Tipi di funzioni}
Le \textbf{funzioni}, o applicazioni, sono le relazioni più importanti fra gli insiemi. Si definiscono formalmente come:
\begin{definition}
    \textbf{Funzione} - $f : X \implies Y$\par
    Siano due insiemi $X,Y$. Un'applicazione $f(X)$ in $Y$ è una corrispondenza $f \subseteq X \times Y$ con la seguente proprietà:\par
    Per ogni elemento $x\in X$, esiste un unico elemento $y\in Y$ tale che $(x,y) \in f$, ovvero che valga quanto segue:
    \begin{center}
        $\forall x,x' \in X .\{(x = x') \implies (f(x) = f(x'))\}$
    \end{center}
    Per indicare l'elemento corrispondente a $x$ scriviamo $f(x) = y$.
\end{definition}
Questa definizione porta tutti nuovi concetti come conseguenze logiche. Essendo che stiamo lavorando su insiemi, diciamo di avere una funzione $f: \mathbb{R} \implies \mathbb{R}$, possiamo associare a questa funzione un numero all'interno dell'insieme per ottenere la sua corrispondenza nello stesso.\par\quad Fin qua tutto chiaro, ma la presenza di n numeri corrisposti implica l'esistenza di un insieme che li contenga tutti. Questo si chiama \textbf{Insieme Immagine}. Andando più nello specifico, possiamo dire che il singolo elemento $f(x)$ è chiamato \textbf{immagine} di $x$ sotto $f$. Inoltre chiameremo l'insieme di partenza $X$ il \textbf{Dominio} e quello della corrispondenza $Y$ il \textbf{Codominio}.\par\quad
Per definizione di funzione non è possibile che ad un elemento $f(x)$ corrispondano più elementi nell'insieme $Y$, tuttavia è possibile che più elementi di $X$ abbiano una stessa immagine. Seguono alcuni casi notevoli:
\begin{itemize}
    \item Funzione costante
    \item Funzione identità
    \item Funzione valore assoluto
    \item Funzione quadratica
    \item Funzione di Dirichlet
\end{itemize}

CONTINUA DA immagine inversa










\begin{itemize}
    \item Funzioni totali
    \item Funzioni parziali
    \item Iniettive
    \item Suriettive
    \item Biunivoche
    \item Funzioni composte
    \item Funzione inversa
    \item Cancellabilità della funzione
\end{itemize}
\section{Relazioni di equivalenza}
Equivalenza, transitività, simmetria/antisimmetria, monotonia, proiezione simmetrica, assiomi di peano.
\section{Partizioni}
\section{Relazioni di ordinamento}
\section{Domande di teoria}
\section{Esercizi}