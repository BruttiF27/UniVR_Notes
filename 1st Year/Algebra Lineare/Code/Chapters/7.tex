\section{Definizione}
§9. Autovalori e autovettori (vedi [GS, Capitolo V])
9.1 Definizione: autovalore e autovettore
9.2 Osservazione: autovettori sono soluzioni di un sistema lineare
\section{Polinomio caratteristico}
9.3 Definizione: polinomio caratteristico
9.4 Teorema: autovalori sono radici e autovettori sono elementi di spazi nulli (autospazi)
9.5 Corollario: matrici su C possiedono autovalori
9.6 Definizioni: autospazio, moltiplicit`a algebrica e geometrica
9.7 Osservazione: se esiste una base B formata di autovettori di A, allora la matrice associata a A
rispetto a B nel dominio e codominio `e diagonale.
9.8 Proposizione: autovettori linearmente indipendenti
9.9 Definizioni: matrici simili, matrice diagonalizzabile


\section{Esercizi}


