\section{Rotazione di un corpo rigido attorno a un asse fisso}
Nello scorso capitolo è stato introdotto il concetto di centro di massa, il quale ci consente di iniziare a vedere i corpi non più come particelle, bensì aventi una forma. Tuttavia, ora che i corpi possiedono uno spessore, avranno anche una dinamica di \textbf{rotazione} che si attiva quando sono in moto.\par
Supponiamo dunque un corpo che ruota, il quale avrà necessariamente un punto alla sua estremità che misura una distanza $R$ dal centro di massa. Ruotando, effettua uno spostamento $\Delta s$ in un intervallo di tempo $\Delta t$ ad una velocità $\frac{\Delta s}{\Delta v}$. Lo spostamento crea un arco; grazie alla geometria euclidea possiamo dire che $\Delta s = R\theta$, dove $\theta$ è l'ampiezza dell'angolo. Ciò fa ricavare quella che è conosciuta come \textbf{velocità angolare}, ovvero la variazione dell'angolo rispetto al tempo: \[v = \frac{\Delta s}{\Delta t} = \frac{R\Delta\theta}{\Delta t} = R\omega\] 
\noindent La velocità angolare è costante, perché non dipende dalla posizione del punto nel raggio. Facendone la derivata otteniamo poi l'\textbf{accelerazione angolare} $\alpha$: \[\alpha = \frac{dv}{dt} = \frac{d(R\omega)}{dt} = R\frac{d\omega}{dt} = R\frac{d^2\theta}{dt^2} = R\alpha\]
\noindent In breve, al posto dello spazio, velocità e accelerazione abbiamo le loro controparti angolari. Sostituendole alle equazioni del moto otteniamo il comportamento relativo agli angoli. Prendiamo per esempio un corpo composto da tante piccole massette $m_i$, ognuna con una posizione $s_i$; possiamo passare dal moto traslazionale all'angolare dicendo $s=R\theta$ e sarà possibile riutilizzare tutte le leggi del moto viste in precedenza.\par
È corretto pensare che, in quanto movimento, un corpo possa resistere alla rotazione. Questa capacità è misurata con il \textbf{momento di inerzia} $I$, in due modi diversi in base al numero di masse del corpo. Se questo è finito, usiamo: \[I = \sum_i m_ir_i^2\]
\noindent Mentre se parliamo di un numero continuo, avremo: \[I = \int r_i^2\cdot dm\]
\noindent dove $r_i$ è la posizione di ogni massa e $dm$ una massa infinitesima. La misura del momento di inerzia dipende anche dalla forma dell'oggetto, vedasi l'esempio di lanciare un coltello, che rotea più facilmente in base al modo in cui è lanciato. Infatti l'energia cinetica dipende anche dal momento di inerzia, ed è descritta come: \[K = \frac{1}{2}I\omega^2\]
\noindent Supponiamo ora di applicare una forza ad un corpo rigido attaccato a un asse fisso da una delle sue estremità; questo allora ruoterà attorno all'asse. In termini simili, chiamiamo la lunghezza del corpo a partire dall'asse \textbf{braccio} $r$, e la capacità di una forza di porre in rotazione un corpo è detta \textbf{momento della forza} $\overrightarrow{\tau}$, determinato da: \[\overrightarrow{\tau} = rFsin\theta = Fd\]
\noindent Dove $r$ è la distanza fra l'asse di rotazione ed il punto in cui è applicata la forza, e $d$ la distanza fra l'asse di rotazione e il vettore della forza. Il $\sin\theta$ è poi l'angolazione che determina l'efficacia della forza. Difatti, esercitando una forza parallela al braccio, non ruota, mentre se è perpendicolare si avrà massimo effetto.\par
A partire dalla definizione di momento della forza, possiamo sostituire la definizione di forza vista nella seconda legge di Newton con le sue componenti angolari: \[\tau_i = r_iF_i = r_im_ia_i = r_im_ir_i\alpha = r_i^2m_i\alpha\]
\noindent In necessità di calcolare il momento totale, si farà la sommatoria dei $\tau$, che risulta essere: \[\tau_{tot} = I\alpha\]
\noindent In parole povere, stiamo sostituendo le variabili da termini di traslazione in termini rotazionali:
\begin{itemize}
	\item $K = \frac{1}{2}mv^2 \implies K = \frac{1}{2}I\omega^2$
	\item $F = ma \implies \tau I\alpha$
\end{itemize}
\noindent Quindi il momento di inerzia prende il posto della massa e il momento di una forza sostituisce la forza. Abbiamo tuttavia un terzo elemento; il \textbf{momento angolare}, usato al posto della quantità di moto: \[\overrightarrow{L} = \overrightarrow{r} \times \overrightarrow{p}\]
\noindent Il momento angolare si conserva, quindi è costante in assenza di forze.

%

\section{Equilibrio statico ed elasticità}

%

\section{Legge di gravitazione universale}

%

\section{Esercizi svolti}