Lo scopo delle statistica inferenziale è, data una popolazione, di capire la sua vera distribuzione a partire dai dati che si possono osservare e studiare in un campione da essa estratto.\par
Sarà quindi necessario formulare un \textbf{modello statistico} di cui è nota la distribuzione ed i dati osservati, ma con i valori dei parametri incogniti. Ciò ridurrà il problema intero a questi due passaggi:
\begin{enumerate}
	\item Scegliere un valore plausibile per i valori dei parametri reali, quindi fare delle inferenze.
	\item Testare ipotesi sui valori per verificarne l'attendibilità.
\end{enumerate}
\noindent Le uniche inferenze statistiche di nostro interesse saranno quelle sulle \textbf{popolazioni normali} e di \textbf{Bernoulli}.

\section{Stima dei parametri}
Supponiamo che i dati siano realizzazioni di una variabile aleatoria $X$ con una  densità discreta o continua $f(x,\theta)$, dove $\theta$ è un parametro incognito. Dobbiamo stimarlo coi dati ottenuti dalle osservazioni su $X$; per farlo si usano due tipi di stima:
\begin{itemize}
	\item \textbf{Stima puntuale}; Dove si ottiene un singolo valore come stima per il valore di $\theta$.
	\item \textbf{Stima intervallare}; Si ottiene un intervallo di valori possibili per $\theta$, associando ad ogni intervallo un livello di fiducia che $\theta$ vi appartenga.
\end{itemize}
\noindent Partiamo dal primo tipo e dalle sue basi; per prima cosa sarà richiesto introdurre qualche concetto fondamentale. Anzitutto, chiamiamo \textbf{campione} di ampiezza $n$ una collezione di variabili aleatorie $X_n$ indipendenti e tutte con la stessa distribuzione $f(x;\theta)$, con $\theta$ parametro incognito.\par 
Una \textbf{Statistica} $T$ è invece una variabile aleatoria ottenuta come funzione del campione, ovvero $T = T(X_1, ..., X_n)$, ed uno \textbf{stimatore} è qualunque $T$ indipendente da $\theta$, è usata per stimarlo. Tendenzialmente, il ruolo è coperto dalla media campionaria per la varianza campionaria.\par 
Infine, chiamiamo \textbf{stima} il valore numerico assunto dallo stimatore sui dati osservati $x_1, ..., x_n$, ovvero, $\theta = T(x_1, ..., x_n)$.\newline

Sia ora $X_1, ..., X_n$ un campione casuale preso da una popolazione di densità $f(x,\theta)$, che dipende dal parametro $\theta$. Se interpretiamo \[f(x_1, ..., x_n; \theta)\]
\noindent come la verosimiglianza che si realizzi la n-upla $x_1, ..., x_n$ di dati quando $\theta$ è il vero valore del parametro, possiamo prendere come sua stima il \textit{valore che rende massima la funzione}, chiamato \textbf{stimatore di massima verosimiglianza}. Consigliato inoltre vederlo come logaritmo, in quanto ha uno stesso massimo e facilita i calcoli.
\begin{eg}
	% TODO
	INSERISCI ESEMPIO PER STIMATORE MAX VEROSIMIGLIANZA PER BERNOULLI E POPOLAZIONE NORMALE.
\end{eg}
\noindent Ma in che modo è possibile scegliere uno stimatore $T = T(X_1, ..., X_n)$? O meglio, come ne si valuta la \textbf{bontà}?\par 
Bisogna cercare di minimizzare la deviazione dal valore reale del parametro attraverso valore atteso e varianza. Non potendo tuttavia essere onniscenti, è inevitabile incappare in errori e per questo si introduce il concetto di \textbf{distorsione} o bias.
\begin{definition}
	\textbf{Bias}\par 
	\noindent Sia $T=T(X_1, ..., X_n)$ uno stimatore di $\theta$. Allora $b(T) = E(T)-\theta$ è detto bias di $T$ come stimatore di $\theta$. Se è nullo, T è detto stimatore corretto di $\theta$.
\end{definition}
\noindent Uno stimatore buono e utile controlla sia varianza che bias in modo contenuto, con lo scopo di fornire un risultato quanto più vicino alla realtà possibile senza essere troppo permissivo.
\begin{eg}
	% TODO
	INSERISCI ESEMPIO PER STIMATORE BUONO
\end{eg}
\noindent Inoltre, sia lo stesso stimatore $T=T(X_1, ..., X_n)$ del parametro $\theta$. Chiamiamo \textbf{errore quadratico medio} il valore atteso del quadrato della differenza fra lo stimatore ed il $\theta$. Si indica con:
\[MSE(T) = E[(T-\theta^2)^2] = Var(T) + b(T)^2\]
\noindent Se $T$ è corretto, allora questo errore quadratico medio sarà uguale alla varianza dello stimatore.\newline

\noindent Passiamo ora alla \textbf{stima intervallare}. Sia un campione estratto da una popolazione. Ci si aspetta che la stima ottenuta valutando lo stimatore sui dati osservati non sia l'effettivo valore di $\theta$, quindi è preferibile produrre un intervallo per il quale abbiamo una certa fiducia che il parametro vi appartenga. In tal merito, diamo le seguenti definizioni:
\begin{definition}
	\textbf{Stimatore intervallare}\par
	\noindent Sia $X_1, ..., X_n$ un campione casuale di una popolazione dove ci interessa stimare un parametro $\theta$. Siano poi $L_1 = L_1(X_1, ..., X_n)$, $L_2 = L_2(X_1, ..., X_n)$ due statistiche non dipendenti da $\theta$, tali che:
	\begin{center}
		$P(L_1 < \theta < L_2) = 1-\alpha$, con $\alpha \in (0,1)$
	\end{center}
	\noindent L'intervallo $(L_1, L_2)$ si dice \textbf{stimatore intervallare} del parametro $\theta$ e per costruirlo è necessario conoscere la distribuzione delle sue statistiche $L_1, L_2$.
\end{definition}
\begin{definition}
	\textbf{Intervallo di confidenza}\par
	\noindent Siano adesso $\hat{l}_1 = L_1(x_1,...,x_n)$ e $\hat{l}_2 = L_2(x_1, ..., x_n)$ i valori assunti dalle statistiche $L_1, L_2$ sui dati osservati $x_1, ..., x_n$.\par 
	Diremo quindi che $(\hat{l}_1, \hat{l}_2)$ è l'\textbf{intervallo di confidenza} di livello $1-\alpha$ per il parametro $\theta$.
\end{definition}
\noindent Notare che $(L_1, L_2)$ è un intervallo aleatorio che contiene il valore di $\theta$, mentre $(\hat{l}_1, \hat{l}_2)$ è una realizzazione del primo; data la sua natura non si presta ad alcuna valutazione probabilistica. Quindi, in sintesi:
\begin{enumerate}
	\item Otteniamo un \textit{campione casuale} $X_1, ..., X_n$ da una popolazione, il quale ci può far ottenere lo \textit{stimatore intervallare} $(L_1, L_2)$, composto da due variabili aleatorie, fra le quali è probabile accada il parametro $\theta$ di nostro interesse, quindi $P(L_1 < \theta < L_2) = 1-\alpha$, che risulta dare il \textit{coefficiente di fiducia}.
	\item Dal campione casuale effettuiamo delle inferenze, ottenendo il \textit{campione osservato} $x_1, ..., x_n$ che può avere solo valori numerici. Da questo possiamo ottenere l'\textit{intervallo di confidenza} $(\hat{l}_1, \hat{l}_2)$.
\end{enumerate}
\noindent Ultima cosa prima di passare ai vari casi di studio; per costruire lo stimatore intervallare è necessario conoscere la \textbf{distribuzione} delle statistiche $L_1, L_2$; quindi vediamo in che modo possono essere distribuite.\par 
Sia un campione estratto da una popolazione normale $X_1, ..., X_n$ e diciamo che ha una media $\mu \in \mathbb{R}$ e varianza $\sigma^2 > 0$. Siamo interessati a studiarne la distribuzione delle statistiche campionarie, ovvero la \textbf{media campionaria} $\overline{X}$ e la \textbf{varianza campionaria} $S^2$, per ottenere rispettivamente $\mu$ ed $\sigma^2$. Abbiamo che:
\begin{itemize}
	\item \textbf{Densità $\chi^2$ a $n-1$ gradi di libertà}\par 
	\noindent Le variabili aleatorie $\overline{X}$, $S^2$ sono indipendenti e per il teorema di limite centrale abbiamo che la media campionaria si distribuisce con una normale di media $\mu$ incognita e varianza $\frac{\sigma^2}{n}$, quindi: $\overline{X} \sim N(\mu, \frac{\sigma^2}{n})$. Inoltre:
	\[\frac{(n-1)S^2}{\sigma^2} \sim \chi^2_{n-1}\]
	\noindent La cui densità è asimmetrica e non nulla solo sui numeri reali positivi.
	\item \textbf{Densità $t$ di student a $n-1$ gradi di libertà}\par
	\noindent Si tratta di una densità con la forma a campana simmetrica rispetto ad $x=0$ e si stima con:
	\[\frac{\overline{X}-\mu}{\sqrt{\frac{S^2}{n}}}\sim t_{n-1}\]
\end{itemize}

%

\section{Intervalli di confidenza}
Passiamo adesso ai vari casi di studio che possiamo trovare nello svolgimento degli esercizi. Premetto che si somigliano tutti abbastanza e sarà utile capire come agire per poter capire anche la sezione seguente e prendere una decisione ponderata riguardo alle richieste.
\begin{itemize}
	\item \textbf{Intervalli di confidenza per la media di una popolazione normale, varianza nota}\par 
	\noindent Sia il campione $X_1, ..., X_n$, la media incognita $\mu \in \mathbb{R}$ e la varianza nota $\sigma^2 > 0$. Con $\alpha \in (0,1)$ dobbiamo ricavare intervalli di confidenza ad un livello $1-\alpha$ per la media $\mu$.\newline
	
	\noindent Diciamo che $\frac{\overline{X}-\mu}{\sqrt{\frac{\sigma^2}{n}}} = Z \sim N(0,1)$ e indichiamo con $z_{\alpha}$ il valore per cui $P(Z < z_{\alpha}) = \alpha$. Usando un pò di algebra noterai facilmente che:
	\[P(Z < z_{\alpha}) = \alpha \implies 1-P(Z>z_{\alpha}) = \alpha \implies 1-\alpha = P(Z < z_{\alpha})\]
	\noindent E che quindi ci servirà capire quell'area della funzione dove $Z < z_{\alpha}$. È probabile che venga richiesto l'intervallo totale (bilaterale) oppure una sola parte (unilaterale), quindi bisogna prendere la metà richiesta o tutto l'intervallo. Più in particolare, le due metà si ottengono con:
	\[1-\alpha = P(-z_{\alpha/2} < Z < z_{\alpha/2}) \implies 1-\alpha = \left(\overline{X}-z_{\alpha/2} \times \frac{\sigma}{\sqrt{n}} < \mu < \overline{X} + z_{\alpha/2}\times \frac{\sigma}{\sqrt{n}}\right)\]
	\noindent Dove il primo elemento della disequazione è $L_1$, il secondo è il parametro della media da stimare e l'ultimo è $L_2$. Sapendo ora di aver osservato dei dati $x_1, ..., x_n$ tali che $\overline{X}(x_1, ..., x_n) = \overline{x}$, abbiamo che ad un livello di confidenza $1-\alpha$, per la media $\mu$, gli intervalli ottenibili sono:
	\begin{itemize}
		\item \textbf{Bilaterale}: $(\overline{x} - z_{\alpha/2}\times \frac{\sigma}{\sqrt{n}}, \overline{x} + z_{\alpha/2}\times \frac{\sigma}{\sqrt{n}})$
		\item \textbf{Unilaterali}: $(\overline{x} - z_{\alpha}\times \frac{\sigma}{\sqrt{n}}, +\infty)$, $(-\infty, \overline{x}+z_{\alpha}\times \frac{\sigma}{\sqrt{n}})$
	\end{itemize}
	\noindent Per gli esercizi effettuare il seguente procedimento:
	\begin{enumerate}
		\item Calcola la media dei dati raccolti $\overline{x}$ e il livello di confidenza $\alpha$.
		\item Calcola il valore di $z_{\alpha}$ oppure $z_{\alpha/2}$.
		\item Prendi il risultato di $1-\alpha$ oppure $1-\frac{\alpha}{2}$ e trova il corrispondente valore nella tavola degli $\phi(x)$
		\item Hai tutto. Scrivi l'intervallo.
	\end{enumerate}
	\item \textbf{Intervalli di confidenza per la media di una popolazione normale, varianza incognita}\par 
	\noindent Sia il campione $X_1, ..., X_n$, la media $\mu \in \mathbb{R}$ e la varianza nota $\sigma^2 > 0$, ambo ignote. Con $\alpha \in (0,1)$ dobbiamo ricavare intervalli di confidenza ad un livello $1-\alpha$ per la media $\mu$.\newline
	
	\noindent Teniamo anzitutto a mente due cose:
	\begin{itemize}
		\item $\frac{\overline{X}-\mu}{{\frac{S}{\sqrt{n}}}} \sim t_{n-1}$, con $S$ deviazione standard campionaria.
		\item La densità $t$ di student ha una forma a campana simmetrica rispetto a $x=0$.
	\end{itemize}
	\noindent Se $X \sim t_n$, allora $t_{\alpha,n} \in \mathbb{R}$ è il valore per cui $P(X > t_{\alpha,n}) = \alpha$. Supponiamo nuovamente di avere dei dati $x_1, ..., x_n$ tali che $\overline{X}(x_1, ..., x_n) = \overline{x}$ ed $S(x_1, ..., x_n) = \hat{s}$.\par
	Allora a livello di confidenza $1-\alpha$ avremo i seguenti intervalli:
	\begin{itemize}
		\item \textbf{Bilaterale}: $(\overline{x}-t_{\alpha/2, n-1}\times \frac{\hat{s}}{\sqrt{n}}, \overline{x}+t_{\alpha/2, n-1}\times \frac{\hat{s}}{\sqrt{n}})$
		\item \textbf{Unilaterali}: $(\overline{x}-t_{\alpha, n-1}\times \frac{\hat{s}}{\sqrt{n}}, +\infty)$, $(-\infty, \overline{x}+t_{\alpha/2, n-1}\times \frac{\hat{s}}{\sqrt{n}})$
	\end{itemize}
	\noindent Per gli esercizi effettuare il seguente procedimento:
	\begin{enumerate}
		\item Calcola media $\overline{x}$ e deviazione standard $\hat{s}$.
		\item Calcola $\alpha$ e $t_{\alpha, n-1}$, con $n$ numero totale di elementi nel campione.
		\item Ottieni il valore di $t_{\alpha, n-1}$ dalla tavola dei valori di $t_n$.
		\item Scrivi l'intervallo.
	\end{enumerate}
	\item \textbf{Intervalli di confidenza per la varianza di una popolazione normale}
	\noindent
	
	% TODO ----- CONTINUA DA QUA
	
	\item \textbf{Intervalli di confidenza per la media di una popolazione di Bernoulli}
\end{itemize}

%

\section{Verifica di ipotesi}

%

\section{Testing su due popolazioni}