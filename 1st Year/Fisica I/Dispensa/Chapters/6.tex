\section{Pressione e profondità}
Fino ad ora abbiamo studiato i meccanismi della materia fisica. Con l'introduzione dei \textbf{fluidi} è necessario cambiare leggermente forma mentis, pur sempre ricordando che le Leggi di Newton regnano sovrane, e quindi introdurre nuovi concetti. Con fluido intendiamo un insieme di molecole disposte in modo casuale, le quali si possono anche allontanare fra di loro; sulle quali è ancora possibile applicare delle forze.\par
Propriamente, non è presente forza di attrito fra i piani delle molecole, infatti una forza è applicabile esclusivamente in senso non parallelo alla superficie del fluido. Negli esercizi verrà poi utilizzato il \textbf{fluido ideale}, ovvero un liquido privo di qualunque forza di attrito, formalmente definito come privo di \textbf{viscosità}.\newline

\noindent Notiamo come la superficie del liquido e le forze, grazie alla massa, su di essa esercitate sono strettamente legate. Ciò significa che all'aumentare della superficie del liquido, si dà la possibilità a più massa di esercitare la forza peso. Questa forza viene definita \textbf{pressione} $P$; è un valore scalare misurato in \textbf{Pascal} $[1Pa = 1N/m^2]$ ed è definito come il rapporto fra la forza esercitata $F$ e l'area $A$ del liquido selezionata: \[P = \frac{F}{A} \implies F = PA\]
\noindent La pressione è un valore che dipende da quanta massa esercita un peso ad una determinata altezza; ciò significa che prendendo una sezione specifica all'interno del liquido, bisognerà considerare la somma delle forze esercitate fino a quel punto\footnote{Insieme, naturalmente, anche alla risultante}. Questa posizione è detta \textbf{profondità}.\par











\begin{comment}
	Concetto di viscosità, che è l'attrito fra le varie molecole. Qui si hanno delle forze dette di taglio, parallele alla superficie del liquido. La viscosità varia in base alla composizione del liquido.
	
	In un fluido ideale non ci sono forze di attrito.
	Pressione: P = F/A [N/m^2 = 1Pascal]
	F è forza, A è area o superficie.
	
	La pressione varia in base alla profondità dentro al liquido. Più profonda, tanto più il valore sarà alto.
	Se prendiamo una fetta del liquido bisognerà considerare la forza peso esercitata, quelle di pressione e poi la risultante che mantiene la sezione in equilibrio. La sommatoria deve naturalmente dare 0.
	
	Insomma: P_0A + mg = PA \implies P = P_0 + \frac{m}{a}g
	
	La massa riscrivibile in termini di densità \rho = m/v, con v volume. Quindi m = \rho v
	
	P = P_0 + \frac{m}{a}g = P_0 + \frac{\rho v}{a}g = P_0 + \rho gh
	Questa è detta anche Legge di Stevino, la pressione dipende solo dalla profondità.
	
	--- Vasi comunicanti
	La legge di stevino dice che la pressione dipende solo dall'altezza; riempiendo i vasi di acqua comunicanti, dunque, si raggiunge la stessa altezza.
	È probabile che venga inserita una valvola nel tubo che li fa comunicare; quando questa viene rimossa si tornerà sempre ad un'altezza uguale fra i due vasi. Se un contenitore è più piccolo dell'altro e l'altezza da raggiungere non è sostenuta, ovviamente, trasborda.
	
	Supponiamo ora di avere due vasi comunicanti, dove uno di questi ha un pistone sulla sua superficie A_1 e l'altro, più stretto, ha una piattaforma su A_2.
	Il pistone, insieme alla forza peso, esercita una forza F_1, spingendo il liquido, rendendo la pressione P = F_1/A_1.
	Dove di base la pressione alla stessa altezza è uguale, bisogna considerare che il secondo vaso ha superficie minore dell'altro e subisce una forza F_2 rivolta verso l'alto. La sua pressione qui è infatti P = F_2/A_2
	
	Per la legge di stevino possiamo dire che F_1/A_1 = F_2/A_2.
	Semplice ricavare le singole forze così, no? posso anche decidere come modularizzarle. Ringrazia il principio delle leve idrauliche.
	Per esempio, puoi avere un vaso più stretto ed esercitare poca forza, per averne una nettamente maggiore sull'altro vaso più grande.
	Insomma, spingi nel piccolo, alzi il grande.
	
	--- Barometro di Torricelli
	Prendiamo un liquido e mettiamolo in un recipiente. Inseriamo un secondo recipiente privo di aria nel primo. In assenza di aria, non c'è neanche pressione esercitata, quindi scendendo, il liquido inizia a salire, fino a quando non rispecchia perfettamente la forza risultante della pressione atmosferica.
\end{comment}

%

\section{Spinta di Archimede}



\begin{comment}
	--- Principio di Archimede
	Supponiamo un recipiente con un liquido; prendiamone una sua sezione. Abbiamo detto che ci sono le forze di pressione che si ribilanciano. P = P_0 + \rho gh
	
	Abbiamo sempre le due forze date da:
	- F_{prf} = PA
	- F_{atm} = P_0A
	
	La forza risultante sarà naturalmente F_{prf} - F_{atm} = \rho ghA = m_{fluido}g
	Le forze si sottraggono perché la pressione è diversa quando si è in più profondità.
	
	Il principio di archimede dice che un oggetto subisce una forza pari al peso del fluido che viene spostato dall'oggetto. // rivedere definizione
	// Alleggerendoti subisci una spinta verso l'alto fino al bilanciamento.
	
	Come capire se un oggetto galleggia o affonda?
	Guarda densità \rho_f del fluido e dell'oggetto \rho_o. 
	
	Spinta di archimede: B = \rho gV
	F_{prf} = \rho_o gV
	
	Il prof dice anche
	- F_{prf} = B
	- \rho_{ogg}gV_{ogg} = \rho_{fld}gV_{fld} \implies \frac{\rho_{ogg}}{\rho_{fld}} = \frac{V_{fld}}{V_{ogg}}
\end{comment}

%

\section{Dinamica dei fluidi}



\begin{comment}
	--- Fluidodinamica
	Guardiamo che succede quando il fluido si sta spostando ad una data velocità. Prenderemo un fluido non viscoso che scorre dentro un tubo. La sua densità è costante, rendendolo omogeneo. Il flusso deve essere poi stazionario, ovvero che in ogni punto la velocità è fissa.
	Inoltre, il flusso deve essere irrotazionale, quindi non ruota. Il suo momento angolare è 0.
	Queste erano le caratteristiche di un fluido ideale.
	
	Prendiamo un tubo con due sezioni diverse.
	La prima, più piccola, A1, mentre la seconda, posta più in alto, A2, più grande. Abbiamo un fluido che passa nel tubo.
	In un certo intervallo di tempo si sposta una quantità di fluido, quindi va da A1 a A2
	
	A_1 \frac{\delta x_1}{\Delta t} = A_2 \frac{\Delta x_2}{\Delta t} // Dato che il fluido è omogeneo
	Ricordi cos'è x/t? Bravo, velocità. Diciamo quindi A_1v_1 = A_2v_2
	Questa si chiama equiazione di continuità dei fluidi
\end{comment}

%

\section{Equazione di Bernoulli e comportamento nei tubi}




\begin{comment}
	Adesso prendiamo un tubo e ragioniamo in termini di energia.
	La prima estremità ha diametro minore della seconda. Come si trasforma l'energia?
	Abbiamo naturalmente due pressioni P1, P2. Queste sono ottenute grazie a P=F/A. Ciò consente eventualmente di ricavare forza e superficie.
	
	Proviamo a calcolarne il lavoro sostituendo la definizione di forza nell'integrale apposito: \[L = \int F\cdot ds = F\cdot s \implies PA\cdot\Delta x\]
	Applicandolo al tubo preso in esame, otteniamo due espressioni differenti:
	- L = F_1s_1 = P_1A_1\Delta x_1
	- L = -P_2A_2\Delta x_2		// Senso inverso perché la pressione che agisce sul tubo all'estremità 2 è contraria all'altra.
	
	Se voglio la variazione dell'energia totale, come al solito data da $E_{tot} = K + U = \frac{1}{2}mv^2 + mgh$, andrò ad osservare il comportamento alle estremità del tubo considerato.
	E per trovare la variazione di energia \Delta E è necessario ricavare l'energia alle due estremità, date da:
	- E_1 = \frac{1}{2}mv_1^2 + mgh		// Fluido che entra
	- E_2 = \frac{1}{2}mv_2^2 + mgh		// Fluido che esce
	
	Molto semplicemente ora sottraiamo l'energia finale a quella iniziale: \[\Delta E = E_2-E_1 = \frac{1}{2}mv_2^2 + mgh - \frac{1}{2}mv_1^2 + mgh = (P_1 - P_2)v\]
	
	Da questa formula possiamo eguagliare le due parti tramite dei passaggi algebrici e dire: E_1 = E_2.
	\begin{equation}
		\begin{split}
			P_1V + \frac{1}{2}mv_1^2 + mgh_1 &= P_2V + \frac{1}{2}mv_2^2 + mgh_2\\
			P_1 + \frac{1}{2}(\frac{m}{V})v_1^2 + \frac{m}{V}gh_1 &= P_2 + \frac{1}{2}(\frac{m}{V})v_2^2 + \frac{m}{V}gh_2\\
			P_1 + \frac{1}{2}\rho v_1^2 + \rho gh_1 &= P_2 + \frac{1}{2}\rho v_2^2 + \rho gh_2
		\end{split}
	\end{equation}
	
	Siamo dunque arrivati al teorema di Bernoulli. L'energia si conserva quindi è uguale in entrambe le estremità. È fondamentalmente il bilanciamento fra fluido, pressione e altezza.
	
	Supponiamo un recipiente di altezza h, fino ad essa riempito d'acqua. Ha un foro alla sua base; velocità di uscita del liquido? Possiamo usare il teorema di Bernoulli perché sappiamo che la differenza di energia è costante per entrambi i punti, ovvero la superficie e il foro.
	\begin{equation}
		\begin{split}
			P_1 + \frac{1}{2}\rho v_1^2 + \rho gh_1 &= P_0 + \frac{1}{2}\rho v_0^2 + \rho gh_0\\
			gh_1 &= v_0^2\\
			v_0 = \sqrt{2gh_1}
		\end{split}
	\end{equation}
	
	La pressione è uguale da ambo le parti perché parliamo di un recipiente circondato dall'atmosfera. Quindi possiamo dire P_1=P_0, rendendoli intercambiabili e addirittura semplificabili.
	Si può inoltre approssimare la velocità della superficie del recipiente a zero, perché è praticamente nulla.
	gh_0 = 0, perché alla base del recipiente, preso come riferimento.
	\rho rimossi grazie all'algebra
\end{comment}

%

\section{Esercizi svolti}





\begin{comment}
	```
	# Esempio di esercizio
	Supponiamo di avere un tubo di 2.5cm di acqua che serve per riempire un secchio di 30l. Serve 1min per riempire il secchio.
	Spostiamo il tubo ad altezza 1m da terra e aggiungiamo alla sua estremità dove esce l'acqua un ugello di 0.5cm^2. Calcolare la distanza coperta dall'acqua che esce dal tubo.
	
	Per risolvere l'esercizio consigliabile ragionare come un proiettile con la sua gittata.
	In questi termini possiamo quindi usare le leggi del moto per i movimenti in due dimensioni:
	- y = y_0 + v_0t - 1/2gt^2
	- x = x_0 + v_0t
	
	Abbiamo già la posizione di y che è 1m da terra; ora bisogna trovare il tempo t, per il quale serve v_{0y}.
	Capiamo che v_{0y} = 0, perché l'acqua è sparata in orizzontale. Dunque: \[y = y_0 + v_0t - 1/2gt^2 \implies 1/2gt^2 = 1 \implies t^2 = 2m/g \implies t = \sqrt{2m/g} = 0.45s\]
	
	Per quanto riguarda l'asse delle x, abbiamo x_0 = 0 dato il punto di riferimento; ma ci manca velocità e tempo.
	Per la velocità bisogna calcolare quella della situazione iniziale dell'ugello da 2.5cm. Dopodiché, siccome A_1v_1 = A_2v_2 sarà semplice sostituire. Procediamo:
	Sappiamo che 1L = 1000cm^3, quindi fa passare in 1min 30000cm^3. La portata è dunque \frac{30L * 1000cm^3}{60s} = 500cm^3/s. Possiamo dirlo grazie alla legge di continuità dei fluidi: \[A_1v_1 = A_2v_2 \implies A_1\frac{s_1}{t_1} = A_2\frac{s_2}{t_2}\]
	
	Quindi dal primo tubo con ugello di 2.5cm escono 500cm^3/sec di acqua.
	Adesso bisogna calcolare v_2, che è v_{0x} nella legge del moto scritta prima.
	Ma tu guarda, basta ricavarlo con passaggi algebrici, lmao: \[A_1v_1 = A_2v_2 \implies v_2 = \frac{A_1v_1}{A_2} = \frac{500cm^3/sec}{0.5cm^2} = 1000cm/sec = 10m/sec\]
	Adesso abbiamo tutto per calcolare la posizione: \[x = x_0 + v_0t \implies x = 0 + (10m/s)(0.45s) \implies x = 4.5m\], che è la soluzione.
	```
\end{comment}