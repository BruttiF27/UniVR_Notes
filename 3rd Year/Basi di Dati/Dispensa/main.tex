\documentclass[12pt, a4paper, oneside]{book}

\usepackage{graphicx}
\usepackage{amsmath}
\usepackage{amsfonts}
\usepackage{mathtools}
\usepackage{amsthm}
\usepackage{amssymb}
\makeatother

\newtheorem{theorem}{Teorema}
\newtheorem{definition}{Definizione}
\newtheorem{eg}{Esempio}
\newtheorem{lemma}{Lemma}
\newtheorem{corollary}{Corollario}
\newcommand*{\blankpage}{
{\newpage \vspace*{5cm}\thispagestyle{empty}\centering \bfseries  \textit{Inserire citazione inerente alla materia}\par}
\vspace{\fill}}

\textwidth=450pt\oddsidemargin=0pt

\begin{document}
	
	\begin{titlepage}
		\begin{center}
			{{\LARGE{\textsc{Università degli Studi di Verona}}}} \rule[0.1cm]{15.8cm}{0.1mm}
			\rule[0.5cm]{15.8cm}{0.6mm}
			{\small{\bf CORSO DI LAUREA IN INFORMATICA}}
		\end{center}
		\vspace{12.5mm}
		
		\begin{center}
			{\Huge{\bf Basi di dati}}
		\end{center}
		\vspace{130mm}
		
		\begin{minipage}[l]{1.8mm}
			\rule[0.1cm]{0.1mm}{2cm}
			\rule[0.1cm]{0.6mm}{2cm}
		\end{minipage}
		\hspace{2mm}
		\begin{minipage}{10cm}	
			{\large{Federico Brutti\\
			\texttt{federico.brutti@studenti.univr.it}}}
		\end{minipage}
	\end{titlepage}
	
	\blankpage
	\tableofcontents
	
	\chapter{Sistemi per la gestione di database}
	Onestamente non ho la benché minima idea di cosa tratti matematica di base; tutti gli argomenti sembrano familiari ma allo stesso tempo estranei. Inoltre sembra una materia di cui si sente la mancanza nell'ordinamento precedente. Iniziamo con la definizione formale di \textbf{Insieme}, elemento della teoria su cui si basa la matematica tutta:
\begin{definition}
    \textbf{Insieme}\par
    Gruppo di elementi aventi una stessa proprietà. Si indica con una lettera maiuscola.
\end{definition}
Pare ovvio che con questi insiemi sia possibile operare in qualche modo; per prima cosa elenchiamo i simboli utilizzati nel corso:\par
\textbf{Connettivi}:
\begin{itemize}
    \item \textbf{Congiunzione}: $\land$\par
    Ritorna vero solo se tutti gli elementi sono veri.
    \item \textbf{Disgiunzione}: $\lor$\par
    Ritorna vero se almeno un elemento è vero.
    \item \textbf{Negazione}: $\neg$\par
    Rende falso il vero e viceversa.
    \item \textbf{Implicazione}: $\implies$\par
    Corrisponde a "Se, allora", ritorna vero nei casi $0 \to 1$ oppure $1 \to 1$, mentre è falso se $1 \to 0$ oppure $0 \to 0$.
    \item \textbf{Doppia Implicazione}: $\iff$\par
    Corrisponde a "se e solo se, allora" e viene rappresentata mediante due implicazioni: $(\phi \to \psi) \land (\psi \to \phi)$.
    \item \textbf{Bottom}: $\bot$\par
    Indica il valore di assurdo, $0$.
    \item \textbf{Top}: $\top$\par
    Indica il valore di verità, $1$.
\end{itemize}
\textbf{Quantificatori}:
\begin{itemize}
    \item \textbf{Esiste}: $\exists$\par
    Indica l'esistenza di un elemento con una determinata proprietà. Normalmente si usa legato ad una proprietà di un elemento, quindi per dimostrarlo serve quest'ultimo e la prova di tale proprietà.
    \item \textbf{Per ogni}: $\forall$\par
    Indica che per ogni caso considerato, esiste un elemento con una data proprietà. Per dimostrarlo serve supporre un elemento e trovare una prova della proprietà ad esso associata.
\end{itemize}
Dai connettivi e i quantificatori abbiamo anche i seguenti assiomi logici:
\begin{itemize}
    \item \textbf{Tautologie}\par
    Formule che risultano vere in ogni istanza presa in esame. Un esempio di tautologia sono le leggi di De Morgan, fondamentali per l'insiemistica.
    \begin{eg}
        \textbf{Tautologia}\par
        \begin{itemize}
            \item \textbf{Semplice}:\par
            Data una formula $P$ abbiamo che $P\implies P$ è sempre vera, quindi una tautologia. Per dimostrarla troviamo una prova di $P$ e hai fatto.
            \item \textbf{Modus Ponens}:\par
            Se $P, Q$ sono due formule, allora $(P\implies Q) \implies (\neg Q \implies \neg P)$ è tautologia. Per dimostrarla è necessario trovare le prove di ambo le ipotesi, dopodiché supponi le prove per $[\neg P := (P \implies \bot)]$ e $[\neg Q := (Q \implies \bot)]$.\par
            Supponi ora $P$. Da $P\implies Q$ traiamo $Q$, dalla quale possiamo trarre $Q \implies \bot$, quindi $\bot$. La formula quindi vale perché dall'assurdo si può derivare qualunque cosa.
        \end{itemize}
    \end{eg}
    \item \textbf{Principio del terzo escluso}:\par
    Data la formula $(P \lor \neg P)$, non c'è nessun altro elemento fra $P \land \neg P$ o $P \lor \neg P$.
    \item \textbf{Eliminazione della doppia negazione}:\par
    Dall'assurdo possiamo derivare qualunque cosa, di conseguenza possiamo derivare una formula $P$ da $\bot$.
    \begin{eg}
        \begin{center}
            $\neg\neg P \implies P := \neg P \implies \bot := (P \implies \bot) \implies \bot$.
        \end{center}
    \end{eg}
\end{itemize}

Ed ora introduciamo tutte le varie operazioni insieme alle loro proprietà.

\section{Operazioni fra gli insiemi}
Distinguiamo inizialmente i due casi in cui è possibile operare con gli insiemi:
\begin{itemize}
    \item \textbf{Coppie}, collezioni di oggetti dove è possibile distinguere il primo elemento dal secondo. Si distinguono in:
    \begin{itemize}
        \item \textbf{Ordinate}: $(A, B) = \{\{x\}, \{x, y\}\}$\par
        Insieme dove gli elementi sono legati da una determinata relazione di ordinamento.
        \item \textbf{Non ordinate}: $(A, B) = (B, A)$\par
        Gli insiemi di questo tipo saranno sempre uguali se contengono gli stessi identici elementi, a prescindere dall'ordine in cui sono scritti.
    \end{itemize}
    \item \textbf{N-uple}, dove sono presenti più di due insiemi, trattato più avanti.
\end{itemize}
Ed ora possiamo iniziare con le operazioni effettive:
\begin{itemize}
    \item \textbf{Appartenenza, contenimento e sottoinsieme}\par
    Diciamo che un elemento $x$ appartiene ad un insieme $A$ quando rispetta i criteri per farne parte, come avere una determinata proprietà o caratteristica.
    \begin{definition}
        \textbf{Appartenenza e non appartenenza}\par
        Data una proprietà $P$ requisito per far parte dell'insieme $A$, definiamo formalmente:
        \begin{itemize}
            \item \textbf{Appartenenza}: $x \in A$, $A = \{x | P(x)\}$\par
            All'insieme $A$ appartiene l'elemento $x$ tale che $x$ abbia una data proprietà P.
            \item \textbf{Non appartenenza}: $y \notin A$\par
            All'insieme $A$ non appartiene $y$.
        \end{itemize}
    \end{definition}
    Diremo poi che un insieme $B$ è sottoinsieme di $A$ quando il primo è interamente contenuto nel secondo. Ciò non necessariamente significa che sia uguale, tuttavia.
    \begin{definition}
        \textbf{Sottoinsiemi}\par
        Dati due insiemi $A$ e $B$ diremo che $B$ possiamo avere i seguenti casi:
        \begin{itemize}
            \item \textbf{Sottoinsieme Improprio}: $B \subseteq A \iff \forall x.(x \in A \implies x \in B)$\par
            Quando ogni elemento appartiene a $B$, appartiene anche ad $A$.
            \item \textbf{Uguaglianza}: $A = B \iff \forall x.(x \in A \iff x \in B)$\par
            Quando due insiemi sono perfettamente uguali.
            \item \textbf{Sottoinsieme Proprio}: $B \subset A$\par
            Quando tutti gli elementi di $B$ appartengono ad $A$ e $A \neq B$.
        \end{itemize}
    \end{definition}
    Abbiamo infine l'elemento neutro, detto \textbf{Insieme Vuoto}, scritto con $A = \emptyset$, il quale indica un insieme privo di elementi.
    \item \textbf{Unione}\par
    L'unione fra due insiemi risulta come un terzo insieme contenente gli elementi di entrambi. Formalmente:
    \begin{definition}
        \textbf{Unione} $A \cup B = \{a | a \in A \lor a \in B\} = C$\par
        Unisce gli elementi di $A$ a quelli di $B$ per creare un nuovo insieme $C$ che contiene tutti gli elementi dei primi due senza ripetizioni. Detiene inoltre le seguenti proprietà:
        \begin{itemize}
            \item $A \cup \emptyset = \emptyset$
            \item $(A \cup B) = (B \cup A)$
            \item $(A \cup B \cup C) = (A \cup B) \cup C$
            \item $A \cup A$
            \item $A \subseteq C \land B \subseteq C = A \cup B \subseteq C$
            \item $A \subseteq C \iff A \cup Z = C$
        \end{itemize}
    \end{definition}
    \item \textbf{Intersezione}\par
    L'intersezione prende solamente gli elementi comuni ad $A$ e $B$.
    \begin{definition}
        \textbf{Intersezione} $A \cap B = \{x|x \in A \land x \in B\}$\par
        Dati due insiemi $A,B$, crea un insieme $C$ che contiene esclusivamente gli elementi comuni ai primi due. Detiene le seguenti proprietà:
        \begin{itemize}
            \item $A \cap \emptyset = \emptyset$
            \item $A \cap B = B \cap A$
            \item $A \cap (B \cap C) = (A \cap C) \cap C$
            \item $A \cap A = A$
            \item $C \subseteq A \land C \subseteq B \implies C \subseteq A \cap B$
            \item $A \subseteq B \iff A \cap B = A$
            \item $A \cap (B \cup C) = (A \cap B) \cup (A \cap C)$
        \end{itemize}
    \end{definition}
    \item \textbf{Prodotto cartesiano}\par
    Il Prodotto Cartesiano è una relazione fra due insiemi dove a partire dagli elementi di $A$, crea tutte le coppie possibili con gli elementi di $B$. Giuro è più semplice a vederlo.

    \begin{definition}
        \textbf{Prodotto Cartesiano} $A \times B = \{(x,y)| x \in A \land y \in B\}$\par
        Dati due insiemi $A,B$, si definisce il loro prodotto cartesiano l'insieme di tutte le coppie ordinate di elementi, indicati da $(a, b)$, tali che il primo elemento a della coppia appartenga all'insieme A e il secondo elemento b della coppia appartenga all'insieme B.
    \end{definition}

    \begin{eg}
        \textbf{Calcolo di un prodotto cartesiano}\par
        Non è molto dissimile da un prodotto di polinomi; moltiplichi ogni elemento di A per ogni elemento di B, come segue:
        \begin{center}
            $A = {1, 2}$, $B = {3, 4}$\par
            $A \times B = C = {(1,3), (1,4), (2,3), (2,4)}$
        \end{center}
    \end{eg}
    \item \textbf{Differenza}\par
    La differenza fra insiemi sottrae gli elementi di $B$ a quelli di $A$.
    \begin{definition}
        \textbf{Differenza} $A \setminus B = \{x|x \in A \land  x \notin B$\}\par
        Dati due insiemi $A,B$, l'operazione differenza sottrae tutti gli elementi di $B$ a quelli di $A$. Nel caso in cui gli insiemi non abbiano elementi in comune, l'operazione non avrà effetto. Detiene le seguenti proprietà:
        \begin{itemize}
            \item $A \setminus \emptyset = A$
            \item $A \setminus A = \emptyset$
            \item $(A \setminus B) \cap B = \emptyset$
            \item $(A \setminus B) \cup A = A$
            \item $A \cup B = (A \setminus B) \cup (A \cap B) \cup (B \setminus A)$
        \end{itemize}
    \end{definition}
    Un'altra operazione molto utile sempre in questo senso è la \textbf{Differenza Simmetrica}, la quale permette di ricavare esclusivamente gli elementi unici da due insiemi.
    \begin{definition}
        \textbf{Differenza Simmetrica} $A \triangle B = (A \setminus B) \cup (B \setminus A)$\par
        Dati due insiemi $A, B$, la differenza simmetrica effettua un'unione fra la differenza $A\setminus B$ e $B\setminus A$, con lo scopo di ottenere Tutti gli elementi appartenenti ai due insiemi che non sono ripetuti. Detiene le seguenti proprietà:
        \begin{itemize}
            \item $A \triangle B = (A \cup B) \setminus (A \cap B)$
            \item $A \triangle B = B \triangle A$
            \item $(A \triangle B) \triangle C = A \triangle (B \triangle C)$
            \item $A \cap (B \triangle C) = (A \cap B) \triangle (A \cap C)$
            \item $A \triangle \emptyset = A$
            \item $A \triangle A = \emptyset$
            \item $(A \triangle B) \cap C = (A \cap C) \triangle (B \cap C)$
        \end{itemize}
    \end{definition}
    \item \textbf{Famiglie di insiemi}\par
    \begin{definition}
        \textbf{Famiglie di insiemi} - $\chi := \{X_i | i \in I\}$\par
        Se ad ogni elemento $i$ di un insieme non vuoto $I$ corrisponde un insieme $X_i, i \to X_i$, allora l'insieme di insiemi $X_i$ è chiamato \textbf{famiglia di insiemi} ed $I$ è il suo insieme di indicizzazione.
    \end{definition}
    INSERISCI ESEMPIO
    \item \textbf{Insieme delle parti}\par
    \begin{definition}
        \textbf{Insieme delle parti} - $P(X) := \{A | A \subseteq X\}$\par
        Definiamo l'insieme delle parti $P(X)$ l'insieme di tutti i sottoinsiemi di $X$.
    \end{definition}
    INSERIRE ESEMPIO
\end{itemize}

\subsection{Leggi di De Morgan}

%

\section{Relazioni fra insiemi}
Ci è possibile mettere in relazione gli elementi di due o più insiemi diversi\footnote{Il totale degli insiemi nella relazione è dato dall'arietà. Se è unaria, sarà per un insieme, se binaria per due e così via.}. Diciamo infatti che se $x\in X$ e $y\in Y$, i due elementi sono in relazione se la loro coppia $(x,y)$ è in una relazione $R$, intesa come sottoinsieme $R$ di $X \times Y$. Segue definizione formale:
\begin{definition}
    \textbf{Corrispondenza}\par
    Una corrispondenza dell'insieme $X$ nell'insieme $Y$ è un qualunque insieme $R . R \subseteq X \times Y$. Se la coppia $(x,y) \in R$ si dice che $x$ corrisponde a $y$ nella corrispondenza $R$. Si scrive anche\par
    \begin{center}
        $xRy :\iff (x,y) \in R$
    \end{center}
\end{definition}
A partire da questo possiamo trovare i seguenti casi:
\begin{itemize}
    \item Un elemento di $X$ può corrispondere a più elementi di $Y$ e viceversa.
    \item Un elemento di $X$ può corrispondere a più elementi di $Y$, i quali a loro volta corrispondono ad altri elementi di $X$.
    \item Relazione vuota, dove in $X$ non ci sono elementi che corrispondono agli elementi di $Y$.
\end{itemize}
\begin{eg}
    INSERISCI ESEMPIO DI RELAZIONE
\end{eg}
Ed ora possiamo definire questi altri casi notevoli


\begin{itemize}
    \item \textbf{Relazione inversa}\par
    \begin{definition}
        \textbf{Relazione inversa} - $R^{-1} \subseteq Y \times X$\par
        La relazione inversa sussiste solamente se abbiamo la certezza che esista la coppia $(x,y) \in R$. La definiamo formalmente come:
        \begin{center}
            $R^{-1} := \{(y,x) \in Y \times X . (y,x) \in R\}$
        \end{center}
    \end{definition}
    INSERISCI ESEMPIO
    \item \textbf{Composizione delle operazioni}\par
    Ti ricorderai il problema delle funzioni composte; è esattamente la stessa cosa: più funzioni messe insieme.
    \begin{definition}
        \textbf{Composizione delle operazioni}\par
        Se $R \subseteq X \times Y \land S \subseteq Y \times Z$, la loro composizione $S \circ R\subseteq X \times Z$ è definita come segue:
        \begin{center}
            $S \circ R := \{(x,z) \in X \times Z . \exists y \in Y . ((x,y) \in R \land (y,z) \in S)\}$
        \end{center}
    \end{definition}
    INSERISCI ESEMPIO CON LA DIAGONALE.
\end{itemize}

%

\section{Principi di dimostrazione}
Il processo di dimostrazione matematica è un algoritmo deduttivo utilizzato per provare la verità di ipotesi arbitrarie basandosi sul ragionamento logico. Esistono più metodi per arrivare ad una stessa soluzione:
\begin{itemize}
    \item \textbf{Dimostrazione per assurdo}\par
    Partiamo dal presupposto che la tesi sia falsa. Se si riesce a concludere il processo senza incappare in contraddizioni si è dimostrato che la tesi è falsa, altrimenti è vera.\par
    \begin{eg}
        \textbf{Dimostrazione per assurdo}\par
        INSERISCI ESEMPIO.
    \end{eg}
    \item \textbf{Dimostrazione per induzione}\par
    Algoritmo basato sul passaggio dallo specifico al generale, si compone di due passi:
    \begin{enumerate}
        \item Passo base, dove si prende un valore comodo per provare la veridicità della tesi nel caso più facile.
        \item Passo induttivo, dove si prova, basandosi sul caso base, che valga anche per tutte le istanze successive.
    \end{enumerate}
    \begin{eg}
        \textbf{Dimostrazione per induzione su $\mathbb{N}$}\par
        INSERISCI ESEMPIO
    \end{eg}
    \begin{eg}
        \textbf{Dimostrazione per induzione su $\mathbb{N}^*$}\par
        Definiamo l'insieme numerico di lavoro come $\mathbb{N}^* = \mathbb{N} \setminus \{0\} = \{1, 2, ..., n\}$\par
        Tesi da provare: $\theta(n) = \forall n \in \mathbb{N}^*.(1+2+...+n=\dfrac{n(n+1)}{2})$
        \begin{itemize}
            \item \textbf{Passo base}:\par
            Testiamo se la tesi vale sostituendo $n$ a $1$\par
            \begin{center}
                $\theta(1) \iff 1 = \dfrac{1(1+1)}{2} = 1$, che è vera.
            \end{center}
            \item \textbf{Passo induttivo}:\par
            Espandiamo il ragionamento per $\theta(n+1)$. Va sostituito $(n+1)$ alla singola $n$ presente nella tesi iniziale.
            \begin{center}
                $\theta(n+1) \iff (1+2+...+n+(n+1)) = \dfrac{(n+1)((n+1)+1)}{2} = \dfrac{(n+1)(n+2)}{2}$
            \end{center}
            Adesso proviamo che il risultato ottenuto è valido:
            \begin{center}
                $\dfrac{n(n+1)}{2}+(n+1) = \dfrac{n(n+1) + 2(n+1)}{2} = \dfrac{n^2+3n+2}{2} = \dfrac{(n+1)(n+2)}{2}$
            \end{center}
        \end{itemize}
        Come volevasi dimostrare.
    \end{eg}
    \item \textbf{Dimostrazione per ricorsione}\par
    La ricorsione si costituisce anch'essa di due casi, ovvero il caso base, da dove inizia tutto, ed il caso ricorsivo, che avanza tenendo conto dei valori precedentemente ottenuti.
    \begin{eg}
        \textbf{Dimostrazione per ricorsione}\par
        Definiamo la funzione $n!$:
        \begin{itemize}
            \item Passo base\par
            Se $n = 0 \implies 1$
            \item Passo ricorsivo\par
            Se $n > 0 \implies (n-1)! \times n$ 
        \end{itemize}
        Proviamo a ragionare come si comporta tale funzione quando sostituiamo alla $n$ i valori presi in esame. Otterremo che:
        \begin{center}
            $0! = (0-1)! \times1 = 1$\par
            $1! = (1-1)!\times1 = 1\times1 = 1$\par
            $2! = (2-1)!\times2 = 1\times2 = 2$\par
            $3! = (3-2)!\times3 = 2\times3 = 6$\par
            $4! = (4-3)!\times4 = 6\times4 = 24$\par
        \end{center}
    \end{eg}
\end{itemize}

\section{Domande di teoria}

\begin{theorem}
    Here goes a theorem.
\end{theorem}

\begin{proof}
        Here goes the proof
\end{proof}

\begin{corollary}
    Here goes a collorary
\end{corollary}

\begin{eg}
    Here goes an example
\end{eg}

\begin{note}
    Here goes a note 
\end{note}

\begin{lemma}
    Here goes a lemma
\end{lemma}

\begin{prop}
    Here goes a proposition
\end{prop}

\begin{definition}
    Here goes a definition 
\end{definition}

\subsection{Esercizi}
	\chapter{Progettazione di una base di dati}
	\section{Sistemi Lineari Tempo invariante}
Possiamo ora dare una definizione rigorosa di sistema: si tratta di un modello matematico o formulazione di un processo o fenomeno fisico che permette di trasformare un'entrata in un'uscita determinata. Ne esistono principalmente di due tipi:
\begin{itemize}
	\item \textbf{Single Input Single Output}
	\item \textbf{Multiple Input Multiple Output}, non visti in questo programma.
\end{itemize}
\noindent In base a questi poi ci sono i sistemi dinamici, ovvero con memoria, i quali salvano i dati grazie agli stati. Rammento infine che, come i segnali, i sistemi possono essere a tempo continuo e discreto.\par
Per quanto riguarda i sistemi LTI, approfondiamo le proprietà menzionate nel capitolo precedente:
\begin{itemize}
	\item \textbf{Linearità}: Partendo da un sistema con due entrate, otterrò in output un risultato equivalente alla combinazione lineare dei singoli input. Formalmente: \[da rivedere\]
	\item \textbf{Tempo-invarianza}: Traslando l'input del sistema prima o dopo nel tempo avrò come output lo stesso risultato traslato prima o dopo nel tempo: \[u(t)\to v(t) \implies u(t+\tau)\to v(t+\tau), \forall \tau \in \mathbb{R}\]
	\item \textbf{Causalità}: L'effetto non precede mai la causa. L'uscita $v(t)$ all'istante $\tau$ deve dipendere esclusivamente dall'ingresso $u(t)$ per $t \leq \tau$. Per comodità considereremo solo sistemi a riposo, ovvero con $\tau = 0$.
\end{itemize}
\noindent Richiamiamo all'attenzione il concetto di stabilità per formalizzarlo definitivamente. Definiamo:\newline

\noindent \textbf{Sistema asintoticamente stabile}\par
\noindent Un sistema si dice tale se dopo aver dato l'input, l'output va a morire fino ad arrivare a $0$. Formalmente:
\begin{center}
	$\exists\tau\in\mathbb{R}$ tale che $u(t) = 0$, $\forall t \geq \tau \implies lim_{t\to \infty}v(t) = 0$
\end{center}

\noindent\textbf{Sistema BIBO-stabile}\par
\noindent

\begin{comment}
	--- Rivedere ---
	
	1. Se ho un sistema con a_1u_1(t)+b_1u_2(t) [la somma degli input u_n crea un altro input.] e li metto dentro al sistema, otterrò un risultato che è combinazione lineare dei singoli input, quindi: a_1u_1(t \to a_2u_2)
	
	
	
	--- Riprendi da qui ---
	
	- Proprietà di stabilità BIBO: Un sistema si dice tale se \exists\tau\in\mathbb{R} e M_u>0\in R tale che se |u(t)| < M_u, \forall t \geq \tau \to \exists M_v > 0: |v(t)| < M_v \forall t \geq \tau. WhatTheFuck
	Diciamo di avere un input che oscilla, ad un certo \tau continua a farlo semprre entro una certa ampiezza. Se lo butto dentro al sistema otterrò un segnale in uscita che sarà dopo \tau all'interno di un certo intervallo di ampiezza.
	Quindi, esiste un \tau istante di tempo ed un vincolo massimo e minimo, ambo in ampiezza, paralleli all'asse delle x. Dopo \tau, l'input rimane all'interno di questi vincoli.
	Se la stessa cosa vale per l'output, si dice bibostabile.
	
	Ogni sistema asintoticamente stabile è necessariamente anche bibostabile, ma non viceversa.
	
	Come modelliamo i sistemi?
	Descriveremo i sistemi tramite equazioni differenziali in input e in output.
	Passaggi da fare:
	- Scrivere tutte le equazioni che descrivono le leggi in atto
	- Scrivere estesa l'equazione differenziale del sistema. Input a dx, Output a sx.
	
	Forma generale dei modelli?
	a_n*\frac{\partial^nv(t)}{\partial t^n}+a_{n-1}*\frac{\partial^{n-1}v(t)}{\partial t}+ ... + \frac{\partial^1v(t)}{\partial t} + a_0v(t) = b_m*\frac{\partial^mu(t)}{\partial t}+b_{m-1}*\frac{\partial^{m-1}u(t)}{\partial t} + ... + b_1*\frac{\partial u(t)}{\partial t} + b_0u(t)
	
	u(t) è l'input, v(t) è l'output.
	b_n, a_n \neq 0 sono due coefficienti normali.
	
	In forma più compatta, sono due sommatorie: \sum_{i=0}^na_i \frac{\partial^i v(t)}{\partial t} = \sum_{j=0}^mb_j\frac{\partial^ju(t)}{\partial t}
	n ed m danno l'ordine delle equazioni differenziali e di solito n\geq m.
	I sistemi si scrivono con questa equazione. La scatoletta, proprio.
	
	Se n>m, si dice strettamente proprio
	Se n\geq m si dice proprio
\end{comment}
	\chapter{Interazione con un database}
	La \textbf{Logica del Primo Ordine}, o dei Predicati è un'espansione della Logica Proposizionale. La novità principale è la presenza di due nuovi simboli detti \textbf{Quantificatori}: \textit{Per ogni} $\forall$ ed \textit{Esiste} $\exists$, con i quali sarà possibile descrivere strutture matematiche. La loro particolarità è che legano più di ogni altro connettivo.

\section{Sintassi}
\section{Sistemi Deduttivi}
formalizzazione, contromodello, struttura di Peano
\section{Semantica}
\section{Esercizi}
\section{Appunti}
\subsection{Strutture e Tipi di Similarità}
\subsection{Derivazione Semantica}
\subsection{Deduzione Naturale}
\subsection{Teoremi di Correttezza e Completezza}
	\chapter{RDBMS e SQL in postgreSQL}
	\section{Insiemi finiti e infiniti}
\section{Equipotenza}
\section{Ordinamento delle cardinalità}
\section{Teorema di Cantor}
\section{Non numerabilità dei reali}
\section{Domande di teoria}
\section{Esercizi}
	\chapter{Query e transazioni}
	\section{Framing}

%

\section{Accesso al canale condiviso}

%

\section{Sottolivello MAC}
Protocolli Aloha, CSMA, CSMA/CD, CSMA/CA

%

\section{Bridge, switch e LAN}

%

\section{Wireless LAN}
	\chapter{Accesso a database da programmi Python}
	\section{Introduzione ad Assembly}

%

\section{Istruzioni e Sintassi}
Stringhe e numeri
%

\section{Debugging e Makefile}

%

\section{Relazione ISA-FSMD su LC-3}

%

\section{Funzioni e passaggio di parametri}

%

\section{Confronto con il C}
	\chapter{DBMS, NoSQL e MongoDB}
	\section{Definizione}
Partiamo dal problema che fa nascere il senso di questo argomento: data un'applicazione lineare ed uno spazio vettoriale sul quale opera noi vogliamo trovare una base $B$ di $V$ tale che la matrice $A$ che rappresenta l'applicazione rispetto alla base sia la più semplice possibile.\par\quad
Questa forma ambita è la matrice quadrata diagonale di ordine $n = dimV$, i cui elementi al di fuori della diagonale principale sono nulli. Per i nostri scopi inseriremo il valore $\lambda$ a tutti gli elementi della diagonale principale. Da questa ricerca della base si arriva all'introduzione dei concetti di questa sezione: \textbf{Autovettori} e \textbf{Autovalori}.\newline

\textbf{- Che cosa sono autovettori e autovalori?}\par
Partiamo dalle definizioni; esiste un significato distinto di entrambi i concetti per applicazioni lineari e matrici ed è bene tenerli entrambi a mente.
\begin{definition}
    \textbf{Autovettori e autovalori di un'applicazione lineare}\par
    Sia $f:V\to V$ un'applicazione lineare. Un vettore tale che $v \in V$ si dice \textbf{autovettore} della funzione se:
    \begin{itemize}
        \item $v$ non è il vettore nullo.
        \item Esiste uno scalare $\lambda \in \mathbb{K}$ tale che $f(v) = \lambda v$.
    \end{itemize}
    Inoltre, lo scalare $\lambda$ è univocamente determinato dal vettore $v$ e si dice \textbf{autovalore} della funzione relativo all'autovettore $v$.
\end{definition}
Ottenendo gli autovalori possiamo dire di poter creare uno spazio che li racchiude tutti; lo chiamiamo \textbf{autospazio}. Per ogni $\lambda \in \mathbb{K}$ poniamo la seguente formula:
\begin{center}
    $V_\lambda = Ker(f-\lambda I) = \{v \in V:f(v) = \lambda v\}$
\end{center}
questo sottospazio vettoriale di $V$ si dice autospazio di $f$ relativo all'autovalore $\lambda$ ed i suoi elementi non nulli sono tutti gli autovettori dell'applicazione relativi all'autovalore a lei dato.\par\quad
Per le matrici il concetto si rivela più semplice e intuitivo, anche se alla fine gli elementi più utilizzati rimarranno gli autovalori.
\begin{definition}
    \textbf{Autovettori e autovalori di una matrice}\par
    Sia $A$ una matrice quadrata di ordine $n$. Un vettore $x \in \mathbb{C}^n$ non nullo si dice \textbf{autovettore} della matrice se esiste uno scalare $\lambda \in \mathbb{C}$ tale che valga:
    \begin{center}
        $Ax = \lambda x$
    \end{center}
    Come per le applicazioni, lo scalare $\lambda$ è univocamente determinato dal vettore $x$ e viene chiamato \textbf{autovalore} di $A$ relativo all'autovettore $x$.
\end{definition}
\textbf{- Come si ottengono autovettori e autovalori?}\par
Palle\newline

\textbf{- A cosa servono autovettori e autovalori?}\par



§9. Autovalori e autovettori (vedi [GS, Capitolo V])
9.1 Definizione: autovalore e autovettore
9.2 Osservazione: autovettori sono soluzioni di un sistema lineare
\section{Polinomio caratteristico}
9.3 Definizione: polinomio caratteristico
9.4 Teorema: autovalori sono radici e autovettori sono elementi di spazi nulli (autospazi)
9.5 Corollario: matrici su C possiedono autovalori
9.6 Definizioni: autospazio, moltiplicit`a algebrica e geometrica
9.7 Osservazione: se esiste una base B formata di autovettori di A, allora la matrice associata a A
rispetto a B nel dominio e codominio `e diagonale.
9.8 Proposizione: autovettori linearmente indipendenti
9.9 Definizioni: matrici simili, matrice diagonalizzabile


\section{Esercizi}
\section{Appunti}
Gli autovettori e gli autovalori si ottengono mediante la ricerca del polinomio caratteristico. Suppongo che il primo \textbf{SUPPOSIZIONE, VERIFICARE} sia lo stesso polinomio trovato, mentre è certo che gli autovalori siano le radici, quindi le soluzioni, dello stesso.\par\quad
La ricerca del polinomio ha luogo con la formula $det(A-\lambda I_n)$, dove:
\begin{itemize}
    \item $A$ è la matrice presa sotto esame.
    \item $\lambda$ è lo scalare che verrà moltiplicato alla matrice identità.
    \item $I_n$ è la matrice identità di ordine $n$.
\end{itemize}
 Per ordine si intende quanto è lunga. Essendo che normalmente si lavora con matrici quadrate $3x3$, l'ordine sarà, appunto, $3$.\par\quad
 Per il calcolo del determinante è consigliato semplificare per la riga o colonna che presenta più zeri, in tal modo da ridurre al minimo i calcoli e quindi i possibili sbagli. Fondamentalmente l'operazione consiste nella sottrazione fra la matrice sotto esame $A$ e una matrice diagonale i cui elementi della diagonale principale sono tutti $-\lambda$.\par\quad
 In termini ancora più semplici parliamo della matrice $A$ con aggiunto il termine $-\lambda$ alla sua diagonale principale.\par\quad
 Fatto ciò è possibile calcolare il determinante. Se ti trovi incasinato con le lambda è normale. Spesso è necessario scomporre il polinomio in forma più leggibile. Prega che non sia un grado sì alto da usare Ruffini. Bleah.\newline

 Trovati gli autovalori sarà sempre richiesto di calcolare molteplicità algebrica e geometrica. La prima è semplicissima, il suo valore corrisponde a quante volte è presente l'autovalore nel polinomio. La molteplicità geometrica invece richiede più passaggi, data la formula $m_g = n -rk(A - \lambda I_n)$:
 \begin{itemize}
     \item $n$ è l'ordine della matrice. Se hai una quadrata sarà uguale per righe e colonne. Per esempio se la matrice è $3x3 \to n = 3$. Semplice.
     \item Calcola il rango della matrice risultante dall'operazione $A - \lambda I_n$ tramite Gauss. Non fare cazzate, di solito è semplice.
 \end{itemize}
La molteplicità geometrica è necessariamente un valore che segue questa restrizione: $m_a \leq 1 \leq m_g$, quindi è necessariamente maggiore o uguale della sua controparte algebrica. Ciò dà un utile sicurezza per il controllo calcoli.\newline

Da questi dati è possibile determinare se una matrice è o meno \textbf{diagonalizzabile}. Qua termina la mia conoscenza.

	\chapter{Architettura interna, DBMS}
	\section{Definizioni e Polinomio caratteristico}
\section{Teoremi, autospazio, moltiplicazione algebrica e geometrica}
\section{Considerazioni su matrici, indipendenza lineare, similarità e matrici diagonalizzabili}
\section{Esercizi}
	\chapter{Interrogazioni in SQL}
	\section{Introduzione al linguaggio Assembly}
Quando collocate nella memoria, le istruzioni macchina hanno una forma binaria, illeggibile dalla persona; per questo ci si affida alla forma simbolica dell'istruzione, chiamata \textbf{Assembly Language}. Si tratta di un linguaggio comune a tutti gli elaboratori, usato come base per creare gli ISA propri della macchina.\par\quad
Consente di accedere ai registri della CPU, Scrivere un codice ottimizzato per una specifica architettura di microprocessore e di ottimizzare sezioni dei programmi a livello hardware. Esistono due tipi di convenzioni o regole:
\begin{itemize}
    \item Quelle che definiscono la forma simbolica del linguaggio che formerà il \textbf{Programma Sorgente}.
    \item Quelle che definiscono la forma numerica che formerà il \textbf{Programma Oggetto} e il passaggio all'altro tipo.
\end{itemize}
Il compito di controllare la correttezza sintattica del codice è affidato all'\textbf{Assembler}, il quale eventualmente genererà la forma numerica corrispondente a quanto scritto sotto forma di file con \textit{estensione ".o"}. L'ISA utilizzato per gli esercizi è \textbf{AT\&T}, la cui unica differenza dal più comune \textbf{Intel 80x86} è l'utilizzo di un \% prima del nome dei registri.
\begin{itemize}
    \item ISA AT\&T:\par\quad
    ADDL \%EAX, \%EBX
    \item ISA INTEL 80X86:\par\quad
    ADD EAX, EBX
\end{itemize}
Utilizzeremo registri dalla dimensione base di 16b. Scrivendo "E" all'inizio del loro nome diventeranno \textit{extended registers}, dalla dimensione di 32b. Si dividono in:\newline

\textbf{- Generici}:
\begin{itemize}
    \item \textit{AX}, Accumulation Register; Accumulatore di operazioni aritmetiche e contenitore del risultato.
    \item \textit{BX}, Base Register; Per le operazioni di indirizzamento alla memoria.
    \item \textit{CX}, Counter Register; Usato per contare, per esempio, l'indice nei cicli.
    \item \textit{DX}, Data Register; Usato nelle operazioni di I/O, divisioni e moltiplicazioni.
\end{itemize}
\textbf{- Di segmento}:
\begin{itemize}
    \item \textit{CS}, Code Segment; Punta alla zona di memoria che contiene il codice e fa accedere all'istruzione successiva. Non può essere modificato.
    \item \textit{DS}, Data Segment; Punta alla zona di memoria che contiene i dati.
    \item \textit{ES}, Extra Segment: Spesso usato come registro ausiliario.
    \item \textit{SS}, Stack Segment; Punta alla zona di memoria dove risiede la stack.
\end{itemize}
\textbf{- Puntatore}:
\begin{itemize}
    \item \textit{SP}, Stack Pointer; Punta alla cima della stack. Viene modificato dalle istruzioni PUSH e POP\footnote{Rispettivamente inserimento ed estrazione di un dato dalla stack}.
    \item \textit{BP}, Base Pointer; Punta alla base della porzione di stack gestita in quel punto dal codice.
    \item \textit{IP}, Instruction Pointer; Punta alla prossima istruzione da eseguire.
\end{itemize}
\textbf{- Indice}:
\begin{itemize}
    \item \textit{SI}, Source Index; Punta alla stringa/Vettore sorgente.
    \item \textit{DI}, Destination Index; Punta alla stringa/vettore destinazione.
    \item \textit{FLAGS}; Memorizza lo stato corrente del processore. Ogni bit fornisce un'informazione diversa.
\end{itemize}

Infine vediamo le modalità di indirizzamento:
\begin{itemize}
    \item \textit{A registro}, [\%EAX, \%EBX]\par\quad
    L'operando è contenuto in un registro il cui nome è specificato nell'istruzione.
    \item \textit{Diretto o Assoluto}, [(\%EAX), \%EBX]\par\quad
    L'operando è contenuto in una locazione di memoria e l'indirizzo di quest'ultimo è specificato nell'istruzione.
\end{itemize}

%

\section{Istruzioni e sintassi}

%%%%%%%%%%%%%%%%%%%%%%%%%%%%%%%%%%%%%%%%%%%%%%%%%%%%%%%%%%%%%%%%%%%

\subsection{Stringhe e numeri}

%%%%%%%%%%%%%%%%%%%%%%%%%%%%%%%%%%%%%%%%%%%%%%%%%%%%%%%%%%%%%%%%%%%

\section{Debugging e Makefile}
Il debugging è una parte fondamentale per il controllo di eventuali problemi nel codice che è stato scritto. Quello che sarà utilizzato è \textbf{GDB}, il debugger di GNU.\par
È necessario assemblare il programma con le seguenti linee di comando da terminale:
\begin{verbatim}
    # Per creare il codice oggetto coi flag di debug
    > as -gstabs -o file.o file.s
    
    # Linking del file oggetto ad un nuovo file eseuibile
    > ld -o file file.o
\end{verbatim}

Seguono comandi utili generali per GDB. È possibile utilizzarli anche per i programmi C se è necessario:
\begin{verbatim}
    # Run del debugger
    > gdb nomeEseguibile

    # Cambio visualizzazione. Utile per osservare il workflow e lo stato dei 
      registri. Dopo aver dato il comando, premere invio per cambiare visuale.
    > lay next

    # Setta un breakpoint al punto indicato
    > break nomeFunzione / numeroRiga

    # Passa alla prossima istruzione
    > next / nexti

    # Va avanti fino al prossimo breakpoint
    > continue

    # Refresha la visuale. Si bugga spesso.
    > ref
\end{verbatim}

Adesso andiamo invece a vedere ulteriori comandi specifici per i programmi Assembly:
\begin{verbatim}
    # Stampa a video i valori contenuti nei vari registri.
    > info registers

    # Stampa il valore nel registro in binario, decimale, hex.
    > p/t $eax
    > p/d $eax
    > p/x $eax

    # Trova l'indirizzo della variabile e stamoa 4B
    > x/4b &variabile

    # Stampa il valore di una variabile.
    > print nomeVariabile

    # Stampa il valore di un registro come stringa.
    > x/s $eax
\end{verbatim}
Tuttavia, attenzione. Assemblare un progamma con i flag di debug peggiora le prestazioni totali; inoltre comporta problemi di sicurezza perché consentirebbe di leggere il codice sorgente tramite debugger.


%%%%%%%%%%%%%%%%%%%%%%%%%%%%%%%%%%%%%%%%%%%%%%%%%%%%%%%%%%%%%%%%%%%

\section{Relazione ISA-FSMD su LC3}

%%%%%%%%%%%%%%%%%%%%%%%%%%%%%%%%%%%%%%%%%%%%%%%%%%%%%%%%%%%%%%%%%%%

\section{Funzioni e passaggio di parametri}
Prima di parlare di funzioni è necessario chiarire il funzionamento dei due tipi di strutture di dato:
\begin{itemize}
    \item \textit{LIFO}; Last In, First Out, si tratta delle dinamiche governanti la stack. Immagina un secchio nel quale stai ponendo dei libri, noterai sicuramente che ti è possibile rimuovere solamente l'ultimo inserito.
    \item \textit{FIFO}; First In, First Out, facilmente comprensibile immaginando un tubo con una bocca che si chiude quando inserisci qualcosa ed un culo sempre aperto. Sarà possibile prelevare solo gli elementi in fondo alla pila.
\end{itemize}
Nei programmi Assembly è possibile utilizzare la stack mediante due comandi:
\begin{verbatim}
    # Mette il valore contenuto in eax nella stack. [32b]
    pushl %eax

    # Preleva l'ultimo valore inserito in stack e lo inserisce in eax. [32b]
    popl %eax
\end{verbatim}
Per il funzionamento della stack, ogni volta in cui si pusha qualche valore su di essa, il registro \%esp si decrementa, perché il valore più alto è alla base, mentre in cima sta 0.\par\quad
Conoscere questa dinamica è importante, perché consente un maggiore spazio di manovra nel salvataggio dei dati. Inoltre, possiamo capire come dare parametri da terminale quando si attiva un programma. La linea di comando conta come una stringa ed è quella il path dell'eseguibile. Tutti gli indirizzi dei parametri (stringhe) dati vengono impilati sulla stack, insieme al loro numero totale, che starà sempre in cima ad essi.\par\quad
Per poterli utilizzare bisogna usare il comando popl due volte per rimuovere prima il totale dei parametri, e poi prelevare il primo valore e salvarlo su qualche registro. Il comando dal terminale è infatti letto dall'assembler da destra a sinistra.\par\quad
Una cosa utile è inoltre la possibilità di ottenere un valore dalla stack in qualunque posizione senza che esso venga prelevato, utilizzando la modalità di \textit{Indirizzamento più spiazzamento}.
\begin{verbatim}
    # Salva in ecx il valore in stack alla posizione +8 rispetto ad esp
      (vista come posizione 0).    
    movl 8(%esp), %ecx
\end{verbatim}

Passiamo alle \textbf{funzioni}; è realisticamente impensabile voler scrivere tutto un programma in un singolo file, sia per motivi di lettura, che di debugging e controllo. Divideremo quindi il progetto in vari sottoprogrammi. Ciò è possibile tramite la seguente scrittura:
\begin{verbatim}
    # Nel file main, si invoca la funzione.
    call nomeFile
    
        # Nel file nomeFile
            # Dichiarazione della funzione
            .type nomeFunzione, @function
            .
            # Blocco di istruzioni
            .
        # Ritorno al file chiamante la funzione
        ret
\end{verbatim}
Con queste conoscenze sei ora pronto a lavorare al progetto richiesto. Se non lo consegnerai entro i termini, perderai ogni cosa e avrai sprecato tempo.

%%%%%%%%%%%%%%%%%%%%%%%%%%%%%%%%%%%%%%%%%%%%%%%%%%%%%%%%%%%%%%%%%%%

\section{Confronto con il linguaggio C}
	\chapter{MongoDB}
	\section{Matrice coniugata, H-trasposta, prodotto interno e norma}
\section{Interpretazione geometrica in $\mathbb{R}^2$}
\section{Definizione di Ortogonale}
Insieme ortogonale è linearmente indipendente.
Coefficienti per base Ortogonale
\section{Definizione di Ortonormale}
C'ha anche il corollario.
\subsection{Algoritmo di Gram Schmidt}
\section{Esercizi}

\end{document}