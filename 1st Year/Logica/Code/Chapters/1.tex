\section{Che cos'è la logica matematica?}
La \textbf{Logica Matematica} ha lo scopo di formalizzare concetti matematici in una lingua artificiale, per dare una certezza di significato al linguaggio naturale, il quale risulterebbe ambiguo. Similmente a quest'ultimo, anche i linguaggi della logica si compongono di grammatica, significato e regole, identificati rispettivamente in:
\begin{itemize}
    \item \textbf{Sintassi}: Definisce la corretta scrittura delle formule logiche.
    \item \textbf{Semantica}: Definisce il significato delle formule logiche.
    \item \textbf{Sistemi Deduttivi}: Strumenti sintattici utili a manipolare formule e costruire dimostrazioni, dette derivazioni.
\end{itemize}
Abbiamo quindi un insieme di parole da assemblare (sintassi) che con diverse combinazioni possono creare sentenze dai diversi significati (semantica) ed il tutto segue un determinato insieme di regole (sistemi deduttivi). Tutto ciò consente di creare teoremi, anch'essi composti da più parti:
\begin{itemize}
    \item \textbf{Enunciato}; Ciò che il teorema vuole esprimere.
    \item \textbf{Dimostrazione}; La prova formalizzata di quanto espresso.
\end{itemize}
Sarà nostro compito dimostrare gli enunciati in base alle richieste degli esercizi. Servirà specificare una procedura che permetta di ottenere solo risultati veri o falsi in linguaggio logico.

%

\section{Connettivi e Quantificatori}
Qui sono elencati tutti i connettivi e i quantificatori utilizzati nel corso, con lo scopo di familiarizzare con loro a prescindere da quando verranno o meno utilizzati.
\begin{itemize}
    \item \textbf{Connettivi}
    \begin{itemize}
        \item \textbf{Congiunzione} $\land$:\par
        Ritorna vero solo se tutti gli elementi sono veri.
        \item \textbf{Disgiunzione} $\lor$:\par
        Ritorna vero se almeno un elemento è vero.
        \item \textbf{Negazione} $\neg$:\par
        Rende falso il vero e viceversa.
        \item \textbf{Implicazione} $\implies$:\par
        Corrisponde a "Se, allora", ritorna vero nei casi $0 \to 1$ oppure $1 \to 1$, mentre è falso se $1 \to 0$ oppure $0 \to 0$.
        \item \textbf{Doppia Implicazione} $\iff$:\par
        Corrisponde a "se e solo se, allora" e si rappresenta tramite due implicazioni: $(\phi \to \psi) \land (\psi \to \phi)$.
        \item \textbf{Bottom} $\bot$:\par
        Indica il valore di assurdo, $0$.
    \end{itemize}
    \item \textbf{Quantificatori}
    \begin{itemize}
        \item \textbf{Esiste} $\exists$:\par
        Indica l'esistenza di un elemento con una determinata proprietà.
        \item \textbf{Per ogni} $\forall$:\par
        indica che per ogni caso considerato, esiste un elemento con una data proprietà.
    \end{itemize}
\end{itemize}

%

\section{Strumenti di lavoro}
Introduciamo in linguaggio naturale il funzionamento di tutti i nostri strumenti e strategie che utilizzeremo per la dimostrazione dei teoremi:
\begin{itemize}
    \item \textbf{Sentenze}: Combinazioni dei singoli elementi dell'alfabeto. Equivalgono alle proposizioni in linguaggio naturale e possono assumere solamente due valori: vero o falso. Ne esistono di due tipi:
    \begin{itemize}
        \item \textbf{Minimale}; Quando non si può scomporre ulteriormente.
        \item \textbf{Composta}; Quando è composta da più sentenze minimali, ed è quindi scomponibile.
    \end{itemize}
    \item \textbf{Induzione Strutturale}: Il principio di induzione è utilizzato per dimostrare la verità dell'enunciato a partire da un passo base semplice, per poi estendere l'ipotesi a ogni istanza successiva. Se l'ipotesi è provata, hai dimostrato l'enunciato.
    \item \textbf{Ricorsione Primitiva}: La ricorsione consiste nella formulazione di funzioni in termini di loro stesse all'interno della loro definizione. Le funzioni di una formula comprendono ogni caso possibile e riportano ad un passo base per terminare il processo.
    \item \textbf{Deduzione Naturale}: Processo di dimostrazione di una data formula a partire da delle ipotesi. Si tratta di un algoritmo basato puramente sulla correttezza sintattica.
    \item \textbf{Tabella di Verità}: Tabella che mostra ogni singolo caso presentabile per un determinato enunciato.
    \item \textbf{Valutazione di Verità}: Nella semantica, il processo di verifica della veridicità di una formula, esaminando il risultato di ogni singola formula presente nell'enunciato.
    \item \textbf{Sostituzione}: La sostituzione con una formula$\psi$ tutte le occorrenze di un simbolo proposizionale o del primo ordine in un'altra formula $\phi$.
    \item \textbf{Modello}: Definizione in termini logico-matematici di una data formula o funzione.
    \item \textbf{Contromodello}: Scrittura in termini logico-matematici che prova la fallacia di una data ipotesi iniziale.
\end{itemize}

%
%Se serve, questo è per inquadrare i teoremi. \begin{theorem}
%    Here goes a theorem.   
%\end{theorem}

%E questo è per la prova conseguente.\begin{proof}
%        Here goes the proof
%\end{proof}

%\begin{corollary}
%    Here goes a collorary
%\end{corollary}

%\begin{eg}
%    Here goes an example
%\end{eg}

%\begin{note}
%    Here goes a note 
%\end{note}

%\begin{lemma}
%    Here goes a lemma
%\end{lemma}

%\begin{prop}
%    Here goes a proposition
%\end{prop}

%\begin{definition}
%    Here goes a definition 
%\end{definition}