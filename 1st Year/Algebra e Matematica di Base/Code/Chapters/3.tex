\section{Principio di induzione sui naturali}
L'insieme dei numeri naturali $\mathbb{N}$ è il più importante di tutta la matematica e la sua dinamica si basa su due principi:
\begin{itemize}
    \item Un numero dato, spesso $0$, dal quale partire.
    \item Una funzione successore $Succ(n)$, che permette di ottenere il numero conseguente.
\end{itemize}
Conosci già gli elementi di $\mathbb{N}$; si tratta di un intervallo che comprende i numeri da $[0, +\infty)$, ne consegue che è possibile dimostrare una proprietà $\phi(n)$ per tutti i numeri naturali usando l'induzione, che in questo caso chiameremo $IND_{\mathbb{N}}$. Il suo funzionamento non differisce da una classica induzione:
\begin{enumerate}
    \item Parti da un caso base $\phi(0)$ e provalo vero.
    \item Supponi un $n \in \mathbb{N}$ e prova l'ipotesi $\phi(n+1)$.
    \item Concludi che $\forall n \in \mathbb{N}$ vale $\phi(n)$.
\end{enumerate}
Per i nostri scopi useremo inoltre i seguenti assiomi, i quali renderanno più semplice la risoluzione degli esercizi:
\begin{prop}
    \textbf{Assiomi di Peano}\par
    \begin{itemize}
    \item $0 \notin \mathbb{N}$, quindi la funzione $Succ(n)$ non è suriettiva.
    \item $Succ(m) = Succ(m) \implies m = n \forall m,n \in \mathbb{N}$, quindi la funzione $Succ(n)$ è iniettiva.
    \item Sia $\phi(n)$ una formula sui numeri naturali, allora vale:\par
    $[\phi(0) \land \forall n \in \mathbb{N}.(\phi(n) \implies \phi(Succ(n))] \implies \forall n \in \mathbb{N}.(\phi(n))$
\end{itemize}
\end{prop}

\begin{eg}
    \textbf{Dimostrazione con $IND_{\mathbb{N}}$}\par
    INSERISCI ESEMPIO con $\forall n \in \mathbb{N}.[(1+2024)^n \geq 1+n\times 2024]$.
\end{eg}

Considera che è possibile operare anche in un insieme dei naturali dove non fa parte lo zero, formalmente definito come $\mathbb{N^*} = \mathbb{N} / \{0\}$. Il procedimento per le dimostrazioni non cambia, semplicemente il passo base sarà con il numero $1$. Abbiamo inoltre un algoritmo equivalente a quanto visto denominato $IND_<$.

\begin{eg}
    \textbf{Dimostrazione con $IND_{\mathbb{N}^*}$}\par
    Definiamo l'insieme numerico di lavoro come $\mathbb{N}^* = \mathbb{N} \setminus \{0\} = \{1, 2, ..., n\}$\par
    Tesi da provare: $\theta(n) = \forall n \in \mathbb{N}^*.(1+2+...+n=\dfrac{n(n+1)}{2})$
    \begin{itemize}
        \item \textbf{Passo base}:\par
        Testiamo se la tesi vale sostituendo $n$ a $1$\par
        \begin{center}
            $\theta(1) \iff 1 = \dfrac{1(1+1)}{2} = 1$, che è vera.
        \end{center}
        \item \textbf{Passo induttivo}:\par
        Espandiamo il ragionamento per $\theta(n+1)$. Va sostituito $(n+1)$ alla singola $n$ presente nella tesi iniziale.
        \begin{center}
            $\theta(n+1) \iff (1+2+...+n+(n+1)) = \dfrac{(n+1)((n+1)+1)}{2} = \dfrac{(n+1)(n+2)}{2}$
        \end{center}
        Adesso proviamo che il risultato ottenuto è valido:
        \begin{center}
            $\dfrac{n(n+1)}{2}+(n+1) = \dfrac{n(n+1) + 2(n+1)}{2} = \dfrac{n^2+3n+2}{2} = \dfrac{(n+1)(n+2)}{2}$
        \end{center}
    \end{itemize}
    Come volevasi dimostrare.
\end{eg}

\begin{eg}
    \textbf{Dimostrazione con $IND_<$}\par
    INSERISCI ESEMPIO.
\end{eg}

%

\section{Principali operazioni ed elementi}
Dove è sempre possibile utilizzare l'induzione per la dimostrazione dei teoremi, risulta particolarmente comodo provare determinate relazioni e operazioni per \textbf{Ricorsione}. Anch'essa è composta di due casi:
\begin{enumerate}
    \item Caso base, da dove inizia la dimostrazione.
    \item Caso ricorsivo, il quale avanza tenendo conto dei valori precedentemente ottenuti.
\end{enumerate}
L'esempio più semplice, spesso usato anche nella programmazione, è la formalizzazione del concetto di fattoriale.
\begin{eg}
    \textbf{Dimostrazione con ricorsione}\par
    \begin{itemize}
        \item Passo base\par
        Se $n = 0 \implies 1$
        \item Passo ricorsivo\par
        Se $n > 0 \implies (n-1)! \times n$ 
    \end{itemize}
    Proviamo a ragionare come si comporta tale funzione quando sostituiamo alla $n$ i valori presi in esame. Otterremo che:
    \begin{center}
        $0! = (0-1)! \times1 = 1$\par
        $1! = (1-1)!\times1 = 1\times1 = 1$\par
        $2! = (2-1)!\times2 = 1\times2 = 2$\par
        $3! = (3-2)!\times3 = 2\times3 = 6$\par
        $4! = (4-3)!\times4 = 6\times4 = 24$\par
    \end{center}
\end{eg}
Adesso è ora di distruggere tutto ciò che è stato insegnato alle scuole elementari sulla matematica; definiamo tutte le operazioni elementari:
\begin{itemize}
    \item \textbf{Funzione Successore}
    \item \textbf{Addizione}
        \begin{itemize}
            \item \textbf{Proprietà commutativa}
            \item \textbf{Proprietà associativa}
        \end{itemize}
    \item \textbf{Ordinamento, concetto di minimo}\par
    Partiamo dal presupposto che per parlare di ordinamento abbiamo bisogno di definire cosa rende un numero maggiore o minore di un altro. Assorbito tal concetto, diciamo che ogni sottoinsieme non vuoto di $\mathbb{N}$ ha necessariamente un elemento minimo, formalmente $\exists n \in N$, dove $n \in A, \forall \alpha \in A.(n <= \alpha)$.
    \begin{proof}
        \textbf{Concetto di minimo $MIN$}\par
        Iniziamo supponendo un insieme $A$ non vuoto ed un elemento $n \in A$. Diciamo che:
        \begin{itemize}
            \item $0 \in A$, se non lo è, passa al numero successivo, altrimenti hai trovato il minimo.
            \item $1 \in A$, se non lo è, passa ancora al successivo, altrimenti hai trovato il minimo. Ripeti il processo fin quando non trovi il numero.
        \end{itemize}
    \end{proof}
    Tuttavia, attenzione: il concetto di minimo non vale per i sovrainsiemi numerici di $\mathbb{N}$, a meno che non venga preso un insieme proprio. Puoi inoltre usare lo stesso ragionamento per trovare il valore massimo in un insieme i cui estremi sono definiti.\par\quad
    Un'altra particolarità del principio di minimo è che implica la validità del principio del terzo escluso nei numeri naturali:
    \begin{theorem}
        $MIN \implies PEM_P$\par 
    \end{theorem}
    \begin{proof}
        \textbf{Ricorsione}\par
        Sia una formula $P$, dimostriamo che vale il principio del terzo escluso, ovvero $(P \lor \neg P)$.\par
        \begin{center}
            $A_P = \{1\} \cup \{x \in \mathbb{N} | x = 0 \land P\}$\par
            $1 \in A_P$, quindi $A_P$ non è vuoto.
        \end{center}
        Non essendo un insieme vuoto, avrà per forza un minimo. CVD.
    \end{proof}
    Possiamo ottenere uno stesso risultato anche induttivamente, effettuando il seguente ragionamento:
    \begin{proof}
        $IND_\mathbb{N}$\par
        Sia ora $n = min(A_P)$, abbiamo che $[n = 0 \lor \exists m \in \mathbb{N}.(n = succ(m))]$.\par
        Chiamiamo il minimo $P$ e dimostriamo che vale $(P \lor \neg P)$:
        \begin{center}
            $0 \in A_P \implies P$\par
            Se il minimo è 1, P non è valida, di conseguenza vale $\neg P$, quindi:\par
             $P_A.(0=0)\implies (0\in A_P) \implies 0 = minA_P$.\par
             Qui hai trovato l'assurdo ricavando che il minimo è $0$.
        \end{center}
        CVD, vale $(P \lor \neg P)$.
    \end{proof}
    \item \textbf{Moltiplicazione}\par
    La moltiplicazione è la seconda operazione elementare che consente di esprimere ogni singolo numero naturale. Viene definita ricorsivamente come segue:
    \begin{definition}
        \textbf{Moltiplicazione} - $m,n \in N$\par
        \begin{itemize}
            \item $m\times0 = 0$
            \item $m\times succ(n) = m\times n+m$
        \end{itemize} 
    \end{definition}
    Vediamo un esempio pratico con la funzione successivo:
    \begin{eg}
        \textbf{Esercizio su moltiplicazione}\par
        AGGIUNGI SPIEGAZIONI - RICOMINCIA DA QUA
        \begin{center}
            $2\times 3 = 2\times succ(2) =$\par
            $2\times 2 + 2 = 2\times succ(1) + 2 =$\par
            $2\times 1 + 2 + 2 = 2\times succ(0) + 2 + 2 =$\par
            $2\times 0 + 2 + 2 + 2 = 0 + 2 + 2 + 2 = 6$.
        \end{center}
    \end{eg}
        \begin{itemize}
            \item \textbf{Proprietà distributiva}
        \end{itemize}
    \item \textbf{Elevamento a potenza}
    \item \textbf{Equivalenza}
\end{itemize}

%

\section{Costruzione di interi e razionali}
\section{Fattorizzazione e teorema fondamentale dell'aritmetica}
\section{Congruenze}
\section{Domande di teoria}
\section{Esercizi}

\section{Appunti}


\begin{eg}
    $1 \times 1 = 1$\par
    $1 * 1 = 1 * succ(0) = 1 * 0 + 1 = 0 + 1 = 1$
\end{eg}

\begin{eg}
    $m * 1 = m$\par
    $m * 1 = m * succ(0) = m * 0 + m = 0 + m = m$
\end{eg}

\begin{eg}
    $1 * m = m$\par
    $\forall m \in N . (1*m = m)$\par
    $\phi 0 = 1 * 0 = 0$\par
    $\phi(m) \implies \phi(succ(m)) \iff 1 * succ(m) = succ(m)$\par
    $1*succ(m) = 1*m + 1 = m+1 = succ(m)$\par
    $\phi(m) \iff \ * m = m$.
\end{eg}

\begin{eg}
    $0 * m = 0$\par
    $\forall m \in N(0*m = 0)$\par
    $\phi(0) \iff 0*0 = 0$ Per definizione\par
    $\phi(succ(m)) \iff 0 * succ(m) = 0$\par
    $0 * succ(m) = 0 * m + 0 = 0+0 = 0$
\end{eg}

La moltiplicazione è distributiva rispetto all'addizione.
\begin{definition}
    Proprietà distributiva\par
    \begin{enumerate}
        \item $m(l+n) = m*l+m*n$
        \item $m(l*n) = (m*l)n$
        \item $m*n = n*m$
        \item $m*n = 0 \implies m = 0 \lor n = 0$
    \end{enumerate}
\end{definition}

\begin{proof}
    $\forall m \in N . [\forall n \in N.(m*n = 0 \implies m= 0 \lor n = 0]$\par
    $\phi(0) \iff \forall n \in N . (0*n = 0 \implies 0=0 \lor n = 0$ Vale banalmente per identità\par
    $\phi(n) \implies \phi(succ(n)), \phi(succ(n)) \iff \forall n \in N . (succ(n)*n = 0 \implies succ(m) = 0 \lor  n = 0$\newline

    $\theta(n) \iff succ(m)*n = 0 \implies succ(m) = 0 \lor n = 0)$\par
    $\theta(0) \iff succ(m) * 0 = 0 \implies succ(m) = 0 \lor 0=0$ vale banalmente per identità.\par
    $\theta(n) \implies \theta(succ(n)), \theta(succ(n)) \iff succ(n = 0 \implies succ(m = 0 \lor succ(n) = 0$\par
    $succ(m) * succ(n) = 0 \iff succ(m)*n + succ(m) = 0 \implies succ(m)*n = 0 \land succ(m) = 0$ Questa è falsa per peano 1. Il successore di 0 è 1.\par
    La formula è dimostrata per il funzionamento del connettivo implica 0->1.
\end{proof}

\begin{eg}
    Dimostrare $m < n$, tenendo conto della definizione ricorsiva della moltiplicazione.\newline

    $m < n \implies m*l < n*l$, dove $l >= 1$\par
    $m < n \iff \exists k \in N^*(n = m+k), k >= 1$\newline

    Per ricorsione:\par
    $n = m+k, k <= 1$\par
    $m*l < n*l$\par
    $n*l = (m+k)l = m*l + k*l, k*l >= 1$\par
    $k >= 1, l >= 1, k*l >= 1$\par
    $k = 1 + \sigma$, $l = 1 + \sigma'$\par
    $k*l = (1+\sigma)(1+\sigma') = 1+\sigma' + \sigma + \sigma * \sigma' >= 1$\newline

    Per induzione:\par
    $\forall l \in N^*[(m*l < n*l)]$, $m,n . m < n$, $\phi(l) = []$\par
    $\phi(1) \iff m*l < n*l \iff m < n$ vale\par
    $\phi(l) \implies \phi(succ(l))$\par
    $\phi(succ(l)) \iff m * succ(l) < n * succ(l)$\par
    $m * succ(l) = m * l + m$\par
    $n * succ(l) = n*l +n$\par
    $m*l < n*l = m < n$
\end{eg}

\begin{eg}
    Dimostrare che $m <= m*n, n >= 1$\par
    $m <= n \iff \exists k \in N . (n = m+k), k >= 0$\newline
    
    Per m fissato:\par
    $\forall n \in N^*.[(m <= m*n)], \theta(n) = []$\par
    $\theta(1) \iff m <= m*1 = m$ vale\par
    $\theta(n) \implies \theta(succ(n)) \iff m <= m *succ(n) \iff m <= m(n+1) = m*n+m$\par
    $\theta(n) \iff m <= m*n <= m*n + n = n*succ(m)$\par
\end{eg}

Per ogni numero naturale definiamo:
\begin{definition} Potenza\par
    \begin{itemize}
        \item $m^0 = 1$
        \item $m^{succ(n)} := m^n * m$
    \end{itemize}
    $2^3 = 2^{succ(2)} = 2^2 * 2 = 2^{succ(1)} * 2 = 2^1 * 2 * 2 = 2^{succ(0)}*2*2 = 2^0*2*2*2 = 1*2*2*2 = 8$.
\end{definition}

- Costruzione dei numeri interi\par
Come definire gli interi? $\mathbb{N} \cup \{0\} \cup \mathbb{N}$\par
$N^* tilde N_*, N^* \cap N_* = void$.\par

\begin{eg}
    Proviamo a rappresentare -1 con i numeri naturali. Ne serviranno due:
    \begin{center}
        $-1 = 0-1$\par
        $= 1-2$ e così via.
    \end{center}
    Puoi provare a farlo anche con 0, 1, 2 e altri. Questo per dire che tutti i numeri si possono rappresentare mediante una relazione di sottrazione. Ci consente di rendere due coppie scritte in modo diverso uguali grazie alla definizione di funzione.
\end{eg}

Supponiamo le coppie $(m,n) tilde (m',n')$, le quali devono essere appunto equivalenti, avremo di conseguenza che $m-n = m'-n'$. Possiamo spostare i termini coi criteri di equivalenza, varrà quindi: $m + n' = m' + n$.\par
\begin{prop}
    \textbf{tilde è un'equivalenza su NxN}.
\end{prop}

\begin{proof}
    Equivalenza\par
    $(m,n) tilde (m,n) \iff m+n = m+n$. La relazione è riflessiva.\par
    $(m,n) tilde (m',n') \implies (m',n') tilde (m,n) \implies m+n' = m'+n \implies m'+n = m+n'$. La relazione è simmetrica.\par
    $ (m,n) tilde (m',n') \iff m+n' = m'+n ! (m',n')tilde(m'',n'') \iff m'+n'' = m''+n'$\par
    $(m,n) tilde (m'',n'') \iff m+n'' = m'' + n$.\par
    $B = m'+n'' = m''+n' \implies m'+n'' + n = m'' +n' +n$\par
    Quanto ottenuto fa ottenere $A = m+n'+n'' = m'' +n'+n \iff m+n'' = m''+n$. La relazione è quindi transitiva.\par
\end{proof}

\begin{definition}
    $\mathbb{Z}:= \mathbb{N}\times \mathbb{N}/tilde, (m,n) tilde(m',n') \iff m+n' = m'+n$\par
    $Z = \{[(m,n)]_{tilde} | (m,n) \in NxN$\par
    $[(m,n)]_{\sim}:= \{(m',n') \in NxN | (m',n') tilde (m,n)\}$\par
    $0_Z .= [(0,0)]_{tilde} = [(1,1)]_{tilde} \iff (0,0) tilde (1,1)$\par
    $1_Z := [(1,0)]_{tilde} = [(2,1)]_{tilde}, (1,0) tilde (2,1)$
\end{definition}

\begin{prop}
    Definiamo $i: N -> Z$ definito da $n|->i(n) := [(n,0)]_{tilde}$\par
    La funzione è iniettiva.\par
    $i(n) = i(m) \implies n = m, \forall n,m \in N$\par
    $i(n) = i(m) \iff [(n,0)]_{tilde} = [(m,0)]_{\sim} \iff (n,0) \sim (m,0) \iff n+0 = m+0 \iff n = m$.
\end{prop}




