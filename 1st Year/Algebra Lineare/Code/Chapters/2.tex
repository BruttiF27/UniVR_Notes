\section{Equazioni, sistemi e operazioni}
Risoluzione di sistemi lineari, inoltre
\section{Metodo di Eliminazione di Gauss}
\section{Rango e Teorema di Roché-Capelli}
\section{Esercizi}
\begin{enumerate}
    \item \textbf{Che cos'è un sistema lineare e come si risolve?}
    
    \item \textbf{Cosa sono i vettori e che operazioni è possibile effettuare con loro? Come puoi scriverci la soluzione di un sistema?}
    \item \textbf{In che modo un sistema lineare di equazioni si può scrivere in una matrice? Che operazioni hanno luogo fra matrici e vettori?}
    \item \textbf{In che modo si trovano le soluzioni di una matrice?}
    \item \textbf{Che cos'è il rango e come si utilizza nel teorema di Rouché-Capelli?}
    \item \textbf{In che cosa consiste l'algoritmo di eliminazione di Gauss?}
\end{enumerate}

\section{Argomenti}
2.1 Sistemi lineari (esempio, definizione, sistema lineare omogeneo, soluzione)
2.1 Matrici (definizione, coefficienti/entrate, Mm×n(C), Mm×n(R))
2.1 Forma matriciale (matrice dei coefficienti, vettore delle incognite, vettore dei termini noti, matrice aumentata)
2.1 Operazioni elementari: Scambiare righe, moltiplicare una riga per uno scalare non nullo, sommare una riga con un multiplo di un altra riga.
2.1 Linee in R2: 1, 0 o ∞ soluzione.
2.2 Metodo dei eliminazione di Gauss (EG). Definizioni di pivot, forma ridotta e colonne dominanti.
2.2 Risoluzione di un sistema lineare. Definizioni di variabile dominante e variabile libera. Ogni sistema ha 1, 0 o ∞ soluzione.
2.3 Definizione rango.
2.3 Teorema di Roche-Capelli