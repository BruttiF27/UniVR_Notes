Onestamente non ho la benché minima idea di cosa tratti matematica di base; tutti gli argomenti sembrano familiari ma allo stesso tempo estranei. Inoltre sembra una materia di cui si sente la mancanza nell'ordinamento precedente. Iniziamo con la definizione formale di \textbf{Insieme}, elemento della teoria su cui si basa la matematica tutta:
\begin{definition}
    \textbf{Insieme}\par
    Gruppo di elementi aventi una stessa proprietà. Si indica con una lettera maiuscola.
\end{definition}
Pare ovvio che con questi insiemi sia possibile operare in qualche modo; introduciamo dunque tutte le varie possibilità insieme alle loro proprietà.

\section{Operazioni fra gli insiemi}
\begin{itemize}
    \item \textbf{Appartenenza, contenimento e sottoinsieme}
    \item \textbf{Unione}
    \item \textbf{Intersezione}
    \item \textbf{Differenza}
    \item \textbf{Insieme Potenza}
    \item \textbf{Prodotto cartesiano}
    \item \textbf{Insieme delle parti}
    \item \textbf{Complementare di un insieme}
    \item \textbf{Generalizzazione di operazioni}
\end{itemize}

\subsection{Leggi di De Morgan}
\section{Relazioni fra insiemi}
\section{Principi di dimostrazione}

\section{Domande di teoria}

\begin{theorem}
    Here goes a theorem.
\end{theorem}

\begin{proof}
        Here goes the proof
\end{proof}

\begin{corollary}
    Here goes a collorary
\end{corollary}

\begin{eg}
    Here goes an example
\end{eg}

\begin{note}
    Here goes a note 
\end{note}

\begin{lemma}
    Here goes a lemma
\end{lemma}

\begin{prop}
    Here goes a proposition
\end{prop}

\begin{definition}
    Here goes a definition 
\end{definition}

\subsection{Esercizi}