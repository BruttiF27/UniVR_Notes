\section{Grafici logaritmici}
Abbiamo notato come ogni segnale può essere rappresentato come una somma di sinusoidi, e che ai sistemi LTI bibostabili, quando dato in input sinusoidale, ritorneranno un output della stessa forma e frequenza, ma con ampiezza e fase diverse.\par
Quella di Bode è una rappresentazione grafica eseguita mediante \textbf{grafici logaritmici} che sfrutta proprio questa caratteristica, mostrando ampiezza e fase della funzione di trasferimento del sistema al variare della frequenza.
\begin{figure}[h]
	\centering
	\includegraphics[width=0.9\linewidth]{Images/graficoLogaritmico_example.png}
	\caption{Grafico logaritmico}
\end{figure}

\noindent Più precisamente, per i \textbf{diagrammi di Bode} vengono utilizzati due grafici semi-logaritmici, i quali presentano le pulsazioni $\omega[rad/s]$ sull'asse delle $x$, mentre sulle $y$ descrivono le ampiezze $A[dB = 20log_{10}(|A|)]$ o le fasi $\phi[deg]$, in due grafici diversi.\par
Essendo i grafici in una scala logaritmica sull'asse delle $x$, avremo intervalli dati dalle potenze di $10$, chiamati \textbf{decadi}; i valori coperti sono di frequenza arbitrariamente piccola o grande, ma non include mai lo zero. Gli intervalli di decadi sono graficamente definiti dallo spazio di $10$ quadrati del grafico.\par
Stiamo ragionando in termini di risposta in frequenza, quindi partiamo da un sistema che divide l'input in due parti per andare in due blocchi diversi. Poi somma questi due output per ritornare l'uscita. Supponiamo che:
\begin{itemize}
	\item $u(t) = \sin(\frac{1}{2}t)$
	\item $v(t) = 2.83\degree\cdot \sin(\frac{1}{2}t - 0.785\degree)$
	\item $\omega = \frac{1}{2}$
\end{itemize}
\noindent Ora applicheremo la trasformata di Laplace passando al dominio dei complessi: \[\frac{V(s)}{U(s)} = \frac{2s + 1}{s} \implies \frac{2j\omega + 1}{j\omega} = 2 - \frac{1}{\omega}j\]
\noindent Ottenuto il comportamento in frequenza, sappiamo che la parte reale è $2$ e l'immaginaria è $-\frac{1}{\omega}j$, quindi prendiamo il piano cartesiano con la parte reale (x) e l'immaginaria (y), andando a scrivere i puntini al variare della frequenza $\omega$. A partire da questo piano possiamo calcolare ampiezza $l$ e fase $\theta$ della funzione, dove:
\begin{itemize}
	\item $l = \sqrt{Re^2 + Im^2} = |\sigma + j\omega| = |\sigma|$
	\item $\theta = Arg(\sigma + j\omega) = atan2(\sigma, j\omega)$
\end{itemize}
\noindent La funzione $atan2$ e quindi la funzione \textbf{argomento} $Arg$ è definita generalmente come: \[Arg(\sigma+j\omega) = \begin{cases}
	\sigma > 0 \implies arctan(\frac{\omega}{\sigma})\\
	\sigma = 0, \omega>0\implies \frac{\pi}{2}\\
	\sigma = 0, \omega<0\implies -\frac{\pi}{2}\\
	\sigma < 0, \omega\geq 0 \implies arctan(\frac{\omega}{\sigma})+\pi\\
	\sigma < 0, \omega<0 \implies arctan(\frac{\omega}{\sigma})-\pi
\end{cases}\]
\noindent Tuttavia, usando la funzione segno, otteniamo una forma di Arg più compatta, definita come: \[sgn(\alpha) = \begin{cases}
	\alpha > 0 \implies 1\\
	0
\end{cases} \implies Arg(\sigma + j\omega) = \begin{cases}
	\sigma > 0 \implies arctan(\frac{\omega}{\sigma})\\
	\sigma = 0, \omega\neq 0 \implies sgn(\omega \cdot \frac{\pi}{2})\\
	\sigma < 0 \implies arctan(\frac{\omega}{\sigma}) + sgn(\omega) \cdot \pi
\end{cases}\]
\noindent Dove nell'ultimo caso specifichiamo che $sgn(\omega) = 1$ se $\omega = 0$. Ci è poi particolarmente utile andare a verificare il comportamento della funzione ai suoi estremi, quindi esaminiamo cosa succede con valori minimi e massimi.\par
Nel caso preso in esame, variando la frequenza e quindi portando il punto all'estremità in basso e in alto, avremo gradi minimi di -90° e 0°. Questi sono i confini per i gradi nel diagramma di Bode, mentre per quanto riguarda le ampiezze, minore è la frequenza, più grande sarà il vettore. Questo metodo, seppur funzionale, tuttavia, risulta essere estremamente lento, perché bisogna verificare tutte le frequenze; ripetere questo processo ogni volta è atroce.\par
La soluzione più agevole vede la trasformazione della risposta nella \textbf{forma di Bode}, che discuteremo nella prossima sezione; ma prima è necessario chiarire alcuni presupposti sui numeri complessi per lavorarci:
\begin{itemize}
	\item \textbf{Moltiplicazione}: $|a\cdot b| = |a|\cdot |b|$, $Arg(a\cdot b) = Arg(a) + Arg(b)$
	\item \textbf{Divisione}: $|\frac{a}{b}| = \frac{|a|}{|b|}$, $Arg(\frac{a}{b}) = Arg(a) - Arg(b)$
	\item \textbf{Esponenziali}: $|a^\mu| = |a|^\mu$, $Arg(a^\mu) = \mu\cdot Arg(a)$
	\item \textbf{Logaritmi}: Leggermente più complicato. Bisogna ricordare \[\log s = x \iff s = e^x; s = \rho e^{j\theta}, x = \sigma + j\omega\]
	\noindent Ne deduciamo quindi che: \[s = e^x \implies \rho e^{j\theta} = e^{\sigma + j\omega} \implies e^{\log\rho}e^{j\theta} = e^\sigma e^{j\omega}\]
	\noindent Arriviamo infine a $\log\rho = \sigma$, $\theta = \omega$. Dunque: \[\log s = x \implies \log\rho + j\omega \implies \log|s| + j\cdot Arg(s)\]
\end{itemize}

%

\section{Forma di Bode}
Chiariti i presupposti delle operazioni nel campo dei complessi, introduciamo la \textbf{Forma di Bode}. Si tratta di una scrittura che ha lo scopo di arrivare agli omonimi diagrammi senza dover sostituire il valore della frequenza per ogni singolo punto.\par
Partiamo dalla funzione di trasferimento in laplace $H(s)$, in una forma dove sono espansi gli zeri e i poli. Ridefiniamo la costante $K$ come $K_b$, dove porteremo ogni singola costante dei fattori, facendoci ottenere:
\[H(s) = K\frac{(s-z_1)(s-z_2) \cdot\cdot\cdot (s-z_m)}{(s-p_1)(s-p_2) \cdot\cdot\cdot (s-p_n)} = K_b\frac{\prod_i(1+s\tau_i')^{\mu_i'} \prod_k\left(1+2\zeta_k'\cdot \frac{s}{\omega_{n,k}'} + \frac{s^2}{(\omega_{n,k}')^2}\right)^{\mu_k'}}{(s^\nu)\prod_i(1+s\tau_i)^{\mu_i}\prod_k\left(1+2\zeta_k\cdot\frac{s}{\omega_{n,k}} + \frac{s^2}{(\omega_{n,k})^2}\right)^{\mu_k}}\]
\noindent Questa forma si può sempre ottenere con i sistemi LTI e presenta le seguenti componenti:
\begin{itemize}
	\item $K_b$: Detto \textbf{guadagno di Bode}, è il termine costante che racchiuderà tutte le altre costanti prese dai fattori.
	\item $s^\nu$: Raggruppa tutte le radici nulle, per esempio $\frac{s^5}{s^4} = s^1$
	\item $(1+s\tau_i)^{\mu_i}$: Raggruppa una radice reale, come $(s+1)$
	\item $\left(1+2\zeta_k \frac{s}{\omega_{n,k}} + \frac{s^2}{(\omega_{n,k})^2}\right)^{\mu_k}$: Raggruppa una radice complessa coniugata, ovvero quel trinomio che non si può scomporre.
\end{itemize}
\noindent Vediamo come è possibile arrivare alla forma di Bode partendo dalla funzione di trasferimento in frequenza $H(s)$. Supponiamo di avere la seguente: \[H(s) = \frac{s^3 - 2s^2 - 8s}{s^4 - 2s^3 + 2s^2}\]
\noindent Come primo passaggio è necessario \textbf{raccogliere} ed eventualmente semplificare le $s$ presenti nella funzione. Qui otteniamo: \[H(s) = \frac{s^3 - 2s^2 - 8s}{s^4 - 2s^3 + 2s^2} = \frac{s(s^2 - 2s - 8)}{s^2(s^2 - 2s + 2)} = \frac{(s-4)(s+2)}{s(s - (1-j))(s - (1+j))}\]
\noindent Abbiamo ottenuto al numeratore due zeri reali, mentre al denominatore abbiamo un polo nullo e una coppia di poli complessi e coniugati. Possiamo procedere con la prossima manipolazione che è il \textbf{raccoglimento delle costanti}.\par
In questo caso, dobbiamo lavorare esclusivamente sugli zeri, poiché sono gli unici fattori che presentano una costante diversa da $1$: \[\frac{(s-4)(s+2)}{s(s - (1-j))(s - (1+j))} = \frac{-4\left(1+\frac{s}{-4}\right)\cdot 2\left(1 + \frac{s}{2}\right)}{s(s - (1-j))(s - (1+j))} = -8\frac{\left(1+\frac{s}{-4}\right)\left(1 + \frac{s}{2}\right)}{s(s - (1-j))(s - (1+j))}\]
\noindent Per ottenere una forma più compatta possiamo dire inoltre che $\tau_1'=-\frac{1}{4}$ e $\tau_2'=\frac{1}{2}$: \[H(s) = -8\frac{(1 + s\tau_1')(1 + s\tau_2')}{s(s - (1-j))(s - (1+j))}\]
\noindent Adesso bisogna manipolare la coppia di poli complessi e coniugati. Sì, l'abbiamo scomposto prima per sicurezza, ma è più semplice raccogliere la costante quando sviluppato. Otteniamo: \[s(s - (1-j))(s - (1+j)) = s^2 -2s +2 = 2\left(1 - s + \frac{s^2}{2}\right)\]
\noindent Vedendo la forma capiamo dall'ultimo termine che $\omega_n^2 = 2 \implies \omega_n = \sqrt{2}$. È evidente poi che $\frac{2\zeta}{\omega_n} = -1 \implies \zeta = -\frac{1}{\sqrt{2}}$ dal secondo termine. Dunque, riscrivendo quanto ottenuto nella forma richiesta: \[\left(1+2\zeta_k \frac{s}{\omega_{n,k}} + \frac{s^2}{(\omega_{n,k})^2}\right)^{\mu_k} \implies 2\left(1 + \left(-\frac{1}{\sqrt{2}}\right)\frac{s}{\sqrt{2}} + \frac{s^2}{(\sqrt{2})^2}\right)\]
\noindent Abbiamo finalmente ottenuto la forma di Bode della funzione di trasferimento richiesta: \[H(s) = -16\frac{\left(1+\frac{s}{-4}\right)\left(1 + \frac{s}{2}\right)}{1\left(1 + \left(-\frac{1}{\sqrt{2}}\right)\frac{s}{\sqrt{2}} + \frac{s^2}{(\sqrt{2})^2}\right)} = K_b\frac{(1 + s\tau_1')(1 + s\tau_2')}{s^\nu\left(1 + 2\zeta_1\frac{s}{\omega_n} + \frac{s^2}{\omega_n^2}\right)}\]

%

\section{Grafici della risposta in frequenza}
Come già menzionato, i diagrammi di Bode sono due, uno per l'ampiezza e l'altro per la fase. Questi possono rappresentare ogni singolo fattore nella relativa forma di Bode che si suppone ottenuta dalla funzione di trasferimento tramite passaggi algebrici. Ciò significa che esistono diagrammi per rappresentare:
\begin{itemize}
	\item \textbf{Guadagno di Bode} $K_b$.
	\item \textbf{Zeri/Poli nulli} come $s^\mu$, con molteplicità $\mu$.
	\item \textbf{Zeri/Poli reali} come $(1+\tau s)^\mu$, con $\tau$ costante di tempo.
	\item \textbf{Zeri/Poli complessi e coniugati} come $\left(1+2\zeta\frac{s}{\omega_n}+ \frac{s^2}{\omega_n^2}\right)^\mu$, dove $\zeta$ è detto \textbf{coefficiente di smorzamento} e $\omega$ la \textbf{pulsazione naturale}. 
\end{itemize}
\noindent Ognuno di loro ha una definizione particolare riguardo al disegno. Studiamoli uno per volta. \begin{itemize}
	\item \textbf{Termine costante}: Il guadagno di Bode è un numero reale tale per cui la parte immaginaria è nulla e si denota come: \[|H(j\omega)| = |K_b|\]
	\noindent Ciò significa che il modulo è in decibel e la funzione giace sull'asse dei reali. \begin{itemize}
		\item \textbf{Ampiezza}: Retta orizzontale data da: \[A(\omega) = \big|H(j\omega)\big|_{dB} = 20\log_{10}|K_b|\]
		\item \textbf{Fase}: Retta orizzontale posizionata in base a segno positivo o negativo: \[\phi(\omega) = 0\degree \lor \phi(\omega) = -180\degree\]
	\end{itemize}
	\item \textbf{Zeri/Poli nulli}: Fissato $\omega\in\mathbb{R}$, il polo/zero nullo è, a seconda della potenza $\nu\in\mathbb{Z}$, o un reale puro, positivo o negativo, o un immaginario puro, positivo o negativo. Viene annotato come: \[H(j\omega) = (j\omega)^\nu\]
	\begin{itemize}
		\item \textbf{Ampiezza}: Retta che cresce o decresce di $20dB$ ogni decade in base al valore della potenza $\nu$ a partire da dove il valore si annulla: \[A(\omega) = |H(j\omega)|_{dB} = 20log_{10}(\omega^\nu) = 20\nu log_{10}(\omega)\]
		\item \textbf{Fase}: Retta orizzontale data da: \[\phi(\omega) = Arg(j\omega)^\nu = \nu Arg(j\omega) = \nu 90\degree\]
	\end{itemize}
	\begin{figure}[h]
		\centering
		\includegraphics[width=0.60\linewidth]{Images/zeriPoliNulliGrafico.png}
		\caption{Grafico di zeri/poli nulli}
	\end{figure}
	\item \textbf{Zeri/Poli reali}: Per agevolare la rappresentazione del termine binomio è consigliato approfondire il relativo comportamento asintotico, per poi rappresentarlo realmente. Si annota come: \[H(j\omega) = (1+j\omega\tau)^\mu \equiv H(s) = (1+s\tau)^\mu\]
	\noindent Definiamo ora per studiare l'asintoticità, la \textbf{pulsazione di spezzata}: \[\omega_0 = \frac{1}{|\tau|}\]
	\begin{itemize}
		\item \textbf{Ampiezza}: Se zero reale hai $\mu>0$, la funzione cresce di $20dB/\text{decade}$, in alternativa, $\mu<0$ e la funzione decresce allo stesso valore. Descritta generalmente da: \[A(\omega) = 20\mu\log_{10}\sqrt{1+(\omega\tau)^2}\]
		\noindent E per la quale possiamo avere i seguenti valori in base al comportamento asintotico:
		% TODO Aggiungi grafici
		\[\begin{cases}
			\omega \ll \omega_0 \quad\implies\quad A(w) = 20log_{10}(1) = 0\\
			\omega = \omega_0 \quad\implies\quad A(w) = 20log_{10}(\sqrt{2}) = \pm3dB\\
			\omega \gg \omega_0 \quad\implies\quad A(w) = 20log_{10}(\frac{\omega}{\omega_0}) = \pm20dB/\text{decade}
		\end{cases}\]
		\item \textbf{Fase}: Descritta generalmente da: \[\phi(\omega) = atan2\left(-\frac{\omega}{\omega_0}\right)\]
		\noindent L'asintoticità genera i seguenti valori: \[\begin{cases}
			\omega \ll \omega_0 \quad\implies\quad \phi(w) = atan2(0) = 0\\
			\omega = \omega_0 \quad\implies\quad \phi(w) = atan2(-1) = -45\degree\\
			\omega \gg \omega_0 \quad\implies\quad \phi(w) = atan2(-\infty) = -90\degree
		\end{cases}\]
	\end{itemize}
	\item \textbf{Zeri/Poli complessi e coniugati}: Anche per quanto riguarda il termine trinomio è consigliato studiare il suo comportamento asintotico. Questi si indicano come: \[H(s) = \left(1+2\zeta\frac{s}{\omega_n}+ \frac{s^2}{\omega_n^2}\right)^\mu\]
	\noindent Dove $\omega_n$ è la pulsazione naturale, mentre $\zeta$ è il \textbf{coefficiente di smorzamento}.
	\begin{itemize}
		\item \textbf{Ampiezza}: Con uno zero, la funzione cresce di $40dB/\text{decade}$, in alternativa decresce dello stesso valore. Data la formula generale: \[A(\omega) = 20\mu\log_{10}(2\zeta)\]
		\noindent L'asintoticità fa risultare: \[\begin{cases}
			\omega \ll \omega_0 \quad\implies\quad A(\omega) = -20log_{10}(1) = 0\\
			\omega \gg \omega_0 \quad\implies\quad A(\omega) = -20log_{10}(\frac{\omega}{\omega_0})^2 = -40\log_{10}(\frac{\omega}{\omega_0})
		\end{cases}\]
		\noindent in particolare, nel caso di $\omega = \omega_0$ abbiamo: \[\begin{cases}
			\zeta = 0.5 \quad\implies\quad A(\omega) = 0\\
			\zeta > 0.5 \quad\implies\quad \text{Nessuna risonanza, grafico passa per } \omega_0\\
			\zeta < 0.5 \quad\implies\quad \text{Picco di risonanza: }\omega_r = 20\mu\log_{10}(2\zeta \sqrt{1-\zeta^2})\\
			\zeta = 0 \quad\implies\quad \text{Discontinuità}
		\end{cases}\]
		\item \textbf{Fase}: Presenta i seguenti casi: \[\begin{cases}
			\omega \ll \omega_0 \quad\implies\quad \phi(\omega) = atan2(0) = 0\degree\\
			\omega = \omega_0 \quad\implies\quad \phi(\omega) = -90\degree\\
			\omega \gg \omega_0 \quad\implies\quad \phi(\omega) = atan2(2\zeta \frac{\omega}{\omega_0}) = -180\degree
		\end{cases}\]
	\end{itemize}
	
	
\end{itemize}











\begin{comment}				
	Lez25 - Sistemi
	
	--- Dalla funzione di trasferimento al diagramma di bode.
	
	Supponiamo H(s) = \frac{s^2(s+1)(s^2+3s+16)}{(2s-1)}
	Passiamo alla forma di Bode:
	\begin{equation}
		\begin{split}
			H(s) &= \frac{s^2(s+1)(s^2+3s+16)}{(2s-1)}\\
			&= \frac{s^2(1)(1+\frac{s}{1})^1 \cdot 16(\frac{16}{16} + \frac{3}{16}s + \frac{s^2}{16})^2}{-1(1 - \frac{2}{1}s)^1}\\
			&= -16 \frac{s^2(1+s)(1+ \frac{3}{16}s + \frac{s^2}{16})}{1 - 2s}
		\end{split}
	\end{equation}
	
	Ottenuta la forma di Bode possiamo andare a prendere le singole parti viste e disegnare i rispettivi diagrammi.
	- Per K_b:
	Calcoliamo ampiezza e fase:
	A = 20log_{10}(|-16|) = 24
	\phi = \begin{cases}
		0° K_b > 0\\
		-180° K_b < 0
	\end{cases} \implies -180°, perché K_b = -16.
	
	- Per gli altri zeri:
	- (s+1)
	Calcoliamo molteplicità \mu e Termine moltiplicato per s \tau. In questo caso \mu=1, \tau=1.
	Ciò significa che frequenza di taglio \omega = \frac{1}{|\tau|} = 1
	
	Dunque prima di \omega, l'ampiezza A=0, mentre dopo è data dalla formula A=20\mulog_{10}(\omega|\tau|). In questo caso cresce di 20dB ogni decade.
	
	Le fasi hanno un comportamento molto simile. Fino a \omega, \phi = 0, mentre dopo \phi=\museg(\tau)*90° = 90°. Unendo le rette una decade prima e dopo di \omega, abbiamo il comportamento asintotico del sistema.
	
	- s^2
	\mu=2, perché è di grado 2. Nel diagramma avremo un punto in (10^0,0).
	Ampiezza A=20\mu\log(\omega) = \frac{40dB}{dec}. Quindi molto semplicemente cresce di 40dB per decade.
	Fase \phi=\mu*90° = 180°. Semplice retta posta su 180°.
	
	- Complesso coniugato (\+ 3/16s + s^2/16):
	\mu=1, poi \frac{2\zeta}{\omega_n}=3/16 e \omega_n^2=16.
	
	Dunque:
	\omega_n = \sqrt{16} = 4
	\frac{2\zeta}{\omega_n}=3/16 \implies \zeta = \frac{3}{16}\cdot\frac{\omega_n}{2} = \frac{3}{8}
	
	Per disegnare, in quanto \omega_n=4, partiamo dalla decade 10^0 e andiamo al quarto quadratino. Là sta \omega_n.
	
	Prima di \omega_n A=0, mentre dopo è data da A=20\mu\log_{10}(\frac{\omega}{\omega_n}) = \frac{40dB}{dec}. Quindi il valore cresce di 40dB a decade. Occhio che non parti dalla prima linea della divisione del grafico, ma bisogna contare sempre 10 quadratini.
	
	\omega_r = \omega_n\sqrt{1-2\zeta^2}
	M_r = 20\mu log_{10}(2\zeta\sqrt{1-\zeta^2})
	
	Le fasi hanno naturalmente un funzionamento simile. \phi=0° prima di \omega_n, mentre dopo è \phi=180\mu\sng(\zeta) = 180°. Quindi abbiamo una retta a 180°.
	Prendi i punti una decade prima e dopo di \omega_n e poi connettili per il comportamento asintotico del sistema.
	
	- Polo 1/(1-2s)
	Definitivamente un polo in quanto molteplicità negativa. Abbiamo dunque:
	\mu=-1, \tau=-2 \implies \omega_n=\frac{1}{|\tau|} = \frac{1}{2} = 0.5.
	
	Dunque l'ampiezza A=0 prima di \omega_n, mentre dopo è data da A=20\mu \log_{10}(\omega|\tau|) = -20dB/dec. Quindi cala di 20dB ad ogni decade.
	
	Per la fase, stessa roba prima di \omega_n, poi \phi=\mu*\sgn(\tau)*90° = 90°.
	Prendendo una decade prima e dopo \omega_n, uniamo i puntini e otteniamo il comportamento asintotico.
\end{comment}

%

\section{Diagramma di Bode totale}



\begin{comment}
	Lez26 - Sistemi
	
	--- Diagramma di bode totale
	Ottenuti i grafici, bisogna mostrare il comportamento totale di Bode. Fondamentalmente si sommano assieme.
	Dovremo scrivere le frequenze \omega_n trovate finora per ogni elemento. Prendiamo come riferimento l'esercizio della Lez25
	- Abbiamo una costante K_b che sta là. È sempre sulla stessa riga. Non influisce.
	- Il primo termine (1+s) ha \omega_n=10^0
	- Il secondo termine s^2 ha \omega_n=10^0, in quanto ha lo zero nullo.
	- Il terzo termine ha \omega_n=4
	- L'unico polo ha \omega_n=0.5
	
	Ora li confiniamo in una tabella dove gli estremi sono dati dalle decadi più vicine al loro valore. In questo caso:
	
	10^{-1} | 0.5 | 10^0 | 4 | 10^1
	
	Adesso calcoliamo il valore di ciascun termine nei valori trovati. Aggiungiamo gli estremi per capire meglio l'andamento.
	- Per le ampiezze: Somma degli andamenti per decade per ogni intervallo (se lo vedi graficamente è più semplice, giuro)
	Blocco antecedente a 10^{-1} = 40dB/dec
	Blocco di 10^{-1} = 40dB/dec
	Blocco di 0.5 = 20dB/dec
	Blocco di 10^0 = 40dB/dec
	Blocco di 4 = 80dB/dec
	Blocco di 10^1 = 80dB/dec
	
	Tendenzialmente si fa con le tabelle sta cosa dell'andamento:
	\begin{tabular}
		\begin{table}[|c|c|c|c|c|c|]
			\hline
			& 10^{-1} & 0.5 & 10^0 & 4 & 10^1\\
			\hline
			K_b & 0 & 0 & 0 & 0 & 0\\
			\hline
			zeroR_1 & 0 & 0 & 0 & 20 & 20\\
			\hline
			zeroNul_1 & 40 & 40 & 40 & 40 & 40\\
			\hline
			zeroCC_1 & 0 & 0 & 0 & 0 & 40\\
			\hline
			poloR_1 & 0 & 0 & -20 & -20 & -20\\
			\hline
			& 40 & 40 & 20 & 40 & 80
		\end{table}
	\end{tabular}
	
	Dopo la tabella si fa il grafico delle ampiezze.
	\begin{tabular}
		\begin{table}[|c|c|c|c|c|c|]
			\hline
			& 10^{-1} & 0.5 & 10^0 & 4 & 10^1\\
			\hline
			K_b & 24 & 24 & 24 & 24 & 24\\
			\hline
			zeroR_1 & 0 & 0 & 0 & 12 & 20\\		[20log(\omega|\tau|)]
			\hline
			zeroNul_1 & -40 & -12 & 0 & 24 & +40\\	[20\mu log(\omega)]
			\hline
			zeroCC_1 & 0 & 0 & 0 & 0 & 15.9\\	[40\mu sng(\zeta)log(\frac{\omega}{\omega_n})]
			\hline
			poloR_1 & 0 & 0 & -6 & -18 & -26\\	[20log(\omega|\tau|)]
			\hline
			& -16 & 12 & 18 & 42 & 74\\
			\hline
		\end{table}
	\end{tabular}
	
	Le formule utilizzate sono quelle viste per i quattro casi di costante, zeri reali, complessi e coniugati e simili. Meglio riprenderle da là piuttosto che trustare quelle scritte nelle [].
	Questi risultati vanno riportati nel diagramma di bode totale. Segna anzitutt gli intervalli dati dalle \omega_n e poi segna i punti dati dai valori delle ampiezze. Per quanto riguarda la fase, segnare nell'intervallo di quanto cresce o decresce la funzione.
	
	Adesso bisogna fare il grafico delle fasi, che segue un ragionamento simile.
	Andiamo a considerare naturalmente il solo valore asintotico.
	
	\begin{tabular}
		\begin{table}[|c|c|c|c|c|c|]
			\hline
			& 10^{-1} & 0.5 & 10^0 & 4 & 10^1\\
			\hline
			K_b & -180 & -180 & -180 & -180 & -180\\
			\hline
			zeroR_1 & 0 & 0 & 0 & 90 & 90\\
			\hline
			zeroNul_1 & 180 & 180 & 180 & 180 & 180\\
			\hline
			zeroCC_1 & 0 & 0 & 0 & 0 & 180\\
			\hline
			poloR_1 & 0 & 0 & 90 & 90 & 90\\
			\hline
			& 0 & 0 & 90° & 180° & 360°
		\end{table}
	\end{tabular}
	
	E potremo disegnare il grafico in base ai valori ottenuti. Occhio a non disegnare spezzate oblique. Queste rette vanno disegnate come fosse campionamento, quindi perfettamente a 90° se cambiano direzione.
\end{comment}

%

\section{Esercizi svolti}

\begin{comment}
	Come arrivare alla forma di Bode
	1. Raccogliere "le costanti"
	- Poli/zeri nulli:
	H(s) = 4\frac{s^3+s^2-2s}{s^3+s^2} = 4\frac{s(s^2+s-2)}{s^2(s+1)} = 4\cdot \frac{1}{s}\cdot\frac{s^2+s-2}{s+1}
	
	- Poli e zeri reali
	H(s) = 4\cdot \frac{1}{s}\cdot\frac{s^2+s-2}{s+1} = 4\cdot\frac{1}{s}\cdot \frac{(s-1)(s+2)}{s+1}
	
	// Adesso bisogna raggiungere la forma (1+s\tau)^\mu. Ciò si ottiene tirando fuori ogni costante per far rimanere come addendo nei poli e negli zeri il valore '1'.
	H(s) = 4\cdot\frac{1}{s}\cdot\frac{-1(1-s)\cdot 2(1+\frac{s}{2})}{1(1+s)}
	
	// Facciamo confluire tutte le costanti nel K_b, che in questo caso è 4.
	H(s) = -8\cdot\frac{1}{s}\cdot \frac{(1-s)(1+\frac{s}{2})}{(1+s)}
	
	// Questa appena ottenuta è la forma di Bode. Basta raccogliere le costanti da ambo i polinomi scomposti. Il \tau è detta poi la costante di tempo e deve essere definita rispetto a questa forma. Si tratta della radice di ogni elemento del polinomio.
	
	- Polo/zero complesso coniugato
	// Ricorda: z*\overline{z} = |z|^2
	
	&= (s-(\sigma+j\omega))^\mu(s-(\sigma-j\omega))^\mu\\
	&= (s^2-s\sigma+sj\omega - s\sigma - sj\omega + |s|^2)\mu\\
	&= (s^2 - 2s\sigma + |s|^2)^\mu\\
	&= |s|^{2\mu}(1-2\frac{\sigma}{|s|^2}s + \frac{s^2}{|s|^2})^\mu\\
	&= |s|^{2\mu}(\+2\xi\cdot \frac{s}{\omega_n} + \frac{\omega_n^2}{\omega_n})^\mu
	
	// Coeff. di smorzamento: \xi = -\frac{\sigma}{|s|} = -\frac{Re(s)}{|s|}
	// Pulsazione naturale: \omega_n = |s|
	
	Esempio - forma di bode numero complesso coniugato
	(s-(2+j))^2(s-(2-j))^2 &= (s^2-4s+5)^2
	&= 5^2(1-\frac{4}{5}s + \frac{s^2}{5})^2		// Raccolta la costante.
	
	Banalmente, \omega_n = \sqrt{5}. 
	
	La forma di bode è data da (1+2\xi_k \frac{s}{\omega_{n,k}} + \frac{s^2}{(\omega_{n,k})^2})^{\mu_k}, quindi ricondiuciamoci ad essa.
	
	\frac{2\xi}{\omega_n} = -\frac{4}{\sqrt{5}} \implies -\frac{4}{\sqrt{5}}\cdot\frac{1}{2} = -\frac{2}{\sqrt{5}}
	
	5^2(1-\frac{4}{\sqrt{5}}s + \frac{s^2}{(\sqrt{5})^2})^2 = 5^2(1+2(-\frac{2}{\sqrt{5}})\frac{5}{\sqrt{5}} + \frac{5^2}{(\sqrt{5})^2})^2
	
	Occhio che non tutti i trinomi sono complessi coniugati. Sebbene sia possibile risparmiare passaggi se trovi un trinomio, occhio a quale sia. Prima scomponi, se possibile, altrimenti puoi usare la forma di bode senza problemi.
	
	---
	
	Poi ci sarà da trasformare H(s) in H(j\omega) perché stiamo facendo analisi in frequenza.
	
	Esempio: H(s) = \frac{s^3-2s^2-8s}{s^4-2s^3+2s^2}
	1. Trova lo zero nullo raccogliendo la s
	H(s) = \frac{2(s^2-2s-8)}{s^2(s^2-2s+2)} = \frac{1}{s}\cdot\frac{s^2-2s-8}{s^2-2s+2}
	
	2. Verifica il discriminante dei polinomi. Se è maggiore di 0 è semplificabile.
	H(s) = \frac{(s-4)(s+2)}{s^2-2s+2}
	
	3. Raccogliamo i termini costanti per moltiplicarli a K_b
	\frac{-4(1-\frac{s}{4})\cdot 2(1+\frac{s}{2})}{2(\frac{s^2}{2}-s+1)}
	&= -4\frac{(1-\frac{s}{4}) \cdot (1+\frac{s}{2})}{(\frac{s^2}{2}-s+1)}
	
	4. Troviamo i \tau (questi sempre positivi devono essere) e \omega_n
	- (1-\frac{s}{4}) \implies \tau_1 = -\frac{1}{4}
	- (1+\frac{s}{2}) \implies \tau_2 = \frac{1}{2}
	- (1-s+\frac{s^2}{2}) \implies \omega_n^2 = 2 \implies \omega_n=\sqrt{2}
	\frac{2\xi}{\omega_n}=-1 \implies \frac{2\xi}{\sqrt{2}}=-1 \implies \xi = \frac{-1\sqrt{2}}{2} = -\frac{\sqrt{2}}{2}
\end{comment}