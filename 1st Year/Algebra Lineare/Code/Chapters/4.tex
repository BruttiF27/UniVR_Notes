\section{Lo spazio Vettoriale}
Iniziamo ora un argomento più complesso e astratto rispetto agli altri; abbiamo visto che i sistemi lineari si possono scrivere sotto forma matriciale e che l'insieme soluzione è esprimibile mediante vettori. Un insieme di questi vettori, spartanamente, viene chiamato \textbf{Spazio Vettoriale} su un determinato campo. Formalmente:
\begin{definition}
    \textbf{Spazio vettoriale sul campo $\mathbb{K}$}\par
    Uno spazio vettoriale sul campo $\mathbb{K}$ è un insieme $V$, i cui elementi sono detti vettori $v$ ed è dotato delle seguenti due operazioni:
    \begin{itemize}
        \item \textbf{Somma di vettori}\par
        Dati due vettori $v,w \in V$, quest'operazione interna associa un terzo vettore $u \in V$ alla somma dei primi due e si scrive $u = v+w$.
        \item \textbf{Moltiplicazione per scalare}\par
        Dato uno scalare $t \in \mathbb{K}$ ed un vettore $v \in V$, quest'operazione esterna associa un nuovo vettore $u \in V$ al prodotto fra i primi due elementi e si scrive $u = t\times v$.
    \end{itemize}
\end{definition}
Come potrai immaginare, la presenza di operazioni implica anche l'esistenza di relative proprietà. Qui le elenchiamo tutte, anche le più ovvie, dove $\alpha, \beta \in \mathbb{K}$, $u,v,w \in V$:\par
\begin{prop}
    \textbf{Proprietà delle operazioni negli spazi vettoriali}
    \begin{itemize}
        \item \textbf{Proprietà commutativa}\par
        L'ordine in cui sono effettuate le operazioni non cambia il risultato.
        \begin{center}
            $v+w = w+v$.
        \end{center}
        \item \textbf{Proprietà associativa}\par
        Ci è possibile effettuare un'operazione prima dell'altra e non cambiare il risultato.
        \begin{center}
             $(v+u)+w = v+(u+w)$, $(\alpha \times \beta)v = \alpha(\beta \times v)$.
        \end{center}
        \item \textbf{Proprietà distributiva}\par
        Nel moltiplicare più elementi per uno scalare, il moltiplicatore viene distribuito a tutti loro.
        \begin{center}
            $\alpha(v+w) = \alpha v + \alpha w$, $(\alpha \times \beta)v = \alpha v + \beta v$.
        \end{center}
        \item \textbf{Elemento neutro}\par
        Elemento tale per cui la sua somma o moltiplicazione per scalare con un vettore restituisce il vettore stesso.
        \begin{center}
            $v + v_0 = v$, $v \times 1 = v$.
        \end{center}
        \item \textbf{Elementi inversi}\par
        Elemento tale per cui se sommato al primo risulta zero.
        \begin{center}
            $w + v = 0 \iff v = -w$.
        \end{center}
    \end{itemize}
\end{prop}
Queste proprietà valgono anche se si volesse spostare il focus su matrici, insiemi numerici, insiemi di polinomi e successioni. Vale addirittura la somma fra vettori geometrica. Puoi provare a dimostrarlo ma è inutilmente tedioso, quindi prendilo per assioma.

%

\section{Combinazione Lineare, Insieme di Generatori}
Quando abbiamo un vettore risultante dalla moltiplicazione di una serie di vettori con una serie di scalari, si dice \textbf{Combinazione lineare} dei vettori $v_1,...,v_n$ con coefficienti $t_1,...,t_n$. Lo definiamo formalmente come:
\begin{definition}
    \textbf{Combinazione lineare}\par
    Sia $V$ uno spazio vettoriale su $\mathbb{K}$ e siano i vettori $v_1,...,v_n \in V$ con $t_1,...,t_n \in \mathbb{K}$. Il vettore risultante dalla loro moltiplicazione è:
    \begin{center}
        $v = t_1v_1 + t_2v_2 + ... + t_nv_n$
    \end{center}
    ed è detto combinazione lineare dei vettori $v_n$ con coefficienti della combinazione $t_n$.
\end{definition}
Risulta un pò difficile da immaginare in senso pratico, quindi propongo il seguente esempio:
\begin{eg}
    \textbf{Combinazione lineare}
    \begin{center}
        $v = \begin{pmatrix}
        1\\
        2\\
        3
        \end{pmatrix}$ è la combinazione lineare dei vettori: $e_1 = \begin{pmatrix}
        1\\
        0\\
        0
        \end{pmatrix}$, $e_2 = \begin{pmatrix}
        0\\
        1\\
        0
        \end{pmatrix}$, $e_3 = \begin{pmatrix}
        0\\
        0\\
        1
        \end{pmatrix}$\newline
        
        I coefficienti di combinazione sono rispettivamente $1,2,3$; infatti risulta che:
        \begin{center}
            $\begin{pmatrix}
            1\\
            2\\
            3
            \end{pmatrix} = 1\begin{pmatrix}
        1\\
        0\\
        0
        \end{pmatrix} +2\begin{pmatrix}
        0\\
        1\\
        0
        \end{pmatrix}+3\begin{pmatrix}
        0\\
        0\\
        1
        \end{pmatrix}$
        \end{center}
    \end{center}
\end{eg}
Hai potuto osservare che la combinazione lineare è \textit{generata} da un insieme di vettori. Nello specifico questo viene chiamato \textbf{Insieme di generatori}. Alcuni spazi vettoriali sono determinati da un insieme finito, come quello nell'esempio, e vengono definiti come spazi \textbf{finitamente generati}. Ci è inoltre possibile dimostrare, partendo da una combinazione lineare, se è o meno un insieme di generatori di un dato spazio $V$. Generalmente:
\begin{prop}
    \textbf{Quando una combinazione lineare è un insieme di generatori?}\par
    Diciamo che la combinazione dei vettori elementari $\{e_1, ..., e_n\}$ è un insieme di generatori dello spazio $\mathbb{K}^n$ se e solo se ogni vettore elementare presenta il valore $1$ in una sola riga, il quale verrà moltiplicato per uno scalare, ed in tutte le altre il valore $0$.
\end{prop}
\begin{eg}
    Dato lo spazio $V = \mathbb{K}^3$ diciamo che il seguente insieme è un insieme di generatori di $V$:
    \begin{center}
        $\left\{
            e_1 =\begin{pmatrix}
                1\\
                0\\
                0
            \end{pmatrix}, e_2 =\begin{pmatrix}
                0\\
                1\\
                0
            \end{pmatrix}, e_3 =\begin{pmatrix}
                0\\
                0\\
                1
            \end{pmatrix}
        \right\}$
    \end{center}
    Ciò vale solamente se vale la seguente operazione:
    \begin{center}
        $v_1e_1 + v_2e_2 + v_3e_3 = V = v_1\begin{pmatrix}
                1\\
                0\\
                0
            \end{pmatrix} + v_2\begin{pmatrix}
                0\\
                1\\
                0
            \end{pmatrix} + v_3\begin{pmatrix}
                0\\
                0\\
                1
            \end{pmatrix} = \begin{pmatrix}
                v_1\\
                v_2\\
                v_3
            \end{pmatrix}$
    \end{center}
    Bisogna infatti prendere i vettori dell'insieme; moltiplicare loro un qualsiasi vettore $V$ e se il risultato postumo alla somma è uguale al vettore iniziale, hai dimostrato che l'insieme è un insieme di generatori di $V$.
\end{eg}
\begin{eg}
    Ma qual è la logica dietro i vettori da moltiplicare scelti? Prova a vedere l'insieme supposto di generatori come un sistema lineare dove ogni caso deve essere uguale a $1$, perché deve ritornare il vettore $V$. Procediamo dunque con un secondo esempio, dato $V = \mathbb{R}^2$ diciamo che il seguente insieme è un insieme di generatori di $V$:
    \begin{center}
        $\left\{
            r_1 =\begin{pmatrix}
                1\\
                3
            \end{pmatrix}, r_2 =\begin{pmatrix}
                1\\
                1
            \end{pmatrix}, r_3 =\begin{pmatrix}
                0\\
                1
            \end{pmatrix}
        \right\}$
    \end{center}
    Quindi, testiamolo. Immaginiamo l'insieme supposto di generatori come un sistema lineare e poniamo ogni equazione ad $1$:
    \begin{center}
        $\begin{cases}
            1x_1+1x_2+0 = 1 \iff x_1 = 0\\
            3x_1+1x_2+1x_3 = 1 \iff x_3 = -3x_1
        \end{cases}$
    \end{center}
    Ora che sappiamo cosa ottenere, rifletti su \textit{come} ottenerli. Noti che la prima riga dell'insieme ha i valori $1+1+0$, quindi per arrivare ad $1$ servirà moltiplicare per un numero negativo. Prendiamo quindi il vettore $(v_1-v_2)$.\par\quad
    Nella seconda riga abbiamo che per ottenere $1$, dobbiamo rimuovere quel $3$, quindi moltiplicheremo una colonna per $-3$, prendiamo il vettore $(v_2-v_1)$. Risolviamo.
    \begin{center}
        $(v_1-v_2)(r_1) + (v_2)(r_2) + 3(v_2-v_1)(r_3) =
        \begin{pmatrix}
            v_1-v_2 + v_2 + 0\\
            3(v_1-v_2) + v_2 + 3(v_2-v_1)
        \end{pmatrix} = \begin{pmatrix}
            v_1\\
            v_2
        \end{pmatrix}$
    \end{center}
    Abbiamo la conferma che è un insieme di generatori.
\end{eg}

% RIPRENDI DA QUA

\section{Il Sottospazio Vettoriale}
Può capitare che uno spazio vettoriale ne contenga altri; in tal caso lo chiameremo \textbf{Sottospazio vettoriale}. Per far sì che siano tali devono valere le seguenti proprietà:
\begin{itemize}
    \item Somma vettoriale e moltiplicazione per scalare.
    \item Deve essere un insieme non vuoto.
\end{itemize}

4.3 Definizione: sottospazio generato da un insieme
4.3 Definizione: intersezione e somma di sottospazi

%

\section{Spazio delle Colonne e Spazio Nullo}
4.4 Definizione: spazio delle colonne
4.4 Proposizione: spazio delle colonne e sistemi lineari
4.4 Definizione: spazio nullo
4.4 Proposizione: spazio nullo `e un sottospazio

%

\section{Esercizi}
