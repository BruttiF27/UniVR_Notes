\section{Elementi di probabilità}
La probabilità è una branca della matematica che si occupa dello studio e descrizione degli \textbf{esperimenti aleatori}, ovvero delle inferenze il cui esito non è del tutto prevedibile. Esistono due metodi per l'espressione del concetto di probabilità:
\begin{itemize}
	\item \textbf{Approccio frequentista}; Determinazione della probabilità mediante esperimenti ripetuti. Risulta quindi come il rapporto fra il totale in cui si è esperito un esito e il totale degli esperimenti.
	\item \textbf{Approccio soggettivista}; Dove la probabilità è vista come un livello di fiducia nel verificarsi di un dato esito. È na roba da filosofi, non fa per noi.
\end{itemize}
Abbiamo parlato di una totalità di esperimenti; questi vengono formalmente chiamati \textbf{eventi} $E$ e detengono informazioni riguardo al loro esito. Ogni evento è un sottoinsieme dello \textbf{spazio campionario} $S$, che li comprende tutti.
\begin{eg}
	\textbf{Spazio campionario}\newline
	Un esempio di spazio campionario è dato dalla totalità dei valori delle facce di un dado, mentre gli eventi sono i singoli valori usciti da un esperimento.
	\begin{center}
		$S = \{1, 2, 3, 4, 5, 6\}$, $E = \{4\}$
	\end{center}
\end{eg}
Alle operazioni logiche sulle affermazioni corrispondono quelle insiemistiche,le quali si mostrano mediante diagrammi di eulero venn. Siano $A, B \subseteqq S$ due eventi:
\begin{itemize}
	\item \textbf{Intersezione}\newline
	Quando "Avviene $A$ e avviene $B$"
	% TODO ----- AGGIUNGI DIAGRAMMA
	\item \textbf{Unione}\newline
	Quando "Avviene $A$ oppure $B$"
	% TODO ----- AGGIUNGI DIAGRAMMA
	\item \textbf{Sottrazione}\newline
	Quando "Avviene $A$, ma non $B$"
	% TODO ----- AGGIUNGI DIAGRAMMA
	\item \textbf{Complementare}\newline
	Quando "Non avviene $A$"
\end{itemize}
Inoltre, se gli insiemi $A,B$ sono tali che la loro intersezione sia vuota, si dicono \textbf{incompatibili}.

%

\section{Calcolo della probabilità}
Fortunatamente esiste una concezione standard sulle caratteristiche assunte dalla probabilità. Associamo infatti ad ogni evento $E$ sullo spazio campionario $S$, un valore denotato con $P(E)$, detto \textbf{probabilità dell'evento} $E$. Il comportamento della funzione è dato dai seguenti \textbf{assiomi di Kolmogorov}:
\begin{definition}
	\textbf{Assiomi di Kolmogorov}\newline
	\begin{enumerate}
		\item $P(A)$ è un valore compreso fra $0$ e $1$.
		\item $P(S) = 1$.
		\item Se $A$ e $B$ sono incompatibili, allora $P(A \cup B) = P(A) + P(B)$.
		\item Siano $A, B$ due eventi tali che $A \subseteqq B$, allora:
		\begin{center}
			$\begin{cases}
				B = S \implies S / A = A^c \implies P(A^c) = 1 - P(A)\\
				P(B / A) = P(B) - P(A)
			\end{cases}$
		\end{center}
		\item Se $A_1, A_2, ..., A_k$ sono eventi a due a due incompatibili, quindi disgiunti, allora:
		\begin{center}
			\[P\left(\bigcup_{i = 1}^k A_i\right) = \sum_{i = 1}^{k} P(A_i)\]
		\end{center}
		\item Siano $A, B$ due eventi generici, allora:
		\begin{center}
			$P(A \cup B) = P(A) + P(B) - P(A \cap B)$
		\end{center}
	\end{enumerate}
\end{definition}
Un primo caso di studio per la probabilità è il suo calcolo ad \textbf{esiti equiprobabili}; ciò significa che ogni evento ha la stessa chance di avvenire rispetto agli altri. L'esempio classico è il lancio di un dado; implicando che questo non sia truccato, ogni faccia ha $\frac{1}{6}$ di possibilità di uscire. Formalmente la definiamo con la seguente scrittura:
\begin{center}
	$P(A) = \dfrac{|A|}{|S|}$
\end{center}
\begin{eg}
	Quali sono le probabilità che lanciando due volte un dado esca il valore 7?\newline
	Innanzitutto dobbiamo chiederci quale sia lo spazio campionario e gli eventi. Sappiamo che è un dado, quindi avremo rispettivamente:
	\begin{itemize}
		\item $S = \{1, 2, 3, 4, 5, 6\}$
		\item $E_1 = \{1\}, ..., E_6 = \{6\}$
	\end{itemize}
	Ora, potremmo fare bruteforcing facendoci del male, ma il trucco per questi esercizi (entro certi limiti) è disegnare una tabella dei risultati, prendere il totale di quante volte si presenta il valore richiesto e poi applicare la formula dell'approccio frequentista. In questo caso:
	\begin{center}
		\begin{tabular}{|c||c|c|c|c|c|c|}
			\hline
			- & 1 & 2 & 3 & 4 & 5 & 6\\
			\hline
			\hline
			1 & 2 & 3 & 4 & 5 & 6 & 7\\
			\hline
			2 & 3 & 4 & 5 & 6 & 7 & 8\\
			\hline
			3 & 4 & 5 & 6 & 7 & 8 & 9\\
			\hline
			4 & 5 & 6 & 7 & 8 & 9 & 10\\
			\hline
			5 & 6 & 7 & 8 & 9 & 10 & 11\\
			\hline
			6 & 7 & 8 & 9 & 10 & 11 & 12\\
			\hline
		\end{tabular}
	\end{center}
	Notiamo che il valore $7$ compare $6$ volte ed il totale degli esiti ottenibili è $6 \times 6 = 36$. Il risultato sarà dato quindi da:
	\begin{center}
		$\dfrac{6}{36} = \dfrac{1}{6}$, soluzione dell'esercizio.
	\end{center}
\end{eg}
Puta caso, devi lavorare con una quantità di dati abnorme; utilizzare la tabella precedente per analizzare è impensabile, hai una vaga idea di quanto grande verrebbe? Viene ad aiutarci il seguente ragionamento, scrivibile tramite \textbf{coefficienti binomiali}:
\begin{eg}
	\textbf{Calcolo combinatorio con coefficiente binomiale}\newline
	Diciamo di avere una gara a cui partecipano 10 atleti. In quanti modi possiamo assegnare i vari posti del podio? Avremo:
	\begin{itemize}
		\item 10 modi per il primo posto.
		\item 9 modi per il secondo, in quanto il primo è già stato assegnato.
		\item 8 modi per il terzo, per la medesima ragione.
	\end{itemize}
	Supponiamo di voler uccidere tutti quelli che perdono e che quindi considereremo solo i tre posti del podio. Allora quali sarebbero i possibili esiti?
	\begin{center}
		ABC, ACB, CAB, CBA, BAC, BCA, quindi $3\times 2\times 1 = 6$ esiti.
	\end{center}
	Questo calcolo si può esprimere più facilmente mediante l'utilizzo dei fattoriali, Gli esiti totali possibili saranno quindi:
	\begin{center}
		$\dfrac{(10\times9\times8)}{(3\times2\times1)} = \dfrac{10!}{7!\times3!}$
	\end{center}
	Perché è sbucato fuori un $7!$ dal nulla? Ebbene, quelli sono tutti i numeri che non ci interessano, in quanto vogliamo solamente i posti del podio.
\end{eg}
\noindent Da questo esempio traiamo dunque una formula generale, in inglese chiamata \textbf{n choose k}, la quale ha due varianti dipendentemente se ci interessa (caso 1) o meno (caso 2) l'ordine dei dati:
\begin{center}
	$\begin{cases}
		\dfrac{n!}{(n-k)!k!}\\
		\dfrac{n!}{(n-k)!}
	\end{cases} \implies \begin{pmatrix}
	n\\
	k
	\end{pmatrix}$
\end{center}

% TODO ----- Aggiungi esempio del mazzo da poker

%

\section{Probabilità condizionata}
Finora abbiamo utilizzato l'approccio frequentista per il calcolo delle probabilità di un evento, considerandole come a loro stanti. È tuttavia possibile che la probabilità di un evento $A$ possa essere influenzata da un altro $B$. Chiamiamo questo concetto \textbf{probabilità condizionata} e prende la formula matematica:
\begin{center}
	$P(A|B) = \dfrac{P(A\cap B)}{P(B)}$
\end{center}
Personalmente leggo la formula come "probabilità di $A$ sotto $B$". Quest'ultimo evento può quindi influenzare il primo positivamente, aumentandone la probabilità, oppure negativamente, diminuendola. Un'altra particolarità riguarda lo spazio campionario; essendo che stiamo valutando un'istanza dove $B$ avviene sicuramente, sarà proprio questo lo spazio. Possiamo infatti spaccare le istanze:
\begin{itemize}
	\item Succedono $A$ e $B$; quindi la probabilità condizionata.
	\item Succede $A$, ma non $B$; quindi la probabilità senza influenze.
\end{itemize}
È necessario conoscere i valori di entrambe le istanze per il calcolo della probabilità effettiva, infatti compone la seguente:
\begin{definition}
	\textbf{Formula delle probabilità totali}\par
	Formula utilizzata per il calcolo delle probabilità di un evento il cui esperire è condizionato da un altro. Siano $A,B$ due eventi generici.
	\begin{center}
		$P(A) = (A|B)P(B) + P(A|B^c)P(B^c)$
	\end{center}
	Dove il primo addendo rappresenta la probabilità condizionata ed il secondo quella di $A$ a sé stante.
\end{definition}

\begin{eg}
	\textbf{Calcolo di probabilità condizionata}\newline
	\noindent Siano due urne tali che:
	\begin{itemize}
		\item A contiene 2 palline rosse e 4 verdi.
		\item B contiene 3 palline rosse e 2 verdi.
	\end{itemize}
	Si lancia ora un dado; se esce 6 si estrae da A, altrimenti da B. Calcolare la probabilità di estrarre una pallina verde.\par
	Introduciamo i seguenti due eventi in base al risultato del dado:
	\begin{enumerate}
		\item $E$, Il dado mostra $6$.
		\item $F$, La pallina estratta è verde.
	\end{enumerate}
	\noindent Potremmo elencare ogni singola permutazione dati i pochi casi, ma useremo la formula della probabilità totale per pulizia. Attualmente deteniamo i seguenti dati:
	\begin{itemize}
		\item $P(F|E) = \frac{4}{6}$, date le $6$ palline di A, di cui $4$ verdi.
		\item $P(F|E^c) = \frac{2}{5}$, date le $5$ palline di B, di cui $2$ verdi.
		\item $P(E) = \frac{1}{6}$, probabilità del dado di far uscire $6$.
		\item $P(E^c) = \frac{5}{6}$, ogni altro numero del dado.
	\end{itemize}
	Ciò che abbiamo è sufficiente per utilizzare la formula della probabilità totale. Risulterà infatti:
	\begin{center}
		$P(F) = P(F|E)P(E) + P(F|E^c)P(E^c) = \dfrac{4}{6}\times \dfrac{1}{6} + \dfrac{2}{5}\times \dfrac{5}{6} = \dfrac{4}{9}$
	\end{center}
\end{eg}
\noindent Se pensi che sia possibile ottenere algebricamente le altre probabilità della formula, hai avuto un'ottima idea. Infatti per le probabilità condizionate abbiamo un nome apposito, che è:
\begin{definition}
	\textbf{Formula e teorema di Bayes}\par
	Necessaria per il calcolo della singola probabilità condizionata di un evento.
	\begin{center}
		$P(A|B) = \dfrac{P(B|A)P(A)}{P(B)}$
	\end{center}
	Sia ora ${B_1, B_2, ..., B_n}$ una partizione dello spazio campionario. Ne segue il teorema:
	\begin{center}
		$P(B_i|A) = \dfrac{P(A|B_i)P(B_i)}{\sum^n_{j=1} P(A|B_j)P(B_j)}$
	\end{center}
\end{definition}

\begin{eg}
	\textbf{Calcolo di probabilità con formula di Bayes}\par
	\noindent Abbiamo un esame a 4 risposte multiple. Gli studenti iscritti si dividono in:
	\begin{itemize}
		\item Preparato, corrispondente all'80\% del totale. Risponde correttamente al 90\%.
		\item Impreparato, il 20\% rimanente, che risponde a caso. Quindi hanno un 25\% di azzeccare la risposta.
	\end{itemize}
	Qual è la probabilità di prendere l'esame di uno studente preparato fra tutti?\newline
	
	\noindent Definiamo gli eventi come $A$, ovvero che lo studente sia preparato, e $B$, quella di azzeccare una risposta, che è necessariamente condizionata dal primo evento. Elenchiamo i dati che abbiamo fin da subito:
	\begin{itemize}
		\item $P(A) = 0,8$
		\item $P(A^c) = 0,2$
		\item $P(B|A) = 0,9$
		\item $P(B|A^c) = 0,25$
	\end{itemize}
	Dobbiamo trovare $P(A|B)$, ma procediamo per passi. Innanzitutto ci serve $P(B)$, ovvero la probabilità di azzeccare la risposta in generale. Usiamo la formula delle probabilità totali:
	\begin{center}
		$P(B) = P(B|A)P(A) + P(B|A^c)P(A^c) = 0,9\times 0,8 + 0,25\times 0,2 = 0,72 + 0,05 = 0,77$
	\end{center}
	Ora possiamo muoverci facendo delle asserzioni algebriche. Considera che $P(A|B)P(B) = P(A\cap B) = P(B\cap a) = P(B|A)P(A)$, da cui otteniamo la formula di Bayes: 
	\begin{center}
		$P(A|B) = \dfrac{P(B|A)P(A)}{P(B)}$
	\end{center}
	La quale, sostituendo le variabili con i loro rispettivi valori, ci darà il risultato:
	\begin{center}
		$P(A|B) = \dfrac{0,72}{0,77} = 0.935$
	\end{center}	
\end{eg}
E se invece avessimo due eventi completamente \textbf{indipendenti}? Questi si dicono tali se, dati per esempio A, B, vale la relazione:
\begin{center}
	$P(A \cap B) = P(A)P(B)$
\end{center}
Per esempio, se lancio due dadi, la probabilità che escano i valori 6 e 5 separatamente è $\frac{1}{36}$, perché mi va bene una sola combinazione. Infatti:
\begin{center}
	$P(A \cap B) = \dfrac{1}{6}\times \dfrac{1}{6} = \dfrac{1}{36}$
\end{center}
Abbiamo appurato che i due dadi non si influenzano fra di loro. Se ci fossero invece tre eventi avremmo le seguenti relazioni, dati A, B, C:
\begin{itemize}
	\item $P(A \cap B \cap C) = P(A)P(B)P(C)$
	\item $P(A \cap B) = P(A)P(B)$
	\item $P(A \cap C) = P(A)P(C)$
	\item $P(B \cap C) = P(B)P(C)$
\end{itemize}
Ovviamente, per casi richiedenti più di tre eventi, si dovranno verificare le istanze per tutti i successivi.

%

\section{Variabili aleatorie}

%

\section{Distribuzioni congiunte}

%

\section{Classi notevoli di variabili aleatorie}

%

\section{Statistiche campionarie}