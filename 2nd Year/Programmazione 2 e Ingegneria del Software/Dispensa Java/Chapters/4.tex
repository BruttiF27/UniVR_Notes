I \textbf{modelli di progettazione} o design pattern sono soluzioni standardizzate a problemi comuni in ambito di sviluppo software. Risultano utili per dare un'idea del workflow adattato ed evita che si vada a reinventare la ruota. Consentono, di conseguenza, di creare codice più mantenibile ed efficiente in una struttura ripetibile. Si dividono in tre categorie:
\begin{itemize}
	\item \textbf{Creazionali}: Rendono un sistema indipendente dal modo in cui gli oggetti sono istanziati ed elaborati.
	\item \textbf{Strutturali}: Scolpiscono una composizione di classi ed oggetti per formare una struttura più grande. Utili per facilitare la progettazione e minimizzare le dipendenze fra le parti.
	\item \textbf{Comportamentali}: Lavorano sull'assegnazione di responsabilità fra gli oggetti, facendo dipendere un sistema dalla composizione e l'interazione fra gli oggetti.
\end{itemize}
\noindent Il motivo per cui utilizzare queste strategie è autoesplicativo: scrivere codice pulito, mantenibile e performante.

%

\section{Singleton}
Appartenente ai pattern creazionali, implica che ogni classe vada a rappresentare un concetto il quale richieda l'implementazione di una singola sua istanza. Questo modello è utilizzato in ambiti dove è richiesto limitare l'istanziazione degli oggetti.
\begin{lstlisting}[language=Java]
	public RingOfPower {
		private static String owner = "Sauron";
		private static RingOfPower instance;
		
		private RingOfPower () {}
		// Blocca eventuali creazioni di istanze e ritorna quella esistente
		public static RingOfPower getInstance () {
			if (instance == null) { instance = new RingOfPower(); }
			return instance;
		}
	}
	public String getOwner () { return owner; }
\end{lstlisting}
\noindent Naturalmente, essendo un metodo static, per ottenere l'istanza sarà necessario chiamarla con \textbf{RingOfPower.getInstance()}.

%

\section{Iterator}

%

\section{Observer}

%

\section{Factory}