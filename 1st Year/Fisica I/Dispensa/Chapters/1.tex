La fisica è una scienza naturale che si occupa dei principi primi che spiegano il funzionamento dell'universo e li esprime tramite il linguaggio matematico; pone quindi le basi per lo studio di tutto ciò che ci circonda ed è vastamente utilizzata anche in ambito ingegneristico. Per lo studio ci serviremo dei \textbf{modelli di analisi}, approssimazioni di fenomeni reali per renderne la comprensione più semplice e riportare ad una stessa dinamica le varie situazioni. Partiamo con alcuni concetti di base.

\section{Grandezze fisiche e analisi dimensionale}
Diciamo \textbf{grandezza fisica} una proprietà misurabile mediante un apposito dispositivo, per esempio, nel misurare il peso di un oggetto ci serviremo di una bilancia. Queste si esprimono con una moltiplicazione fra un valore numerico e la relativa unità di misura: $[1g]$. Distinguiamo le:
\begin{itemize}
	\item \textbf{Fondamentali}: Concetti indipendenti l'uno dall'altro indefinibili in termini di altre grandezze.
	\item \textbf{Derivate}: Definibili mettendo in relazione le grandezze fondamentali.
\end{itemize}
\begin{table}[h]
	\centering
	\begin{tabular}{c | c}
		\hline
		\textbf{Grandezze fondamentali} & \textbf{Grandezze derivate}\\
		\hline
		Lunghezza $[m]$ & Superficie $[L^2]$ \\
		Massa $[Kg]$ & Volume $[L^3]$\\
		Tempo $[s]$ & Velocità $[L/t]$\\
		Intensità di corrente $[i]$ & Accelerazione $[L/t^2]$\\
		Temperatura assoluta $[T]$ & Forza $[M\times L/t^2]$\\
		& Pressione $[(M\times L/t^2)/L^2]$\\
		\hline
	\end{tabular}
\end{table}
\noindent Quello utilizzato da noi per le misure è detto \textbf{sistema internazionale}, caratterizzato dalla semplicità per ottenere multipli e sottomultipli, attraverso moltiplicazioni e divisioni per $10$ rispettivamente. Gli eventuali risultati si scriveranno poi in base al numero di \textbf{cifre significative} richiesto, ovvero il totale delle cifre decimali entro le quali deve essere espresso il valore; tuttavia, in presenza di numeri molto grandi o piccoli, è possibile usare la \textbf{notazione scientifica}, una scrittura più compatta.\par
Essendo poi che stiamo lavorando su valori espressi come una moltiplicazione, è necessario prestare attenzione alle unità di misura in gioco. Ciò si fa mediante l'\textbf{analisi dimensionale}, un semplice algoritmo che funge da accertamento.
\begin{esempio}
	\textbf{Analisi dimensionale}\par
	\noindent Prendiamo la seguente formula indicante una velocità: $v = at$. Per controllare se è dimensionalmente corretta, si sostituiscono ai valori nell'equazione le loro unità di misura. Se le misure sono concordanti, la formula sarà corretta. Abbiamo quindi:
	\begin{center}
		$v = \frac{L}{T}$, $a = \frac{L}{T^2}$, $t = T$
	\end{center}
	\begin{equation}
		\begin{split}
			v & = at \implies \frac{L}{T} = \frac{L}{T^2}\times T \implies \frac{L}{T} = \frac{L}{T}
		\end{split}
	\end{equation}
	\noindent Notiamo che il risultato è un'identità, quindi la misura è corretta.
\end{esempio}

%

\section{Vettori}
I \textbf{vettori}, indicati con $\overrightarrow{A}$, sono oggetti in un piano di riferimento definiti mediante due misure: la distanza da un punto detto \textbf{origine} e la direzione orientata relativamente ad un asse di riferimento. Li utilizziamo per studiare la posizione di un punto materiale in più dimensioni, mediante le siddette \textbf{coordinate cartesiane} $(x,y)$ e \textbf{coordinate polari} $(r,\theta)$, strettamente legate fra loro.\par
Finora, per esprimere i valori è stato utilizzato puramente un numero; chiamiamo questa una \textbf{grandezza scalare}, ma è possibile specificare valori anche con una direzione, creando le \textbf{grandezze scalari}. Essendo queste ultime non necessariamente sovrapposte agli assi, è possibile scomporle in parti ad essi associate. Scriviamo infatti in forma generale: \[\overrightarrow{A} = x\hat{i} + y\hat{j} + z\hat{k}\]
\noindent Dove $x,y,z$ sono le grandezze dei vettori, dette \textbf{moduli} nel piano rispettivo e $\hat{i},\hat{j},\hat{k}$ i vettori unitari che danno loro la direzione. In particolare, è possibile introdurre l'aritmetica legata ai vettori. Dove esistono metodi grafici, ci concentreremo sulle apposite formule:
\begin{itemize}
	\item \textbf{Somma algebrica fra vettori}: $\overrightarrow{A} = \overrightarrow{B}+\overrightarrow{C} = (B_x+C_x)\hat{i} + (B_y+C_y)\hat{j}$
	\item \textbf{Moltiplicazione con scalare}: $n\overrightarrow{A}$
	\item \textbf{Coordinate cartesiane in funzione delle polari}: $x = rcos(\theta)$, $y = rsin(\theta)$.
	\item \textbf{Coordinate polari in funzione delle cartesiane}: $\theta = tan^{-1}\left(\dfrac{y}{x}\right)$, $r = \sqrt{x^2 + y^2}$.
\end{itemize}

%

\section{Esercizi svolti}
\textbf{Esercizio 1: Passaggio fra tipi di coordinate}\par
\noindent Supponiamo di avere due punti in coordinate cartesiane $A = (2.00, -4.00)m; B = (-3.00, 3.00)m$. Vogliamo passare a coordinate polari.\newline

\noindent Ricordiamo che la forma polare è espressa nella formula $(r,\theta)$, dove $r$ è il raggio che passa per l'origine e $\theta$ l'ampiezza dell'angolo da esso formato. Abbiamo che: \[r = \sqrt{x^2 + y^2}; \theta = arctan\left(\frac{y}{x}\right)\]
\noindent Quindi sostituiamo i valori richiesti alle variabili per il punto $A$ e $B$:
\begin{itemize}
	\item $r_A = \sqrt{x^2 + y^2} = \sqrt{2^2 - 4^2} = 4.47m$
	\item $\theta_A = arctan\left(\dfrac{y}{x}\right) = \theta = arctan\left(\dfrac{-4}{2}\right) = \theta = arctan(-2) = -63.4°$
	\item $r_B = \sqrt{-3^2 + 3^2} = 4.24m$
	\item $\theta_B = arctan\left(\dfrac{3}{-3}\right) = arctan(-1) = -45°$
\end{itemize}
\noindent Notare che il punto $B$ risiede necessariamente nel secondo quadrante e che quindi dovremo sottrarre $180 - 45 = 135$ per ottenere l'effettivo valore in gradi.\newline

\noindent\textbf{Esercizio 2: Somma nella stessa direzione}\par
\noindent % TODO INSERISCI ESERCIZIO

\noindent\textbf{Esercizio 3: Somma in direzioni diverse}\par
\noindent % TODO INSERISCI ESERCIZIO