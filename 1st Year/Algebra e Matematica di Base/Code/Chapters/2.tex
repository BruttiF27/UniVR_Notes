\section{Tipi di funzioni}
Le \textbf{funzioni}, o applicazioni, sono le relazioni più importanti fra gli insiemi. Si definiscono formalmente come:
\begin{definition}
    \textbf{Funzione} - $f : X \implies Y$\par
    Siano due insiemi $X,Y$. Un'applicazione $f(X)$ in $Y$ è una corrispondenza $f \subseteq X \times Y$ con la seguente proprietà:\par
    Per ogni elemento $x\in X$, esiste un unico elemento $y\in Y$ tale che $(x,y) \in f$, ovvero che valga quanto segue:
    \begin{center}
        $\forall x,x' \in X .\{(x = x') \implies (f(x) = f(x'))\}$
    \end{center}
    Per indicare l'elemento corrispondente a $x$ scriviamo $f(x) = y$.
\end{definition}
Questa definizione porta tutti nuovi concetti come conseguenze logiche. Essendo che stiamo lavorando su insiemi, diciamo di avere una funzione $f: \mathbb{R} \implies \mathbb{R}$, possiamo associare a questa funzione un numero all'interno dell'insieme per ottenere la sua corrispondenza nello stesso.\par\quad Fin qua tutto chiaro, ma la presenza di n numeri corrisposti implica l'esistenza di un insieme che li contenga tutti. Questo si chiama \textbf{Insieme Immagine}. Andando più nello specifico, possiamo dire che il singolo elemento $f(x)$ è chiamato \textbf{immagine} di $x$ sotto $f$. Inoltre chiameremo l'insieme di partenza $X$ il \textbf{Dominio} e quello della corrispondenza $Y$ il \textbf{Codominio}.\par\quad
Per definizione di funzione non è possibile che ad un elemento $f(x)$ corrispondano più elementi nell'insieme $Y$, tuttavia è possibile che più elementi di $X$ abbiano una stessa immagine. Seguono alcuni casi notevoli:
\begin{itemize}
    \item \textbf{Funzione costante} - $f_{y_0} := \{(x,y_0) | x \in X\}$\par
    Si tratta di una funzione definita dalla regola $f_{y_0}(x) = y_0$, la quale vale per ogni $x \in X$.\par
    INSERISCI IMMAGINE.
    \item \textbf{Funzione identità} - $Diag(X) := \{(x, x)|x \in X\}$\par
    La funzione diagonale o identità, denotata con $id_X(x) = x$ per ogni $x \in X$, restituisce lo stesso valore che le è stato assegnato. Se vuoi scriverla, fai appello alla scrittura dell'insieme delle parti, solo saranno considerati parte dell'insieme le coppie con ambo i numeri uguali.\par
    \item \textbf{Funzione valore assoluto} - $|.| := \{(x, x) | x ≥ 0\} \cup \{(x, −x) | x < 0\}$\par
    Definita unicamente su intervalli positivi, possiamo dire che "specchia" ogni valore che si sarebbe trovato negli intervalli negativi. Si definisce con la seguente regola:\begin{center}
        $|x|=
        \begin{cases} 
            x & ,x \geq 0 \\
            -x & ,x < 0
        \end{cases}$
    \end{center}
    INSERIRE IMMAGINE.
    \item \textbf{Funzione quadratica} - $qu := \{(x,x^2)|x \in \mathbb{R}\}$\par
    Più comunemente conosciuta come la parabola fra le funzioni elementari. Hai che $qu(x) = x^2$.\par
    INSERIRE IMMAGINE.
    \item \textbf{Funzione di Dirichlet} - $Dir = \{(x, 0)|x \in Q\} \cup \{(x, 1)|x \in R / Q\}$.\par
    Curiosa funzione dalla difficile integrazione. Always bet on Lebesgue. Si definisce con:
    \begin{center}
        $Dir(x) =
        \begin{cases}
            1 & ,x \in \mathbb{Q}\\
            0 & ,x \notin \mathbb{Q}
        \end{cases}$
    \end{center}
\end{itemize}
La possibilità di ottenere risultati tramite applicazioni implica l'esistenza di immagini; infatti definiamo formalmente:
\begin{definition}
    \textbf{Immagine e Controimmagine}:\par
    Sia $f:X\to Y$ un'applicazione. Se $A\subseteq X$, diremo che l'immagine di $A$ secondo $f$ è il seguente insieme:
    \begin{center}
        $f(A):= \{f(x)|x \in A\} = \{y \in Y|\exists x \in A(f(x) = y)\}$
    \end{center}
    L'immagine di tutto il dominio è poi detta immagine dell'applicazione $f$ ed è l'insieme $f(X) := \{f(x)|x \in X\}$.\newline
    
    Se invece abbiamo $B \subseteq Y$, definiamo controimmagine di $B$ secondo $f$ l'insieme:
    \begin{center}
        $f^{-1}(B) := \{x \in X|f(x) \in B\}$.
    \end{center}
\end{definition}

CONTINUA DA PAGINA 31 DOCUMENTO PAGINA 33 EFFETTIVA.








\begin{itemize}
    \item Funzioni totali
    \item Funzioni parziali
    \item Iniettive
    \item Suriettive
    \item Biunivoche
    \item Funzioni composte
    \item Funzione inversa
    \item Cancellabilità della funzione
\end{itemize}
\section{Relazioni di equivalenza}
Equivalenza, transitività, simmetria/antisimmetria, monotonia, proiezione simmetrica, assiomi di peano.
\section{Partizioni}
\section{Relazioni di ordinamento}
\section{Domande di teoria}
\section{Esercizi}