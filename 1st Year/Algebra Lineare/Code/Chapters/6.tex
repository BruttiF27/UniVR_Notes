\section{Dipendenza e Indipendenza Lineare}
Partiamo subito con un concetto necessario per introdurre l'argomento della sezione; è possibile che gli insiemi di generatori abbiano a loro volta dei sottoinsiemi. Prendiamo infatti lo spazio dei valori reali $\mathbb{K} = \mathbb{R}$, lo spazio vettoriale $V = \mathbb{R}^2$ e il seguente insieme di generatori:
\begin{center}
    $C = \left\{\begin{pmatrix}
        0\\
        1
    \end{pmatrix}, \begin{pmatrix}
        1\\
        0
    \end{pmatrix}, \begin{pmatrix}
        2\\
        3
    \end{pmatrix}\right\}$
\end{center}
I coefficienti per generare i vettori $x,y$ non sono univocamente determinati, infatti puoi arrivare ad una soluzione con più strade, basta pensare a come sono definiti gli insiemi numerici: ogni numero può essere rappresentato come somma o sottrazione di altri.\par\quad
Se si trovano soluzioni più efficienti per arrivare al vettore richiesto, ci si trova di fronte ad un \textbf{sottoinsieme di generatori}, i cui elementi devono necessariamente essere capaci di ricreare i vettori dell'insieme non utilizzati. In tal merito diciamo che $C$ è \textbf{linearmente dipendente}, poiché un suo vettore può essere espresso come combinazione lineare degli altri. Facciamo ora una piccola osservazione:
\begin{prop}
    Se $\{v_1,...,v_n\}$ è un insieme di generatori di uno spazio vettoriale $V$ e se $v_n$ è combinazione lineare di $v_1,...,v_{n-1}$, allora quest'ultimo insieme è un insieme di generatori di $V$. Di conseguenza, ciò che viene applicato agli insiemi di generatori vale anche per i loro sottoinsiemi.
\end{prop}
Ora possiamo definire formalmente l'argomento della sezione:
\begin{definition}
    \textbf{Dipendenza e Indipendenza Lineare}\par
    Dati i vettori di uno spazio vettoriale, quindi $v_1,...,v_n \in V$, il loro insieme si dice linearmente \textbf{dipendente} se almeno uno di quei vettore è una combinazione lineare degli altri. Ne consegue inoltre che se un insieme non è linearmente dipendente, sarà linearmente \textbf{indipendente}.
\end{definition}
La cosa comoda è che per dimostrare l'indipendenza lineare di un insieme esistono tre teoremi perfettamente equivalenti

\begin{theorem}
    PALLEPISELLOSONOQUASOLOPERSEPARAREGLIARGOMENTI.
\end{theorem}

6.3 Teorema: linear indipendenza
6.4 Definizione: base
6.5 Osservazione: base `e come un sistema di coordinate
6.6 Base di C(U ), U una matrice in forma ridotta
11/04/24
6.7 Proposizione: base, insieme di generatori minimo, insieme massimamente linearmente indipendente
6.8 Teorema: esistenza della base
6.9 Teorema di Steinitz: ogni insieme linearmente indipendente pu`o essere completato a una base
6.10 Corollario: ogni base ha lo stesso numero di elementi
6.11 Definizione: dimensione
6.12 Corollario
6.13 Proposizione: dimensioni di sottospazi

% ----- CONTINUA DA QUA LEZ. 21 -----

\section{Applicazioni Lineari}
§7. Applicazioni lineari (vedi [GS, Capitolo II])
7.1 Definizione: applicazione lineare
7.2 Esempi e matrice associata a un’applicazione lineare (rispetto alla base canonica)
15/04/24
7.3 Definizione: isomorfismo
7.4 Definizione: applicazione delle coordinate rispetto a una base
7.5 Applicazione delle coordinate Kn → Kn
7.6 Teorema: l’applicazione delle coordinate `e un isomorfismo
7.7 Osservazione: isomorfismi e dimensione
7.8 Corollario: due spazi vettoriali sono isomorfismi se e solo se hanno la stessa dimensione
18/04/24
7.9 Teorema e definizione: matrice del cambio di base
7.10 Teorema e definizione: matrice associata a f rispetto a basi
\section{Rango e Nullità}
§8. Rango e nullit`a (vedi [GS, Capitolo II])
8.1 Spazio nullo e immagine di un’applicazione lineare
22/04/24
8.2 Teorema: nullit`a + rango
8.3 Dimensione di C(A)
8.4 Dimensione di N (A)
8.5 Procedimento per determinare basi di C(A) e N (A)
8.6 Proposizione e definizione: rango di un’applicazione lineare
8.7 Teorema: insieme di soluzioni di un sistema lineare
\section{Esercizi}