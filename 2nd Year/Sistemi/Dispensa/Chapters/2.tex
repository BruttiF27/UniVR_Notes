\section{Sistemi Lineari Tempo invariante}
Possiamo ora dare una definizione rigorosa di sistema: si tratta di un modello matematico o formulazione di un processo o fenomeno fisico che permette di trasformare un'entrata in un'uscita determinata. Ne esistono principalmente di due tipi:
\begin{itemize}
	\item \textbf{Single Input Single Output}
	\item \textbf{Multiple Input Multiple Output}, non visti in questo programma.
\end{itemize}
\noindent In base a questi poi ci sono i sistemi dinamici, ovvero con memoria, i quali salvano i dati grazie agli stati. Rammento infine che, come i segnali, i sistemi possono essere a tempo continuo e discreto.\par
Per quanto riguarda i sistemi LTI, approfondiamo le proprietà menzionate nel capitolo precedente:
\begin{itemize}
	\item \textbf{Linearità}: Partendo da un sistema con due entrate, otterrò in output un risultato equivalente alla combinazione lineare dei singoli input.
	\item \textbf{Tempo-invarianza}: Traslando l'input del sistema prima o dopo nel tempo avrò come output lo stesso risultato traslato prima o dopo nel tempo: \[u(t)\to v(t) \implies u(t+\tau)\to v(t+\tau), \forall \tau \in \mathbb{R}\]
	\item \textbf{Causalità}: L'effetto non precede mai la causa. L'uscita $v(t)$ all'istante $\tau$ deve dipendere esclusivamente dall'ingresso $u(t)$ per $t \leq \tau$. Per comodità considereremo solo sistemi a riposo, ovvero con $\tau = 0$.
\end{itemize}
\noindent Richiamiamo all'attenzione il concetto di stabilità per formalizzarlo definitivamente. Definiamo:\newline

\noindent \textbf{Sistema asintoticamente stabile}\par
\noindent Un sistema si dice tale se dopo aver dato l'input, l'output va a morire fino ad arrivare a $0$. Formalmente:
\begin{center}
	$\exists\tau\in\mathbb{R}$ tale che $u(t) = 0$, $\forall t \geq \tau \implies lim_{t\to \infty}v(t) = 0$
\end{center}

\noindent\textbf{Sistema BIBO-stabile}\par
\noindent Supponiamo di avere un segnale che oscilla da un certo istante $\tau$ sempre all'interno di un certo intervallo di ampiezza. Se in output vale la stessa dinamica, il sistema si dice BIBO-stabile. Formalmente:
\begin{enumerate}
	\item Se $(\exists\tau \land M_u>0)\in\mathbb{R}$ tali che $|u(t)| < M_u$
	\item Allora: $\forall t\geq\tau \implies \exists M_v > 0$
	\item Infine, quest'ultimo è tale che: $|v(t)| < M_v \forall t \geq \tau$.
\end{enumerate}
\noindent Prendere nota del fatto che ogni sistema asintoticamente stabile è necessariamente anche BIBO-stabile, in quanto suo caso particolare, ma non vale il contrario.\par
Ora che conosciamo la teoria, bisogna capire come si modellano effettivamente i sistemi. Come già detto, vengono descritti mediante equazioni differenziali per input e output. Un buon iter di lavoro generale è prima scrivere tutte le equazioni che descrivono le leggi in atto, per poi estendere la scrittura con tutte le componenti richieste. Va ordinata con output a sinistra e input a destra. L'equazione differenziale generale per la rappresentazione dei modelli è: \[a_n\frac{\partial^nv(t)}{\partial t^n}+a_{n-1}\frac{\partial^{n-1}v(t)}{\partial t}+ ... + \frac{\partial^1v(t)}{\partial t} + a_0v(t) = b_m\frac{\partial^mu(t)}{\partial t}+b_{m-1}\frac{\partial^{m-1}u(t)}{\partial t} + ... + b_1\frac{\partial u(t)}{\partial t} + b_0u(t)\]
\noindent Qui, $b_n, a_n \neq 0$ sono due normali coefficienti. La buona notizia è che esiste una forma più compatta, grazie alla sommatoria: \[\sum_{i=0}^na_i \frac{\partial^i v(t)}{\partial t} = \sum_{j=0}^mb_j\frac{\partial^ju(t)}{\partial t}\]
\noindent Qui invece $n,m$ danno l'ordine delle equazioni differenziali, con generalmente $n\geq m$. Più precisamente, se $n>m$, si dice strettamente proprio e se $n\geq m$ si dice proprio. Nello schema con la black box, nel sistema si scrive proprio questa seconda equazione.

%

\section{Calcolo delle risposte}
Come precedentemente menzionato, i sistemi sono definiti tramite equazioni differenziali, dove la forma generale mostra a sinistra l'output e a destra l'input. A noi interessa studiare il comportamento del sistema, quindi le sue risposte. Definiamo quindi \textbf{risposta totale}: \[v = v_l + v_f\]
\noindent Dove le componenti sono:
\begin{itemize}
	\item \textbf{Risposta libera}: Comportamento del sistema che dipende solo dalle condizioni iniziali.
	\item \textbf{Risposta forzata}: Comportamento del sistema che dipende solo da un determinato input $u$.
\end{itemize}
\noindent Il procedimento da effettuare si traduce nella risoluzione dell'equazione differenziale con specifici vincoli, per ottenere le due componenti ed infine descrivere il comportamento del sistema.

\subsection{Calcolo della risposta libera}
I passaggi sono equivalenti a quelli per la risoluzione di un'equazione differenziale associata ad un problema di Cauchy. Quindi:
\begin{enumerate}
	\item Ottenimento dell'equazione omogenea.
	\item Ottenimento del polinomio caratteristico.
	\item Risoluzione del suddetto polinomio per ottenere radici, ordine di grandezza e molteplicità algebrica.
	\item Sostituzione dei valori nelle opportune variabili nella formula della soluzione generale.
	\item Applicazione delle condizioni di Cauchy.
	\item Scrittura della risposta libera.
\end{enumerate}
\begin{esempio}
	\textbf{Calcolare la risposta libera dell'equazione differenziale}:
	\begin{center}
		$\begin{cases}
			v(0) = 0\\
			v'(0) = 1
		\end{cases}$; $v''(t) + 3v'(t) - 4v(t) = 5u'(t) - u(t)$
	\end{center}
	\noindent \textbf{1. Ottenimento equazione omogenea}\par
	\noindent L'equazione omogenea associata a quella presa in esame è semplicemente la parte di nostro interesse posta uguale a zero, in questo caso l'output, quindi: \[v''(t) + 3v'(t) - 4v(t) = 0\]
	\noindent \textbf{2. Ottenimento polinomio caratteristico}\par
	\noindent Per ottenere la forma polinomiale bisogna sostituire alla $t$ una variabile $s$, il cui grado è il numero di derivata dell'elemento: \[s^2 + 3s - 4 = 0\]
	\noindent \textbf{3. Risoluzione del polinomio ottenuto}\par
	\noindent Non difficile. Sono i classici passaggi algebrici per le equazioni di secondo grado: \[s^2 + 3s - 4 = 0 \implies (s+4)(s-1) = 0\]
	\noindent Da questo calcolo ricaviamo due radici distinte, quindi $r=2$, i cui valori sono: $\lambda_1=-4, \lambda_2=1$, e che le loro molteplicità sono: $\mu_1=\mu_2=1$.\newline
	
	\noindent \textbf{4. Sostituzione nella formula generale}\par
	\noindent Abbiamo già detto che la formula della soluzione generale è data da: \[v_l(t) = \sum_{i=1}^r \sum_{l=0}^{\mu_i-1} c_{i,l}\times e^{\lambda_i t}\times\frac{t^l}{l!}\]
	\noindent Dove $i,l$ sono coefficienti ottenuti con i vincoli di Cauchy. Sostituiamo:
	\begin{equation}
		\begin{split}
			v_l(t) &= c_1\times e^t\times\frac{t^0}{0!} + c_2\times e^{-4t}\times \frac{t^0}{0!}\\
				&= c_1e^t + c_2e^{-4t}
		\end{split}
	\end{equation}
	\noindent Questa equazione sarà la base sulla quale andremo ad applicare le condizioni e derivare ove necessario.\newline
	
	\noindent \textbf{5. Applicazione delle condizioni di Cauchy}\par
	\noindent Osservando le condizioni, è chiaro che saranno applicate prima alla forma normale, poi alla derivata prima. Mettiamo a sistema le due formule, sostituiamo a $t$ il valore del vincolo e poniamo il tutto al valore richiesto: \[\begin{cases}
		v_l(t) = c_{1,0}e^t + c_{2,0}e^{-4t}\\
		v_l'(t) = 1c_{1,0}e^t - 4c_{2,0}e^{-4t}
	\end{cases} \implies \begin{cases}
		c_{1,0}1 + c_{2,0}1 = 0\\
		c_{1,0} - 4c_{2,0} = 1
	\end{cases} \implies \begin{cases}
		c_{1,0} + c_{2,0} = 0\\
		c_{1,0} - 4c_{2,0} = 1
	\end{cases}\]
	\noindent Risolvendo il sistema con passaggi algebrici otteniamo infine che:
	\[\begin{cases}
		c_{1,0}+c_{2,0} = 0 \implies c_{1,0} = 1/5\\
		c_{1,0} - 4c_{2,0} = 1 \implies c_{2,0} = -1/5
	\end{cases}\]
	\noindent Ottenendo così i valori delle $c$ da sostituire alla formula generale.\newline
	
	\noindent \textbf{6. Scrittura della risposta libera}\par
	\noindent Nulla di più semplice, una sostituzione delle $c$ nella formula generale: \[v_l(t) = \frac{1}{5}e^t - \frac{1}{5}e^{-4t}\]
\end{esempio}
\noindent Una cosa interessante che si può ricavare dalla descrizione dell'equazione differenziale è il \textbf{modo elementare}, descritto come: \[m(t) = e^{\lambda t}\frac{t^l}{l!}\]
\noindent Risulta utile considerare i suoi tre casi di studio, perché grazie ad essi possiamo trarre conclusioni sulla stabilità di un sistema:
\begin{itemize}
	\item Il limite per $t$ che tende ad infinito del modo è uguale a zero solamente se ogni radice del polinomio caratteristico è minore di zero. In tal caso, il sistema è \textbf{asintoticamente stabile}. \[lim_{t\to \infty} m(t) = 0 \iff \lambda_i < 0\]
	\item Il modo $m(t)$ è limitato, quindi può prendere solo determinati valori, sull'intervallo $[0,+\infty)$ solamente se le radici del polinomio caratteristico sono minori o uguali a zero. In tal caso il sistema è detto \textbf{semplicemente stabile}.
	\item Se il limite per $t$ che tende ad infinito è uguale ad infinito non rientriamo nei primi due casi ed il sistema è detto \textbf{instabile}.
\end{itemize}

\subsection{Calcolo della risposta forzata}
Per il calcolo della risposta forzata ci serviremo di due strumenti: la \textbf{risposta impulsiva} e l'\textbf{operatore di convoluzione}. La prima per poter prendere un qualunque valore della funzione per descriverla, il secondo invece consente, in presenza di due segnali, di calcolare l'area sottesa del loro prodotto girando rispetto all'asse delle $y$ uno dei due segnali per poi scorrerlo sull'altro. La risposta forzata $v_f$ è quindi definita come:
\begin{equation}
	\begin{split}
		v_f = (u*v)(t) &= \int_{-\infty}^{\infty} u(\tau)v(t-\tau) d\tau\\
			&= \int_{-\infty}^{\infty} v(\tau)u(t-\tau) d\tau
	\end{split}
\end{equation}
\noindent Questo operatore detiene anche la proprietà commutativa, associativa e distributiva rispetto alla somma, consentendo di segliere a piacere il segnale fermo. Inoltre vedremo che l'uscita $v(t)$ di un sistema LTI inizialmente a riposo, in corrispondenza di un ingresso $u(t)$ è data proprio dalla convoluzione fra quest'ultimo e la risposta impulsiva $h(t)$.
\begin{equation}
	\begin{split}
		v(t) = (u*h)(t) &= \int_{-\infty}^{+\infty} h(\tau)u(t-\tau) d\tau\\
		&= \int_{-\infty}^{+\infty} u(\tau)h(t-\tau) d\tau
	\end{split}
\end{equation}
\noindent Supponendo sempre un sistema LTI causale e quindi inizialmente a riposo, abbiamo la risposta impulsiva $h(t)=0$ quando $t<0$, perché lo stimolo impulsivo $\delta(t)=0$ per $t<0$. In questi termini è possibile considerare l'integrale in un determinato intervallo di tempo, semplificando eventualmente i calcoli:
\begin{equation}
	\begin{split}
		v(t) = (u*h)(t) &= \int_{0^{-}}^{+\infty} h(\tau)u(t-\tau) d\tau\\
		&= \int_{-\infty}^{t^{+}} u(\tau)h(t-\tau) d\tau
	\end{split}
\end{equation}
\noindent \textbf{- Calcolo della risposta impulsiva}\par
\noindent Dato un sistema a riposo, la risposta impulsiva $h(t)$ è l'output in corrispondenza di un impulso unitario $\delta(t)$. La formula generale è: \[h(t) = d_0\delta(t) + \left[\sum_{i=1}^r\sum_{l=0}^{\mu_1-1} d_{i,l} \frac{t^l}{l!}e^{\lambda_i t}\right] \delta_{-1}(t)\]
\noindent I cui elementi sono:
\begin{itemize}
	\item $d_0$: Coefficiente. Risulta uguale a zero esclusivamente se il sistema è proprio, quindi quando gli indici delle due sommatorie $n,m$ sono uguali.
	\item $\delta(t)$: Impulso unitario, già menzionato.
	\item $d_{i,l}$: Coefficienti specifici della risposta impulsiva.
	\item $\frac{t^l}{l!}e^{\lambda_i t}$: Modi elementari, già menzionati.
	\item $\delta_{-1}(t)$: Funzione gradino per garantire la causalità dell'uscita.
\end{itemize}
\begin{esempio}
	\textbf{Calcolare la risposta impulsiva dell'equazione differenziale} \[v'(t) + 2v(t) = u(t) + u(t)\]
	\noindent\textbf{1. Equazione omogenea, polinomio caratteristico e calcolo delle radici}\par
	\noindent Notiamo anzitutto che entrambi i lati dell'equazione presentano un numero di coefficienti uguale, quindi è confermato che il sistema è proprio e $d_0\neq 0$. Chiarito ciò poniamo l'output a zero e risolviamo il polinomio caratteristico: \[v'(t) + 2v(t) = 0\implies s+2=0\implies \lambda_1 = -2\]
	\noindent Da questo risultato notiamo inoltre che abbiamo una sola radice, quindi $r=1$ e molteplicità $\mu_1=1$.\newline
	
	\noindent \textbf{2. Sostituzione valori nei modi elementari}\par
	\noindent Puri passaggi algebrici: \[\frac{t^l}{l!}e^{\lambda_i t} \implies \frac{t^0}{0!}e^{-2t} \implies e^{-2t}\]
	
	\noindent \textbf{3. Scrittura risposta impulsiva e calcolo derivate necessarie}\par
	\noindent Prendiamo la formula generale di prima e sostituiamo il valore ottenuto. Poi, in base all'ordine di grandezza dell'equazione, sarà necessario derivare il tutto per ricavare ogni singolo termine. In questo caso, dobbiamo derivare una sola volta. 
	\begin{equation}
		\begin{split}
			h(t) &= d_0\delta(t) + \sum_{i=1}^r\sum_{l=l}^{\mu_1-1} d_{i,l} e^{\lambda_i t} \frac{t^l}{l!} \delta_{-1}(t)\\
				&= d_0\delta(t) + d_{1,0}e^{-2t}\delta_{-1}(t)\\\\
			h'(t) &= d_0\delta'(t) - 2d_{1,0}e^{-2t}\delta_{-1}(t) + d_{1,0}e^{-2t}\delta(t)
		\end{split}
	\end{equation}
	
	\noindent \textbf{4. Riscrittura equazione iniziale in termini di impulso}\par
	\noindent Quindi dobbiamo porre $v(t)=h(t)$ e $u(t)=\delta(t)$ e sostituirli all'equazione originale. \[h'(t) + 2h(t) = \delta'(t) + \delta(t)\] 
	\begin{equation}
		\begin{split}
			d_0\delta'(t) - 2d_{1,0}e^{-2t}\delta_{-1}(t) + d_{1,0}e^{-2t}\delta(t) + 2(d_0\delta(t) + d_{1,0}e^{-2t}\delta_{-1}(t)) &= \delta'(t) + \delta(t)\\
			d_0\delta'(t) + 2d_0\delta(t) + d_{1,0}e^{-2t}\delta(t) &= \delta'(t) + \delta(t)
		\end{split}
	\end{equation}
	
	\noindent \textbf{5. Valutazione delle funzioni in $t=0$}
	\begin{equation}
		\begin{split}
			d_0\delta'(t) + 2d_0\delta(t) + d_{1,0}\delta(t) &= \delta'(t) + \delta(t)\\
			d_0\delta'(t) + 2d_0\delta(t) + d_{1,0}\delta(t) - \delta'(t) -\delta(t) &= 0\\
			(d_0-1)\delta'(t) + (d_{1,0} + 2d_0 -1)\delta(t) &= 0
		\end{split}
	\end{equation}
	
	\noindent \textbf{6. Risoluzione del sistema di equazioni e soluzione finale}\par
	\noindent Dall'algebra bisogna ricordarsi il concetto di indipendenza lineare. Possiamo porre a sistema i singoli ordini di grandezza e risolverli per ottenere i valori effettivi di nostro interesse. \[\begin{cases}
		(d_0-1)\delta'(t) = 0\\
		(d_{1,0} + 2d_0 -1)\delta(t) = 0
	\end{cases} = \begin{cases}
		d_0 = 1\\
		d_{1,0} = -1
	\end{cases}\]
	\noindent Chiaro che $d_0=1, d_{1,0}=-1$ siano i coefficienti da sostituire nell'equazione originale. Una volta fatto ciò, il processo è concluso.
	\begin{equation}
		\begin{split}
			h(t) &= d_0\delta(t) + d_{1,0}e^{-2t} \delta_{-1}(t)\\
			&= \delta(t) -e^{-2t}\delta_{-1}(t)
		\end{split}
	\end{equation}
\end{esempio}

\noindent \textbf{- Utilizzo dell'operatore di convoluzione}\par
\noindent Ottenuta la risposta impulsiva, è ora possibile ottenere la forzata, la quale dipende esclusivamente da input e per i sistemi LTI causali è descritta come:
\begin{equation}
	\begin{split}
		v_f(t) = (u*h)(t) &= \int_{0^{-}}^{t^{+}} h(\tau)u(t-\tau) d\tau\\
		&= \int_{0^{-}}^{t^{+}} u(\tau)h(t-\tau) d\tau
	\end{split}
\end{equation}
\begin{esempio}
	\textbf{Calcolare la risposta forzata $v_f$ a partire dai seguenti $h(t), u(t)$} \[h(t) = \frac{1}{2}\delta_0(t) + \frac{7}{4}e^{\frac{1}{2}t} \delta_{-1}(t); u(t) = 3\delta_{-1}(t)\]
	\noindent Quindi notiamo che l'input è una funzione gradino di altezza $3$. Procediamo con la risoluzione dell'integrale. Le forme possibili sono:
	\begin{equation}
		\begin{split}
			v(f) &= (h*u)(t) = (h*3\delta_{-1})(t)\\
				&= \int_{-\infty}^{+\infty} h(\tau) 3\delta_{-1}(t-\tau) d\tau\\
				&= \int_{-\infty}^{+\infty} 3\delta_{-1}(\tau) h(t-\tau) d\tau
		\end{split}
	\end{equation}
	\noindent Volendo preservare la nostra sanità mentale, terremo ferma la funzione più complessa, ovvero la risposta impulsiva. I passaggi continuano come segue:
	\begin{equation}
		\begin{split}
			(h*u)(t) &= \int_{-\infty}^{+\infty} h(\tau) 3\delta_{-1}(t-\tau) d\tau\\
				&= \int_{-\infty}^{+\infty}\left[\frac{1}{2} \delta_0(\tau) + \frac{7}{4} e^{\frac{1}{2} \tau} \delta_{-1}(\tau)\right] 3\delta_{-1}(t-\tau) d\tau\\
				&= \int_{-\infty}^{+\infty} \frac{1}{2}\delta_0(\tau) 3\delta_{-1}(t-\tau)d\tau + \int_{-\infty}^{+\infty} \frac{7}{4}e^{\frac{1}{2} \tau}\delta_{-1}(\tau) 3\delta_{-1}(t-\tau)d\tau\\
				&= \frac{3}{2}\int_{0^-}^{0^+} \delta_0(\tau)\delta_{-1}(t-\tau)d\tau + \frac{21}{4}\int_{0}^{t^+} e^{\frac{1}{2}\tau}\delta_{-1}(\tau)\delta_{-1}(t-\tau)d\tau\\
				&= \frac{3}{2}\times1 + \frac{21}{4}\int_{0}^{t^+}e^{\frac{1}{2} \tau} d\tau\\
				&= \frac{3}{2} + \frac{21}{4}\left[\frac{1}{2} e^{\frac{1}{2}\tau}\right]_0^t = \frac{3}{2} + \frac{21}{4}\left(\frac{1}{2} e^{2t} - \frac{1}{2}\right) = \frac{3}{2} + \frac{21}{8} e^{\frac{1}{2} t} - \frac{21}{8} = v_f
		\end{split}
	\end{equation}
\end{esempio}
\noindent Notare come gli estremi di integrazione variano ad un certo punto. Questo accade grazie al focus del nostro interesse. Per esempio, nel primo blocco, abbiamo l'intervallo di definizione dell'impulso, per il quale risulta essere sempre uguale ad $1$. Così è stato sostituito. Il secondo blocco riguarda invece gli istanti di tempo dopo l'impulso, quindi a partire da $0$ all'istante $t$.\par
Un'ultima cosa che è utile menzionare, più perché è richiesta all'esame che altro, è saper definire la stabilità di un sistema. Già menzionato come una stabilità \textbf{asintotica} è confermata se la parte reale di ogni radice del polinomio caratteristico è minore di zero, ma per definire un sistema \textbf{BIBO-stabile} è necessario verificare che la risposta impulsiva sia integrabile e con un risultato finito.