\section{Programmazione orientata agli oggetti}
Il corso di programmazione II ha lo scopo di fornire un modus operandi differente; passeremo infatti dalla programmazione imperativa ad una \textbf{orientata agli oggetti}.\par

% TODO COS'È LA OOP?

Lo scopo e l'utilità della OOP è quello di gestire appropriatamente un programma con molte linee di codice. Rende il codice estremamente modulare, riutilizzabile e di conseguenza risulta più facile mantenerlo. Per il modo in cui sono trattati gli oggetti, inoltre, il codice è reso particolarmente sicuro.\newline

\noindent Familiarizziamo ora con i due concetti base della OOP: \textbf{classi} e \textbf{oggetti}. Le prime definiscono una struttura dati, alla quale è possibile associare dati, detti \textbf{attributi}, e funzioni, chiamate \textbf{metodi}. Dalle classi, che in parole povere fungono da tipo di dato, è possibile ottenere gli oggetti, delle loro istanze, con stessi dati e metodi. Infatti, il processo di creazione di un oggetto è detto \textbf{istanziazione}.\par
Il workflow della OOP si basa sulle interazioni fra gli oggetti; queste avvengono grazie alle loro \textbf{interfacce}, set di messaggi che l'oggetto può ricevere, mappate ad un metodo nello stesso. Se si riceve un messaggio al di fuori dell'interfaccia, è detto illegale e viene bloccato. Sicuro, no? Ora, l'idea generale per una corretta programmazione a oggetti sono:
\begin{enumerate}
	\item Identificare i componenti.
	\item Definire l'interfaccia dei componenti.
	\item Definire le modalità con cui le interfacce consentono l'interazione fra oggetti.
	\item Minimizzare le relazioni fra i componenti.
\end{enumerate}

% TODO Definizione di OOP, classi, oggetti e UML.
\section{Il linguaggio Java}
% TODO Java, JVM, struttura generale dei programmi, compilazione ed esecuzione.

















Unified Modeling Language UML - Linguaggio di modellazione per definire graficamente e documentare un sistema orientato agli oggetti. Ci interessano i diagrammi di classe.




Incapsulamento e information hiding
Ereditarietà