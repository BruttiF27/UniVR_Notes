Cominciamo lo studio effettivo della Logica Matematica attraverso la \textbf{Logica Proposizionale}. Fondamentalmente è una palestra da utilizzare come trampolino di lancio verso argomenti più complessi, come la Logica del Primo Ordine, che verrà vista in seguito.\par\quad
Al termine del capitolo bisognerà saper valutare la veridicità di una formula e saper utilizzare la deduzione naturale per derivare sentenze.

\section{Sintassi}
In questo ambito il ragionamento si compone di entità linguistiche collegate dalla relazione "segue da". Queste si chiamano \textbf{Sentenze}. Come visto nell'introduzione possono essere minimali o composte, ma entrambe, se esprimono un pensiero completo si dicono \textbf{dichiarative}. Queste devono necessariamente avere un valore di vero o falso per essere considerate tali.
\begin{eg}
    \textbf{Tipi di sentenze}\par
    \begin{itemize}
        \item Sentenza minimale: $\phi$\par
        Entità irriducibile.
        \item Sentenza composta: $\phi \lor \psi$\par
        Entità riducibile.
    \end{itemize}
\end{eg}
Ogni linguaggio logico è basato su un determinato \textbf{Alfabeto}, determinante i simboli utilizzabili in esso. Infatti, i linguaggi sono definiti come $L \subseteq A^*$, dove quest'ultimo insieme è quello delle \textbf{Stringhe}, ovvero tutte le sequenze di simboli generabili tramite l'alfabeto. \par\quad
Le stringhe appartenenti ad $L$ si dicono sintatticamente corrette. In particolare, definiamo ora l'alfabeto della logica proposizionale che consente la loro composizione:
\begin{definition}
    \textbf{Alfabeto} $L^{PROP}$\par
    L'alfabeto della Logica Proposizionale è definito tramite tre unità sintattiche distinte:
    \begin{itemize}
        \item Insieme di simboli proposizionali $AT^{PROP}$, i cui elementi saranno indicati con lettere greche.
        \item Connettivi $\land$, $\lor$, $\to$, $\bot$.
        \item Simboli ausiliari "(" e ")". Non variano il senso della frase, ma aiutano a renderla più leggibile.
    \end{itemize}    
\end{definition}
Abbiamo constatato che il linguaggio proposizionale è costituito dall'insieme delle stringhe sintatticamente corrette. Si indica con $PROP$ e si definisce induttivamente come segue:
\begin{definition}
    \textbf{Insieme delle proposizioni PROP}
    \begin{itemize}
        \item $\bot \in PROP$;
        \item Se p è un simbolo proposizionale, allora $p \in PROP$;
        \item Se $\phi, \psi \in PROP$, allora valgono le seguenti relazioni:\par
        $(\phi \land \psi) \in PROP$, $(\phi \lor \psi) \in PROP$, $(\phi \implies \psi) \in PROP$, $(\neg\phi) \in PROP$.
        \item Null'altro appartiene all'insieme PROP.
    \end{itemize}
\end{definition}
Per renderci la vita un pochino più semplice sono state ideate delle convenzioni sulla scrittura delle proposizioni; si possono infatti omettere alcune parentesi nelle formule e fissare un ordine di precedenza tra i connettivi:
\begin{itemize}
    \item Si omettono le parentesi più esterne di una formula, così al posto di scrivere "$(\phi \land \psi)$", si scrive solo "$\phi \land \psi$".
    \item Il connettivo $\neg$ è il più forte e associa il primo simbolo presente a destra, perciò "$\neg\phi \lor \psi$" sarebbe uguale a "$(\neg\phi) \lor \psi$".
    \item Si assume che $\land$ e $\lor$ siano connettivi più forti rispetto a $\implies$ e $\iff$, quindi scrivere "$\phi_1 \land \psi_1 \implies \phi_2 \lor \psi_2$" è equivalente a intendere "$(\phi_1 \land \psi_1) \implies (\phi_2 \lor \psi_2)$".
    \item I connettivi $\implies, \land, \lor$ associano a destra, perciò scrivere "$\phi \implies \psi \implies \sigma$" equivale a dire: "$(\phi \implies (\psi \implies \sigma))$", mentre "$\phi \land \psi \lor \sigma$" sta per "$(\phi \land (\psi \lor \sigma))$".
\end{itemize}
Anche la logica matematica si basa sulla teoria degli insiemi; questo significa che valgono i teoremi inerenti, ma anche il funzionamento delle proprietà di un insieme. Diciamo infatti
\begin{prop}
    \textbf{Proprietà P su un insieme A}\par
    Se abbiamo un insieme $P$ ed è sottoinsieme proprio o improprio di un altro $A$ abbiamo che:
    \begin{center}
        $P \subseteq A \implies$ proprietà $P$ su $A$.
    \end{center}
    Di conseguenza, se abbiamo un elemento $a \in A$ possiamo dire che gode della proprietà se vale anche la relazione $a \in P$.
\end{prop}
Finora abbiamo visto i soli elementi; ora di vedere cosa si può fare per dimostrare gli enunciati. Gli strumenti prendono la forma di \textbf{Induzione strutturale} e \textbf{Ricorsione primitiva}.
\begin{theorem}
    \textbf{Induzione strutturale su PROP}\par
    Considera una proprietà $P \subseteq PROP$. Supponendo veri i punti:
    \begin{itemize}
        \item $\forall \phi \in AT$, vale $P[\phi]$\par
        Se la proprietà vale per ogni formula atomica...
        \item $\forall \phi, \psi \in PROP$, valgono $P[\phi], P[\psi]$\par
        E che se la proprietà vale per ogni formula in $PROP$...
    \end{itemize}
    Allora varranno le relazioni $P[\phi \land \psi], P[\phi \lor \psi], P[\phi \implies \psi]$. Di conseguenza vale che $\forall \phi \in PROP$, vale $P[\phi]$.
\end{theorem}
\begin{eg}
    \textbf{Costruzione induttiva della formula} $P[\phi \implies (\phi \lor \psi)]$\par
    Fondamentalmente devi scomporre l'intera formula. Parti dalle formule atomiche
    \begin{center}
        $P[\phi], P[\psi]$
    \end{center}
    Entrambe valgono, quindi puoi prendere una o l'altra. Puoi inoltre introdurre l'implicazione.
    \begin{center}
        $P[\phi] \implies P[\phi \lor \psi] := P[\phi \implies (\phi \lor \psi)$
    \end{center}
    E hai finito.
\end{eg}
Parliamo ora di ricorsione. Premetto che tutte le volte in cui si usa un linguaggio definito induttivamente, è possibile descriverne gli elementi anche ricorsivamente. Sempre. Questo algoritmo genera funzioni in termini di sé stesse all'interno della propria definizione. Risulta più semplice con l'esempio del fattoriale:
\begin{eg}
    \textbf{Funzione fattoriale}\par
    \begin{center}
        $fatt = 
        \begin{cases}
            0! = 1\\
            n! = n\times(n-1)!
        \end{cases}$
    \end{center}
    Come si può vedere, la funzione comprende ogni singolo caso che si può presentare. Se alla funzione è dato il valore $0$ sei nel caso base, mentre in qualunque altra istanza sostituisci il numero dato alla $n$ e poi svolgi i calcoli.
\end{eg}
Terminiamo con due ultimi concetti sintattici definiti ricorsivamente:
\begin{definition}
    \textbf{Rango di una formula}\par
    Definiamo $r : PROP \to \N$ ricevente in input una formula, il valore del suo rango. Si tratta di una quantità statistica utile al fine di vedere una definizione per ricorsione.
    \begin{itemize}
        \item $r[\phi] = 0$, se $\phi \in AT$
        \item $r[(\neg\phi)] = 1 + r[\phi]$
        \item $r[(\phi \circ \psi)] = 1 + max\{r[\phi], r[\psi]\}$, con $\circ \in \{\lor, \land, \implies\}$
    \end{itemize}
\end{definition}

\begin{definition}
    \textbf{Sottoformula}\par
    Si definisce sottoformula $sub: PROP \to 2^{PROP}$ la funzione che prende in input una formula proposizionale e ne restituisce l'insieme delle parti. Compone l'insieme di tutte le sottoformule presenti in una certa formula data.
    \begin{itemize}
        \item $sub[\phi] = \{\phi\}$, se $\phi \in AT$
        \item $sub[(\neg\phi)] = \{(\neg\phi)\} \cup sub[\phi]$
        \item $sub[(\phi \circ \psi)] = {(\phi \circ \psi)} \cup sub[\phi] \cup sub[\psi]$, con $\circ \in \{\land, \lor, \implies\}$
    \end{itemize}
\end{definition}

%

\section{Semantica}
Finora abbiamo studiato il valore sintattico delle frasi, ma non dimentichiamo che, come nella lingua italiana, una composizione di frasi può avere un significato; nel linguaggio logico è necessario analizzare gli enunciati per determinarne eventualmente verità, col valore di $1$,  o falsità, col valore di $0$. Come fare?
\begin{definition}
    \textbf{Valutazione di verità}\par
    Sia la funzione $V:PROP\to \{0,1\}$, ovvero $V \in PROP$ che può ritornare uno dei due valori fra $0,1$. Questa è tale che:
    \begin{itemize}
        \item $V[\bot] = 0$
        \item $V[\neg\psi] = 1 \iff V[\psi] = 0$
        \item $V[\psi \land \phi] = 1 \iff V[\psi] = 1 \land V[\phi] = 1$
        \item $V[\psi \lor \phi] = 1 \iff V[\psi] = 1 \lor V[\phi] = 1$
        \item $V[\psi \to \phi] = 1 \iff V[\psi] = 0 \lor V[\phi] = 1$
        \item $V[\psi \iff \phi] = 1 \iff V[\psi] = V[\phi]$
    \end{itemize}
    Si ricorda più facilmente pensando alle tavole di verità delle varie funzioni. Difatti, è la loro definizione ricorsiva.
\end{definition}
Ora, considera le variabili che abbiamo usato finora; queste possono indicare composizioni di formule al loro interno, quindi non essere atomiche, si definiscono infatti \textbf{metavariabili}, e per conoscere il loro valore semantico è necessario valutare i loro simboli proposizionali. Ciò avviene grazie alla \textbf{Valutazione atomica}, la base per definire tutte le valutazioni di verità.\par\quad
Il concetto è definito in modo astratto, ma fondamentalmente devi scomporre la formula fin quando ogni suo elemento diventa atomico. Dopo aver valutato i singoli valori, ricava il risultato in base ai connettivi usati.
\begin{definition}
    \textbf{Valutazione atomica}\par
Sia la seguente funzione:
    \begin{center}
        $V : AT \to \{1, 0\}$ tale che $v[\bot] = 0$.
    \end{center}
Data una valutazione atomica $v$, esiste ed è unica una valutazione:
    \begin{center}
        $[\cdot]_v : PROP \to \{1, 0\}$, Tale che $[\alpha]_v = v[\alpha]$ per $\alpha \in AT$.
    \end{center}
\end{definition}
Nel caso in cui nell'esercizio non venga richiesto se cercare verità o falsità, si valuta ogni singolo caso. Vediamo ora un esempio:
\begin{eg}
    Effettuare valutazione di verità sulla formula: $(\phi \land \psi) \implies (\neg\psi \lor \phi)$.\par
    Partiamo da un ragionamento semplice: avendo un'implicazione, le uniche istanze che possono riportare un valore vero, data la formula $P \implies Q$, sono $0\to 1$, $1\to 1$, $0\to 0$.\par\quad Muoviamoci ora per determinare i valori delle singole formule atomiche, una metavariabile alla volta, in base alle istanze discusse:
    \begin{itemize}
        \item $P = [(\phi \land \psi)]_v = 0 \iff [\phi]_v = 0$ OR $[\psi]_v = 0$.
        \item $Q = [(\neg\psi \lor \phi)]_v = 1 \iff [\neg\psi]_v = 1$ OR $[\phi]_v = 1 \iff [\psi]_v = 0$ OR $[\phi]_v = 1$.
    \end{itemize}
    Nella formula $P$ abbiamo ricavato che almeno una delle due variabili deve avere un valore di falsità. Guardando infatti la formula $Q$, notiamo che $[\psi]_v = 0$. L'ipotesi è confermata e non andando incontro a contraddizioni, abbiamo dimostrato che la formula è \textbf{vera}.
\end{eg}
Si possono inoltre presentare alcuni casi notevoli davanti alla valutazione del significato di un enunciato, come:
\begin{definition}
    \textbf{Concordanza e Tautologia}
    \begin{itemize}
        \item Dato un insieme finito di proposizioni $P$, se due valutazioni differenti risultano uguali, si dicono concordanti su $P$.
        \item Se una formula proposizionale è vera in ogni caso, si dice \textbf{Tautologia} e si indica con $[\alpha]_v = 1$.
    \end{itemize}
\end{definition}
Negli esercizi verranno utilizzate le seguenti scritture, con $\alpha\in PROP$ generica:
\begin{itemize}
    \item $\vDash \alpha:$ Tautologia.
    \item $\vdash \alpha:$ Teorema.
\end{itemize}
Per dimostrare una tautologia è necessario controllare che sia vera per ogni singolo caso; se in un'istanza si rivela falsa, non è tale.
\begin{theorem}
    \textbf{Tautologie Notevoli}
    \begin{itemize}
        \item Leggi di De Morgan:
        \begin{center}
            $\vDash \neg(\phi \land \psi) \iff (\neg\phi \lor \neg\psi)$, $\vDash \neg(\phi \lor \psi) \iff (\neg\phi \land \neg\psi)$.
        \end{center}
        \item Negazione Involutiva:
        \begin{center}
            $\vDash \phi \iff \neg\neg\phi$.
        \end{center}
        \item Leggi Commutative:
        \begin{center}
            $\vDash(\phi \land \psi) \iff (\psi \land \phi)$, $(\phi \lor \psi) \iff (\psi \lor \phi)$.
        \end{center}
        \item Leggi Distributive:
        \begin{center}
            $\vDash\phi \land (\psi \lor \gamma) \iff ((\phi \land \psi) \lor (\phi \land \gamma))$, $\phi \lor (\psi \land \gamma) \iff ((\phi \lor \psi) \land (\phi \lor \gamma))$.
        \end{center}
        \item Legge Associativa:
        \begin{center}
            $\vDash\phi \land(\psi \land \gamma) \iff (\phi\land\psi)\land\gamma$, $\phi\lor(\psi\lor\gamma) \iff (\phi\lor\psi)\lor\gamma$.
        \end{center}
    \end{itemize}
\end{theorem}
L'algoritmo di valutazione della verità può essere esteso anche ad un insieme determinato di proposizioni, che chiameremo $\Gamma$. Definiamo quindi:
\begin{definition}
    \textbf{Soddisfacibilità di una formula e conseguenza logica}\par
    La valutazione di verità si estende ad un insieme di proposizioni arbitrario $\Gamma$ tale che:
    \begin{center}
        $[\Gamma]_v = 1 \iff \forall\alpha \in \Gamma$ | $[\alpha]_v = 1$.
    \end{center}
    Attenzione, se $[\Gamma]_v \neq 1$, significa che c'è \textit{almeno una} valutazione dove risulta falsa, ma non che lo sono tutte. Se si afferma invece che $[\Gamma]_v = 0$, tutte le valutazioni saranno false.\par\quad
    Da questo insieme di proposizioni è possibile anche trarre delle conclusioni, dette \textbf{conseguenze logiche}. Si dice che una proposizione $\phi$ segue logicamente da un insieme $\Gamma$ se la verità delle proposizioni in esso contenute implica la verità di $\phi$.\par\quad
    Dato infatti l'insieme $\Gamma$ di formule, si dice $\psi$ conseguenza logica delle formule di $\Gamma$ e si scrive: 
    \begin{center}
        $\Gamma \vDash \psi$, ovvero in questo caso: $[\Gamma]_v = 1 \implies [\psi]_v = 1$.
    \end{center}
    Inoltre, se un insieme vuoto implica la verità di una formula, abbiamo dimostrato una tautologia.
\end{definition}
Diremo infine che una formula $\psi$ è detta \textbf{soddisfacibile} se esiste una valutazione $v$ per cui risulti vera, altrimenti è \textbf{insoddisfacibile}. Il concetto si può estendere agli insiemi di proposizioni; $\Gamma$ si dice soddisfacibile quando una sua formula $\psi$ è sempre vera data una certa valutazione, altrimenti è insoddisfacibile. Formalmente:
\begin{eg}
    \textbf{Soddisfacibilità di un insieme $\Gamma$}
    \begin{itemize}
        \item Se $\{p\} \in \Gamma$, $[p]_v = 1$ e quindi l'insieme è soddisfacibile.
        \item Se $\{p, \neg p\} \in \Gamma$, non tutte le valutazioni di \textit{p} sono vere in quanto contraddittorie e quindi l'insieme non è soddisfacibile. 
    \end{itemize}
\end{eg}
Le proprietà della conseguenza logica saranno riprese in seguito nella sezione dei sistemi deduttivi, in quanto saranno usate per la deduzione naturale. Parliamo ora invece di \textbf{Sostituzione proposizionale}, una tecnica che consente, in una formula $\phi$, di sostituire tutte le occorrenze di una formula $p_1$ con un'altra $\psi$.
\begin{definition}
    \textbf{Sostituzione proposizionale}
    \begin{center}
        Sia la formula: $\phi = ((p_1 \to (p_5 \lor p_1)) \land p_3)$.
    \end{center}
    Qui è possibile sostituire tutti i simboli $p_1$ con $\psi$. Il passaggio intermedio si scrive $\phi[\psi/p_1]$, ovvero che $\psi$ sostituirà il valore $p_1$. Risulterà:
    \begin{center}
        $\phi = ((\psi \to (p_5 \lor \psi)) \land p_3)$.
    \end{center}
    Se un termine nella formula presa in esame non compare affatto, è ovvio che la sostituzione non avrà alcuno scopo o effetto.
\end{definition}
Seguono ora le proprietà dell'algoritmo appena discusso:
\begin{prop}
    Sia la funzione $[\psi/p]:PROP\to PROP$ tale che:
    \begin{itemize}
        \item $\phi[\psi/p] = \bot$ se $\phi = \bot$.
        \item $\phi[\psi/p] = \phi$ se $\phi \in AT \land \phi \neq p$.
        \item $\phi[\psi/p] = \psi$ se $\phi = p$.
        \item $\neg(\phi)[\psi/p] = \neg(\phi[\psi/p])$.
        \item $(\phi_1 \circ \phi_2)[\psi/p] = (\phi_1[\psi/p] \circ \phi_2[\psi/p])$, dove $\circ = \{\land, \lor, \implies\}$.
    \end{itemize}
\end{prop}
E ora sblocchiamo il vero potenziale della sostituzione. Ragiona: se due formule risultano uguali, avranno egual valore nelle valutazioni. Se sono tali, sono intercambiabili, o anche meglio, scriverle allo stesso modo. Ciò ci consente di raggruppare più formule \textbf{equivalenti} sotto uno stesso termine, rendendo l'enunciato nettamente più semplice, infatti:
\begin{prop}
    \textbf{Sostituzione di formule equivalenti}\par
    Siano due formule $\phi_1\, \phi_2$ e $p\in AT$. Se le prime due formule sono equivalenti sintatticamente, ovvero $\vDash \phi_1 \longleftrightarrow \phi_2$, prendendo una formula generica $\psi$ possiamo scrivere:
    \begin{center}
        $\psi[\phi_1/p] \longleftrightarrow \psi[\phi_2/p]$.
    \end{center}
    La relazione di \textbf{equivalenza sintattica} $\longleftrightarrow$ è data dalle seguenti proprietà:
    \begin{center}
        Siano $\vDash \phi_1 \longleftrightarrow \phi_1'$, $\vDash \phi_2 \longleftrightarrow \phi_2'$
    \end{center}
    Varranno le seguenti relazioni:
    \begin{itemize}
        \item $\vDash \phi_1 \implies \phi_2 \longleftrightarrow \phi_1' \implies \phi_2$.
        \item $\vDash \phi_1 \land \phi_2 \longleftrightarrow \phi_1' \land \phi_2'$.
        \item $\vDash \phi_1 \lor \phi_2 \longleftrightarrow \phi_1' \lor \phi_2'$.
    \end{itemize}
\end{prop}
Formalizziamo infine il concetto di \textbf{Relazione di equivalenza}. Molto semplicemente, i simboli $\longleftrightarrow, \iff$ Rappresentano rispettivamente equivalenza sintattica e semantica; sono una scrittura compatta del seguente concetto: 
\begin{center}
    $\phi \iff \psi = (\phi \to \psi) \land (\psi \to \phi)$.
\end{center}
Restituisce il valore di vero solamente quando entrambe le formule sono vere.\footnote{Ciò implica anche che se le formule sono equivalenti e la valutazione di una delle due è vera, lo sarà per forza anche l'altra.}\par
Una relazione di equivalenza R si scrive $\approx$ ed è tale se vengono rispettati questi tre criteri:
\begin{enumerate}
    \item \textbf{Riflessività}; $\forall a \in A$ | $aRa$.
    \item \textbf{Transitività}; $\forall a, b, c \in A$ | $(aRb \land bRc) \to aRc$.
    \item \textbf{Simmetria}; $\forall a, b \in A$ | $aRb \to bRa$.
\end{enumerate}
\begin{theorem}
    \textbf{Relazione di equivalenza}
    \begin{center}
        Date due formule $\phi, \psi \in PROP$, se $\vDash \phi \leftrightarrow \psi$, si scriverà $\phi \approx \psi$.
    \end{center}
\end{theorem}

%

\section{Sistemi Deduttivi}
Prima di iniziare a lavorare, è necessario introdurre il concetto di \textbf{Sistema Deduttivo}. Si tratta di un costrutto che fornisce una struttura formale di regole inferenziali, le quali permettono di derivare una conclusione a partire da alcune premesse.\par\quad
Nell'enunciare un teorema, è necessario fornire anche la sua dimostrazione, poiché, senza di essa, definiamo il primo \textbf{congettura}. La dimostrazione di un teorema è definita come un algoritmo: sequenza di passi non ambigui mirati ad arrivare alla tesi partendo dalle ipotesi. I passi sono tutto lo svolgimento che effettueremo per dimostrare l'enunciato e si chiama \textbf{deduzione}. 
\begin{definition}
    \textbf{Deduzione Naturale}\par
    Metodo di ragionamento puramente sintattico. Si effettua stabilendo:
    \begin{itemize}
        \item \textbf{Assunzioni}; Ragionamenti temporanei.
        \item \textbf{Derivazioni}; Dalle formule è possibile effettuare derivazioni attraverso determinate leggi. Ogni passaggio si dice \textbf{Passo di derivazione}.
        \item \textbf{Scaricamento}; La chiusura delle assunzioni fatte all'inizio.
    \end{itemize}
    Ogni derivazione è rappresentata tramite \textbf{alberi}, dove le \textbf{foglie} sono le ipotesi e la \textbf{radice} l'enunciato. Una dimostrazione si dice completa quando ogni foglia è chiusa, ovvero quando si è scaricata ogni ipotesi; si indica con $Hp[D] = \emptyset$.\par
    La scrittura significa che l'insieme delle ipotesi della derivazione $D$ è vuoto. Possono inoltre esistere più derivazioni per un singolo teorema.
\end{definition}
Passiamo ora alle \textbf{Regole deduttive}. Si tratta di algoritmi utili per la derivazione delle formule nella deduzione naturale. Si dividono in due tipi:
\begin{itemize}
    \item \textbf{Introduzione}; Premesse collegate introducendo un connettivo logico.
    \item \textbf{Eliminazione}; Dalle premesse è possibile rimuovere un connettivo logico.
\end{itemize}
Negli esercizi, le regole deduttive consentono di effettuare i passi di derivazione e sono rappresentati in forma di frazione. Sopra stanno le premesse, sotto la conclusione e a fianco la regola utilizzata. Le formule all'interno di parentesi quadre sono poi ipotesi assunte/scaricate da una determinata regola di derivazione, la quale dovrà essere appropriatamente numerata. Vediamo come fare insieme alla rappresentazione delle regole a nostra disposizione:
\begin{itemize}
    \item \textbf{Regole dell'implicazione} $(\to)$
    \begin{itemize}
        \item \textbf{Introduzione dell'implicazione}\par
        Se dall’ipotesi $\alpha$ segue $\beta$, con una certa derivazione D, allora si deriva $\alpha \to \beta$ tramite l’introduzione dell’implicazione. Questa regola assume/scarica come ipotesi l’implicante dell'implicazione.
        \begin{prooftree}
            \AxiomC{$[\beta]^1$}
            \RightLabel{$\to I^1$}
            \UnaryInfC{$\alpha \to \beta$}
        \end{prooftree}
        \item \textbf{Eliminazione dell'implicazione}\par
        Chiamato anche \textit{modus ponens}, se da due derivazioni $D_1, D_2$ otteniamo $\alpha$ e $\alpha \to \beta$, possiamo concludere $\beta$ eliminando il connettivo.
        \begin{prooftree}
            \AxiomC{$\alpha$}
            \AxiomC{$\alpha \to \beta$}
            \RightLabel{$\to E^1$}
            \BinaryInfC{$\beta$}
        \end{prooftree}
        \item \textbf{Indebolimento}\par
        
    \end{itemize}
    \item \textbf{Regole della congiunzione} $(\land)$\par
    \begin{itemize}
        \item \textbf{Introduzione dell'AND}\par

        \item \textbf{Eliminazione dell'AND}\par
        
    \end{itemize}
    \item \textbf{Regole del bottom} $(\bot)$\par
    \begin{itemize}
        \item \textbf{Riduzione ad assurdo (RAA)}\par

        \item \textbf{Eliminazione del bottom}\par
        
    \end{itemize}
    \item \textbf{Regole della disgiunzione} $(\lor)$\par
    \begin{itemize}
        \item \textbf{Introduzione dell'OR}\par

        \item \textbf{Eliminazione dell'OR}\par
        
    \end{itemize}
    \item \textbf{Regole della doppia implicazione} $(\leftrightarrow)$\par
    \begin{itemize}
        \item \textbf{Introduzione della doppia implicazione}\par

        \item \textbf{Eliminazione della doppia implicazione}\par
        
    \end{itemize}
    \item \textbf{Assiomi}\par
    
\end{itemize}

% ----- Continua da qua -----

\section{Correttezza e Completezza}
\section{Esercizi}