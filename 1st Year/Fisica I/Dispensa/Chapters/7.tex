\section{Concetto di temperatura}
Quando parliamo di termodinamica, stiamo considerando un concetto applicabile ai tre stati della materia, quindi solido, liquido e gassoso. Analizzeremo quest'ultimo all'interno di appositi recipienti, dentro ai quali si espande fino a coprire tutta l'area interna.\par
Ora che abbiamo la definizione per ogni stato della materia, possiamo definire il concetto di \textbf{temperatura}. Si tratta di un valore misurato in gradi \textbf{Celsius} o \textbf{Kelvin}, dove in particolare il valore minimo di temperatura ottenibile in assenza di energia, detto anche \textbf{Zero Assoluto}, è dato da $0K = -273.15^\circ C$.\par
Formalmente, è ciò che risulta dal movimento delle molecole in un oggetto; supponiamo infatti di avere una sbarra di metallo con una data lunghezza $L$, muovendosi, le particelle dilatano il materiale che finisce per scaldarsi; quindi la variazione della lnghezza è proporzionale alla temperatura e si definisce come: \[\Delta L = \alpha L_i\Delta T \implies L_f = L_i(1+\alpha\Delta T)\]
\noindent Dove $\alpha$ è il \textbf{coefficiente di dilatazione termica}, $L_i$ la lunghezza iniziale e $\Delta T$ la \textbf{variazione di temperatura}. In particolare, il coefficiente è un valore che dipende dalla composizione del materiale e si presenta in forma lineare e solida. In questi ultimi termini, infatti, si eleva tutto al cubo: \[L_f^3 = (L_i+\alpha\Delta T)^3\]
\noindent In ogni caso, gli $\alpha$ sono numeri dal valore molto basso, abbastanza da poterli non considerare quando sono al quadrato o al cubo, rendendoli uguali a zero. Ciò ci permette di sviluppare il cubo come un binomio:
\begin{equation}
	\begin{split}
		L_f^3 &= (L_i+\alpha\Delta T)^3\\
		L_f^3 &= L_i^3 + 3L_i^2(\alpha\Delta T) + 3L_i(\alpha\Delta T)^2 + (\alpha\Delta T)^3\\
		L_f^3 &= L_i^3 + 3\alpha L_i^2\Delta T + 3\alpha^2 L_i(\Delta T)^2 + \alpha^3(\Delta T)^3\\
		L_f^3 &= L_i^3 + 3\alpha L_i^3\Delta T\\
		L_f^3 &= L_i^3(1+3\alpha\Delta T)
	\end{split}
\end{equation}
\noindent Quindi, fatta molto più semplicemente; il coefficiente di dilatazione termica cubica è \textbf{tre volte quello lineare}, mentre per la superficie parliamo di $2\alpha$.

%

\section{Legge di stato dei gas perfetti}
Iniziamo ora a discutere dei gas; nello studio e negli esercizi faremo uso del concetto di \textbf{gas perfetto}, uno stato in cui le molecole non interagiscono mai fra di loro. Sebbene questo sia poco realistico, nella vita vera potremmo rinchiudere il gas in un recipiente particolarmente grande ed ottenere questo risultato. Di conseguenza, in base al volume, vale la seguente \textbf{legge di stato dei gas perfetti}: \[PV = nRT\]
\noindent Nella formula sono usate la \textbf{pressione} $P$, il \textbf{volume} $V$, il \textbf{numero di moli} $n$, la \textbf{costante universale dei gas} $R = 0.314[J/m]$ e la \textbf{temperatura} $T$.\par
In alternativa si può scrivere anche $PV = NK_BT$, dove $K_B = 1.3806\cdot 10^{-23}[J/K]$ è la \textbf{costante di Boltzmann} ed $N$ il \textbf{numero di molecole}.

\begin{comment}
	--- Teoria cinetica dei gas
	Possiamo studiare il movimento del gas considerando le molecole come particelle (e usare quindi le leggi di Newton viste in precedenza).
	Le molecole sono a miliardi, però, per questo si utilizza la statistica e le relative distribuzioni per darci una mano.
	
	Supponiamo un cubo dentro al quale stanno molte molecole. Prendiamone una, la quale sbatte sulle pareti. Abbiamo tre velocità in quanto siamo in 3D.
	La quantità di moto p=mv determina quella iniziale e finale:
	- p_i = m_0v_{xi}
	- p_f = -m_0v{xi}	// Sbatte e cambia direzione mantenendo stessa v
	
	Proviamo a vedere la variazione della quantità di moto
	\Delta p = p_f - p_i = - m_0v_{xi} - m_0v_{xi} = -2m_0v_{xi}
	
	Ricordando che v=s/t; possiamo sostituire la distanza al posto dello spostamento per ottenere il valore scalare, ed eventualmente ricavare il tempo. \[v = s/t = 2d/\Delta t \implies \Delta t = \frac{2d}{v_{xi}}\]
	
	Ora voglio la forza F\Delta T = \Delta p. Cosa ne possiamo ottenere?
	In base a quello che abbiamo appena ottenuta diciamo: F = \frac{\Delta p}{\Delta t} = -\frac{2m_0v_{xi}}{2d}v_{xi} = -\frac{m_0v_{xi}^2}{d}
	Questa era la variazione della quantità di moto; se voglio invece la forza di azione e reazione data dalla parete basta invertire il segno. Grazie, Newton.
	
	Questa era però la forza di una singola molecola. Quindi bisogna fare la sommatoria di tutte le forze delle molecole.
	
	Velocità? More like velocità media: v_x^2 = \frac{\sim v_{xi}^2}{N}
	Quindi per la velocità media delle tre direzioni bisogna sommarle tutte; è come sommare 1/3 per componente x, y e z.
	Ragion per cui (per qualche motivo) è valido dire 3\overline{v_x} = \overline{v^2}
	
	Cosa cambia per la forza? Che si scrive: \[F = \frac{m_0}{d}\frac{N}{3}\overline{v^2}\]
	
	Questa formula ci fa ottenere le componenti per quella della pressione, dato che serve la forza e l'area. Con passaggi algebrici otteniamo \[P = \frac{N}{3V}m_0\overline{v^2} \implies PV = \frac{N}{3}m_0\overline{v^2}\]
	
	Please stay with me; che cosa succese dividendo per 2 m_0 e \overline{v^2}? Esatto, la formula dell'energia cinetica: 1/2mv^2. Quindi questa formula, isolando questa parte, consente di ottenere il valore di K per ogni singola molecola.
	
	\[PV = \frac{N}{3}\frac{1}{2}mv^2 \implies \frac{2N}{3}(\frac{1}{2}mv^2) = NK_BT \implies \frac{1}{2}mv^2 = \frac{3}{2}K_BT\]
	Quindi la temperatura è un'espressione macroscopica dell'energia cinetica.
\end{comment}

%

\section{Concetto di calore}
definizione di calore, calore specifico,



\begin{comment}
	Lez21 - Fisica 1
	
	Temperatura è l'aumentare della sbarra, con un determinato coefficiente, misurabile in celsius e kelvin.
	La temperatura è strettamente collegata all'energia cinetica. All'aumentare della velocità delle molecole così fa anche la temperatura.
	
	--- Energia di un gas
	Energia cinetica delle molecole, come detto il loro movimento è collegato alla temperatura. Infatti: \[\frac{3}{2}k_BT = \frac{1}{2}mv^2\]
	La temperatura occhio che non è l'energia, bensì un side-effect.
	
	```Principio zero della termodinamica
	Dati due oggetti con temperature T_1, T_2 rispettivamente, quando messi a contatto, c'è uno scambio di energia, detto calore, e si arriva ad una temperatura di equilibrio, detta equilibrio termico, uguale per entrambi.
	```
	
	Il calore si indica con Q, e quando è presente si ha uno scambio di energia cinetica fra un oggetto più freddo e uno più caldo. Si misura in Joule, ma anche in Calorie. [1Cal = 4184J]. È definito come \[Q = mc\Delta T\]
	m = massa che cambia la temperatura
	c = calore specifico
	\Delta T = differenza di temperatura
	Dunque in base all'oggetto considerato, varia il calore specifico, misurato in [Joule/kg*°C]. La formula dice quanto calore serve per passare ad una certa temperatura.
	
	```Esempio di esercizio
	Abbiamo un lingotto di metallo di m=0.05Kg ed è scaldato a T_m=200°C, per poi essere immerso in 0.04Kg di acqua, a temperatura T_a=20°C. La temperatura finale T_f = 22.4°C; trovare il calore specifico.
	
	Il lingotto scambia calore con l'acqua, naturalmente, per raggiungere l'equilibrio che presenta la T_f = 22.4°C, uguale per entrambi. Da una parte si dona, dall'altra si prende. Scriviamo quindi \[Q_m = -Q_a\]
	\noindent Da qua possiamo espandere l'equazione con la definizione per oggetto:
	\begin{equation}
		\begin{split}
			m_{met}c_{met}(T_f - T_{met, i}) &= -m_{acq}c_{acq}(T_f - T_{acq, i})\\
			c_{met} = -\frac{m_{acq}c_{acq}(T_f - T_{acq, i})}{m_{met}(T_f - T_{met, i})}\\
		\end{split}
	\end{equtation}
	```
	
	Ci sono più modi per scambiare calore, ognuno di questi lo traferisce con velocità diverse:
	- Convezione: Nello scambio di calore un corpo perde la sua forma
	- Conduzione: Scambio di calore da corpo caldo a corpo freddo per raggiungere equilibrio
	- Irraggiamento: Trasferimento di energia mediante radiazioni (un'onda colpisce la materia e fa muovere le molecole aumentando la temperatura)
	
	Quindi:
	- Q: Calore
	- U: Energia interna
	
	Ragioniamo ancora in termini energetici e sostituiamo la forza (che è pressione per comodità, stiamo comunque lavorando con fluidi) nell'integrale apposito: \[\int Fdy \implies -\int PA dy\]
	\noindent Questa cosa funziona perché supponendo un cilindro con un pistone che preme al suo interno, andiamo a modificare la sezione A del gas, creando inevitabilmente uno scambio di energia muovendo le molecole.
	Quindi per trasferire energia abbiamo il calore oppure la compressione del gas in un contenitore. Con quest'ultima dinamica, definiamo il primo principio della termodinamica: \[\Delta U = Q + L\]
	\noindent Ovvero che la variazione dell'energia interna è data dalla somma del calore con il lavoro eseguito.
	
	Esiste anche un altro tipo di calore, usato per ottenere il cambio di fase: \[Q = \Delta m L\]
	L è il calore latente, dove può essere di fusione (solido->liquido) o evaporazione (liquido->gas)
	
	In tal merito, per le trasformazioni ne abbiamo di più tipi:
	- Con temperatura T costante, è detta isoterma
	- Con calore Q a 0 non c'è scambio di calore con l'esterno, simile all'esempio del pistone. Si dice adiabatica.
	- Con Variazione di volume \Delta V = 0, si dice isocora
	- Con pressione P costante, si dice isobara
\end{comment}

%

\section{Statistica termodinamica, trasformazioni ed entropia}
cenni di statistica termodinamica, trasformazioni termodinamiche, entropia.





\begin{comment}
	Lez22 - Fisica 1
	
	--- Applicazioni della termodinamica
	Diciamo trasformazione quando si vanno a modificare le proprietà del liquido o gas con il quale stiamo lavorando. Queste possono essere reversibili o irreversibili. Quando una trasformazione avviene, si considera un calore specifico in base alla dinamica; ce n'è uno per il volume costante come anche per la pressione costante.
	Per comodità le trasformazioni vengono indicate con un piano cartesiano, dove sulle y sta la pressione, mentre sulle x sta il volume.
	
	- Trasformazione Isobara (Pressione costante)
	\[V = m\frac{RT}{P}\], le variabili rimangono solo V e T, mentre il resto son costanti.
	Il grafico rappresenta una retta verticale che si può spostare solo sul piano delle x. Infatti si modifica solo il volume.
	
	- Trasformazione Isocora (volume costante)
	\[P = m\frac{RT}{V}\]
	\noindent No variazione di volume? Allora il Lavoro = 0. Di conseguenza, \Delta E = Q + 0.
	Il grafico mostra una situazione uguale alla precedente, ma ad assi invertiti.
	
	Diciamo che \Delta E è una variabile di stato, quindi dipende esclusivamente dallo stato dell'energia iniziale e da quello della finale.
	
	- Trasformazione isoterma (Temperatura costante)
	Quindi PV = mRT è tutto costante. Essendo poi la pressione espressa come \[P = \frac{const}{V} \implies y = \frac{const}{x}\]
	\noindent Avremo un grafico di 1/x, che come ricorderai è un'iperbole, ma caghiamo solo il primo quadrante.
	Quindi P e V cambiano, ma in tal modo che il loro prodotto rimanga costante.
	Abbiamo visto prima, inoltre, che la variazione dell'energia dipende dalla variazione di temperatura: \[\Delta E = nc_V\Delta T\]
	\noindent Essendo in questo caso la temperatura costante, non ha variazione e \Delta T = 0, rendendo nulla anche la variazione di energia.
	
	- Trasformazione Adiabatica (Calore Q = 0)
	L'oggetto è completamente isolato dall'esterno e non c'è scambio di energia con esso, quindi nessun calore. \Delta E = 0 + L. La variazione di energia corrisponderà esattamente al lavoro compiuto.
	Vedremo poi che per questa trasformazione PV^\gamma = const, con \gamma = \frac{c_P}{c_V}
\end{comment}

\begin{comment}
	Lez24 - Fisica 1
	
	Abbiamo già visto che il calore si indica con Q = mc\Delta T, con c calore specifico del materiale.
	Per quanto riguarda i gas perfetti abbiamo due casi:
	- Volume costante: Q = mC_v\Delta T
	- Pressione costante Q = mC_p\Delta T
	
	Cerchiamo di capire cosa siano questi due calori specifici. Se voglio passare da una temperatura T_1 a una T_2 sarà necessario effettuare una trasformazione, la quale è esguibile mantenendo vostante il volume o la pressione. Dato PV = mRT e \Delta E = Q + L, vediamo i due casi:
	
	1. Volume costante: \Delta V = 0; \Delta E = Q + L = mC_v\Delta T
	// Ricorda che il lavoro qua è 0 perché L=P\Delta V e non essendoci variazione di volume, si annulla tutto.
	
	Abbiamo che l'energia totale è data da:
	\[E_{tot} = \frac{3}{2}NK_BT = \frac{3}{2}\frac{N}{N_A}RT = \frac{3}{2}mRT\]
	\noindent Dove costante di Boltsmann K_B = R/N_A e numero di moli m = \frac{N}{N_A}
	Essendo poi che la variazione dell'energia ha lavoro nullo, possiamo dire che la variazione dell'energia totale è data da \[\Delta E_{tot} = \frac{3}{2}mR\Delta T\]
	\noindent Ora, siccome la variazione di energia si può scrivere nei due modi visti prima, è possibile eguagliare quanto ottenuto, ricavando finalmente il calore specifico per i gas perfetti a volume costante: \[mC_v\Delta T = \frac{3}{2}mR\Delta T \implies C_v = \frac{3}{2}R\]
	
	2. Pressione costante: P = const; \Delta E = Q+L
	Sostituiamo alla formula della variazione di energia le definizioni delle sue componenti, ottenendo: \[\Delta E = Q+L \implies mC_v\Delta T = mC_p\Delta T - P\Delta V\]
	\noindent Inoltre, con un ragionamento simile a quello di prima, se PV = mRT, allora P\Delta V = mR\Delta T. Ciò è comodo perché consente di effettuare semplificazioni per ricavare il calore specifico a pressione costante: \[mC_v\Delta T = mC_p\Delta T - mR\Delta T \implies C_v = C_p - R\]
	\noindent Ora è semplice ottenere il calore specifico a pressione costante, no? \[C_p = C_v + R = \frac{3}{2}R + R = \frac{5}{2}R\]
	
	-- Trasformazione adiabatica: Q=0 // Boh qui ha fatto un casino, guarda il libro
	Con calore nullo abbiamo: \[\Delta E = 0 + L \implies mC_v\Delta T = -P\Delta V\]
	Attenzione però che la pressione qui non è costante, quindi non possiamo dire P\Delta V = mR\Delta T. Non sappiamo come cambiano pressione e volume, quindi si può passare alle derivate: \[d(PV) = d(mRT) \implies d(PV) + PdV = dT\cdot mR \implies PdV = mRdT - dPV\]
	
	Però sappi che alla fine devi raggiungere: \[\frac{dP}{P} = -\gamma\frac{dV}{V} = 0 \implies \log(P) + \gamma\log(V) = const \implies PV^\gamma = const\]
	
	Riassumendo, per l'adiabatica:
	- \Delta E = Q+L = mC_v\Delta T
	- PV = mRT
	- C_v = 3/2R, C_p = 5/2R
	- C_p - C_v = R
	- \gamma = C_p/C_v
	
	Per l'isocora: V=const, L=0, \Delta E = Q = mC_v\Delta T
	Per l'isoterma: T=const, \Delta E = 0, Q = -L, PV=mRT, P\Delta V = mRT, L = \int \frac{mRT}{V}dV
	
	--- Macchina termica
	Generano energia grazie al calore, funzionano in base alle trasformazioni. la logica è muoversi e tornare alla posizione iniziale ripetutamente.
	In queste condizioni cicliche e ricordando che l'energia è una variabile di stato, abbiamo che la variazione dell'energia interna è zero e Q=-L. Un esempio di questo evento sono i motori delle auto.
	Possiamo tuttavia anche dare del lavoro e creare calore. Questa cosa avviene con i caloriferi.
	
	Attenzione che quando si parla di macchine termiche una parte del calore viene dissipata. Quindi il lavoro è dato da \[L = |Q_c| - |Q_f|\]
	Ovvero la sottrazione fra il calore caldo e il calore freddo.
	Tuttavia quanto calore viene trasformato in lavoro? In condizioni ideali tutto, ma non è verosimile; questo valore è dato da e = \frac{L}{Q_c} = \frac{|Q_c| - |Q_f|}{|Q_c|} = 1 - \frac{|Q_f|}{|Q_c|} ed è detto efficienza. I motori migliori dissipano meno energia.
	Inoltre, togliendo i valori assoluti, possiamo ragionare sulla dinamica del calore totale in gioco, Dicendo: \[\frac{Q_f}{T_f} = - \frac{Q_c}{T_c} \implies \frac{Q_f}{T_f} + \frac{Q_c}{T_c} = 0\]
	
	La macchina termica migliore (non è realizzabile) è data dal ciclo di carnot, vede due trasformazioni adiabatiche e due trasformazioni isoterme. La sua particolarità è che l'efficienza è \[e = 1 - \frac{T_f}{T_c}\]
	
	La variazione del calore rispetto alla temperatura è detta Entropia e si indica con dS = \frac{dQ}{T}. In un processo reversibile rimane uguale a 0. Generalmente è comunque un valore sempre maggiore o uguale a zero.
\end{comment}