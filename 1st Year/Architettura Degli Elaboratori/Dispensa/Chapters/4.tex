\section{Modello di Von Neumann e Unità funzionali del calcolatore}
Il \textbf{Modello di Von Neumann} è un'architettura composta da tre componenti principali interconnesse mediante un dispositivo detto \textbf{BUS}. Si tratta del modello sul quale si basano tutte le architetture dei calcolatori. Le sue parti sono:
\begin{itemize}
	\item \textbf{Processore}: Atto all'elaborazione dei dati.
	\item \textbf{Memoria}: Atta al salvataggio dei dati.
	\item \textbf{Dispositivi I/O}: Periferiche varie come microfoni o tastiere.
\end{itemize}

%

\section{CPU - Central Processing Unit}
\subsection{CPU Cablata}
\subsection{CPU Microprogrammata}
\subsection{Microistruzioni della CPU}

%

\section{Metodi di I/O, Segnale Interrupt}

%

\section{Direct Memory Access, BUS e Arbitraggio}

%

\section{Stati di un processo}

%

\section{Pila e gestione Interrupt}

%

\section{Tipi di Memoria RAM}
Static RAM, Dynamic RAM, Banchi di memoria ed esercizi.

%

\section{Caratteristiche delle memorie con relativa gerarchia}

%

\section{Memoria Cache e Virtuale}

%

\section{Pipelining}

%

\section{Architettura LC-3}

%

\section{Modello CISC e RISC}

%

\section{Architetture parallele}