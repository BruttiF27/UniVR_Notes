\section{Concetto di temperatura}
La termodinamica studia il comportamento dei sistemi macroscopici e le trasformazioni energetiche che avvengono nei diversi stati della materia: solido, liquido e gassoso. In particolare, nel caso dei gas, tendono ad occupare completamente il volume del recipiente che li contiene.\par
Una delle grandezze fondamentali della termodinamica è la \textbf{temperatura}, la quale descrive lo stato energetico microscopico di un sistema. È legata all’energia cinetica media delle particelle che compongono la materia ed è misurata in gradi \textbf{Celsius} o \textbf{Kelvin}, dove in particolare il minimo teorico ottenibile, detto anche \textbf{Zero Assoluto}, è dato da $0K = -273.15^\circ C$.\newline

\noindent Una conseguenza diretta della temperatura è la \textbf{dilatazione termica}, un fenomeno che vede i corpi allargarsi o restringersi in base al loro valore di temperatura. Supponiamo una sbarra di metallo con una data lunghezza $L$; scaldandolo, aumenta l'agitazione delle particelle, le quali dilatano il materiale. La variazione della lunghezza è dunque proporzionale alla variazione di temperatura e si definisce come: \[\Delta L = \alpha L_i\Delta T \implies L_f = L_i(1+\alpha\Delta T)\]
\noindent Dove $\alpha$ è il \textbf{coefficiente di dilatazione termica}, $L_i$ la lunghezza iniziale e $\Delta T$ la \textbf{variazione di temperatura}. In particolare, il coefficiente è un valore che dipende dalla composizione del materiale e si presenta in forma lineare, superficiale e volumica per le tre rispettive dimensioni. In questi ultimi termini, infatti, si eleva tutto al cubo: \[L_f^3 = (L_i+\alpha\Delta T)^3\]
\noindent In ogni caso, gli $\alpha$ sono numeri dal valore molto basso, abbastanza da poterli non considerare quando sono al quadrato o al cubo, rendendoli uguali a zero. Ciò ci permette di sviluppare il cubo come un binomio:
\begin{equation}
	\begin{split}
		L_f^3 &= (L_i+\alpha\Delta T)^3\\
		L_f^3 &= L_i^3 + 3L_i^2(\alpha\Delta T) + 3L_i(\alpha\Delta T)^2 + (\alpha\Delta T)^3\\
		L_f^3 &= L_i^3 + 3\alpha L_i^2\Delta T + 3\alpha^2 L_i(\Delta T)^2 + \alpha^3(\Delta T)^3\\
		L_f^3 &= L_i^3 + 3\alpha L_i^3\Delta T\\
		L_f^3 &= L_i^3(1+3\alpha\Delta T)
	\end{split}
\end{equation}
\noindent Quindi, fatta molto più semplicemente; il coefficiente di dilatazione termica cubica è \textbf{tre volte quello lineare}, mentre per la superficie parliamo di $2\alpha$.

%

\section{Legge di stato dei gas perfetti}
Iniziamo ora a discutere dei gas; nello studio e negli esercizi faremo uso del concetto di \textbf{gas perfetto}, un modello ideale in cui le molecole non interagiscono tra loro se non tramite urti elastici. Sebbene questo modello non descriva esattamente i gas reali, a basse pressioni e alte temperature il loro comportamento si avvicina a quello ideale. In tali condizioni, le grandezze macroscopiche del gas sono legate dalla seguente \textbf{legge di stato dei gas perfetti}: \[PV = nRT\]
\noindent Nella formula compaiono la \textbf{pressione} $P$, il \textbf{volume} $V$, il \textbf{numero di moli} $n$, la \textbf{costante universale dei gas} $R = 8.314[\frac{J}{mol\cdot K}]$ e la \textbf{temperatura assoluta} $T$. In alternativa, la legge di stato può essere scritta nella forma microscopica \[PV = Nk_BT\]
\noindent dove $k_B = 1.3806\cdot 10^{-23}[J/K]$ è la \textbf{costante di Boltzmann} ed $N$ il \textbf{numero di molecole}.\newline

\noindent Con ciò possiamo introdurre la \textbf{teoria cinetica dei gas}, la quale fornisce una descrizione microscopica del loro comportamento, modellando le molecole come particelle puntiformi che obbediscono alle leggi della meccanica classica e quindi alle leggi di Newton. Tuttavia, poiché un gas è costituito da un numero enorme di molecole, non è possibile descriverne il moto in modo deterministico, ragion per cui si ricorre a strumenti statistici e a grandezze medie.\par
Consideriamo anzitutto un gas contenuto in un recipiente cubico. Una singola molecola, muovendosi all’interno del recipiente, urta le pareti. Analizzando l’urto con una parete perpendicolare all’asse $x$, la componente $v_x$ della velocità cambia segno, mentre il suo modulo resta invariato. La variazione della quantità di moto associata a un urto è quindi: \[\Delta p_x=-2mv_x\]
\noindent Poiché la forza è definita come variazione della quantità di moto nel tempo $F\Delta t = \Delta p$, l’urto della molecola con la parete genera una forza. La pressione esercitata dal gas nasce dalla somma delle forze esercitate da tutte le molecole che urtano le pareti del recipiente. In tal merito, infatti, non conoscendo la velocità delle singole molecole e volendo rappresentare un modello completo, si introducono i valori medi. Per simmetria del moto, le tre componenti del vettore della velocità contribuiscono in modo equivalente, facendo valere la relazione: \[\overline{v^2} = 3\overline{v^2_x}\]
\noindent Tenendo conto del numero totale di molecole $N$ e del volume $V$, si ottiene un’espressione per la pressione: \[P = \frac{N}{3V}m\overline{v^2} \implies PV = \frac{N}{3}m\overline{v^2}\]
\noindent Ragionando poi in termini energetici con la velocità media, possiamo dire che l'energia cinetica media di una particella è espressa dalla formula $\frac{1}{2}m\overline{v^2}$. Riconduciamo a questa forma tramite alcuni passaggi algebrici: \[PV = \frac{N}{3}\cdot\frac{1}{2}m\overline{v^2} \implies \frac{2N}{3}\left(\frac{1}{2}m\overline{v^2}\right) = Nk_BT \implies \frac{1}{2}m\overline{v^2} = \frac{3}{2}k_BT\]
\noindent Ne segue che la temperatura è direttamente proporzionale all’energia cinetica media delle molecole e rappresenta quindi una grandezza macroscopica che descrive lo stato energetico microscopico del gas.

%

\section{Concetto di calore}
Supponiamo adesso due corpi con rispettive temperature $T_1, T_2$. Quando questi sono messi a contatto scambiano energia fino a raggiungere una temperatura comune. Tale stato prende il nome di \textbf{equilibrio termico} ed è caratterizzato dall’assenza di ulteriori scambi energetici. Questo enunciato è detto \textbf{principio zero della termodinamica}.\par
Definiamo lo scambio di energia fra i sistemi come \textbf{calore} $Q$, la cui unità di misura nel Sistema Internazionale è il \textbf{Joule}. Avviene quando sussiste una differenza di temperatura e fluisce spontaneamente da un corpo più caldo a uno più freddo. Si definisce con la relazione: \[Q = mc\Delta T\]
\noindent Dove $m$ è la massa del corpo, $c$ il calore specifico del materiale e $\Delta T$ la variazione di temperatura. In particolare, il \textbf{calore specifico} è un valore che dipende dal materiale ed esprime la quantità di calore necessaria per aumentare di un grado la temperatura di un’unità di massa. In un sistema isolato, durante uno scambio di calore, vale il principio di conservazione dell’energia: \[Q_{ceduto} + Q_{assorbito} = 0\]
\noindent Esistono più meccanismi fondamentali per il trasferimento del calore: \begin{itemize}
	\item \textbf{Conduzione}: Trasferimento di calore attraverso due mezzi materiali senza modifiche alla materia.
	\item \textbf{Convezione}: Trasferimento di calore associato al moto di un fluido.
	\item \textbf{Irraggiamento}: Trasferimento di energia mediante onde elettromagnetiche, che non richiedono un mezzo materiale.
\end{itemize}
\noindent Diamo nome all'energia totale associata al moto e alle interazioni microscopiche delle particelle di un sistema: \textbf{Energia interna} $U$, la quale può variare sia tramite scambio di calore, sia tramite lavoro meccanico, come nel caso della compressione o espansione di un gas. In tal merito possiamo introdurre il \textbf{primo principio della termodinamica}, affermando che la variazione dell'energia interna è data dalla somma del calore con il lavoro eseguito: \[\Delta U = Q + L\]
\noindent Dato un certo calore, è possibile andare a modificare lo stato fisico della materia. Chiamiamo questo evento \textbf{cambio di fase}, dove la temperatura resta costante, ed il calore scambiato modifica il corpo. In questo caso vale la relazione: \[Q = \Delta mL\]
\noindent Dove $L$ è detto \textbf{calore latente} e può essere di fusione o evaporazione. Nella prossima sezione approfondiremo nel dettaglio le dinamiche delle seguenti trasformazioni termodinamiche: \begin{itemize}
	\item \textbf{Isoterma}: A temperatura costante.
	\item \textbf{Isocora}: A volume costante.
	\item \textbf{Isobara}: A pressione costante.
	\item \textbf{Adiabatica}: In assenza di scambio di calore.
\end{itemize}

%

\section{Trasformazioni termodinamiche}
Si definisce \textbf{trasformazione termodinamica} un processo mediante il quale un sistema passa da uno stato di equilibrio iniziale a uno stato di equilibrio finale, variando una o più grandezze di stato. Una trasformazione può essere reversibile o irreversibile, a seconda che il sistema possa essere riportato allo stato iniziale senza lasciare effetti sull’ambiente esterno.\par
Nel caso dei gas perfetti, lo stato del sistema è descritto dalle grandezze di stato pressione $P$, volume $V$ e temperatura	$T$, legate dalla legge di stato. Inoltre, il valore del calore specifico $c$ dipende dal tipo di trasformazione effettuata; definiamo infatti $c_V$ e $c_P$ calore specifico a \textbf{volume costante} e \textbf{pressione costante} rispettivamente, per i quali valgono le relazioni: \[\begin{cases}
	Q = nc_V\Delta T\\
	Q = nc_P\Delta T
\end{cases}\]
\noindent Il diverso vincolo modifica il modo in cui l’energia fornita al sistema viene ripartita tra variazione dell'energia interna e lavoro meccanico, usando come base di definizione il primo principio della termodinamica. Infine, il tipo di trasformazione sarà rappresentato sul piano cartesiano, con la pressione in ordinata $y$ e il volume in ascissa $x$.\newline

\noindent Come già menzionato, esistono quattro tipi di trasformazione termodinamica: \begin{itemize}
	\item \textbf{Trasformazione isocora}\par
	\noindent In una trasformazione isocora il volume rimane costante, mentre variano pressione e temperatura, quindi: \[P = \frac{mRT}{V}\]
	\noindent Poiché non vi è variazione di volume, il lavoro compiuto dal gas è nullo e quindi, per il primo principio della termodinamica la variazione di energia interna è uguale al calore: \[\Delta U = Q\]
	\noindent Per un gas perfetto monoatomico, l'energia interna è data da: \[U = \frac{3}{2}nRT \implies \Delta U = \frac{3}{2}nR\Delta T\]
	\noindent Ed in quanto $\Delta U = Q = nc_V\Delta T$ possiamo uguagliare quanto ottenuto, ricavando il calore specifico a volume costante: \[nc_V\Delta T = \frac{3}{2}nR\Delta T \implies c_V = \frac{3}{2}R\]
	\noindent Infine, nel piano cartesiano si rappresenta con una retta verticale.
	\item \textbf{Trasformazione isobara}\par
	\noindent In una trasformazione isobara la pressione del gas rimane costante. Dalla legge di stato dei gas perfetti segue: \[V = \frac{nRT}{P}\]
	\noindent Dunque, volume e temperatura sono direttamente proporzionali, mentre gli altri fattori sono valori costanti. Il lavoro compiuto è infatti non nullo: \[L = -P\Delta V\]
	\noindent Il primo principio della termodinamica assume la forma: \[\Delta U = Q+L \implies nc_V\Delta T = nc_P\Delta T - P\Delta V\]
	\noindent Dalla legge dei gas perfetti abbiamo: \[PV = nRT \implies P\Delta V = nR\Delta T\]
	\noindent Sostituendo, possiamo infine ricavare il calore specifico a pressione costante: \[nc_V\Delta T = nc_P\Delta T - nR\Delta T\implies c_P = c_V + R = \frac{5}{2}R\]
	\noindent Nel piano cartesiano si rappresenta con una retta orizzontale.
	\item \textbf{Trasformazione isoterma}\par
	\noindent In una trasformazione isoterma la temperatura del gas rimane costante. Di conseguenza, dalla legge di stato: \[PV = \text{const}\]
	\noindent Inoltre, poiché l’energia interna di un gas perfetto dipende solo dalla temperatura, in questa trasformazione si ha: \[\Delta U = 0\]
	\noindent Dal primo principio della termodinamica segue quindi che il calore scambiato è uguale al lavoro compiuto: \[Q = L\]
	\noindent Nel piano cartesiano si rappresenta tramite un'iperbole equilatera.
	\item \textbf{Trasformazione adiabatica}\par
	\noindent In una trasformazione adiabatica non vi è scambio di calore con l’ambiente esterno, dunque: \[Q = 0\]
	\noindent Ed il primo principio della termodinamica pone la variazione dell'energia interna uguale al lavoro compiuto: \[\Delta U = L\]
	\noindent Nei gas perfetti, una trasformazione adiabatica reversibile soddisfa la relazione: \[PV^\gamma = \text{const},\quad \gamma = \frac{c_P}{c_V}\]
	Durante una trasformazione adiabatica una variazione di volume comporta una variazione di temperatura e nel piano cartesiano è rappresentata come una curva adiabatica definita dalla relazione appena vista.
\end{itemize}

%

\section{Applicazioni della termodinamica ed entropia}
Una prima applicazione della termodinamica è data dalle \textbf{macchine termiche}, dispositivi che trasformano calore in lavoro meccanico operando ciclicamente tramite trasformazioni termodinamiche. Poiché il sistema ritorna allo stato iniziale, la variazione di energia interna su un ciclo completo è nulla: \[\Delta U = 0 \implies Q = -L\]
\noindent In particolare, durante un ciclo, il sistema: \begin{itemize}
	\item Assorbe un calore $Q_c$ da una sorgente calda.
	\item Cede un calore $Q_f$ a una sorgente fredda.
	\item Produce un lavoro: $L = Q_c - |Q_f|$
\end{itemize}
\noindent Tuttavia non tutto il calore viene trasmesso; una parte viene infatti dissipata, ed in tal merito definiamo \textbf{efficienza} della macchina: \[e = \frac{L}{Q_c} = 1 - \frac{|Q_f|}{Q_c}\]
\noindent La macchina termica ipoteticamente perfetta è data dal \textbf{ciclo di Carnot}, composto da due trasformazioni isoterme e due adiabatiche. Vede la sua efficienza dipendere esclusivamente dalle temperature delle sorgenti: \[e = 1 - \frac{T_f}{T_c}\]
\noindent Infine, definiamo la variazione del calore rispetto alla temperatura come la grandezza fisica di \textbf{entropia}, espressa come: \[dS = \frac{dQ}{T}\]
\noindent La quale, in particolare: \begin{itemize}
	\item Nei processi reversibili: $\Delta S = 0$
	\item Nei processi reali: $\Delta S \geq 0$
\end{itemize}

%

\section{Esercizi svolti}
\begin{comment}
	```Esempio di esercizio
	Abbiamo un lingotto di metallo di m=0.05Kg ed è scaldato a T_m=200°C, per poi essere immerso in 0.04Kg di acqua, a temperatura T_a=20°C. La temperatura finale T_f = 22.4°C; trovare il calore specifico.
	
	Il lingotto scambia calore con l'acqua, naturalmente, per raggiungere l'equilibrio che presenta la T_f = 22.4°C, uguale per entrambi. Da una parte si dona, dall'altra si prende. Scriviamo quindi \[Q_m = -Q_a\]
	\noindent Da qua possiamo espandere l'equazione con la definizione per oggetto:
	\begin{equation}
		\begin{split}
			m_{met}c_{met}(T_f - T_{met, i}) &= -m_{acq}c_{acq}(T_f - T_{acq, i})\\
			c_{met} = -\frac{m_{acq}c_{acq}(T_f - T_{acq, i})}{m_{met}(T_f - T_{met, i})}\\
		\end{split}
	\end{equtation}
	```
\end{comment}