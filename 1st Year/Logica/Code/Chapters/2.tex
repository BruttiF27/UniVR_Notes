Cominciamo lo studio effettivo della Logica Matematica attraverso la \textbf{Logica Proposizionale}. Fondamentalmente è una palestra da utilizzare come trampolino di lancio verso argomenti più complessi, come la Logica del Primo Ordine, che verrà vista in seguito.\par\quad
Al termine del capitolo bisognerà saper valutare la veridicità di una formula e saper utilizzare la deduzione naturale per derivare sentenze.

\section{Sintassi}
In questo ambito il ragionamento si compone di entità linguistiche collegate dalla relazione "segue da". Queste si chiamano \textbf{Sentenze}. Come visto nell'introduzione possono essere minimali o composte, ma entrambe, se esprimono un pensiero completo si dicono \textbf{dichiarative}. Queste devono necessariamente avere un valore di vero o falso per essere considerate tali.
\begin{eg}
    \textbf{Tipi di sentenze}\par
    \begin{itemize}
        \item Sentenza minimale: $\phi$\par
        Entità irriducibile.
        \item Sentenza composta: $\phi \lor \psi$\par
        Entità riducibile.
    \end{itemize}
\end{eg}
Ogni linguaggio logico è basato su un determinato \textbf{Alfabeto}, determinante i simboli utilizzabili in esso. Infatti, i linguaggi sono definiti come $L \subseteq A^*$, dove quest'ultimo insieme è quello delle \textbf{Stringhe}, ovvero tutte le sequenze di simboli generabili tramite l'alfabeto. \par\quad
Le stringhe appartenenti ad $L$ si dicono sintatticamente corrette. In particolare, definiamo ora l'alfabeto della logica proposizionale che consente la loro composizione:
\begin{definition}
    \textbf{Alfabeto} $L^{PROP}$\par
    L'alfabeto della Logica Proposizionale è definito tramite tre unità sintattiche distinte:
    \begin{itemize}
        \item Insieme di simboli proposizionali $AT^{PROP}$, i cui elementi saranno indicati con lettere greche.
        \item Connettivi $\land$, $\lor$, $\to$, $\bot$.
        \item Simboli ausiliari "(" e ")". Non variano il senso della frase, ma aiutano a renderla più leggibile.
    \end{itemize}    
\end{definition}
Abbiamo constatato che il linguaggio proposizionale è costituito dall'insieme delle stringhe sintatticamente corrette. Si indica con $PROP$ e si definisce induttivamente come segue:
\begin{definition}
    \textbf{Insieme delle proposizioni PROP}
    \begin{itemize}
        \item $\bot \in PROP$;
        \item Se p è un simbolo proposizionale, allora $p \in PROP$;
        \item Se $\phi, \psi \in PROP$, allora valgono le seguenti relazioni:\par
        $(\phi \land \psi) \in PROP$, $(\phi \lor \psi) \in PROP$, $(\phi \implies \psi) \in PROP$, $(\neg\phi) \in PROP$.
        \item Null'altro appartiene all'insieme PROP.
    \end{itemize}
\end{definition}
Per renderci la vita un pochino più semplice sono state ideate delle convenzioni sulla scrittura delle proposizioni; si possono infatti omettere alcune parentesi nelle formule e fissare un ordine di precedenza tra i connettivi:
\begin{itemize}
    \item Si omettono le parentesi più esterne di una formula, così al posto di scrivere "$(\phi \land \psi)$", si scrive solo "$\phi \land \psi$".
    \item Il connettivo $\neg$ è il più forte e associa il primo simbolo presente a destra, perciò "$\neg\phi \lor \psi$" sarebbe uguale a "$(\neg\phi) \lor \psi$".
    \item Si assume che $\land$ e $\lor$ siano connettivi più forti rispetto a $\implies$ e $\iff$, quindi scrivere "$\phi_1 \land \psi_1 \implies \phi_2 \lor \psi_2$" è equivalente a intendere "$(\phi_1 \land \psi_1) \implies (\phi_2 \lor \psi_2)$".
    \item I connettivi $\implies, \land, \lor$ associano a destra, perciò scrivere "$\phi \implies \psi \implies \sigma$" equivale a dire: "$(\phi \implies (\psi \implies \sigma))$", mentre "$\phi \land \psi \lor \sigma$" sta per "$(\phi \land (\psi \lor \sigma))$".
\end{itemize}
Anche la logica matematica si basa sulla teoria degli insiemi; questo significa che valgono i teoremi inerenti, ma anche il funzionamento delle proprietà di un insieme. Diciamo infatti
\begin{prop}
    \textbf{Proprietà P su un insieme A}\par
    Se abbiamo un insieme $P$ ed è sottoinsieme proprio o improprio di un altro $A$ abbiamo che:
    \begin{center}
        $P \subseteq A \implies$ proprietà $P$ su $A$.
    \end{center}
    Di conseguenza, se abbiamo un elemento $a \in A$ possiamo dire che gode della proprietà se vale anche la relazione $a \in P$.
\end{prop}
Finora abbiamo visto i soli elementi; ora di vedere cosa si può fare per dimostrare gli enunciati. Gli strumenti prendono la forma di \textbf{Induzione strutturale} e \textbf{Ricorsione primitiva}.
\begin{theorem}
    \textbf{Induzione strutturale su PROP}\par
    Considera una proprietà $P \subseteq PROP$. Supponendo veri i punti:
    \begin{itemize}
        \item $\forall \phi \in AT$, vale $P[\phi]$\par
        Se la proprietà vale per ogni formula atomica...
        \item $\forall \phi, \psi \in PROP$, valgono $P[\phi], P[\psi]$\par
        E che se la proprietà vale per ogni formula in $PROP$...
    \end{itemize}
    Allora varranno le relazioni $P[\phi \land \psi], P[\phi \lor \psi], P[\phi \implies \psi]$. Di conseguenza vale che $\forall \phi \in PROP$, vale $P[\phi]$.
\end{theorem}
\begin{eg}
    \textbf{Costruzione induttiva della formula} $P[\phi \implies (\phi \lor \psi)]$\par
    Fondamentalmente devi scomporre l'intera formula. Parti dalle formule atomiche
    \begin{center}
        $P[\phi], P[\psi]$
    \end{center}
    Entrambe valgono, quindi puoi prendere una o l'altra. Puoi inoltre introdurre l'implicazione.
    \begin{center}
        $P[\phi] \implies P[\phi \lor \psi] := P[\phi \implies (\phi \lor \psi)$
    \end{center}
    E hai finito.
\end{eg}
Parliamo ora di ricorsione. Premetto che tutte le volte in cui si usa un linguaggio definito induttivamente, è possibile descriverne gli elementi anche ricorsivamente. Sempre. Questo algoritmo genera funzioni in termini di sé stesse all'interno della propria definizione. Risulta più semplice con l'esempio del fattoriale:
\begin{eg}
    \textbf{Funzione fattoriale}\par
    \begin{center}
        $fatt = 
        \begin{cases}
            0! = 1\\
            n! = n\times(n-1)!
        \end{cases}$
    \end{center}
    Come si può vedere, la funzione comprende ogni singolo caso che si può presentare. Se alla funzione è dato il valore $0$ sei nel caso base, mentre in qualunque altra istanza sostituisci il numero dato alla $n$ e poi svolgi i calcoli.
\end{eg}
Terminiamo con due ultimi concetti sintattici definiti ricorsivamente:
\begin{definition}
    \textbf{Rango di una formula}\par
    Definiamo $r : PROP \to \N$ ricevente in input una formula, il valore del suo rango. Si tratta di una quantità statistica utile al fine di vedere una definizione per ricorsione.
    \begin{itemize}
        \item $r[\phi] = 0$, se $\phi \in AT$
        \item $r[(\neg\phi)] = 1 + r[\phi]$
        \item $r[(\phi \circ \psi)] = 1 + max\{r[\phi], r[\psi]\}$, con $\circ \in \{\lor, \land, \implies\}$
    \end{itemize}
\end{definition}

\begin{definition}
    \textbf{Sottoformula}\par
    Si definisce sottoformula $sub: PROP \to 2^{PROP}$ la funzione che prende in input una formula proposizionale e ne restituisce l'insieme delle parti. Compone l'insieme di tutte le sottoformule presenti in una certa formula data.
    \begin{itemize}
        \item $sub[\phi] = \{\phi\}$, se $\phi \in AT$
        \item $sub[(\neg\phi)] = \{(\neg\phi)\} \cup sub[\phi]$
        \item $sub[(\phi \circ \psi)] = {(\phi \circ \psi)} \cup sub[\phi] \cup sub[\psi]$, con $\circ \in \{\land, \lor, \implies\}$
    \end{itemize}
\end{definition}

%

\section{Semantica}

% ----- Continua da qua -----

\section{Sistemi Deduttivi}
\section{Correttezza e Completezza}
\section{Esercizi}



\section{Appunti}
\subsection{Semantica}
Finora abbiamo studiato il valore sintattico delle frasi, ma non dimentichiamo che, come nella lingua italiana, una composizione di frasi può avere un significato; qui nel linguaggio logico è necessario analizzare gli enunciati per determinarne eventualmente verità o falsità. Come fare?
\begin{definition}
    \textbf{Valutazione di verità}\par
    Per comodità indichiamo il valore di verità con 1 e quello di falsità con 0; quindi sia la funzione $V : PROP \to \{0, 1\}$, tale che valgano le seguenti relazioni:
    \begin{enumerate}
        \item $V[\bot] = 0$
        \item $V[\neg\psi] = 1 \iff V[\psi] = 0$
        \item $V[\psi \land \phi] = 1 \iff V[\psi] = 1 \land V[\phi] = 1$
        \item $V[\psi \lor \phi] = 1 \iff V[\psi] = 1 \lor V[\phi] = 1$
        \item $V[\psi \to \phi] = 1 \iff V[\psi] = 0 \lor V[\phi] = 1$
        \item $V[\psi \iff \phi] = 1 \iff V[\psi] = V[\phi]$
    \end{enumerate}
\end{definition}
Queste relazioni sono riassumibili in tabelle di verità, molto utili in ogni caso ci si presenti davanti. Ora, considera le variabili che abbiamo usato finora; queste possono indicare composizioni di formule al loro interno, quindi non essere atomiche, si definiscono infatti \textbf{metavariabili}, e per conoscere il loro valore semantico è necessario valutare i loro simboli proposizionali. Ciò avviene grazie alla \textbf{Valutazione atomica}, la base per definire tutte le valutazioni di verità.
\begin{definition}
    \textbf{Valutazione atomica}\par
    Sia la funzione $V : AT \to \{1, 0\}$ tale che $v[\bot] = 0$\newline

    Seguirà la valutazione come risultato:\par
    $[\cdot]_v : PROP \to \{1, 0\}$ Tale che $[\alpha]_v = v[\alpha]$ per $\alpha \in AT$.
\end{definition}
Si possono presentare alcuni casi notevoli davanti alla valutazione del significato di un enunciato, come:
\begin{definition}
    \textbf{Concordanza}\par
    Dato un insieme finito di proposizioni P, se due valutazioni sono uguali, si dicono concordanti su P. Nel caso in cui una formula proposizionale è vera in ogni caso, si dice \textbf{Tautologia} e si indica con $[\alpha]_v = 1$.
\end{definition}
Negli esercizi verranno utilizzate le seguenti scritture, con $\alpha\in PROP$ generica:
\begin{itemize}
    \item $\vDash \alpha:$ Tautologia.
    \item $\vdash \alpha:$ Teorema.
\end{itemize}
Per dimostrare una tautologia è necessario controllare che sia vera per ogni singolo caso; se in un'istanza si rivela falsa, non è tautologia.
\begin{theorem}
    \textbf{Tautologie Notevoli} - Inserire formule
    \begin{itemize}
        \item Legge di De Morgan
        \item Negazione Involutiva
        \item Legge Commutativa
        \item Legge Distributiva
        \item Legge Associativa
    \end{itemize}
\end{theorem}
Questo concetto e ragionamento può essere esteso anche ad un insieme determinato di proposizioni, che chiameremo $\Gamma$. Definiamo quindi:
\begin{definition}
    \textbf{Soddisfacibilità di una formula e conseguenza logica}\par
    La valutazione di verità si estende ad un insieme di proposizioni arbitrario $\Gamma$ tale che: $[\Gamma]_v = 1 \iff \forall\alpha \in \Gamma$ | $[\alpha]_v = 1$\footnote{Inoltre, se $[\Gamma]_v \neq 1$, significa che c'è almeno una valutazione dove risulta falsa, non che lo sono tutte. Se si afferma invece che $[\Gamma]_v = 0$, tutte le valutazioni saranno false.}.\newline

    Da questo insieme di proposizioni è possibile anche trarre delle conclusioni, dette \textit{conseguenze logiche}. Si dice che una proposizione $\psi$ segue logicamente da un insieme $\Gamma$ se la verità delle proposizioni in esso contenute implica la verità di $\psi$; formalmente:\par
    Dato l'insieme $\Gamma$ di formule, si dice $\psi$ conseguenza logica delle formule di $\Gamma$ e si scrive $\Gamma \vDash \psi$. Se $[\Gamma]_v = 1 \to [\psi]_v = 1$. Se l'insieme vuoto implica una valutazione positiva di $\alpha$, abbiamo una tautologia.
\end{definition}
Diremo infine che una formula $\psi$ è detta \textit{soddisfacibile} se esiste una valutazione \textit{v} per cui risulti vera, altrimenti è \textit{insoddisfacibile}. A sua volta, questo concetto si può applicare agli insiemi di proposizioni. $\Gamma$ si dice soddisfacibile quando una sua formula $\psi$ è sempre vera data una certa valutazione, altrimenti è insoddisfacibile.
\begin{eg}
    \textbf{Soddisfacibilità di un insieme $\Gamma$}\par
    \begin{itemize}
        \item Se $\{p\} \in \Gamma$, $[p]_v = 1$ e quindi l'insieme è soddisfacibile.
        \item Se $\{p, \neg p\} \in \Gamma$, non tutte le valutazioni di \textit{p} sono vere in quanto contraddittorie e quindi l'insieme non è soddisfacibile. 
    \end{itemize}
\end{eg}

La \textbf{Sostituzione} è invece un'utile tecnica per sostituire variabili sotto un unico simbolo e rendere la scrittura più leggibile e semplice. Mettiamo di avere diverse variabili, tutte scritte in modo diverso, ma alcune di queste hanno lo stesso identico valore di altre; qui è possibile usufruire di questa tecnica.
\begin{definition}
    \textbf{Sostituzione Proposizionale}\par
    Sia la formula: $\phi = ((p1 \to (p5 \lor p1)) \land p3)$. Qui è possibile sostituire tutti i p1 con $\psi$. Il passaggio intermedio si scrive $\phi[\psi/p1]$\footnote{Significa che $\psi$ sostituirà il valore p1.} e risulta con:\par
    $\phi = ((\psi \to (p5 \lor \psi)) \land p3)$.\newline

    Se un termine nella formula presa in esame non compare affatto, è ovvio che la sostituzione non avrà alcuno scopo.
\end{definition}
\begin{theorem}
    \textbf{Proprietà della sostituzione}\par
    Siano $\psi_1, \psi_2$ due formule e $p \in AT$. Se $\vDash \psi_1 \iff \psi_2$, prendendo una formula generica $\phi$ otteniamo che:\par
    $\vDash \phi[\psi_1/p] \iff \phi[\psi_2/p]$.\newline

    Ci è quindi possibile tirar fuori dal cappello magico nuove formule, perché questa scrittura è equivalente alla prima.
\end{theorem}

Formalizziamo infine il concetto di \textbf{Relazione di equivalenza}. Molto semplicemente, i simboli $\leftrightarrow, \iff$ Rappresentano rispettivamente equivalenza semantica e sintattica; sono una scrittura compatta del seguente concetto: $\phi \iff \psi = (\phi \to \psi) \land (\psi \to \phi)$. Restituisce il valore di vero solamente quando entrambe le formule sono vere.\footnote{Ciò implica anche che se le formule sono equivalenti e la valutazione di una delle due è vera, lo sarà per forza anche l'altra.}\par
Una relazione di equivalenza R si scrive $\approx$ ed è tale se vengono rispettati questi tre criteri:
\begin{enumerate}
    \item \textit{Riflessività}; $\forall a \in A$ | $aRa$.
    \item \textit{Transitività}; $\forall a, b, c \in A$ | $(aRb \land bRc) \to aRc$.
    \item \textit{Simmetria}; $\forall a, b \in A$ | $aRb \to bRa$.
\end{enumerate}
Diremo infine che date due formule $\phi, \psi \in PROP$, se $\vDash \phi \leftrightarrow \psi$, si scriverà $\phi \approx \psi$.

%

\subsection{Lavorare con la Logica Proposizionale}
Prima di iniziare a lavorare, è necessario introdurre il concetto di \textbf{Sistema Deduttivo}. Si tratta di un costrutto che fornisce una struttura formale di regole inferenziali che permettono di derivare una conclusione a partire da alcune premesse.\par\quad
Che cosa consente tutto ciò? Ora possiamo effettuare lavori di deduzione per verificare la veridicità di una formula data.

%

\subsection{Induzione e Ricorsione}
Le dimostrazioni per \textbf{induzione strutturale} e \textbf{ricorsione primitiva} sono concetti fondamentali che ti verranno sicuramente richiesti all'esame; quindi vediamoli nello specifico.
\begin{theorem}
    \textbf{Principio di Induzione strutturale sull'insieme PROP}\par
    Considera una proprietà P da provare su PROP, quindi $P \subseteq PROP$ supponendo veri ambo gli enunciati:
    \begin{itemize}
        \item $\forall\psi\in AT$, vale $P[\phi]$. 
        \item $\forall\psi,\phi\in PROP$, valgono $P[\psi], P[\phi]$
    \end{itemize}
    Varranno quindi le relazioni $(\psi \land \phi), (\psi \lor \phi), (\psi \to \phi)$.
\end{theorem}
Allora, noi dobbiamo provare la formula P su PROP. Supponiamo che per ogni formula atomica e per ogni elemento di PROP valga la proprietà. Ne consegue che sarà valida anche su $(\psi \land \phi), (\psi \lor \phi), (\psi \to \phi)$, comportando al fatto che $\forall\psi\in PROP$ valga $P[\psi]$.
\begin{proof}
    Basta dimostrare che $PROP \subseteq P$. Nota che P gode delle tre proprietà di PROP e di conseguenza, per minimalità di PROP, si ha la dimostrazione.
\end{proof}
Questo era un esempio teorico, vediamo un esercizio:
\begin{eg}
    Si dimostri per induzione strutturale la formula $p_2\to(p_3\lor p_2)$ che gode della proprietà P\par
    \begin{enumerate}
        \item $P[p_2]$, $P[p_3 \lor p_2]$
        \item $P[p_2 \to (p_3 \lor p_2)]$ - Finito.
    \end{enumerate}
\end{eg}
Se si ha un linguaggio definito tramite induzione è sicuramente possibile utilizzare anche la ricorsione primitiva, che vedremo adesso. Se hai prestato attenzione alle lezioni di programmazione, ti sarà tutto molto più comprensibile.\par
La ricorsione si occupa di partire da una determinata formula per arrivare ad un caso base; vediamo un esempio messo a paragone con il linguaggio C:
\begin{eg}
    Cifre pari nel numero $\alpha = 25781$\par
    Definiamo la proprietà P, la quale ricevuto un numero $\alpha$, riporta quante sono le sue cifre pari.
    \begin{enumerate}
        \item $P[\alpha] = n = P[\alpha]$, se $\alpha \% 2 \neq 0$.
        \item $P[\alpha] = n = P[\alpha] + 1$, se $\alpha \% 2 = 0$.
        \item $P[\alpha] = \alpha / 10$, se $\alpha > 10$
        \item $P[\alpha] = n$, se $\alpha < 10$
    \end{enumerate}
\end{eg}
In C, questa funzione ricorsiva riceverebbe in input il numero 25781 e mediante l'operazione "modulo 2" è possibile capire se un numero è pari o dispari dipendentemente dal resto della divisione per 2. Se il numero è maggiore di 10, significa che ci sono altre cifre da controllare, quindi ritorna il valore del numero diviso 10 per passare alla successiva a sinistra. Se il numero è minore di 10, dopo l'ultimo controllo ritornerà il valore di n, ovvero le cifre pari di cui abbiamo bisogno.\par\quad
Descrivere il caso base appropriatamente è di fondamentale importanza; poiché senza di esso, la ricorsione andrebbe avanti ad infinitum.
\begin{proof}
    Si trovi quante cifre pari sono contenute nel numero $\alpha = 25781$
    \begin{enumerate}
        \item $25781 \%2 \neq 0 \to n = 0$, $\alpha = \alpha / 10$
        \item $2578 \%2 = 0 \to n = 0+1$, $\alpha = \alpha / 10$
        \item $257 \%2 \neq 0 \to n = 1+0$, $\alpha = \alpha / 10$
        \item $25 \%2 \neq 0 \to n = 1+0$, $\alpha = \alpha / 10$
        \item $2 \%2 = 0 \to n = 1+1$, $\alpha = \alpha / 10$, arrivati al caso base.
    \end{enumerate}
    Numeri pari $n \in \alpha = 2$
\end{proof}

\subsection{Tabelle di Verità}

%

\subsection{Deduzione Naturale}
La \textbf{Deduzione naturale} è un metodo di ragionamento puramente sintattico. Si effettua stabilendo:
\begin{itemize}
    \item \textit{Assunzioni}; Ragionamenti temporanei.
    \item \textit{Derivazioni}; Dalle formule è possibile effettuare derivazioni attraverso determinate leggi. Ogni passaggio si dice \textbf{Passo di derivazione}.
    \item \textit{Scaricamento}; La chiusura delle assunzioni fatte all'inizio.
\end{itemize}
Tutte le derivazioni sono rappresentate graficamente tramite \textbf{alberi}, dove in cima stanno le \textit{foglie}, ovvero le assunzioni e alla base la \textit{radice}, che è la formula da derivare. Una derivazione è conclusa solamente quando ogni foglia è stata \textit{chiusa}, quindi scaricata.
\begin{definition}
    \textbf{Teorema}\par
    Una determinata formula $\psi$ si dice teorema se esiste una derivazione per cui l'insieme delle ipotesi $Hp[D] = \varnothing$, ovvero quando è vuoto poiché tutte le foglie sono state chiuse. Possono inoltre esistere più derivazioni per un singolo teorema.
\end{definition}
Passiamo ora alle \textbf{Regole deduttive}. Si tratta di algoritmi utili per la derivazione delle formule nella deduzione naturale. Si dividono in due tipi:
\begin{itemize}
    \item \textit{Introduzione}; Premesse collegate introducendo un connettivo logico.
    \item \textit{Eliminazione}; Dalle premesse è possibile rimuovere un connettivo logico.
\end{itemize}
Negli esercizi, le regole deduttive consentono di effettuare i passi di derivazione e sono rappresentati in forma di frazione. Sopra stanno le premesse, sotto la conclusione e a fianco la regola utilizzata. Le formule all'interno di parentesi quadre sono poi ipotesi assunte/scaricate da una determinata regola di derivazione, la quale dovrà essere appropriatamente numerata. Vedremo come fare insieme alla rappresentazione delle regole a nostra disposizione:
\begin{itemize}
    \item \textbf{Regole dell'implicazione} $(\to)$\par
    \begin{itemize}
        \item \textbf{Introduzione dell'implicazione}\par
        Se dall’ipotesi $\alpha$ segue $\beta$, con una certa derivazione D, allora si deriva $\alpha \to \beta$ tramite l’introduzione dell’implicazione.\par
        Questa regola assume/scarica come ipotesi l’implicante dell'implicazione, indicando quella formula tra le parentesi quadre e un indice.\par
        \begin{center}
            \Large$\frac{[\beta]^1}{\alpha \to \beta}I^1$
        \end{center}
        \item \textbf{Eliminazione dell'implicazione}\par
        
        \item \textbf{Indebolimento}\par
        
    \end{itemize}
    \item \textbf{Regole della congiunzione} $(\land)$\par
    \begin{itemize}
        \item \textbf{Introduzione dell'AND}\par

        \item \textbf{Eliminazione dell'AND}\par
        
    \end{itemize}
    \item \textbf{Regole del bottom} $(\bot)$\par
    \begin{itemize}
        \item \textbf{Riduzione ad assurdo (RAA)}\par

        \item \textbf{Eliminazione del bottom}\par
        
    \end{itemize}
    \item \textbf{Regole della disgiunzione} $(\lor)$\par
    \begin{itemize}
        \item \textbf{Introduzione dell'OR}\par

        \item \textbf{Eliminazione dell'OR}\par
        
    \end{itemize}
    \item \textbf{Regole della doppia implicazione} $(\leftrightarrow)$\par
    \begin{itemize}
        \item \textbf{Introduzione della doppia implicazione}\par

        \item \textbf{Eliminazione della doppia implicazione}\par
        
    \end{itemize}
    \item \textbf{Assiomi}\par
    
\end{itemize}


assiomi, prove indirette, derivabilità.