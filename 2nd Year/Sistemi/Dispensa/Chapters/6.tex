\section{Serie di Fourier}



\begin{comment}
	Lez27 - Sistemi
	
	--- Trasformata di Fourier, introduzione
	Ringrazia Fourier se puoi rappresentare i segnali come somme di integrali di sinusoidi.
	Partiremo dal segnale completo e potremo distinguere le varie componenti della somma.
	
	Insomma qua serve capire che cos'è, a cosa serve e come usarla. Sembra essere una lezione di introduzione con meno informazioni utili di quanto si pensi. Buona introduzione, ma non tanti appunti richiesti.
	
	Partire da un segnale e scomporlo in n componenti, che sarebbero i fasori.
	Capiamo che la somma dei fasori rappresenterà non perfettamente, ma quasi il segnale originale completo.
\end{comment}

%

\section{Trasformata di Fourier}

%

\section{Trasformate notevoli e proprietà della TdF}

%

\section{Campionamento e replicazione}

%

\section{Esercizi svolti}